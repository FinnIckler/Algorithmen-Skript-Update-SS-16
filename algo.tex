 % Copyright \copyright\ 2010  Volodymyr Piven und Alexander Peltzer.
% Es wird die Erlaubnis gegeben, dieses Dokument unter den Bedingungen der von der Free
% Software Foundation veröffentlichen  GNU Free
% Documentation License (Version 1.2 oder neuer) zu kopieren, verteilen und/oder
% zu verändern. Eine Kopie dieser Lizenz ist unter
% http://www.gnu.org/copyleft/fdl.txt erhältlich.
%
% Copyright der Aktualisierung und Überarbeitung \copyright\ 2013  Simon Kalt, Jan-Peter Hohloch, Tobias Fabritz
% Es wird die Erlaubnis gegeben, dieses Dokument unter den Bedingungen der von der Free
% Software Foundation veröffentlichen  GNU Free
% Documentation License (Version 1.2 oder neuer) zu kopieren, verteilen und/oder
% zu verändern. Eine Kopie dieser Lizenz ist unter
% http://www.gnu.org/copyleft/fdl.txt erhältlich.
%
% Zusätzlich muss jede Kopie/Aktualisierung wieder über die Seite
% der Fachschaft Informatik der Uni Tübingen
% den Studenten zur Verfügung gestellt werden
% http://www.fsi.uni-tuebingen.de/

\documentclass[a4paper, 10pt]{scrreprt}
%Finn ; Ich verwende LuaTeX, wegen Fonts, wenn man was pdfLaTeX verwendet, dann standard Times 
\usepackage{ifluatex}
\ifluatex
\usepackage{fontspec}
\setmainfont{Roboto}[BoldFont = Source Code Pro]
\renewcommand{\B}{\mathbb{B}} % |B
\else
\usepackage{times}
\newcommand{\B}{\mathbb{B}} % |B
\fi
\usepackage{etex}
\usepackage{ngerman}
\usepackage{epsfig}
\usepackage{amsmath}
\usepackage{amssymb}
\usepackage{alltt}
\usepackage{enumerate}
\usepackage{color}
\usepackage{cancel}	%Kürzen in Brüchen
\usepackage{remreset} % Section-Zähler soll nicht zurückgesetzt werden
\usepackage{tikz}
\usetikzlibrary{trees,arrows,automata,calc,shapes,chains,positioning,decorations}
\usepackage{pgf}
\let\SavedRightarrow=\Rightarrow
  \usepackage{marvosym}
\let\Rightarrow=\SavedRightarrow
\usepackage{makeidx}
\usepackage[all]{xy}

\renewcommand{\glqq}{„}
\renewcommand{\grqq}{“}
\newcommand{\q}[1]{\glqq#1\grqq}

\newcommand{\LO}{\ensuremath{\mathcal{O}}}
\newcommand{\set}[1]{\{#1\}}


% Bäume
\usepackage{qtree}
\renewcommand{\qroofpadding}{0.6cm} % Makes roofs with math-labels work
\qroofx=2
\qroofy=1

\usepackage{cancel}
\usepackage{wasysym}
\usepackage[linktocpage,colorlinks=true,
           filecolor=rltgreen,
           linkcolor=rltred,
           pdftitle={Algorithmen},
           pdfsubject={Algorithmen},
           pdfkeywords={Algorithmen, skript, Kaufmann},
           pdfproducer={luaLaTeX},
           pagebackref,
           pdfpagemode=None,
           bookmarksopen=true]
           {hyperref}
% Verlinkt Abbildungen etc. am oberen Rand
\usepackage{hypcap}

%%%%%%%%%%%%% Quellcode einbinden%%%%%%%%%%%%%%
\usepackage{listings}
% \lstset{numbers=left, numberstyle=\tiny, numbersep=5pt}
%\lstset{language=C}
%%%%%%%%%%%%%%%%%%%%%%%%%%%%%%%%%%%%%%%%%%%%%%%%
\usepackage{algorithm}
\usepackage{algorithmicx}
\usepackage{algpseudocode} %Pseudocode
\makeatletter
\renewcommand{\ALG@name}{Algorithmus}
\makeatother
\renewcommand{\listalgorithmname}{Algorithmenverzeichnis}
\algrenewcommand{\algorithmiccomment}[1]{\hfill// #1}

% Autoref-Namen
\def\algorithmautorefname{Algorithmus}
\def\figureautorefname{Abbildung}



% \var{variable} sets a variable in italics in mathmode
\newcommand{\var}[1]{\ensuremath{\text{\emph{#1}}}}
\newcommand{\True}{\ensuremath{\text{\textbf{true}}}}
\newcommand{\False}{\ensuremath{\text{\textbf{false}}}}
\newcommand{\algto}{\textbf{to}}
\newcommand{\malgto}{\ensuremath{\text{ \algto{} }}}

%
\renewcommand{\labelitemi}{\guillemotright}
\renewcommand{\implies}{\Rightarrow} % better =>
\renewcommand{\iff}{\Leftrightarrow} % better <=>
\newcommand{\define}[2]{\emph{#1}: #2}

%\usepackage{txfonts}
\usepackage{scrtime}
\definecolor{rltred}{rgb}{0.75,0,0}
\definecolor{rltgreen}{rgb}{0,0.5,0}
\definecolor{rltblue}{rgb}{0,0,0.75}

% F�r Absolut und Norm
\providecommand{\abs}[1]{\lvert#1\rvert} 
\providecommand{\norm}[1]{\lVert#1\rVert}

%\newcommand{\qed}{\mbox{}\hfill\ensuremath{\Box}}
% Definition von Operatoren
\DeclareMathOperator{\car}{char}
\DeclareMathOperator{\real}{Re}
\DeclareMathOperator{\imag}{Im}
\DeclareMathOperator{\Id}{Id}
\DeclareMathOperator{\sgn}{sgn}
\DeclareMathOperator{\const}{const}
\DeclareMathOperator{\Rang}{Rang}
\DeclareMathOperator{\im}{im}
\DeclareMathOperator{\Hom}{Hom}
\DeclareMathOperator{\Ker}{Ker}
\DeclareMathOperator{\coker}{coker}
\DeclareMathOperator{\Vol}{Vol}
\DeclareMathOperator{\Abb}{Abb}
\DeclareMathOperator{\ggT}{ggT}
\DeclareMathOperator{\Arsinh}{Arsinh}
\DeclareMathOperator{\id}{id}
\DeclareMathOperator{\codim}{codim}
\DeclareMathOperator{\Alt}{Alt}
\DeclareMathOperator{\End}{End}
\DeclareMathOperator{\Spur}{Spur}
\DeclareMathOperator{\Bil}{Bil}
\DeclareMathOperator{\Gl}{Gl}
\DeclareMathOperator{\Int}{int}
\DeclareMathOperator{\dist}{dist}
\DeclareMathOperator{\Bild}{Bild}
\DeclareMathOperator{\grad}{grad}
\DeclareMathOperator{\Aff}{Aff}
\DeclareMathOperator{\rot}{rot}
\DeclareMathOperator{\Div}{div}
\DeclareMathOperator{\Aut}{Aut}
\DeclareMathOperator{\supp}{supp}
\DeclareMathOperator{\Ob}{Ob}
\DeclareMathOperator{\dom}{dom}
\DeclareMathOperator{\cod}{cod}
\DeclareMathOperator{\Sets}{Sets}
\DeclareMathOperator{\Grps}{Grps}
\DeclareMathOperator{\vol}{vol}
\DeclareMathOperator{\diam}{diam}
\DeclareMathOperator{\card}{card}
\DeclareMathOperator{\loc}{loc}
%\DeclareMathOperator{\var}{var}
\DeclareMathOperator{\ba}{ba}
\DeclareMathOperator{\ca}{ca}
\DeclareMathOperator{\rca}{rca}
\DeclareMathOperator{\rba}{rba}
\DeclareMathOperator{\conv}{conv}
\DeclareMathOperator{\Span}{span}

% \usepackage{empheq}
% \let\SavedRightarrow=\Rightarrow
%   \usepackage{marvosym}
% \let\Rightarrow=\SavedRightarrow
%\usepackage[hyperref,amsmath,thmmarks,thref]{ntheorem}
\usepackage{amsthm}

% \makeatletter
% \newtheoremstyle{numberempty}%
% 	{\item[\theorem@headerfont \llap{##2}\hskip\labelsep \theorem@separator]}%
% 	{\item[\theorem@headerfont \llap{##2}\hskip\labelsep ##3\theorem@separator]}
% \makeatother


%%
%%    Neue Umgebungen
%%


%%
%%
%%

% Definition der Umgebungen
%\theoremheaderfont{\normalfont\bfseries}
%\setlength{\theorempreskipamount}{3ex}
\theoremstyle{marginbreak}
\newtheorem{lemma}{Lemma}[section]
\newtheorem{satz}[lemma]{Satz}
\theoremstyle{nonumberbreak}
\newtheorem{nnsatz}{Satz}


\theoremstyle{margin}
\newtheorem{theorem}[lemma]{Theorem}
\newtheorem{korollar}[lemma]{Korollar}

%\theorembodyfont{\upshape}
\newtheorem{example}[lemma]{Beispiel}
\newtheorem{examples}[lemma]{Beispiele}
\newtheorem{definition}[lemma]{Definition}

\newtheorem{folgerung}[lemma]{Folgerung}
\newtheorem{bemerkung}[lemma]{Bemerkung}
\newtheorem{problem}[lemma]{Problem}
\newtheorem{exkurs}[lemma]{Exkurs}
\newtheorem{motivation}[lemma]{Motivation}
\newtheorem{erinnerung}[lemma]{Erinnerung}
\newtheorem{verfahren}[lemma]{Verfahren}
\newtheorem{dummy}[lemma]{}
\newtheorem{anwendung}[lemma]{Anwendung}
\newtheorem{sundd}[lemma]{Satz und Definition}

\theoremstyle{numberempty}
\newtheorem{nokeyword}[lemma]{}

\theoremstyle{nonumberplain}
\newtheorem{bsp}{Beispiel}
\newtheorem{annahme}{Annahme}
\newtheorem{bspe}{Beispiele}
\newtheorem{nnbemerkung}{Bemerkung}
\newtheorem{nnziel}{Ziel}
\newtheorem{nnanwendung}{Anwendung}
\newtheorem{nnproblem}{Problem}
\newtheorem{nnfolgerung}{Folgerung}
\newtheorem{nnkorollar}{Korollar}
\newtheorem{nnmotivation}{Motivation}
\newtheorem{notmiss}{Notationsmissbrauch}
\newtheorem{nndefinition}{Definition}
\newtheorem{nnlemma}{Lemma}
\newtheorem{nnbeobachtung}{Beobachtung}
\newtheorem{nnfragen}{Fragen}
\newtheorem{nnfrage}{Frage}
\newtheorem{nnnotation}{Notation}
\newtheorem{nnbehauptung}{Behauptung}
\newtheorem{nndummy}{}

\renewcommand{\proofname}{Beweis}




%\theoremheaderfont{\sc}
\theoremstyle{nonumberplain}
%\theoremsymbol{\ensuremath{\diamond}}
\newtheorem{zbeweis}{Zwischenbeweis}

%\theoremlisttype{allname}

% Damit hyperref auch zufrieden ist:
% \makeatletter
% \providecommand*{\toclevel@sundd}{0}
% \providecommand*{\toclevel@example}{0}
% \providecommand*{\toclevel@proof}{0}
% \providecommand*{\toclevel@lemma}{0}
% %\providecommand*{\toclevel@satz}{0}
% \providecommand*{\toclevel@theorem}{0}
% \providecommand*{\toclevel@korollar}{0}
% \providecommand*{\toclevel@definition}{0}
% \providecommand*{\toclevel@bsp}{0}
% \providecommand*{\toclevel@bspe}{0}
% \providecommand*{\toclevel@axiom}{0}
% \providecommand*{\toclevel@bemerkung}{0}
% \providecommand*{\toclevel@folgerung}{0}
% \providecommand*{\toclevel@nnbemerkung}{0}
% \providecommand*{\toclevel@problem}{0}
% \providecommand*{\toclevel@nnproblem}{0}
% \providecommand*{\toclevel@notmiss}{0}
% \providecommand*{\toclevel@nnfolgerung}{0}
% \providecommand*{\toclevel@nbeweis}{0}
% \providecommand*{\toclevel@nnkorollar}{0}
% \providecommand*{\toclevel@nnmotivation}{0}
% \providecommand*{\toclevel@exkurs}{0}
% \providecommand*{\toclevel@nndefinition}{0}
% \providecommand*{\toclevel@nnlemma}{0}
% \providecommand*{\toclevel@motivation}{0}
% \providecommand*{\toclevel@erinnerung}{0}
% \providecommand*{\toclevel@dummy}{0}
% \providecommand*{\toclevel@nnbeobachtung}{0}
% \providecommand*{\toclevel@examples}{0}
% \providecommand*{\toclevel@nnfragen}{0}
% \providecommand*{\toclevel@nnfrage}{0}
% \providecommand*{\toclevel@verfahren}{0}
% %\providecommand*{\toclevel@nnsatz}{0}
% \providecommand*{\toclevel@nnnotation}{0}
% \providecommand*{\toclevel@anwendung}{0}
% \providecommand*{\toclevel@zbeweis}{0}
% \providecommand*{\toclevel@nndummy}{0}
% \providecommand*{\toclevel@nnziel}{0}
% \providecommand*{\toclevel@nnbehauptung}{0}

% \makeatother



% Einige Anstrengungen, um den § vor die Section-Nummer zu stellen
% \renewcommand{\thesection} allein führt zu einem Konflikt mit ntheorem-hyper
%%%%%%%%%%%%%%%%%%%%%%%%%%%%%%%%%%%%%%%%%%%%%%%%%%%%%%%%%%%%%%%%%%%%%%%%%%%
%\makeatletter
%\def\@seccntformat#1{\@ifundefined{#1@cntformat}%
%{\csname the#1\endcsname\quad}% default
%{\csname #1@cntformat\endcsname}% individual control
%}
%\def\chapter@cntformat{§\@arabic\c@section\quad}
%\@removefromreset{section}{chapter}		% Section-Zähler soll durchlaufen
%\makeatother
%\renewcommand{\thesection}{\arabic{section}}
%%%%%%%%%%%%%%%%%%%%%%%%%%%%%%%%%%%%%%%%%%%%%%%%%%%%%%%%%%%%%%%%%%%%%%%%%%%

\newcommand{\K}{\mathbb{K}} % |K
\newcommand{\Q}{\mathbb{Q}} % (Q
\newcommand{\R}{\mathbb{R}} % |R
\newcommand{\C}{\mathbb{C}} % (C
\newcommand{\N}{\mathbb{N}} % |N
\newcommand{\Z}{\mathbb{Z}} % Z/
\newcommand{\gdw}{genau dann, wenn }
\newcommand{\whitedot}{\color{white}.\color{black}}
\newcommand{\md}{\text{ mod }}
\newcommand{\btl}{\left|}
\newcommand{\bbtl}{\left\|}
\newcommand{\bbtr}{\right\|}
\newcommand{\btr}{\right|}
\newcommand{\Lsg}{\mathbb{L}} % L für Lösungsmenge
\newcommand{\eps}{\varepsilon}
\newcommand{\I}{\mathcal{I}}
\newcommand{\nullvec}{\overrightarrow{0}}
\newcommand{\Lra}{\Leftrightarrow}
\newcommand {\bpm} { \begin{pmatrix} }
\newcommand {\epm} { \end{pmatrix} }
\newcommand{\Ra}{\Rightarrow}
\newcommand{\ra}{\rightarrow}
               
              


%%%%%%%%%%%%%%%%%%%%%%%%%%%%%%%%%%%%%%%%%%

\setcounter{secnumdepth}{2}
\setlength{\parindent}{0pt}
\setlength{\parskip}{1ex}
\newcommand{\Index}[1]{#1\index{#1}}
\makeindex

\title{Algorithmen}
\author{Prof. M. Kaufmann\thanks{Mitschrift: Volodymyr Piven, Alexander Peltzer, Korrektur: Demen Güler, Sebastian Nagel, Aktualisierungen und Korrektur 2013 von Tobias Fabritz, Simon Kalt, Jan-Peter Hohloch, kleine Erweiterung 2015 von Lennard Boden und David-Elias Künstle. Korrigiert und erweitert 2016 von Finn Ickler}}
\date{Sommersemester 2016\\
\begin{scriptsize}Stand: \today , \thistime \end{scriptsize} \bigskip\\
 \Dontwash \quad \NoTumbler \quad \NoIroning }


\begin{document}
\maketitle
\tableofcontents
\newpage

% Copyright \copyright\ 2010  Volodymyr Piven und Alexander Peltzer.
% Es wird die Erlaubnis gegeben, dieses Dokument unter den Bedingungen der von der Free
% Software Foundation veröffentlichen  GNU Free
% Documentation License (Version 1.2 oder neuer) zu kopieren, verteilen und/oder
% zu verändern. Eine Kopie dieser Lizenz ist unter
% http://www.gnu.org/copyleft/fdl.txt erhältlich.
%
% Copyright der Aktualisierung und Überarbeitung \copyright\ 2013  Simon Kalt, Jan-Peter Hohloch, Tobias Fabritz
% Es wird die Erlaubnis gegeben, dieses Dokument unter den Bedingungen der von der Free
% Software Foundation veröffentlichen  GNU Free
% Documentation License (Version 1.2 oder neuer) zu kopieren, verteilen und/oder
% zu verändern. Eine Kopie dieser Lizenz ist unter
% http://www.gnu.org/copyleft/fdl.txt erhältlich.
%
% Zusätzlich muss jede Kopie/Aktualisierung wieder über die Seite
% der Fachschaft Informatik der Uni Tübingen
% den Studenten zur Verfügung gestellt werden
% http://www.fsi.uni-tuebingen.de/

\section*{Vorwort}
\addcontentsline{toc}{chapter}{Vorwort}
Beim vorliegenden Text handelt es sich um eine inoffizielle Mitschrift.
Als solche kann sie selbstverständlich Fehler enthalten und ist daher für
Übungsblätter und Klausuren nicht zitierfähig.

Korrekturen und Verbesserungsvorschläge sind jederzeit willkommen und können als Request im Repository\footnote{SS13:\url{https://bitbucket.org/urm3l/algoskript/}}\footnote{SS15:\url{https://bitbucket.org/de_kuenstle/algoskript-ss15}} eingereicht werden. Alternativ kann gerne unter Beachtung der Lizenz ein Fork erstellt werden.

Der Text wurde mit \LaTeX2e in Verbindung mit dem \AmS-\TeX-Paket für die mathematischen
Formeln gesetzt.

{\scriptsize
Copyright \copyright\ 2010  Volodymyr Piven und Alexander Peltzer.
Es wird die Erlaubnis gegeben, dieses Dokument unter den Bedingungen der von der Free
Software Foundation veröffentlichen  GNU Free
Documentation License (Version 1.2 oder neuer) zu kopieren, verteilen und/oder
zu verändern. Eine Kopie dieser Lizenz ist unter
http://www.gnu.org/copyleft/fdl.txt erhältlich.
}
\section*{Einleitung}
\addcontentsline{toc}{chapter}{Einleitung}
Dozent: 
\begin{itemize}
\item \index{Kaufmann, Prof. Dr.}Professor Dr. Michael Kaufmann \\
      Tel.: +49-7071-2977404 \\
      Email: mk@informatik.uni-tuebingen.de
\end{itemize}
Vorlesungszeiten:
\begin{itemize}
\item Dienstag 14-16 Uhr 
\item Donnerstag 10-12 Uhr 
\end{itemize}
Literatur:
\begin{itemize}
 \item Algorithmen - Kurz gefasst, Schöning, Springer
\end{itemize}


% Copyright \copyright\ 2010  Volodymyr Piven und Alexander Peltzer.
% Es wird die Erlaubnis gegeben, dieses Dokument unter den Bedingungen der von der Free
% Software Foundation veröffentlichen  GNU Free
% Documentation License (Version 1.2 oder neuer) zu kopieren, verteilen und/oder
% zu verändern. Eine Kopie dieser Lizenz ist unter
% http://www.gnu.org/copyleft/fdl.txt erhältlich.
%
% Copyright der Aktualisierung und Überarbeitung \copyright\ 2013  Simon Kalt, Jan-Peter Hohloch, Tobias Fabritz
% Es wird die Erlaubnis gegeben, dieses Dokument unter den Bedingungen der von der Free
% Software Foundation veröffentlichen  GNU Free
% Documentation License (Version 1.2 oder neuer) zu kopieren, verteilen und/oder
% zu verändern. Eine Kopie dieser Lizenz ist unter
% http://www.gnu.org/copyleft/fdl.txt erhältlich.
%
% Zusätzlich muss jede Kopie/Aktualisierung wieder über die Seite
% der Fachschaft Informatik der Uni Tübingen
% den Studenten zur Verfügung gestellt werden
% http://www.fsi.uni-tuebingen.de/

\chapter{Komplexität}
    \section{$\LO$-Notation}
    $f(n) = \LO(g(n))$ (obere Schranke), wenn 
    $$
        \exists \ c ,\ n_0 > 0 \ \forall \ n \geq n_0: 0 \leq f(n) \leq c \cdot g(n)
    $$\\
    $f(n) = \Omega(g(n))$ (untere Schranke), wenn
    $$
    		\exists \ c ,\ n_0 > 0 \ \forall \ n \geq n_0: 0\leq c\cdot g(n)\leq f(n)
    $$\\
    $f(n) = \Theta(g(n))$ (scharfe Schranke), wenn
    $$
    		\exists \ c_1,c_2 ,\ n_0 > 0 \ \forall \ n \geq n_0: 0\leq c_1\cdot g(n)\leq f(n)\leq c_2\cdot g(n)
    $$
    \textit{Anmerkung:} \\
 	 Mit $f(n) = \LO(g(n))$ ist gemeint $f(n) \in \LO(g(n))$ \\
     $\Theta(g(n))=\LO(g(n))\cap \Omega(g(n))$
    
    \section{Nützliche Rechenregeln}
        Logarithmen:
        
        \begin{align*}
            \log(a \cdot b) &= \log a + \log b \\
            \log_a b &= \frac{\log b}{\log a} \\
            \log a^b &= b \cdot \log a \\
            b^{\log_b a} &= a 
        \end{align*}

        Andere nützliche Regeln:
        \begin{itemize}
            \item Stirling-Approximation:
            $$
                n! = \sqrt{2 \pi n} \cdot \left(\frac{n}{e}\right)^n \left(1+\LO\left(\frac{1}{n}\right)\right)
            $$
            Eine einfachere Abschätzung ist:
            $$
                \left(\frac{n}{2}\right)^{\frac{n}{2}} \leq n! \leq n^n
            $$
            \item Gauß'sche Summenformel:
            $$
                \sum \limits_{k=1}^n k = \frac{n(n+1)}{2}
            $$
            \item geometrische Reihe:
            \begin{align*}
                &\sum \limits_{k=0}^n q^k = \frac{q^{n+1}-1}{q-1} \\
                &\sum \limits_{k=0}^\infty q^k = \frac{1}{1-q} \ \ \text{  für } \abs{q} < 1
            \end{align*}
            Also zB für $q= \frac{1}{2}: 1 + \frac{1}{2} + \frac{1}{4} + \frac{1}{8} + \ldots = 2$ 
            \item Ebenfalls gilt (Ableitung):
            $$
                \sum \limits_{k=0}^\infty k \cdot q^{k-1} = \frac{1}{(1-q)^2} \ \ \text{  für } \abs{q} < 1
            $$ 
            \item n-te harmonische Zahl:
            $$
                H_n = 1 + \frac{1}{2} + \frac{1}{3} + \frac{1}{4} + \ldots + \frac{1}{n} = \sum \limits_{k=1}^n \frac{1}{k} = \ln n + \LO(1)
            $$
        \end{itemize}


% Copyright \copyright\ 2010  Volodymyr Piven und Alexander Peltzer.
% Es wird die Erlaubnis gegeben, dieses Dokument unter den Bedingungen der von der Free
% Software Foundation veröffentlichen  GNU Free
% Documentation License (Version 1.2 oder neuer) zu kopieren, verteilen und/oder
% zu verändern. Eine Kopie dieser Lizenz ist unter
% http://www.gnu.org/copyleft/fdl.txt erhältlich.
%
% Copyright der Aktualisierung und Überarbeitung \copyright\ 2013  Simon Kalt, Jan-Peter Hohloch, Tobias Fabritz
% Es wird die Erlaubnis gegeben, dieses Dokument unter den Bedingungen der von der Free
% Software Foundation veröffentlichen  GNU Free
% Documentation License (Version 1.2 oder neuer) zu kopieren, verteilen und/oder
% zu verändern. Eine Kopie dieser Lizenz ist unter
% http://www.gnu.org/copyleft/fdl.txt erhältlich.
%
% Zusätzlich muss jede Kopie/Aktualisierung wieder über die Seite
% der Fachschaft Informatik der Uni Tübingen
% den Studenten zur Verfügung gestellt werden
% http://www.fsi.uni-tuebingen.de/
\chapter{Rekursionen}
\begin{itemize}
\item
Merge-Sort: 
$$
T(n) = \underbrace{2 \cdot T(\frac{n}{2})}_{\text{2 Teilprobleme}} + \underbrace{n}_{\text{Aufwand des Aufteilens}} \text{ und } T(1)=1
$$
\item 
Schaltkreise: $C(n) \leq C(\frac{2}{3}n) + \frac{n}{3}$ 
\item
parallele Algorithmen: $T(n) = T(\frac{9}{10}) + \LO(\log n)$ 
\item
Selection: $T(n) \leq \frac{2}{n} \cdot \sum \limits_{k=\frac{n}{2}}^{n-1} T(k) + \LO(n)$ 
\end{itemize}

\section{3 Methoden zur Berechnung}

\subsection{Schätzen und Beweisen}
\begin{itemize}
\item
$T(n) = 2 \cdot T\left( \frac{n}{2}\right) + n$ \\
Schätzen: $T(n) = \LO(n \log n)$ \\
Beweisen: Behauptung $T(n) \leq c \cdot n \log n$ für c groß genug. \\ %Warum +1?
Induktion: \\
Induktionsanfang: 
$$
T(1) = 1\ \checkmark
$$ 
Induktionsschritt: \\
Es gilt 
$$\begin{aligned}
T(n) &\leq 2 \cdot c \cdot \frac{n}{2} \cdot \log \left(\frac{n}{2}\right) + n \\
&= 2 \cdot c \cdot \frac{n}{2} \cdot (\log n - \log 2) + n \\
&= c \cdot n \cdot ((\log n) -1) + n \leq c \cdot n \log n 
\end{aligned} $$
für $c \geq 1$
\item
$C(n) \leq C ( \frac{2}{3} n) + \frac{n}{3}$ \\
Schätzen: $C(n) = \LO(n)$
\begin{proof}
$C(n) \leq d \cdot n$ für konstante d 
\[C(1) = 1 \leq d \text{ für }d \geq 1\]
$$C(n) \leq d \cdot \frac{2}{3} n + \frac{n}{3} \leq d(\frac{2}{3}n + \frac{n}{3})  \text{ für } d \geq 1 =d \cdot n$$ \end{proof}

\item
$A(n) = 2 \cdot A(\frac{n}{2}) + \sqrt{n}$ \\
Schätzen:$A(n) = \LO(n)$ \\
Versuchter Beweis:\\
$A(n) \leq c \cdot n$ für c groß.\\
$A(1) = 1 $ \\
$A(n) = 2 \cdot c \cdot \frac{n}{2} + \sqrt{n} \nleq c \cdot n$ so nicht!
\end{itemize}  

\subsection{Ausrechnen} 
\begin{align*}
A(n) &= 2 \cdot A\left(\frac{n}{2}\right) + \sqrt{n} \\
&= 2\cdot \left( 2 \cdot A\left(\frac{n}{4}\right) + \sqrt{\frac{n}{2}} \right) + \sqrt{n} \\
&= 4 \cdot A\left(\frac{n}{4}\right) + 2 \cdot \sqrt{\frac{n}{2}} + \sqrt{n} \\
&= 8 \cdot A\left(\frac{n}{8}\right) + 4 \sqrt{\frac{n}{4}} + 2 \sqrt{\frac{n}{2}} + \sqrt{n} \\
&\ldots \\
&= 2^i \cdot A \left( \frac{n}{2^i}\right) + \sum \limits_{j=0}^{i-1} 2^j \cdot \sqrt{\frac{n}{2^j}}  \ \tag{i-ter Schritt}\\
&= 2^i \cdot A\left(\frac{n}{2^i}\right) + \sqrt{n} \cdot \sum \limits_{j=0}^{i-1} 2^{\frac{j}{2}} 
\end{align*}
Schritt ist durch Induktion zu zeigen:
\begin{align*}
&= 2^i \cdot \left(2\cdot A\left(\frac{n}{2^{i+1}}\right) +\sqrt{\frac{1}{2^i}}\right) 
+ \sqrt{n} \cdot \sum \limits_{j=0}^{i-1} 2^{\frac{j}{2}}\\
&=2^{i+1}\cdot A\left(\frac{n}{2^{i+1}}\right) + \sqrt{n} \cdot \sum \limits_{j=0}^{i} 2^{\frac{j}{2}} \ \checkmark
\end{align*}

Sei $i = \log n$, da dies der maximale Fall ist, wenn $n$ eine Zweierpotenz: 
\begin{align*}
&2^{\log n} \cdot A(1) + \sqrt{n} \sum \limits_{j=0}^{\log n-1} \sqrt{2}^j \\
&= n + \sqrt{n} \cdot \LO(\sqrt{n}) \\
&= \LO(n)
\end{align*}

\subsection{Master Theorem}
Löse $T(n) = a \cdot T(\frac{n}{b}) + f(n) \ \ (a \geq 1,\ b > 1,\  f(n) \geq 0)$ \\
Beispielsweise für Merge-Sort: $a=b=2, f(n) = n$ \\

\begin{satz}
Sei $a \geq 1, b > 1$ konstant, $f(n) , T(n)$ nicht negativ mit 
$$
T(n) = a \cdot T \left(\frac{n}{b}\right) + f(n)
$$
wobei $\frac{n}{b}$ für $\lceil \frac{n}{b} \rceil$ oder $\lfloor \frac{n}{b} \rfloor$ steht. Dann gelten
\begin{enumerate}[1.]
\item Für $f(n) = O\left(n^{\log_b a - \epsilon}\right)\ , \epsilon > 0 \Rightarrow T(n) = \Theta(n^{\log_b a})$ 
\item Für $f(n) = \Theta \left(n^{\log_b a}\right) \Rightarrow T(n) = \Theta (n^{\log_b a} \cdot \log n)$ 
\item Für $f(n) = \Omega \left(n^{\log_b a + \epsilon}\right)\ , \epsilon > 0$ und $a \cdot f(\frac{n}{b}) \leq c \cdot f(n)$\\
für $c < 1$ und $n$ genügend groß, so ist: $T(n) = \Theta(f(n))$
\end{enumerate}
\end{satz}

Beispiel:
\begin{itemize}
\item
$T(n) = 9 \cdot T(\frac{n}{3}) +n$, $a=9, b=3, f(n) = n$ \\
$\log_3 9 = 2 \Ra $Fall 1 $\Ra T(n) = \Theta (n^2)$
\item
$T(n) = T(\frac{n}{2}) +1$, $a=1, b=2, f(n) = 1$ \\
$\log_b a= \log_2 1  = 0 \Ra $Fall 2 $\Ra T(n) = \Theta(\log n)$
\item $T(n) = 4 \cdot T(\frac{n}{2}) + n^4\\
f(n) = n^4 \geq \Theta(n^{2+2}) \checkmark\\
\frac{n^4}{4} = 4 \cdot (\frac{n}{2})^4 \leq c \cdot n^4 \checkmark$ Fall 3\\
$\Rightarrow T(n) = \Theta(n^4)$
\item
$T(n) = 2 \cdot T(\frac{n}{2}) + n \log n$, $a=2, b=2, f(n) = n \log n $ \\
$log_2 2 = 1$ \\
$n \log n < n^{1+\epsilon} \Ra $ nicht anwendbar! (da \q{zwischen} Fall 2 und 3!)
\end{itemize}
$T(n) = a \cdot T(\frac{n}{b}) + f(n)$, $a\geq 1,\ b > 1,\ f(n) \geq 0$ \\
\emph{Beweis}:\\
Annahme: $n=b^i, i \in \N$ \\
das heißt, Teilprobleme haben Größe $b^i, b^{i-1}, \ldots, b, 1$ \\
\begin{lemma} 
Sei $a,b \geq 1, f(n) \geq 0, n=b^i, i \in \N$ mit 
$$
T(n) = 
\begin{cases} 
\Theta (1) 					& \text{\normalfont für } n=1 \\ 
a \cdot T(\frac{n}{b}) + f(n) 	& \text{\normalfont für } n=b^i 
\end{cases}
$$

Dann gilt $T(n) = \Theta (n^{\log_b a}) + \sum \limits_{j=1}^{(\log_b n)-1} a^j f(\frac{n}{b^j})$

\begin{proof}
\begin{align*} 
T\left(n\right) &= f\left(n\right) + a \cdot T\left(\frac{n}{b}\right) \\
&= f\left(n\right) + a \left(f\left(\frac{n}{b}\right) + a \cdot T\left( \frac{n}{b^2}\right)\right) \\
&= f\left(n\right) + a \cdot f\left(\frac{n}{b}\right) + a^2 \cdot T\left(\frac{n}{b^2}\right) \\
&= f\left(n\right) + a \cdot f\left(\frac{n}{b}\right) 
+ a^2 \cdot T\left(\frac{n}{b^2}\right) + a^3 \cdot T\left(\frac{n}{b^3}\right) \\
&= \dots \\
&= f\left(n\right) + a \cdot f\left(\frac{n}{b}\right) + a^2 \cdot f\left(\frac{n}{b^2}\right) 
+ \ldots + a^i T\left(\frac{n}{b^i}\right) \\
&= f\left(n\right) + a \cdot f\left(\frac{n}{b}\right) + \ldots + \underbrace{a^{\log_b n} 
\cdot T\left(1\right)}_{=  \Theta\left(n^{\log_b a}\right)} \\
&= \Theta\left(n^{\log_b a}\right) 
+ \sum \limits_{j=0}^{\left(\log_b n\right)-1} a^j f\left(\frac{n}{b^j}\right)
\end{align*}
\end{proof}
\end{lemma}
\begin{figure}[htp]
\centering
\begin{tikzpicture}[->,node distance=3cm,tree node/.style={draw=none,fill=none}]
  \node[tree node] (1) {$f(n)$};
  \node[tree node] (2) [below left of=1] {$f(n/b)$};
  \node[tree node] (3) [below right of=1] {$f(n/b)$};
  \node[tree node] (4) [below left of=2] {$f(n/b^2)$};
  \node[tree node] (5) [below right of=3] {$f(n/b^2)$};
  \node[tree node] (6) at ($(2)!0.5!(3)$) {\dots $a$-mal \dots};
  \node[tree node] (7) [right of=5] {$a^2 \cdot f(n/b^2)$};
  \node[tree node] (8) [above of=7, yshift=-0.88cm] {$a \cdot f(n/b)$};
  \node[tree node] (9) [above of=8, yshift=-0.88cm] {$f(n)$};
  
  \path (1) edge node [left] {} (2)
            edge node [right] {} (3);
  \path (2) edge node [left] {} (4);
  \path (3) edge node [right] {} (5);
\end{tikzpicture}

\caption{Rekursionsbaum} % Bessere Caption?
\label{diag1:recursion-tree}
\end{figure}
3 Fälle, je nachdem wie groß $f(n)$ ist. \\
\textit{Anschaulich:}\\
Entweder dominiert die Wurzel (Fall 3) oder das Blattlevel (Fall 1) 
oder es fallen alle Terme gleich ins Gewicht (Fall 2, $\log_b n$ Terme) 
\begin{lemma}
Sei $a, b \geq 1$, $f(n)$ definiert auf Potenzen von b und sei $g(n) = \sum \limits_{j=0}^{\log_b n -1} a^j f\left(\frac{n}{b^j}\right)$
Ist $f(n) = \LO\left(n^{\log_b a - \epsilon}\right)$, $\epsilon \geq 0$, so ist $g(n) = \LO\left(n^{\log_b a}\right)$.

\begin{proof}
Es gilt:
\begin{align*}
f\left(\frac{n}{b^j}\right) &= \LO \left(\left(\frac{n}{b^j}\right)^{\log_b a -\epsilon}\right) \\
\Ra g(n) &= \LO \left(\sum \limits_{j = 0}^{\log_b n-1} a^j \cdot \left(\frac{n}{b^j}\right)^{\log_b a - \epsilon} \right) \\
&= \LO \left( n^{\log_b a - \epsilon} 
		\cdot \sum \limits_{j=0}^{\log_b n-1} \left(\frac{a \cdot b^\epsilon}{b^{\log_b a}}\right)^j \right) \\
&= \LO \left( n^{\log_b a - \epsilon} \cdot \sum \limits_{j=0}^{\log_b n-1} \left(b^\epsilon\right)^j \right) \tag{geom. Reihe}\\
&= \LO \left( n^{\log_b a - \epsilon} \cdot \frac{\left(b^\epsilon\right)^{\log_b n -1}}{b^\epsilon -1}\right)\\
&= \LO \left( n^{\log_b a - \epsilon} \cdot \underbrace{\frac{n^{\epsilon -1}}{b^\epsilon -1}}_{ \in\LO( n^\epsilon) }\right)\\ 
&= \LO \left( n^{\log_b a - \epsilon} \cdot n^\epsilon \right) = \LO \left(n^{\log_b a} \right)
\end{align*}    
\end{proof}
\end{lemma}


% Copyright \copyright\ 2010  Volodymyr Piven und Alexander Peltzer.
% Es wird die Erlaubnis gegeben, dieses Dokument unter den Bedingungen der von der Free
% Software Foundation veröffentlichen  GNU Free
% Documentation License (Version 1.2 oder neuer) zu kopieren, verteilen und/oder
% zu verändern. Eine Kopie dieser Lizenz ist unter
% http://www.gnu.org/copyleft/fdl.txt erhältlich.
%
% Copyright der Aktualisierung und Überarbeitung \copyright\ 2013  Simon Kalt, Jan-Peter Hohloch, Tobias Fabritz
% Es wird die Erlaubnis gegeben, dieses Dokument unter den Bedingungen der von der Free
% Software Foundation veröffentlichen  GNU Free
% Documentation License (Version 1.2 oder neuer) zu kopieren, verteilen und/oder
% zu verändern. Eine Kopie dieser Lizenz ist unter
% http://www.gnu.org/copyleft/fdl.txt erhältlich.
%
% Zusätzlich muss jede Kopie/Aktualisierung wieder über die Seite
% der Fachschaft Informatik der Uni Tübingen
% den Studenten zur Verfügung gestellt werden
% http://www.fsi.uni-tuebingen.de/

\chapter{Suchen in geordneten Mengen}

    Gegeben ist ein Universum $U$, mit linearer Ordnung $(U, <)$. Sei nun $S \subseteq U$ die Menge der zu verwaltenden Schlüssel. Wir betrachten nun verschiedene Algorithmen in $S$ ein Element zu suchen.
    \begin{enumerate}[1.] 
        \item \emph{lineare Suche:} $S$ liege als geordnetes Array vor. Nun sei next, das Element, das als nächstes betrachtet wird. \\
        Starte mit next $\leftarrow~0$\\
        Schleife: next $\leftarrow$ next $+~1$\\
        Abfrage: Ist $a~<~S\lbrack\text{next}\rbrack$?\\
        Laufzeit:\\
        best case: $\LO(1)$\\
        worst case: $\LO(n), n \text{ Iterat.}, |S| = n$\\
        average case: $\frac{n}{2}$ Iterat. $= \LO(n)$
        \item \emph{binäre Suche}: $S$ liege wieder als sortiertes Array vor. 2 Grenzen oben , unten als Indizes. Wir suchen wie in \autoref{alg:binarySearch} beschrieben.
            \begin{algorithm}
        		\caption{Binary Search}
        		\label{alg:binarySearch}
        		\begin{algorithmic}[1]
        			\Function{Suche}{$a,S$}
        			    \State $\text{unten} \gets 0$
        			    \State $\text{oben} \gets (n-1)$ \Comment{Array enthält $n$ Elemente}
        				\While{$\text{unten} < \text{oben}$}
        				    \State{$\text{next} \gets \lceil \frac{\text{oben} - \text{unten}}{2} \rceil + \text{unten}$}
                            \If{ $S[\text{next}] = x$ }
                                \State{\Return $\text{next}$}
                            \ElsIf{$S[\text{next}] < x$}
                                \State{$\text{unten} \gets \text{next}+1$}
                            \Else 
                                \State{$\text{oben} \gets \text{next}-1$}
                            \EndIf    
        				\EndWhile
        				\State \Return $-1$
        			\EndFunction
        		\end{algorithmic}
        	\end{algorithm}
        Laufzeit (best case): $T(n) = 1$ (zu suchendes Element genau in der Mitte)\\
        Laufzeit (worst case): 
        \begin{align*}
            T(n)~&=~1~+~T \left( \frac{n}{2} \right)\\
            &=~1~+~1~+~T \left( \frac{n}{4} \right)\\
            &=~1~+~1~+~1~+~T \left( \frac{n}{8} \right)\\
            &=~i~+~T \left( \frac{n}{2^{i}} \right)\tag{i-ter Schritt}\\
            &=~\log n~+~T(1)~=~O \left( \log n \right)
        \end{align*}
    \end{enumerate}

    \section{Interpolationssuche}

        next $\leftarrow~\left\lfloor \frac{a - S \lbrack \text{unten} \rbrack}{S \lbrack \text{oben} \rbrack - 
        S \lbrack \text{unten} \rbrack} \cdot \left( \text{oben} - \text{unten} \right)\right\rfloor + \text{unten}$\\
        \begin{tabbing}
            Laufzeit: \= worst case: $\LO(n)$\\
            \> average case: $\LO(\log \log n)$
        \end{tabbing}

        \begin{figure}
            \centering
            \begin{tikzpicture}[->,node distance=1cm, box/.style={draw,shape=rectangle,minimum size=1cm,font=\large},line width=0.5pt]
    \node[box] (1) at (0,0) {2};
    \node[box] (2) [right of=1] {4};
    \node[box] (3) [right of=2] {7};
    \node[box] (4) [right of=3] {9};
    \node[box] (5) [right of=4] {12};
    \node[box] (6) [right of=5] {21};
    \node[box] (7) [right of=6] {26};
    \node[box] (8) [right of=7] {31};
    \node[box] (9) [right of=8] {37};
    \node[box] (10) [right of=9] {44};
    \node[box] (11) [right of=10] {63};
    \node[box] (12) [right of=11] {80};

    \draw (4,-1.5) -- (4,-0.5);
    \draw (4,0.5) .. controls +(90:1) and +(90:1) .. (6,0.5);
    \draw (6,-0.5) .. controls +(-90:1) and +(-90:1) .. (7,-0.5);
\end{tikzpicture}

            \caption{Interpolationssuche}
            \label{diag2:interpolationsearch}
        \end{figure}
        
        Interpolieren und Springen sukzessive in Sprüngen der Länge $\sqrt{\text{oben} - \text{unten}}$, bis das richtige Suchintervall gefunden ist.
        Suchintervall hat Größe $\sqrt{\text{oben} - \text{unten}}$.

    
    \section{Erweiterung der Suche in geordneten Mengen}
        $\btl S \btr = n$ \\
        \begin{enumerate}[a.)]
            \item Lineare Suche $\LO(n)$ 
            \item Binäre Suche $\LO(\log n)$
            \item Interpolationssuche $\LO(\log \log n)$ im Mittel.
        \end{enumerate}
      
        \subsection{Variante von Reingold}
            $S$ als Array $S\lbrack 1 \rbrack, \ldots, S \lbrack
            n \rbrack$ füge künstliche Elemente ein $S \lbrack 0 \rbrack, \ldots S
            \lbrack n+1 \rbrack$ \\
            next $\leftarrow$ (unten -1) + $\lceil \frac{a-S(\text{unten}
            -1)}{S(\text{oben} +1) - S (\text{unten}-1)} \cdot (\text{oben - unten} + 1) \rceil$ \\
            Starte mit $n \leftarrow$ oben $-$ unten $+1$\\
            \begin{enumerate}[(1)]
                \item
                Interpoliere: $a : S[next]$ 
                \item 
                falls ``='' fertig \\
                falls $>$: lineare Suche in $S[next+\sqrt{n}], S[next+2 \cdot \sqrt{n}] \cdots \ \text{ bis } a < S[next+(i-1)\sqrt{n}]$ \\
                Es gilt : Nach $i$ Vergleichen gilt: $S[next+(i-2)\sqrt{n}) \leq a < S[next+(i-1)\sqrt{n}]$ \\
                falls $<$: analoges Vorgehen
                \item
                rekursiv im Suchintervall der Größe $\sqrt{n}$ weitersuchen. \\
            \end{enumerate}
        
        \subsection{Analyse}
            Sei $C$ die mittlere Anzahl von Vergleichen um auf Teilproblem der Größe $\sqrt{n}$ zu kommen, dann: \\
            \begin{align*}
                T(n) &\leq C + T(\sqrt{n}) \\
                T(1) &= 1 \\
                T(n) &= \LO(\log \log n) \text{falls } C \text{ konstant.}
            \end{align*}
            Berechnung:
            \begin{align*}
                T(n) &= C + T(\sqrt{n})  \\
                &= C + C + T\left(n^{\frac{1}{4}}\right) \\
                &= ... = \\
                &= i \cdot C + T\left(n^{\frac{1}{2^i}}\right) \overset{!}{=} a \cdot C + T(2)
            \end{align*}
            Nun soll berechnet werden für welches $a$ gilt:
            \begin{align*}
                n^{\frac{1}{2^{a}}} &= 2 \\
                \log n^{\frac{1}{2^{a}}} &= log 2 \\
                \frac{1}{2^{a}}\cdot \log n &= 1 \\
                \log n &= 2^{a} \\
                \log{ \log n} &= a
            \end{align*}
        
            \subsubsection{Wahrscheinlichkeitsannahme}
                Elemente $a_1 \ldots a_n$ und auch $a$  sind zufällige Elemente aus ($a_0, a_{n+1}$) gezogen nach Gleichverteilung. \\
                Sei $p_i$ Wahrscheinlichkeit, dass $\geq i$ Vergleiche nötig sind. \\
                $\Ra C_i = \sum_i i \cdot  \underbrace{(p_i - p_{i+1})}_{\text{Prob(genau i Vergleiche.)}} 
                		=1(p_1 - p_2) + 2(p_2 - p_3) + 3(p_3-p_4) = \sum \limits_{i \geq 1} p_i$ \\
                \newline
                $p_1 = 1 = p_2$ \\
                Was ist jetzt $p_i$ ? \\
                Falls $i$ Vergleiche nötig sind, weicht der tatsächliche Rang von $a$ in der Folge $a_1 , \ldots, a_n$ um mindestens 
                		$(i-2) \cdot \sqrt{n}$ von next ab. \\
                (Rang von a: Anzahl von $a_i$ mit $a_i < a$) \\
                das heißt: $p_i \leq \text{ prob } ((Rang(a) - next) \geq (i-2) \sqrt{n})$ \\
                bedeutet $\text{ prob } ((y - \mu (y)) \geq (i-2) \cdot \sqrt{n})$ mit unten = 1, oben = n, next = $p \cdot n$ \\
                $p = \frac{a-S[unten -1 ]}{S[oben+1] - S[unten - 1 ]}$\\
                \newline
                next ist die erwartete Anzahl der Elemente $\leq a$ \\
                also der erwartete Rang von $a$. \\
                Sei $y$ Zufallsvariable, die Rang von $a$ in $a_1, \ldots, a_n$ angibt, sowie $\mu(y)$ Erwartungswert. \\\\
                \textit{Tschebyscheff'sche Ungleichung:}
                $$\text{ prob } ((y- \mu (y)) > t) \leq \frac{Var(y)}{t^2}$$
                $Var(x) = E((x-E(x))^{2})$ 
            

                \begin{tabbing}
                    Es gilt: \= Erwartungswert: $p \cdot n$\\
                    \> Varianz: $(1-p)\cdot p \cdot n$
                \end{tabbing}
                    Erwartungswert $\mu$\\
                prob( genau j Elemente < a) = $\begin{pmatrix}
                                n \\ j
                                           \end{pmatrix} p^{j} (1-p)^{n-j}$\\
                Deshalb: 
                \begin{align*}
                    \mu &= \sum \limits_{j=1}^{n} j \cdot \begin{pmatrix}
                    n \\ j
                    \end{pmatrix} p^{j} (1-p)^{n-j}\\
                    &= n \cdot p \cdot \sum \limits_{i=1}^{n} \begin{pmatrix}
                    n-1 \\ j-1
                    \end{pmatrix} p^{j-1} (1-p)^{n-j}\\
                    &= n \cdot p \cdot \sum \limits_{j=0}^{n-1} \begin{pmatrix}
                    n-1 \\ j
                    \end{pmatrix} p^{j} (1-p)^{n-(j+1)}\\
                    &= n \cdot p \cdot 1\\
                    &= n \cdot p
                \end{align*}
                $p_i \leq$ prob $\left( \underbrace{Rang}_{y}~-~\underbrace{next}_{\mu}~\geq~\underbrace{(i-2) \sqrt{n}}_{t} \right)$\\
                $\leq \frac{Var(y)}{t^2} = \frac{p(1-p)\cdot n}{(i-2)^{2} \cdot n} 
                	\overset{p\cdot(1-p)\leq \frac{1}{4}}{\underset{\text{da } 0\leq p\leq 1}{\leq}} \frac{1}{4(i-2)^{2}}$
            
                Es gilt
                $$
                    C \leq \sum_i p_{i} \leq 2 + \sum \limits_{i \geq 3} \frac{1}{4(i-2)^{2}} 
                    \leq 2+\frac{1}{4}\sum_{i\geq 1}{\left(\frac{1}{i}\right)^2}
                    \leq 2 + \frac{\pi^{2}}{24} \approx 2.4
                $$
                Im Mittel brauchen wir also $\LO(\log\log n)$ Versuche, da $T(n)=2.4 \log \log (n)$ \\
                \underline{worst case:} \\
                $T_{\text{worst}} (n) = T_{\text{worst}} (\sqrt{n}) + (\sqrt{n} +1) = 
                O\left(\sqrt{n}+ \sqrt{\sqrt{n}} + \sqrt{\sqrt{\sqrt{n}}}+\ldots \right) = \LO(\sqrt{n})$\\
                Es sind also $\sqrt{n}$ Sprünge nötig um Problem auf Größe $\sqrt{n}$ zu bringen.
                aber: Annahme der Gleichverteilung kritisch.


% Copyright \copyright\ 2010  Volodymyr Piven und Alexander Peltzer.
% Es wird die Erlaubnis gegeben, dieses Dokument unter den Bedingungen der von der Free
% Software Foundation veröffentlichen  GNU Free
% Documentation License (Version 1.2 oder neuer) zu kopieren, verteilen und/oder
% zu verändern. Eine Kopie dieser Lizenz ist unter
% http://www.gnu.org/copyleft/fdl.txt erhältlich.
%
% Copyright der Aktualisierung und Überarbeitung \copyright\ 2013  Simon Kalt, Jan-Peter Hohloch, Tobias Fabritz
% Es wird die Erlaubnis gegeben, dieses Dokument unter den Bedingungen der von der Free
% Software Foundation veröffentlichen  GNU Free
% Documentation License (Version 1.2 oder neuer) zu kopieren, verteilen und/oder
% zu verändern. Eine Kopie dieser Lizenz ist unter
% http://www.gnu.org/copyleft/fdl.txt erhältlich.
%
% Zusätzlich muss jede Kopie/Aktualisierung wieder über die Seite
% der Fachschaft Informatik der Uni Tübingen
% den Studenten zur Verfügung gestellt werden
% http://www.fsi.uni-tuebingen.de/

\chapter{Sortieren}
    $A: 11,3,4,5,4,12 \quad \rightarrow \quad B:3,4,4,5,11,12$\\
    \begin{tabbing}
        Verfahren: \= 1.Min\=imumsuchev + Austausch mit 1. Element usw. \\
		\> \> $T(n) = \LO(n) + T(n-1) \underset{\text{kleiner Gauss}}{=} \LO(n^{2})$\\
        \> 2.Insertionsort: Füge immer an richtige Stelle ein. Mache Platz\\
		\> \>  $T(n) = T(n-1) + O(n) = \LO(n^{2})$\\
        \> 3.Bubble Sort: In $i$-ter Iteration sind die ersten $i-1$ Stellen \\
		\> schon sortiert. Tausche jeweils 'groß' gegen 'klein' aus $S[i,n]$\\
		\>\>$T(n)= O(n) + T(n-1) = \LO(n^{2})$\\
        \> 4.Quicksort, Heapsort, MergeSort, BucketSort
    \end{tabbing}
        

    \section{Quicksort(Divide and Conquer)}
        Eingabe $S$ \ \ $a_1, \ldots , a_n$
        \begin{enumerate}
           \item Wähle Pivotelement $a_1$ \\
           Teile $S \verb=\= \{ a_1 \}$ auf in $A= \{a_i \mid a_i \leq a_1 \text{ für } i \geq 2 \}$, 
           	$B = \{a_i \mid a_i > a_1 \text{ für } i \geq 2 \}$ \\
           Speichere A in $S[1], \ldots, S[i]$ $a_1$ in $S[j+1]$, $B$ in $ S[j+2], \ldots S[n]$ 
           \item Sortiere A und B rekursiv. 
        \end{enumerate}

        Analyse: Schlechtester Fall: wenn $a_1$ minimal $(A = \varnothing)$ oder $a_1$ maximal $(B = \varnothing)$. \\
        $\Ra \#$ Vergleiche: $(n-1)+(n-2)+(n-3) + \ldots + 1 = \frac{n(n-1)}{2} = \LO(n^2)$ \\
        Mittlere Analyse: Modelliere Elemente paarweise verschieden, jede der $n!$ möglichen Permutationen der Eingabe ist gleich wahrscheinlich\\
        oBdA: $S = \{1,2,\ldots, n \} \\ S[1]$ mit Wahrscheinlichkeit $\frac{1}{n}$ für $k=1,2,\ldots, n$ \\
        Teilprobleme der Größe $k-1$ bzw. $n - k$ erfüllen wieder die W' annahmen. \\\\
        Sei $\overline{QS}(n)$ die mittlere Anzahl von Vergleichen an Feld der Grö\ss e $n$. Es gilt:
        $$\overline{QS} (0) = \overline{QS}(1) = 0$$
        sowie für $n \geq 2$: 
        \begin{align*}
            \overline{QS}(n) &= n + E(\overline{QS}(A)+ \overline{QS}(B))  \\
            &= n + \frac{1}{n} \cdot \sum \limits_{k-1}^n \overline{QS} (k-1) + \overline{QS}(n-k) \\
            &= n + \frac{2}{n} \cdot \sum \limits_{k=1}^{n-1} \overline{QS} (k)
        \end{align*} 
        Es lassen sich nun die Formeln
        \begin{align*}
            n \cdot \overline{QS}(n) &= n^2 + 2 \sum \limits_{k-1}^{n-1} \left( \overline{QS}(k) \right) \tag{i} \\
            (n+1) \cdot \overline{QS} (n+1) &= (n+1)^2 + 2 \cdot \sum \limits_{k-1}^n \left( \overline{QS}(k) \right) \tag{ii}
        \end{align*}
        bilden und damit gilt
        \begin{align*}
            \text{(ii) - (i)} &= (n+1) \cdot \overline{QS}(n+1) - n \cdot \overline{QS}(n) = (n+1)^2 -n^2 +2 \cdot \overline{QS}(n) \\
            (n+1) \cdot \overline{QS} (n+1) &= 2n+1~+~(n+2)\cdot \overline{QS} (n) \\
            \overline{QS} (n+1) & \overset{\frac{2n+1}{n+1}\le 2}{\le} 2 + \frac{n+2}{n+1} \cdot \overline{QS} (n) \\
            &= 2 + \frac{n+2}{n+1}\cdot \left( 2+ \frac{n+1}{n} \cdot \overline{QS} (n-1) \right) \\
            &= 2 + (n+2) \cdot \left( \frac{2}{n+1} + \frac{2}{n} + \frac{2}{n-1} + \ldots + \frac{2}{2} + \frac{2}{1} \right) \\
            \overline{QS} (n) &= 2 + 2 \cdot (n+2) 
            	\cdot \underbrace{\left( \frac{1}{n} + \frac{1}{n-1} + \frac{1}{n-2} + \ldots + 1 \right)}_{\text{harmonische Reihe}} \\
            &< 2 + 2 \cdot (n+2) \cdot \left( \ln n + 1 \right) \\
            &\Rightarrow \LO(n \cdot \log(n))
        \end{align*}


    \section{Heap Sort}
        Ähnlich zu \q{Sortieren durch Min-Suche}. \\
        \begin{definition}
            Heap ist Binärer Baum:\\
            Er ist entweder leer, oder hat Wurzelknoten mit linkem und rechtem Teilbaum.
        \end{definition}
        
        Dabei gilt für Baum $T$ und Knoten $v$ in $T$: 

        \begin{align*}
            \text{Tiefe}(v) &= \begin{cases}
                                0 &, \text{ falls $v$ Wurzel von $T$} \\
                                \text{Tiefe}(\text{Parent}($v$)) + 1 &, \text{ sonst}
                            \end{cases} \\
            \text{Höhe}(v) &= \begin{cases}
                                0 &, \text{ falls $v$ Blatt} \\
                                \text{max}(\{\text{Höhe($u$)} \mid \text{$u$ Kind von $v$} \}) + 1 &, \text{ sonst}
                            \end{cases} 
        \end{align*}

    %    momentan kein Bsp
    %    Beispiel Höhe$(v) = 1$ \\
    %    Beispiel Höhe$(u) = 3$ \\
        Allgemein gilt für Baum T:
        % Tiefe$(T) = max \{ \text{Tiefe } (v) \mid v \in T \}$ \\
        % Höhe$(T) = \text{Höhe } \{ \text{ Wurzel }(T) \} = \text{max(Höhe(v) für } v \in T)$ \\

        \begin{align*}
            \text{Tiefe}(T) &= \text{max}(\{\text{Tiefe}(v) \mid v \in T\}) \\ 
            \text{Höhe}(T)  &= \text{Höhe}(\text{Wurzel}(T)) = \text{max}(\{\text{Höhe}(v) \mid v \in T\}) 
        \end{align*}


        Heap wird durch binären Baum dargestellt, wobei jedes Element aus $S$ einem Knoten aus $h$ zugeordnet wird. \\
        Heapeigenschaft: für Knoten mit $u,v$ mit $\text{parent}(v) = u$ gilt $S [u] \leq S [v]$ \\
        
        Beobachtung: \\ 
        Beim Heap stet das Minimum in der Wurzel. Sortieren durch Min-Suche. \\
        1. Suche: Min $\ra$ Wurzel $\LO(1)$ \\
        2. Lösche Min aus Suchmenge und repariere Heap. \\

        Ablauf: \\
        \begin{itemize}
            \item Nimm Blatt und füge es zu Wurzel hinzu
            \item Lass es \q{nach unten sinken} (vertausche mit kleinstem Kind)
            \item \# Iterationen $\leq$ Höhe$(T)$ 
        \end{itemize}
        Höhe sollte also klein sein\\
        $\Rightarrow$ Ausgewogene Bäume:
        \begin{itemize}
       	\item Alle Blätter haben Tiefe $k$ oder $k+1$ 
        	\item Blätter auf Tiefe $k+1$ stehen ganz links. 
        \end{itemize}
        \q{Kanonische Form}: Es gilt auf Tiefe $1,2, \ldots, k$ liegen $2^1, \ldots, 2^k$ Knoten.

        \begin{figure}[htp]
            \Tree [.1 [.5 8 6 ]
          [.2 [.4 7 ]
              6 ] ]

            \caption{Unausgewogener Heap Baum}
            \label{diag3:unbalanced-heap-tree}
        \end{figure}

        In Abb. \ref{diag3:unbalanced-heap-tree} ist ein Beispiel-Heap als Baum dargestellt.
        Dieser ist noch nicht ausgewogen. Eine mögliche ausgewogene Variante für diesen Baum
        ist in Abb. \ref{diag3.5:balanced-heap-tree} dargestellt.

        
        \begin{figure}[htp]
            \Tree [.$1_1$ [.$5_2$ [.$6_4$ $7_8$ ] $8_5$ ]
          [.$2_3$ $4_6$ $6_7$ ] ]

            \caption{Ausgewogener Heap Baum mit Nummerierung der Elemente}
            \label{diag3.5:balanced-heap-tree}
        \end{figure}

        Es gilt für Knoten mit Nummer $i$: $parent(i)$ hat die Nummer
        $\left\lfloor \frac{i}{2} \right\rfloor$, Kinder haben $2i$ und $2i+1$.
        Der Heap wird nur konzeptuell als Baum dargestellt. Implementiert
        wird dieser als Array mit einer Indizierung ab 1. Der Heap wird wie
        in Algorithmus \ref{alg:init-heap} beschrieben initialisiert.

        \begin{algorithm}[htp]
            \caption{Initialisierung des Heaps $h$}
            \label{alg:init-heap}
            \begin{algorithmic}
                \ForAll{$a \in S$}
                    \State \Call{insert}{$a$, $h$}
                \EndFor

                \Function{insert}{$a$, $h$}
                    \State $n \gets $ size(h) $ + 1$
                    \State $A(n) \gets a$
                    \State $i \gets n$
                    \State $j \gets n$
                    \State $\var{done} \gets (i = 1)$
                    \While{$\lnot \var{done}$}
                        \State $j \gets \left\lfloor\frac{j}{2}\right\rfloor$
                        \If{$A(j) > A(i)$}
                            \State swap($A(i), A(j)$)
                            \State $i \gets j$
                        \Else
                            \State $\var{done} \gets \True$
                        \EndIf
                    \EndWhile
                \EndFunction
            \end{algorithmic}
        \end{algorithm}

        \subsection{Laufzeit}
        Initialisierung: $\LO(n \cdot \text{Höhe}(H))$ \\
        eigentliche Sortierung: $\LO(n \cdot \text{Höhe}(H)$ \\
        Höhe$(H)$ ist $\LO(\log n)$, da Baum ausgewogen ist. \\
        \begin{satz}    
            Lemma: Hat ein ausgewogener Baum die Höhe $k$, so hat er mindestens $2^k$ Knoten.
        \end{satz}
        \begin{proof}
            Der kleinste Baum der Höhe $k$ hat nur ein Blatt mit Abstand $k$ zur Wurzel. Auf Stufe $i$ gibt es $2^i$ Knoten für $0 \leq i \leq k-1$ plus einen Knoten auf Stufe $k$. Damit ist die Anzahl der Knoten
            $$
                \sum^{k-1}_{i=0}{2^i} + 1 = 2^k
            $$
        \end{proof}        
        Damit ist $n \geq 2^k$ und $\log n \geq k = \text{Höhe}(H)$
        \begin{satz}
            HeapSort läuft in Zeit $\LO(n \log n)$, auch im Worst-Case. \\
        \end{satz}
        Beachte $\overline{QS} \leq 2 \cdot n \log n$ \\
        Frage: Wie ist HeapSort-Konstante? \\
        (Diese Frage wurde von Prof. Kaufmann offen gelassen ;) )

    \section{MergeSort}
        Prinzip: mische 2 sortierte Folgen zusammen zu einer. \\
        \begin{math}
	        \left.
	        	\begin{array}{c}
	        		(1, 2, 4, 7)\\
	        		(5,6,8) 
	        	\end{array}
	        \right\rbrace \rightarrow (1,2,4,5,6,7,8) \\
        \end{math}
        Es werden immer die aktuell minimalen Elemente verglichen $(\LO(1))$ und dann das kleinste ausgeschrieben. \\
        \underline{Idee:} \\
        Starte mit Teilfolgen der Länge 1. ($n$ Teilfolgen) \\
        Mache daraus Teilfolgen der Länge 2 $(\frac{n}{2}$ Teilfolgen) \\
        usw. bis aus 2 Teilfolgen eine entsteht. \\
        \underline{Laufzeit:}\\
        Teilfolgen haben nach der $i$-ten Iteration die Länge $\leq 2^i$. Ist $2^i \geq n \Ra i \geq \log n$ Iterationen\\
        Pro Iteration mache $\LO(n)$ Vergleiche. $\Ra$ Insgesamt $\LO(n \log n)$ Vergleiche.\\
        \emph{Allgemeiner:} \\
        \q{m-Wege-Mischen}: $\LO(\log_m n)$ Iterationen. Dabei fügt man immer m Folgen zusammen.
    


    \section{Bucket Sort}
        (Sortieren durch Fachverteilung)\\
        \emph{Gegeben:} Wörter über Alphabet $\Sigma$. Sortiere lexikographisch. Als Ordnung gilt
        \begin{align*}
            a < B \\
            a < aa < aba \\ 
            \text{mit } \abs{\Sigma}=m
        \end{align*}
        \begin{enumerate}
            \item Fall: Alle Wörter haben Länge 1. Stelle $m$ Fächer bereit (für $a,b,c, \ldots, z$). Diese Fächer müssen sortiert sein.
            Wirf jedes Wort in entspr. Fach. Konkateniere die Inhalte der Fächer.\\
            $\Rightarrow$ Laufzeit: $\LO(n) + \LO(m)$\\
            Implemetiere einzelne Fächer als lin. Listen.\\
            
            \item Fall: Alle Wörter haben Länge $k$. Sei Wort $a^{i} = a_{1}^{i} a_{2}^{i} \cdots a_{k}^{i}$ 
            das $i$-te Wort, $1 \leq i \leq n$\\
            \emph{Idee:} Sortiere zuerst nach dem letzten Zeichen, dann dem vorletzten, usw. Damit sind Elemente, die zum 
            Schluss ins gleiche Fächer fallen, richtig geordnet.\\
            \emph{Laufzeit:} $\LO((n+m)k)$\\
            \emph{Problem:} Wollen leere Listen überspringen beim Aufsammeln der Listen. Also Ziel: statt 
            $\LO((n+m)k)$ besser $\LO(n\cdot k + m)$.\\
            Erzeugen Paare $\left( j, a^{i}_{j} \right),~1 \leq i \leq n,~1 \leq j \leq k$.\\
            Sortiere nach 2. Komp., dann nach der ersten, dann liegen im $j$-ten Fach die Buchstaben sortiert, die an der 
            $j$-ten Stelle vorkommen.\\
            $\Rightarrow$ leere Fächer werden übersprungen.\\
            $\Rightarrow$ Laufzeit: $\LO(n \cdot k + m)$\\
            
            \item Fall: (Der allgemeine Fall) \\
            Wort $a^{i}$ habe Länge $l_{i}$ Sei
            \begin{align*}
                L = \sum \limits_{i=1}^{n} l_{i}\\
                R_{max} = \max \lbrace l_{i} \rbrace\\
            \end{align*}
            \emph{Idee:} Sortiere Wörter der Länge nach und beziehen zuerst die langen Wörter ein. \\
            \emph{Algorithmus:}             
            \begin{enumerate}
                \item Erzeuge Paare $\left( l_{i}, \text{ Verweis auf } a^{i} \right)$\\
                \item Sortiere Paare durch Bucketsort nach 1.Komp\\
                \item Erzeuge $L$ Paare $\left( j, a_{j}^{i} \right), 1 \leq i \leq n, 1 \leq j \leq l_{i}$ 
                    und sortiere zuerst nach 2. Komp., dann nach 1. Komp. Diese liefert lineare Listen $\text{Nichtleer}(j)$, $1 \leq j \leq r_{max}$\\
                \item Sortiere nun $a^{i}$s durch Bucketsort wie oben unter Berücksichtigung von $L(k)$
            \end{enumerate}
        \end{enumerate}

        
        

        \begin{satz}
            Bucketsort sortiert $n$ Wörter der Gesamtlänge $L$ über Alphabet $\sum,~\btl \sum \btr = m$ in Zeit $\LO(m + L)$
        \end{satz}
        Beispiel für den 2. Fall:        
        \begin{bsp}
            124, 223, 324, 321, 123
            \begin{table}[!htbp]
                \centering
                \begin{tabular}{c|c|c|c|c}
                    Fächer & 1 & 2 & 3 & 4\\
                    \hline
                    1.Iter. & 321 & & 223 & 124\\
                    & & & 123 &\\
                    \hline
                    2.Iter. & & 321 & &\\
                    & & 223 & &\\
                    & & 123 & &\\
                    & & 124 & &\\
                    & & 224 & &\\
                    \hline
                    3.Iter. & 123 & 223 & 321 & \\
                    & 124 & 224 &&
                \end{tabular}
            \end{table}
        \end{bsp}

% 2013 nicht behandelt
%    \section{Hybridsort}
%        Jetzt reelle Zahlen aus $\rbrack 0,1 \rbrack$:  \\
%        Hybridsort: Sei $k = c \cdot n$ mit $c \in \R$ fest. \\
%        \begin{enumerate}[1.)]
%            \item Wirf $a_i$ in Fach $\lceil k \cdot a_i \rceil$ für $1 \leq c \leq n$ 
%            \item Wende HeapSort auf jedes Fach an und konkateniere die Fächer.
%        \end{enumerate}
%        Laufzeit:\\
%        Worst case: $\LO(n \log n)$ \\
%        average case: Sind $a_i$ unabhängig und gleichverteilt , so hat Hybrid Sort Laufzeit $\LO(n)$. \\


    \section{Auswahlproblem} 
        Finde den Median (mittleres Element) \\
        \emph{formal}:\\ Gegeben Menge $S$ von $n$ Zahlen und Zahl $i$, $1 \leq i \leq n$. Finde $x$ mit $\btl \{ y | y> x \} \btr = i - 1$
       
        \begin{enumerate}[1.]
            \item Idee: Sortiere und ziehe dann das $i$-te Element $\Ra \LO(n \log n)$


               \item Idee: Wie Quicksort \\
                Random. Partiion $(A, q,r)$ zerlegt Array $A$ zwischen Indizes $p$ und $r$ an zufälliges Element $q$ in zwei Mengen 
               $A \lbrack p, q-1 \rbrack$ und $A \lbrack q+1, r \rbrack$ 
        \end{enumerate}
            \underline{Random-Select} $(A,p,r,i)$ \\
\begin{algorithm}
\caption{RandomSelect}
\label{alg:randomSelect}
\begin{algorithmic}[1]
\Function{RandomPartition}{A,p,r}
\State A zerteilt in p und r an zufälligen q in Mengen A[p,q-1], A[q+1,r]
\EndFunction
\Function{RandomSelect}{$A,p,r,i$}
	\If{$p = r$}
		\Return $A[p]$
	\EndIf
	\State $q \gets$ \Call{Random Partition}{A,p,r} ($\in \LO(r-p))$
	\State $k \gets q-p+1$
	\If{$i = k$}
		$A[q]$
	\Else
		\If{$i < k$}
		\State\Return\Call{RandomSelect}{$A,p,q-1,i$}
	\Else\ 
		\State\Return\Call{RandomSelect}{A,q+1,r,i}
	\EndIf
	\EndIf
\EndFunction
\end{algorithmic}
\end{algorithm}
           worst case: $\LO(n^2)$, ideal case: $n + \frac{n}{2} + \frac{n}{4} + \ldots + 2 + 1 = \LO(2n) \in \LO(n)$ \\
            average case: Sei $T(n)$ Zufallsvar. für die Laufzeit von Random Selection. \\
            Random Partition zerlegt die Menge der Größe $n$ mit Wahrscheinlichkeit $\frac{1}{n}$ in Teilmengen der Größen $k$ und $n-k$. \\
            Also definiere für $k=1,2, \ldots, n$ \\
           $x_k = \begin{cases} 1 & \text{ falls } A \lbrack p,q \rbrack  k \text{ Elemente hat} \\
            0 & \text{ sonst}
            \end{cases} \Ra E(X_k) = \frac{1}{n}$
            \underline{Beobachtung}: 
            \begin{itemize}
                \item $T(n)$ ist monoton.
                \item Laufzeit ist größer, falls $i$ in größerer Teilmenge.
            \end{itemize}
            $\Ra $
            \begin{eqnarray*}
                T(n) & \leq &  \sum \limits_{k=1}^n X_k \cdot (T(\max(k-1, n-1))+ \LO(n)) \\
                &=& \sum \limits_{k=1}^n X_k \cdot T(\max(k-1, n-k)) + \LO(n) 
            \end{eqnarray*}
            und damit : \\
            \begin{eqnarray*}
                E(T(n)) &=& E( \sum \limits_{k=1}^n X_k \cdot T(\max (k-1, n-k)) + \LO(n)) \\
                &=& E(\sum \limits_{k=1}^n X_k \cdot T(\max(k-1, n-k)) + \LO(n) \\
                &=& \sum E(X_k) \cdot E(T_{\max} (k-1, n-k)) + \LO(n) \\
                &=& \sum \limits_{k=1}^n \frac{1}{n} \cdot E(T_{\max} (k-1, n-k)) + \LO(n) \\
                &\leq& \frac{2}{n} \sum \limits_{k= \lfloor \frac{n}{2} \rfloor}^{n-1} E(T(k)) + \LO(n)
            \end{eqnarray*}
            Da $X_k$ und $T_{\max}$ unabhängig sind.
        
        \underline{Behauptung}: $E(T(n)) \leq c \cdot n$ für konstantes $c$. \\
        $T(n) = \LO(1)$ falls $n$ konstant $\surd$. 
        Induktion: \\
          
        \begin{eqnarray*}
            E(T(n)) &\leq& \frac{2}{n} \sum \limits_{k=\lfloor \frac{n}{2} \rfloor}^n c \cdot k + a \cdot n \text{  (a ist geeignete Konstante.) } \\
            &=& \frac{2c}{n} ( \sum \limits_{k=1}^{n-1} k - \sum \limits_{k=1}^{\lfloor \frac{n}{2} \rfloor -1} k) + a \cdot n \\
            &\leq& \frac{2c}{n} \cdot (\frac{n^2}{2} - \frac{(\frac{n}{2})^2}{2} ) + a \cdot n \\
            &\leq & \frac{2n}{n} \cdot ( \frac{3}{8} n^2) + a \cdot n \\
            &=& \frac{3}{4} \cdot c \cdot n + a \cdot n \leq c \cdot n \text{ für } a \leq \frac{c}{4}
        \end{eqnarray*}
          
        Bessere Version:
        \begin{enumerate}
            \item Teile $n$ Elemente in $\frac{n}{5}$ Gruppen der Größe $5$, sortiere sie und bestimme Median jeder Gruppe.
            \item Suche rekursiv den Median $x$ der Mediane.
            \item Teile Menge $S$ bezüglich $x$ in 2 Mengen $S_1, S_2$. Sei $\btl S_1 \cup x \btr = k$ und $\btl S_2 \btr = n-k$
            \item 
            \begin{algorithmic}
				\begin{algorithm}
				\If{i = k}\State \Return $x$
				\Else 
					\If{$i < k$}
				\State \Call{Auswahl}{$S_1, i$}
				\Else
				\State \Call{Auswahl}{$S_2, i-k$}
				\EndIf
				\EndIf
				\end{algorithm}
            \end{algorithmic}
        \end{enumerate}
        Wieso Mediane? \\
        $\frac{1}{2} \cdot \frac{n}{5}$ der Mediane sind größer als $x$. Die Gruppen dieser Mediane liefern jeweils $3$ Elemente, die größer sind als $x$; also sind 
        mindestens $\frac{3}{10} n $ Elemente größer als $x$ und genauso $\frac{3}{10}n$ kleiner. \\
        $\Ra \btl S_1 \btr, \btl S_2 \btr \leq \frac{6n}{10}$ \\
        Um $\frac{6}{10}$ schrumpft die Menge rekursiv mindestens. \\
        \emph{Laufzeit:} \\
        $T(n) = \begin{cases} 
        \LO(1) & \text{ falls n konstant ist} \\
        T(\frac{n}{5}) + T(\frac{5n}{10}) + c \cdot n & \text{ sonst}
        \end{cases}$ \\
        Behauptung: $T(n)  = c \cdot n$ für $c$ konstant, also $T(n) = \LO(n)$ \\    

 2013 nicht angeschrieben
        Induktion: 
        \begin{eqnarray*}
            T(n) & \leq & c \cdot \frac{n}{5} + c \cdot \frac{7n}{10} + a \cdot n \\
            &=& \frac{9}{10} c \cdot n + a \cdot n \leq c \cdot n \text { für } a \leq \frac{c}{10}
        \end{eqnarray*} $\surd$

        \begin{satz}
            In zeit $\LO(n)$ können wir das $i$-te größte element aus einer ungeordneten Menge bestimmen.
        \end{satz}


% Copyright \copyright\ 2010  Volodymyr Piven und Alexander Peltzer.
% Es wird die Erlaubnis gegeben, dieses Dokument unter den Bedingungen der von der Free
% Software Foundation veröffentlichen  GNU Free
% Documentation License (Version 1.2 oder neuer) zu kopieren, verteilen und/oder
% zu verändern. Eine Kopie dieser Lizenz ist unter
% http://www.gnu.org/copyleft/fdl.txt erhältlich.
%
% Copyright der Aktualisierung und Überarbeitung \copyright\ 2013  Simon Kalt, Jan-Peter Hohloch, Tobias Fabritz
% Es wird die Erlaubnis gegeben, dieses Dokument unter den Bedingungen der von der Free
% Software Foundation veröffentlichen  GNU Free
% Documentation License (Version 1.2 oder neuer) zu kopieren, verteilen und/oder
% zu verändern. Eine Kopie dieser Lizenz ist unter
% http://www.gnu.org/copyleft/fdl.txt erhältlich.
%
% Zusätzlich muss jede Kopie/Aktualisierung wieder über die Seite
% der Fachschaft Informatik der Uni Tübingen
% den Studenten zur Verfügung gestellt werden
% http://www.fsi.uni-tuebingen.de/

\chapter{Dynamische Mengen}
    Operationen: Einfügen, Streichen, Vereinigen, Spalten, etc. \\
    
    \section{Suchbäume}
        Beispiel: $\{ 2,3,5,7,11,13,17 \}$ (\autoref{diag4:search-tree},\autoref{diag5:leaf-oriented}).

        Knoten haben: 
        \begin{itemize}
            \item Key-Element
            \item Verweise zu Kindern (lson, rson)
            \item oft Verweis zu parent
        \end{itemize}
        
        
         \subsection{Knotenorientierte Speicherung} 
            \begin{itemize}
                \item 1 Element pro Knoten
                \item Alle Elemente im linken Teilbaum von Knoten v sind kleiner als alle im rechten Teilbaum
                \item keine Elemente in den Blättern
            \end{itemize}
            Dies ist die weiterhin verwendete Speicherform in diesem Skript.
            
            \begin{figure}[htp]
            	\Tree [.5 [.2 $\square$ 3 ]
          [.7 $\square$
              [.13 11 17 ] ] ]

            	\caption{Knotenorientierter Suchbaum}
            	\label{diag4:search-tree}
            \end{figure}


        \subsection{EXKURS: Blattorientierte Speicherung} 
            1 Element pro Blatt, Elemente aus linkem Teilbaum $\leq$ (Hilfs-) Schlüssel an $v$ $\leq$ Element aus rechtem Teilbaum.
            Ein Beispiel für diese Speicherung befindet sich in \autoref{diag5:leaf-oriented}.
            
            \begin{figure}[htp]
                \centering
                
% \begin{tikzpicture}[anchor=mid,>=latex',line join=bevel,]
\begin{tikzpicture}[-,line join=bevel,]
  \pgfsetlinewidth{1bp}
%%
\pgfsetcolor{black}
  % Edge: c -> f
  \draw [->] (339bp,144bp) .. controls (339bp,144bp) and (339bp,125.49bp)  .. (339bp,103bp);
  % Edge: d -> h
  \draw [->] (57bp,77bp) .. controls (57bp,77bp) and (48.079bp,59.752bp)  .. (35.357bp,35.156bp);
  % Edge: c -> g
  \draw [->] (389bp,144bp) .. controls (389bp,144bp) and (396.81bp,129.72bp)  .. (409.14bp,107.19bp);
  % Edge: e -> k
  \draw [->] (228bp,77bp) .. controls (228bp,77bp) and (232.19bp,60.787bp)  .. (238.66bp,35.776bp);
  % Edge: f -> m
  \draw [->] (363bp,77bp) .. controls (363bp,77bp) and (369.92bp,60.275bp)  .. (380.19bp,35.465bp);
  % Edge: f -> l
  \draw [->] (315bp,77bp) .. controls (315bp,77bp) and (315bp,61.286bp)  .. (315bp,36.088bp);
  % Edge: e -> j
  \draw [->] (186bp,77bp) .. controls (186bp,77bp) and (181.81bp,60.787bp)  .. (175.34bp,35.776bp);
  % Edge: a -> b
  \draw [->] (257bp,206bp) .. controls (257bp,206bp) and (214.1bp,184.25bp)  .. (186bp,170bp);
  % Edge: b -> e
  \draw [->] (207bp,144bp) .. controls (207bp,144bp) and (207bp,125.49bp)  .. (207bp,103bp);
  % Edge: a -> c
  \draw [->] (307bp,206bp) .. controls (307bp,206bp) and (339.06bp,185.75bp)  .. (364bp,170bp);
  % Edge: d -> i
  \draw [->] (99bp,77bp) .. controls (99bp,77bp) and (99bp,61.286bp)  .. (99bp,36.088bp);
  % Edge: b -> d
  \draw [->] (154bp,157bp) .. controls (154bp,157bp) and (106.21bp,123.04bp)  .. (78bp,103bp);
  % Node: a
\begin{scope}
  \definecolor{strokecol}{rgb}{0.0,0.0,0.0};
  \pgfsetstrokecolor{strokecol}
  \draw (248bp,207bp) -- (248bp,231bp) -- (317bp,231bp) -- (317bp,207bp) -- cycle;
  \draw (267bp,207bp) -- (267bp,231bp);
  \draw (297bp,207bp) -- (297bp,231bp);
  \draw (257bp,213bp) node { };
  \draw (282bp,213bp) node {10};
  \draw (307bp,213bp) node {};
\end{scope}
  % Node: c
\begin{scope}
  \definecolor{strokecol}{rgb}{0.0,0.0,0.0};
  \pgfsetstrokecolor{strokecol}
  \draw (330bp,145bp) -- (330bp,169bp) -- (399bp,169bp) -- (399bp,145bp) -- cycle;
  \draw (349bp,145bp) -- (349bp,169bp);
  \draw (379bp,145bp) -- (379bp,169bp);
  \draw (339bp,151bp) node { };
  \draw (364bp,151bp) node {15};
  \draw (389bp,151bp) node {};
\end{scope}
  % Node: b
\begin{scope}
  \definecolor{strokecol}{rgb}{0.0,0.0,0.0};
  \pgfsetstrokecolor{strokecol}
  \draw (155bp,145bp) -- (155bp,169bp) -- (217bp,169bp) -- (217bp,145bp) -- cycle;
  \draw (175bp,145bp) -- (175bp,169bp);
  \draw (198bp,145bp) -- (198bp,169bp);
  \draw (165bp,151bp) node { };
  \draw (186bp,151bp) node {3};
  \draw (207bp,151bp) node {};
\end{scope}
  % Node: e
\begin{scope}
  \definecolor{strokecol}{rgb}{0.0,0.0,0.0};
  \pgfsetstrokecolor{strokecol}
  \draw (176bp,78bp) -- (176bp,102bp) -- (238bp,102bp) -- (238bp,78bp) -- cycle;
  \draw (196bp,78bp) -- (196bp,102bp);
  \draw (219bp,78bp) -- (219bp,102bp);
  \draw (186bp,84bp) node { };
  \draw (207bp,84bp) node {5};
  \draw (228bp,84bp) node {};
\end{scope}
  % Node: d
\begin{scope}
  \definecolor{strokecol}{rgb}{0.0,0.0,0.0};
  \pgfsetstrokecolor{strokecol}
  \draw (47bp,78bp) -- (47bp,102bp) -- (109bp,102bp) -- (109bp,78bp) -- cycle;
  \draw (67bp,78bp) -- (67bp,102bp);
  \draw (90bp,78bp) -- (90bp,102bp);
  \draw (57bp,84bp) node { };
  \draw (78bp,84bp) node {2};
  \draw (99bp,84bp) node {};
\end{scope}
  % Node: g
\begin{scope}
  \definecolor{strokecol}{rgb}{0.0,0.0,0.0};
  \pgfsetstrokecolor{strokecol}
  \draw (418bp,90bp) ellipse (27bp and 18bp);
  \draw (418bp,90bp) node {17};
\end{scope}
  % Node: f
\begin{scope}
  \definecolor{strokecol}{rgb}{0.0,0.0,0.0};
  \pgfsetstrokecolor{strokecol}
  \draw (305bp,78bp) -- (305bp,102bp) -- (373bp,102bp) -- (373bp,78bp) -- cycle;
  \draw (324bp,78bp) -- (324bp,102bp);
  \draw (354bp,78bp) -- (354bp,102bp);
  \draw (315bp,84bp) node { };
  \draw (339bp,84bp) node {11};
  \draw (363bp,84bp) node {};
\end{scope}
  % Node: i
\begin{scope}
  \definecolor{strokecol}{rgb}{0.0,0.0,0.0};
  \pgfsetstrokecolor{strokecol}
  \draw (99bp,18bp) ellipse (27bp and 18bp);
  \draw (99bp,18bp) node {3};
\end{scope}
  % Node: h
\begin{scope}
  \definecolor{strokecol}{rgb}{0.0,0.0,0.0};
  \pgfsetstrokecolor{strokecol}
  \draw (27bp,18bp) ellipse (27bp and 18bp);
  \draw (27bp,18bp) node {2};
\end{scope}
  % Node: k
\begin{scope}
  \definecolor{strokecol}{rgb}{0.0,0.0,0.0};
  \pgfsetstrokecolor{strokecol}
  \draw (243bp,18bp) ellipse (27bp and 18bp);
  \draw (243bp,18bp) node {7};
\end{scope}
  % Node: j
\begin{scope}
  \definecolor{strokecol}{rgb}{0.0,0.0,0.0};
  \pgfsetstrokecolor{strokecol}
  \draw (171bp,18bp) ellipse (27bp and 18bp);
  \draw (171bp,18bp) node {5};
\end{scope}
  % Node: m
\begin{scope}
  \definecolor{strokecol}{rgb}{0.0,0.0,0.0};
  \pgfsetstrokecolor{strokecol}
  \draw (387bp,18bp) ellipse (27bp and 18bp);
  \draw (387bp,18bp) node {13};
\end{scope}
  % Node: l
\begin{scope}
  \definecolor{strokecol}{rgb}{0.0,0.0,0.0};
  \pgfsetstrokecolor{strokecol}
  \draw (315bp,18bp) ellipse (27bp and 18bp);
  \draw (315bp,18bp) node {11};
\end{scope}
%
\end{tikzpicture}
\begin{flushleft}
	\textit{Anmerkung:}\\
		Beispielsweise 10 ist \textbf{nicht} Element dieses Baumes. sie wird lediglich als Hilfsschlüssel verwendet.
		11 dagegen ist sowohl Schlüssel, als auch Element.
\end{flushleft}
                \caption{Blattorientierte Speicherung}
                \label{diag5:leaf-oriented}
            \end{figure}
        
       
        
        \subsection{Suchen in Suchbäumen}  
            Beschrieben in \autoref{alg:searchSearchtree}
    	\begin{algorithm}
        		\caption{Suchen in Suchbäumen}
        		\label{alg:searchSearchtree}
        		\begin{algorithmic}[1]
        			\Function{Suchen}{$x$}
        			    \State $u \gets \text{Wurzel}$
        			    \State $\text{found} \gets \text{false}$
        				\While{$u$ exists and !found}
        				    \If{\Call{key}{$u$} = x}
        				        \State $\text{found} \gets \text{true}$
        				    \Else
        				        \If{\Call{key}{$u$} $> x$}
        				            \State $u \gets $ \Call{lson}{$u$}
        				        \Else 
        				            \State $u \gets $ \Call{rson}{$u$}
        				        \EndIf
        				    \EndIf
        				\EndWhile
        				\State \Call{Return}{found}
        			\EndFunction
        		\end{algorithmic}
        	\end{algorithm}

        \subsection{Einfügen in Suchbäumen}
            Einfügen (x) : zuerstSuchen(x) \\
            Sei $u$ der zuletzt besuchte Knoten $u$ hat $\leq 1$ Kind (oder $x \in S$).\\
            Falls $x < $ key($u$), erzeuge neues Kind $v$ von $u$ mit key($v$) = x als lson($u$), andernfalls als rson ($u$). \\
            Beispielhaft (Füge 10 ein): \\
            % Graphic for TeX using PGF
% Title: /home/alex/wuala/WualaDrive/myfiles/Uni/4 Semester/Algorithmen/Skript/algorithmen/Diagramm6.dia
% Creator: Dia v0.97.1
% CreationDate: Tue May  4 14:44:28 2010
% For: alex
% \usepackage{tikz}
% The following commands are not supported in PSTricks at present
% We define them conditionally, so when they are implemented,
% this pgf file will use them.
\ifx\du\undefined
  \newlength{\du}
\fi
\setlength{\du}{15\unitlength}
\begin{tikzpicture}
\pgftransformxscale{1.000000}
\pgftransformyscale{-1.000000}
\definecolor{dialinecolor}{rgb}{0.000000, 0.000000, 0.000000}
\pgfsetstrokecolor{dialinecolor}
\definecolor{dialinecolor}{rgb}{1.000000, 1.000000, 1.000000}
\pgfsetfillcolor{dialinecolor}
\pgfsetlinewidth{0.100000\du}
\pgfsetdash{}{0pt}
\pgfsetdash{}{0pt}
\pgfsetmiterjoin
\definecolor{dialinecolor}{rgb}{1.000000, 1.000000, 1.000000}
\pgfsetfillcolor{dialinecolor}
\fill (12.750000\du,3.900000\du)--(12.750000\du,4.900000\du)--(14.750000\du,4.900000\du)--(14.750000\du,3.900000\du)--cycle;
\definecolor{dialinecolor}{rgb}{0.000000, 0.000000, 0.000000}
\pgfsetstrokecolor{dialinecolor}
\draw (12.750000\du,3.900000\du)--(12.750000\du,4.900000\du)--(14.750000\du,4.900000\du)--(14.750000\du,3.900000\du)--cycle;
% setfont left to latex
\definecolor{dialinecolor}{rgb}{0.000000, 0.000000, 0.000000}
\pgfsetstrokecolor{dialinecolor}
\node[anchor=west] at (13.400000\du,4.600000\du){13};
\pgfsetlinewidth{0.100000\du}
\pgfsetdash{{1.000000\du}{1.000000\du}}{0\du}
\pgfsetdash{{1.000000\du}{1.000000\du}}{0\du}
\pgfsetbuttcap
{
\definecolor{dialinecolor}{rgb}{0.000000, 0.000000, 0.000000}
\pgfsetfillcolor{dialinecolor}
% was here!!!
\definecolor{dialinecolor}{rgb}{0.000000, 0.000000, 0.000000}
\pgfsetstrokecolor{dialinecolor}
\draw (9.800000\du,1.650000\du)--(12.750000\du,3.900000\du);
}
\pgfsetlinewidth{0.100000\du}
\pgfsetdash{}{0pt}
\pgfsetdash{}{0pt}
\pgfsetmiterjoin
\definecolor{dialinecolor}{rgb}{1.000000, 1.000000, 1.000000}
\pgfsetfillcolor{dialinecolor}
\fill (8.950000\du,6.150000\du)--(8.950000\du,8.700000\du)--(12.900000\du,8.700000\du)--(12.900000\du,6.150000\du)--cycle;
\definecolor{dialinecolor}{rgb}{0.000000, 0.000000, 0.000000}
\pgfsetstrokecolor{dialinecolor}
\draw (8.950000\du,6.150000\du)--(8.950000\du,8.700000\du)--(12.900000\du,8.700000\du)--(12.900000\du,6.150000\du)--cycle;
\pgfsetlinewidth{0.100000\du}
\pgfsetdash{}{0pt}
\pgfsetdash{}{0pt}
\pgfsetmiterjoin
\definecolor{dialinecolor}{rgb}{1.000000, 1.000000, 1.000000}
\pgfsetfillcolor{dialinecolor}
\fill (4.950000\du,10.150000\du)--(4.950000\du,13.100000\du)--(9.050000\du,13.100000\du)--(9.050000\du,10.150000\du)--cycle;
\definecolor{dialinecolor}{rgb}{0.000000, 0.000000, 0.000000}
\pgfsetstrokecolor{dialinecolor}
\draw (4.950000\du,10.150000\du)--(4.950000\du,13.100000\du)--(9.050000\du,13.100000\du)--(9.050000\du,10.150000\du)--cycle;
\pgfsetlinewidth{0.100000\du}
\pgfsetdash{}{0pt}
\pgfsetdash{}{0pt}
\pgfsetbuttcap
{
\definecolor{dialinecolor}{rgb}{0.000000, 0.000000, 0.000000}
\pgfsetfillcolor{dialinecolor}
% was here!!!
\pgfsetarrowsend{stealth}
\definecolor{dialinecolor}{rgb}{0.000000, 0.000000, 0.000000}
\pgfsetstrokecolor{dialinecolor}
\draw (12.750000\du,4.900000\du)--(10.925000\du,6.150000\du);
}
\pgfsetlinewidth{0.100000\du}
\pgfsetdash{}{0pt}
\pgfsetdash{}{0pt}
\pgfsetbuttcap
{
\definecolor{dialinecolor}{rgb}{0.000000, 0.000000, 0.000000}
\pgfsetfillcolor{dialinecolor}
% was here!!!
\pgfsetarrowsend{stealth}
\definecolor{dialinecolor}{rgb}{0.000000, 0.000000, 0.000000}
\pgfsetstrokecolor{dialinecolor}
\draw (8.950000\du,8.700000\du)--(7.000000\du,10.150000\du);
}
\pgfsetlinewidth{0.100000\du}
\pgfsetdash{}{0pt}
\pgfsetdash{}{0pt}
\pgfsetbuttcap
{
\definecolor{dialinecolor}{rgb}{0.000000, 0.000000, 0.000000}
\pgfsetfillcolor{dialinecolor}
% was here!!!
\pgfsetarrowsend{stealth}
\definecolor{dialinecolor}{rgb}{0.000000, 0.000000, 0.000000}
\pgfsetstrokecolor{dialinecolor}
\draw (12.900000\du,8.700000\du)--(14.650000\du,10.400000\du);
}
\pgfsetlinewidth{0.100000\du}
\pgfsetdash{}{0pt}
\pgfsetdash{}{0pt}
\pgfsetbuttcap
{
\definecolor{dialinecolor}{rgb}{0.000000, 0.000000, 0.000000}
\pgfsetfillcolor{dialinecolor}
% was here!!!
\definecolor{dialinecolor}{rgb}{0.000000, 0.000000, 0.000000}
\pgfsetstrokecolor{dialinecolor}
\draw (8.800000\du,7.550000\du)--(13.000000\du,7.600000\du);
}
\pgfsetlinewidth{0.100000\du}
\pgfsetdash{}{0pt}
\pgfsetdash{}{0pt}
\pgfsetbuttcap
{
\definecolor{dialinecolor}{rgb}{0.000000, 0.000000, 0.000000}
\pgfsetfillcolor{dialinecolor}
% was here!!!
\definecolor{dialinecolor}{rgb}{0.000000, 0.000000, 0.000000}
\pgfsetstrokecolor{dialinecolor}
\draw (4.950000\du,11.625000\du)--(9.050000\du,11.625000\du);
}
% setfont left to latex
\definecolor{dialinecolor}{rgb}{0.000000, 0.000000, 0.000000}
\pgfsetstrokecolor{dialinecolor}
\node[anchor=west] at (10.675000\du,7.175000\du){11 ($u$)};
% setfont left to latex
\definecolor{dialinecolor}{rgb}{0.000000, 0.000000, 0.000000}
\pgfsetstrokecolor{dialinecolor}
\node[anchor=west] at (6.750000\du,11.125000\du){10 ($v$)};
\pgfsetlinewidth{0.100000\du}
\pgfsetdash{}{0pt}
\pgfsetdash{}{0pt}
\pgfsetbuttcap
{
\definecolor{dialinecolor}{rgb}{0.000000, 0.000000, 0.000000}
\pgfsetfillcolor{dialinecolor}
% was here!!!
\pgfsetarrowsend{stealth}
\definecolor{dialinecolor}{rgb}{0.000000, 0.000000, 0.000000}
\pgfsetstrokecolor{dialinecolor}
\draw (4.950000\du,13.100000\du)--(4.050000\du,13.850000\du);
}
\pgfsetlinewidth{0.100000\du}
\pgfsetdash{}{0pt}
\pgfsetdash{}{0pt}
\pgfsetbuttcap
{
\definecolor{dialinecolor}{rgb}{0.000000, 0.000000, 0.000000}
\pgfsetfillcolor{dialinecolor}
% was here!!!
\pgfsetarrowsend{stealth}
\definecolor{dialinecolor}{rgb}{0.000000, 0.000000, 0.000000}
\pgfsetstrokecolor{dialinecolor}
\draw (9.050000\du,13.100000\du)--(10.050000\du,14.100000\du);
}
\end{tikzpicture}
 \\
        
        \subsection{Löschen in Suchbäumen}
            \begin{alltt}
			Streichen (x); zuerst Suchen(x), 
			Suchen endet in Knoten \( u, x \in S \)
			1. u ist Blatt
			streiche u, rson, lson, Verweis von parent(v) auf undef.
			2. u hat 1 Kind
			Streiche u, setze w als Kind von parent(u) an die Stelle von u.
			3. u hat 2 Kinder
			Suche v mit größtem Info-Element im linken Teilbaum von u, 
			(einmal nach links, dann immer nach rechts! \( \Ra \) v hat keinen rson)
			Ersetze u durch v & Streiche v unten (wie Fall 1 oder 2)
            \end{alltt}
            
            Laufzeit $\LO(h+1)$ , wobei h Höhe des Suchbaums ist.\\

        \subsection{Diskussion}
            Ein Suchbaum kann sich sehr schlecht verhalten. Werden etwa aufsteigende Zahlen eingefügt wie beschrieben,
            so ergibt sich eine entartete Baumform, die stark an eine Liste erinnert. Beispiel: \\
            % Graphic for TeX using PGF
% Title: /home/alex/wuala/WualaDrive/myfiles/Uni/4 Semester/Algorithmen/Skript/algorithmen/Diagramm8.dia
% Creator: Dia v0.97.1
% CreationDate: Tue May  4 15:00:19 2010
% For: alex
% \usepackage{tikz}
% The following commands are not supported in PSTricks at present
% We define them conditionally, so when they are implemented,
% this pgf file will use them.
\ifx\du\undefined
  \newlength{\du}
\fi
\setlength{\du}{15\unitlength}
\begin{tikzpicture}
\pgftransformxscale{1.000000}
\pgftransformyscale{-1.000000}
\definecolor{dialinecolor}{rgb}{0.000000, 0.000000, 0.000000}
\pgfsetstrokecolor{dialinecolor}
\definecolor{dialinecolor}{rgb}{1.000000, 1.000000, 1.000000}
\pgfsetfillcolor{dialinecolor}
\pgfsetlinewidth{0.100000\du}
\pgfsetdash{}{0pt}
\pgfsetdash{}{0pt}
\pgfsetmiterjoin
\definecolor{dialinecolor}{rgb}{1.000000, 1.000000, 1.000000}
\pgfsetfillcolor{dialinecolor}
\fill (6.500000\du,6.100000\du)--(6.500000\du,7.100000\du)--(8.500000\du,7.100000\du)--(8.500000\du,6.100000\du)--cycle;
\definecolor{dialinecolor}{rgb}{0.000000, 0.000000, 0.000000}
\pgfsetstrokecolor{dialinecolor}
\draw (6.500000\du,6.100000\du)--(6.500000\du,7.100000\du)--(8.500000\du,7.100000\du)--(8.500000\du,6.100000\du)--cycle;
\pgfsetlinewidth{0.100000\du}
\pgfsetdash{}{0pt}
\pgfsetdash{}{0pt}
\pgfsetmiterjoin
\definecolor{dialinecolor}{rgb}{1.000000, 1.000000, 1.000000}
\pgfsetfillcolor{dialinecolor}
\fill (10.850000\du,9.750000\du)--(10.850000\du,10.750000\du)--(12.850000\du,10.750000\du)--(12.850000\du,9.750000\du)--cycle;
\definecolor{dialinecolor}{rgb}{0.000000, 0.000000, 0.000000}
\pgfsetstrokecolor{dialinecolor}
\draw (10.850000\du,9.750000\du)--(10.850000\du,10.750000\du)--(12.850000\du,10.750000\du)--(12.850000\du,9.750000\du)--cycle;
\pgfsetlinewidth{0.100000\du}
\pgfsetdash{}{0pt}
\pgfsetdash{}{0pt}
\pgfsetmiterjoin
\definecolor{dialinecolor}{rgb}{1.000000, 1.000000, 1.000000}
\pgfsetfillcolor{dialinecolor}
\fill (15.300000\du,13.550000\du)--(15.300000\du,14.550000\du)--(17.450000\du,14.550000\du)--(17.450000\du,13.550000\du)--cycle;
\definecolor{dialinecolor}{rgb}{0.000000, 0.000000, 0.000000}
\pgfsetstrokecolor{dialinecolor}
\draw (15.300000\du,13.550000\du)--(15.300000\du,14.550000\du)--(17.450000\du,14.550000\du)--(17.450000\du,13.550000\du)--cycle;
\pgfsetlinewidth{0.100000\du}
\pgfsetdash{}{0pt}
\pgfsetdash{}{0pt}
\pgfsetbuttcap
{
\definecolor{dialinecolor}{rgb}{0.000000, 0.000000, 0.000000}
\pgfsetfillcolor{dialinecolor}
% was here!!!
\pgfsetarrowsend{stealth}
\definecolor{dialinecolor}{rgb}{0.000000, 0.000000, 0.000000}
\pgfsetstrokecolor{dialinecolor}
\draw (8.500000\du,7.100000\du)--(10.850000\du,9.750000\du);
}
\pgfsetlinewidth{0.100000\du}
\pgfsetdash{}{0pt}
\pgfsetdash{}{0pt}
\pgfsetbuttcap
{
\definecolor{dialinecolor}{rgb}{0.000000, 0.000000, 0.000000}
\pgfsetfillcolor{dialinecolor}
% was here!!!
\pgfsetarrowsend{stealth}
\definecolor{dialinecolor}{rgb}{0.000000, 0.000000, 0.000000}
\pgfsetstrokecolor{dialinecolor}
\draw (12.850000\du,10.750000\du)--(15.300000\du,13.550000\du);
}
\pgfsetlinewidth{0.100000\du}
\pgfsetdash{}{0pt}
\pgfsetdash{}{0pt}
\pgfsetbuttcap
{
\definecolor{dialinecolor}{rgb}{0.000000, 0.000000, 0.000000}
\pgfsetfillcolor{dialinecolor}
% was here!!!
\pgfsetarrowsend{stealth}
\definecolor{dialinecolor}{rgb}{0.000000, 0.000000, 0.000000}
\pgfsetstrokecolor{dialinecolor}
\draw (6.500000\du,7.100000\du)--(5.600000\du,8.200000\du);
}
\pgfsetlinewidth{0.100000\du}
\pgfsetdash{}{0pt}
\pgfsetdash{}{0pt}
\pgfsetbuttcap
{
\definecolor{dialinecolor}{rgb}{0.000000, 0.000000, 0.000000}
\pgfsetfillcolor{dialinecolor}
% was here!!!
\pgfsetarrowsend{stealth}
\definecolor{dialinecolor}{rgb}{0.000000, 0.000000, 0.000000}
\pgfsetstrokecolor{dialinecolor}
\draw (10.850000\du,10.750000\du)--(9.650000\du,12.350000\du);
}
\pgfsetlinewidth{0.100000\du}
\pgfsetdash{}{0pt}
\pgfsetdash{}{0pt}
\pgfsetbuttcap
{
\definecolor{dialinecolor}{rgb}{0.000000, 0.000000, 0.000000}
\pgfsetfillcolor{dialinecolor}
% was here!!!
\pgfsetarrowsend{stealth}
\definecolor{dialinecolor}{rgb}{0.000000, 0.000000, 0.000000}
\pgfsetstrokecolor{dialinecolor}
\draw (15.300000\du,14.550000\du)--(14.050000\du,16.100000\du);
}
\pgfsetlinewidth{0.100000\du}
\pgfsetdash{}{0pt}
\pgfsetdash{}{0pt}
\pgfsetbuttcap
{
\definecolor{dialinecolor}{rgb}{0.000000, 0.000000, 0.000000}
\pgfsetfillcolor{dialinecolor}
% was here!!!
\pgfsetarrowsend{stealth}
\definecolor{dialinecolor}{rgb}{0.000000, 0.000000, 0.000000}
\pgfsetstrokecolor{dialinecolor}
\draw (16.823709\du,14.600031\du)--(18.700000\du,16.900000\du);
}
% setfont left to latex
\definecolor{dialinecolor}{rgb}{0.000000, 0.000000, 0.000000}
\pgfsetstrokecolor{dialinecolor}
\node[anchor=west] at (19.450000\du,18.500000\du){.....};
\pgfsetlinewidth{0.100000\du}
\pgfsetdash{}{0pt}
\pgfsetdash{}{0pt}
\pgfsetmiterjoin
\pgfsetbuttcap
{
\definecolor{dialinecolor}{rgb}{0.000000, 0.000000, 0.000000}
\pgfsetfillcolor{dialinecolor}
% was here!!!
\pgfsetarrowsend{stealth}
\definecolor{dialinecolor}{rgb}{0.000000, 0.000000, 0.000000}
\pgfsetstrokecolor{dialinecolor}
\pgfpathmoveto{\pgfpoint{8.650000\du}{4.850000\du}}
\pgfpathcurveto{\pgfpoint{10.450000\du}{2.500000\du}}{\pgfpoint{21.200000\du}{15.300000\du}}{\pgfpoint{20.400000\du}{16.500000\du}}
\pgfusepath{stroke}
}
% setfont left to latex
\definecolor{dialinecolor}{rgb}{0.000000, 0.000000, 0.000000}
\pgfsetstrokecolor{dialinecolor}
\node[anchor=west] at (15.950000\du,8.500000\du){n};
\end{tikzpicture}
 \\
            Idee, dies zu lösen: \\
            \begin{enumerate}[1]
                \item Hoffe, dass es nicht vorkommt. (Unwahrscheinlich, dass ein solcher Input erfolgt)
                \item Baue Baum von Zeit zu Zeit neu auf.
                \item Balancierte Bäume.
            \end{enumerate}

    \section{Balancierte Bäume}
        Sei u Knoten im Suchbaum. Die Balance von u sei definiert als 
        $$
            Bal(u) = \text{Höhe}(\text{rson}) - \text{Höhe}(\text{lson})
        $$
        Setze Höhe des undefinierten TBs $= -1$.\\
        Beispiele s. \autoref{diag26:balance}

        
        \begin{figure}[htp]
			\centering
			\begin{tikzpicture}[grow via three points={
                         one child at (-1,-1) and two children at (-1,-1) and (1,-1)}]
\node at (0,0) (u) {u}
     child{node (v) {v}};

\node at (10,0) (x) {x}
	child{node (y) {y}
		child{node (z) {z}}};

\end{tikzpicture}
			\caption{$Bal(u) = 0 -1 = -1$, $Bal(x) = 0-2 = -2$}
			\label{diag26:balance}
        \end{figure}

        \begin{definition}
            Ein binärer Baum $T$ heißt AVL-Baum, falls gilt 
            $$
                \forall u \in T : \abs{Bal(u)} \leq 1 
            $$
        \end{definition}

        \begin{figure}[htp]
            \centering
            \begin{tikzpicture}[xscale=2,yscale=2,thick]
  \draw (0,0) -- (-135:1cm) -- ++(-120:1cm) -- ++(-105:1cm) ++(-0.15,0) node {0};;
  \draw (-135:1cm) -- ++(-60:1cm) ++(0.15,0) node {0};;
  \draw (0,0) -- (-45:1cm) -- ++(-60:1cm) ++(0.15,0) node {0};
  \draw (0,0.2) node {$-1$};
  \draw (-135:1cm) ++(-0.3,0) node {$-1$};
  \draw (-135:1cm) ++(-120:1cm) ++(-0.3,0) node {$-1$};
  \draw (-45:1cm) ++(0.2,0)  node {$1$};
\end{tikzpicture}
            \caption{Ein AVL Baum. An den Knoten ist die Balance des jeweiligen Teilbaums angegeben.}
            \label{diag9:avl-balance}
        \end{figure}

        \begin{definition}
            Fibonacci-Bäume (\autoref{diag10:fibonacci-tree})
            $$T_0, T_1, T_2, \ldots$$ sind definiert durch $T_0 = $ leer, $T_1 = \bullet$, $T_2 = /$, \\
            $T_h$ siehe Abbildung \ref{diag10:fibonacci-tree} \\
            Fibonacci-Zahlen: 
            $$F_0 = 0, F_1 = 1, F_h = F_{h-1} + F_{h-2}$$
        \end{definition}

        \begin{figure}[htp]
            \centering
            \resizebox{0.5\linewidth}{!}{%
\Tree [. \qroof{$T_{h-1}$}. \qroof{$T_{h-2}$}. !{\qbalance} ]
}
            \caption{Fibonacci-Baum}
            \label{diag10:fibonacci-tree}
        \end{figure}
        
        \emph{Behauptung:} $T_h$ enthält genau $F_h$ Blätter (für alle $h \geq 0$) \\
        Beweis per Induktion. (klar)\\
        \emph{Behauptung:} AVL-Bäume der Höhe $h$ haben $\geq F_h$ Blätter. \\
        \begin{proof}     
            \begin{itemize}
                \item $h=0$ \ $\bullet \ \ \text{ gilt für einen Knoten}$ 
                \item $h=1$ \ / oder \verb=\= oder /\verb=\=
                \item $h \geq 2$ : Erhalten den blattminimalen AVl-Baum der Höhe h durch Kombination von blattmin. AVL-Bäumen der Höhe $h-1$ und $h-2$.

                $\Ra \geq F_{h-1} + F_{h-2} = F_h$ Blätter. \\
                $F_h = h$-te Fibonacci -Zahl. 
                $$F_h = \frac{\alpha^h - \beta^h}{\sqrt{5}}$$ mit $\alpha = \frac{1+ \sqrt{5}}{2}, \beta = \frac{1-\sqrt{5}}{2}$.   
            \end{itemize}
        \end{proof}

        \begin{lemma}
            AVL-Bäume mit $n$ Knoten haben Höhe $\LO(\log n)$ 
        \end{lemma}

        \begin{proof}
            Baum hat $\leq n$ Blätter, also $F_h \leq n$, damit gilt 
            $$
                \frac{\alpha^h - \beta^h}{\sqrt{5}} \leq n
            $$ 
            Da $\btl \beta \btr < 1$ gilt: 
            \begin{align*}            
                \frac{\alpha^h - \beta^h}{\sqrt{5}} &\geq \frac{\alpha^h}{2 \sqrt{5}} \\
                \Ra \alpha^h &\leq 2 \sqrt{5} \cdot n. \\
                h \cdot \log \alpha &< \log (2\sqrt{5}) + \log n \\
                h &\leq \frac{\log 2\sqrt{5} + \log n}{\log \alpha} = \LO(\log n)
            \end{align*}
        \end{proof}

        Beim Einfügen/Streichen kann die Balance gestört werden, z.B. auf -2 oder 2 (siehe anschließendes Beispiel). Wir fordern aber für einen AVL Baum, dass ständig gilt 
        $$
            \forall u \in T : \abs{Bal(u)} \leq 1 
        $$
        und wissen von Fibonacci-Bäumen, dass die Höhe $\LO(\log n)$ beträgt. Es müssen also bei Operationen am Baum Korrekturen ausgeführt werden,
        dass die AVL-Eigenschaften erhalten bleiben.
   
        \subsection{Einfügen($w$) in AVL-Bäumen}
            Angenommen es wurde die AVL-Bedingung zerstört und es ist eine Balance von $\pm 2$ entstanden. Dann sei $u$ der tiefste dieser "`unbal."' Knoten. O.B.d.A. sei $Bal(u)=2$ \\
            
            Sei $w$ der neue Knoten. Der wurde rechts eingefügt. Sei $v$ rechtes Kind von $u$\\
            \begin{enumerate}[\text{Fall} 1)]
                \item Bal(v) = 1

                \begin{tabbing}
                    Es gilt: \= 1) Links-Rechts Ordnung wird aufrechterhalten:\\
                    \> $T_{L} \leq u \leq T_{A} \leq v \leq T_{B}$\\
                    \> 2) Nach Rotation haben $u$ und $v$ Balance 0.\\
                    \> Knoten in $T_{A}, T_{B}, T_{L}$ behalten ihre alte Balance.\\
                    \> 3) Höhe($v$) nach Rotation = Höhe($u$) vor Einfügen($w$).
                \end{tabbing}
                \item Bal(v) = -1\\
                Nach Einfügen von $w$ folge dem Pfad von $w$ zur Wurzel.\\
                Berechne auf diesem Pfad alle Balancen neu. \\
                Tritt Bal. -2/2 auf, führe Rotation/Doppelrotation aus.\\
                Laufzeit: $\LO(\log n)$ (Rot./Doppelrot. in $\LO(1)$)
                \item Bal(v) = 0\\
                Kann nicht vorkommen. AVL-Bedingung vorher schon verletzt.
            \end{enumerate}
            
            \subsubsection{Beispiel für Einfügeoperation}
                Es soll in den folgenden Baum das Element 3 eingefügt werden:
                \begin{center}
                    \begin{tikzpicture}[
    edge from parent path=
    {(\tikzparentnode.south) .. controls +(0,-.5) and +(0,.5)
        .. (\tikzchildnode.north)},
    every node/.style={draw,circle},
    label distance=-1mm,
    level/.style={sibling distance = 5cm/#1,
    level distance = 1.5cm}]
    \node [label=330:$1$]{1}
        child {node[label=330:$-1$] {$\square$}}
        child {node[label=330:$0$] {$2$}
            child {node[label=330:$-1$] {$\square$}}
            child {node[label=330:$-1$] {$\square$}}
        };
\end{tikzpicture}
                \end{center}
                Dieser hat jedoch nach einer Einfügung die nachfolgende Gestalt
                \begin{center}
                    \begin{tikzpicture}[
    edge from parent path=
    {(\tikzparentnode.south) .. controls +(0,-.5) and +(0,.5)
        .. (\tikzchildnode.north)},
    every node/.style={draw,circle},
    label distance=-1mm,
    level/.style={sibling distance = 5cm/#1,
    level distance = 1.5cm}]
    \node [label=330:$2$]{1}
        child {node[label=330:$-1$] {$\square$}}
        child {node[label=330:$1$] {2}
            child {node[label=330:$-1$] {$\square$}}
            child {node[label=330:$0$] {3}
                child {node[label=330:$-1$] {$\square$}}
                child {node[label=330:$-1$] {$\square$}}
            }
        };
\end{tikzpicture}
                \end{center}
                Da nun am obersten Knoten die Balance 2 entstanden ist, muss an der Wurzel rotiert werden. Dies führt zu folgendem Baum:
                \begin{center}
                    \begin{tikzpicture}[
    edge from parent path=
    {(\tikzparentnode.south) .. controls +(0,-.5) and +(0,.5)
        .. (\tikzchildnode.north)},
    every node/.style={draw,circle},
    label distance=-1mm,
    level/.style={sibling distance = 5cm/#1,
    level distance = 1.5cm}]
    \node [label=330:$0$]{2}
        child {node[label=330:$0$] {1}
            child {node[label=330:$-1$] {$\square$}}
            child {node[label=330:$-1$] {$\square$}}
        }
        child {node[label=330:$0$] {3}
            child {node[label=330:$-1$] {$\square$}}
            child {node[label=330:$-1$] {$\square$}}
        };
\end{tikzpicture}
                \end{center}
                Es sind dabei an allen Knoten die jeweiligen Balancen angegeben.

        
        \subsection{Streichen($w$) in AVL-Bäumen}
			\begin{algorithmic}
\begin{algorithm}
\Function{Streiche}{AVL}
	\State key $\gets$ \Call{Suche}{x}
	\If{\textsc{isBlatt}{u}}
		\State \Call{streiche}{u}
		\State \Call{rson}{parent(u)} $\gets$ \Call{lson}{parent}{u}
		$\gets$ \textbf{void}
	\EndIf
	\If{\textsc{numberOfChildren}{u} = 1}\Comment{o.B.d.A sei $u = rson(parent(u))$}
		\State \Call{rson}{parent(u)} $\gets$ w
		\State \Call{streiche}{u}
	\EndIf
	\If{\textsc{numberOfChildren}{u} > 1}
		\State $v \gets$  \Call{FindMaxKey}{\textsc{lson}(u)}
		\State \Call{ersetze}{u,v}
		\State \Call{streiche}{v}
	\EndIf
\EndFunction
	
\end{algorithm}
\end{algorithmic}
            Annahme: es gibt Knoten mit Bal $\pm 2$.\\
            Sei $u$ der tiefste solche Knoten.\\
            O.B.d.A. sei $Bal(u) = 2$.\\
            Sei $v$ das rechte Kind von $u$
            \begin{enumerate}[\text{Fall} 1)]
                \item Bal($v$) = 0\\
                Höhe insgesamt hat sich nicht geändert gegenüber vor Streichen. Können hier abbrechen!
                \item Bal($v$) = 1\\
                Beobachte: Der Teilbaum hat jetzt kleinere Höhe als zuvor. Balancen oben drüber ändern sich!\\
                $\Rightarrow$ Iteriere Rebal.-Prozess weiter oben.
            \end{enumerate}
            \begin{satz}
                Balancierte Bäume (AVL) erlauben Suchen/Einfügen/Streichen in Zeit $\LO(\log n)$, wobei $n$ Zahl der Knoten.
            \end{satz}
        
        \subsection{Anwendung (Schnitt von achsenparallelen Liniensegmenten)}
            Ziel: Anzahl der Schnittpunkte von achsenparallelen Liniensegmenten berechnen. Die Lage wird als allgemein angenommen.
            Ein naiver Algorithmus, der paarweise alle Elemente vergleicht hätte eine Laufzeit von $\LO(n^{2})$. 
            Wir wählen also eine bessere Alternative:
            
            \subsubsection{PlaneSweep}
                \emph{x-Struktur:} geordnete Liste von Endpunkten nach x-Koor. statisch, durch Sortieren erzeugt.\\
                \emph{y-Struktur:} repräsentiert einen Zustand der Sweepline $L$ (dynamisch). Speichern in $L$ horiz. Segmente, 
                die im Moment von $L$ gekreuzt werden geordnet nach y-Koordinate.\\
                $\rightarrow$ AVL-Baum unterstützt Einf./Streichen in Zeit $\LO(\log n)$\\
                
                \emph{Vertikale Segmente:} $(x, y_{u}, y_{o})$\\
                Wollen berechnen \# horiz. Segmente mit y-Koord. zwischen $y_{u}$ und $y_{o}$\\
                $\rightarrow$ berechne Rang($y_{u}$), Rang($y_{o}$). (\# Elemente, die kleiner sind)\\
                Rang($y_{o}$) - Rang($y_{u}$) = \# Schnittpunkt auf vert. Segment.\\
                zu tun: Bestimme Rang(x) in AVL-Baum.\\

                Merke in jedem Knoten die Zahl der Knoten im linken Teilbaum (l count).\\
                Suchen(x): Beim Rechtsabbiegen erhöhe Rangzähler um (l count + 1) $\rightarrow$ Rang(x) in $\LO(\log n)$\\
                $\Rightarrow$ Gesamtlaufzeit: $\LO(n \cdot \log n)$ \\
                Wollen wir die Kreuzungen explizit bestimmen, dann brauchen wir eine Laufzeit von $\LO(n \cdot \log n + k)$

    \section{(2,4)-Bäume (blattorientiert)}
        (2,4)-Bäume gehören zu den (a,b)-Bäumen.
        
        \begin{definition}
            Seien $a,b \in \N$ mit $a \geq 2$, $b \geq 2a-1$. T heißt (a,b)-Baum falls gelten:
            \begin{enumerate}[a)]
                \item alle Blätter von T haben gleiche Tiefe
                \item alle Knoten haben $\leq b$ Kinder
                \item alle Knoten (außer Wurzel) haben $\geq a$ Kinder
                \item Wurzel hat $\geq 2$ Kinder
            \end{enumerate}
        \end{definition}
        
        Das Ziel ist nun die Menge $S = \{x_1 < x_2 < ... < x_n\}$ abzuspeichern. Dabei sollen die Schlüssel in den Blättern aufsteigend von links
        nach rechts geordnet abgespeichert werden.
        
        \begin{enumerate}[1.]
            \item Schlüssel in Blättern.
            \item Innere Knoten $v$ mit $d$ Kindern hat Elemente $K_1(v), \ldots, K_{d-1}(v)$ \\
            $k_i(v)$ = Inhalt des rechtesten Blattes im i-ten Unterbaum. 
        \end{enumerate}


        \begin{bsp}
            $$S=\{2,5,7,11,15,17,19 \}$$ 
            Klar: Tiefe ist $\LO(\log n)$
            \begin{figure}
            	% Graphic for TeX using PGF
% Title: /home/alex/WualaDrive/myfiles/Uni/4 Semester/Algorithmen/Skript/algorithmen/Diagramm18.dia
% Creator: Dia v0.97.1
% CreationDate: Tue May 18 14:27:21 2010
% For: alex
% \usepackage{tikz}
% The following commands are not supported in PSTricks at present
% We define them conditionally, so when they are implemented,
% this pgf file will use them.
\ifx\du\undefined
  \newlength{\du}
\fi
\setlength{\du}{15\unitlength}
\begin{tikzpicture}
\pgftransformxscale{1.000000}
\pgftransformyscale{-1.000000}
\definecolor{dialinecolor}{rgb}{0.000000, 0.000000, 0.000000}
\pgfsetstrokecolor{dialinecolor}
\definecolor{dialinecolor}{rgb}{1.000000, 1.000000, 1.000000}
\pgfsetfillcolor{dialinecolor}
\pgfsetlinewidth{0.100000\du}
\pgfsetdash{}{0pt}
\pgfsetdash{}{0pt}
\pgfsetbuttcap
\pgfsetmiterjoin
\pgfsetlinewidth{0.100000\du}
\pgfsetbuttcap
\pgfsetmiterjoin
\pgfsetdash{}{0pt}
\definecolor{dialinecolor}{rgb}{1.000000, 1.000000, 1.000000}
\pgfsetfillcolor{dialinecolor}
\pgfpathmoveto{\pgfpoint{12.508333\du}{2.700000\du}}
\pgfpathlineto{\pgfpoint{15.941667\du}{2.700000\du}}
\pgfpathcurveto{\pgfpoint{16.415711\du}{2.700000\du}}{\pgfpoint{16.800000\du}{3.158908\du}}{\pgfpoint{16.800000\du}{3.725000\du}}
\pgfpathcurveto{\pgfpoint{16.800000\du}{4.291092\du}}{\pgfpoint{16.415711\du}{4.750000\du}}{\pgfpoint{15.941667\du}{4.750000\du}}
\pgfpathlineto{\pgfpoint{12.508333\du}{4.750000\du}}
\pgfpathcurveto{\pgfpoint{12.034289\du}{4.750000\du}}{\pgfpoint{11.650000\du}{4.291092\du}}{\pgfpoint{11.650000\du}{3.725000\du}}
\pgfpathcurveto{\pgfpoint{11.650000\du}{3.158908\du}}{\pgfpoint{12.034289\du}{2.700000\du}}{\pgfpoint{12.508333\du}{2.700000\du}}
\pgfusepath{fill}
\definecolor{dialinecolor}{rgb}{0.000000, 0.000000, 0.000000}
\pgfsetstrokecolor{dialinecolor}
\pgfpathmoveto{\pgfpoint{12.508333\du}{2.700000\du}}
\pgfpathlineto{\pgfpoint{15.941667\du}{2.700000\du}}
\pgfpathcurveto{\pgfpoint{16.415711\du}{2.700000\du}}{\pgfpoint{16.800000\du}{3.158908\du}}{\pgfpoint{16.800000\du}{3.725000\du}}
\pgfpathcurveto{\pgfpoint{16.800000\du}{4.291092\du}}{\pgfpoint{16.415711\du}{4.750000\du}}{\pgfpoint{15.941667\du}{4.750000\du}}
\pgfpathlineto{\pgfpoint{12.508333\du}{4.750000\du}}
\pgfpathcurveto{\pgfpoint{12.034289\du}{4.750000\du}}{\pgfpoint{11.650000\du}{4.291092\du}}{\pgfpoint{11.650000\du}{3.725000\du}}
\pgfpathcurveto{\pgfpoint{11.650000\du}{3.158908\du}}{\pgfpoint{12.034289\du}{2.700000\du}}{\pgfpoint{12.508333\du}{2.700000\du}}
\pgfusepath{stroke}
% setfont left to latex
\definecolor{dialinecolor}{rgb}{0.000000, 0.000000, 0.000000}
\pgfsetstrokecolor{dialinecolor}
\node at (14.225000\du,3.925000\du){11,19};
\pgfsetlinewidth{0.100000\du}
\pgfsetdash{}{0pt}
\pgfsetdash{}{0pt}
\pgfsetbuttcap
\pgfsetmiterjoin
\pgfsetlinewidth{0.100000\du}
\pgfsetbuttcap
\pgfsetmiterjoin
\pgfsetdash{}{0pt}
\definecolor{dialinecolor}{rgb}{1.000000, 1.000000, 1.000000}
\pgfsetfillcolor{dialinecolor}
\pgfpathmoveto{\pgfpoint{7.283333\du}{7.250000\du}}
\pgfpathlineto{\pgfpoint{10.816667\du}{7.250000\du}}
\pgfpathcurveto{\pgfpoint{11.304518\du}{7.250000\du}}{\pgfpoint{11.700000\du}{7.742487\du}}{\pgfpoint{11.700000\du}{8.350000\du}}
\pgfpathcurveto{\pgfpoint{11.700000\du}{8.957514\du}}{\pgfpoint{11.304518\du}{9.450000\du}}{\pgfpoint{10.816667\du}{9.450000\du}}
\pgfpathlineto{\pgfpoint{7.283333\du}{9.450000\du}}
\pgfpathcurveto{\pgfpoint{6.795482\du}{9.450000\du}}{\pgfpoint{6.400000\du}{8.957514\du}}{\pgfpoint{6.400000\du}{8.350000\du}}
\pgfpathcurveto{\pgfpoint{6.400000\du}{7.742487\du}}{\pgfpoint{6.795482\du}{7.250000\du}}{\pgfpoint{7.283333\du}{7.250000\du}}
\pgfusepath{fill}
\definecolor{dialinecolor}{rgb}{0.000000, 0.000000, 0.000000}
\pgfsetstrokecolor{dialinecolor}
\pgfpathmoveto{\pgfpoint{7.283333\du}{7.250000\du}}
\pgfpathlineto{\pgfpoint{10.816667\du}{7.250000\du}}
\pgfpathcurveto{\pgfpoint{11.304518\du}{7.250000\du}}{\pgfpoint{11.700000\du}{7.742487\du}}{\pgfpoint{11.700000\du}{8.350000\du}}
\pgfpathcurveto{\pgfpoint{11.700000\du}{8.957514\du}}{\pgfpoint{11.304518\du}{9.450000\du}}{\pgfpoint{10.816667\du}{9.450000\du}}
\pgfpathlineto{\pgfpoint{7.283333\du}{9.450000\du}}
\pgfpathcurveto{\pgfpoint{6.795482\du}{9.450000\du}}{\pgfpoint{6.400000\du}{8.957514\du}}{\pgfpoint{6.400000\du}{8.350000\du}}
\pgfpathcurveto{\pgfpoint{6.400000\du}{7.742487\du}}{\pgfpoint{6.795482\du}{7.250000\du}}{\pgfpoint{7.283333\du}{7.250000\du}}
\pgfusepath{stroke}
% setfont left to latex
\definecolor{dialinecolor}{rgb}{0.000000, 0.000000, 0.000000}
\pgfsetstrokecolor{dialinecolor}
\node at (9.050000\du,8.550000\du){2,5,7};
\pgfsetlinewidth{0.100000\du}
\pgfsetdash{}{0pt}
\pgfsetdash{}{0pt}
\pgfsetbuttcap
\pgfsetmiterjoin
\pgfsetlinewidth{0.100000\du}
\pgfsetbuttcap
\pgfsetmiterjoin
\pgfsetdash{}{0pt}
\definecolor{dialinecolor}{rgb}{1.000000, 1.000000, 1.000000}
\pgfsetfillcolor{dialinecolor}
\pgfpathmoveto{\pgfpoint{16.566667\du}{7.100000\du}}
\pgfpathlineto{\pgfpoint{20.033333\du}{7.100000\du}}
\pgfpathcurveto{\pgfpoint{20.511980\du}{7.100000\du}}{\pgfpoint{20.900000\du}{7.581294\du}}{\pgfpoint{20.900000\du}{8.175000\du}}
\pgfpathcurveto{\pgfpoint{20.900000\du}{8.768706\du}}{\pgfpoint{20.511980\du}{9.250000\du}}{\pgfpoint{20.033333\du}{9.250000\du}}
\pgfpathlineto{\pgfpoint{16.566667\du}{9.250000\du}}
\pgfpathcurveto{\pgfpoint{16.088020\du}{9.250000\du}}{\pgfpoint{15.700000\du}{8.768706\du}}{\pgfpoint{15.700000\du}{8.175000\du}}
\pgfpathcurveto{\pgfpoint{15.700000\du}{7.581294\du}}{\pgfpoint{16.088020\du}{7.100000\du}}{\pgfpoint{16.566667\du}{7.100000\du}}
\pgfusepath{fill}
\definecolor{dialinecolor}{rgb}{0.000000, 0.000000, 0.000000}
\pgfsetstrokecolor{dialinecolor}
\pgfpathmoveto{\pgfpoint{16.566667\du}{7.100000\du}}
\pgfpathlineto{\pgfpoint{20.033333\du}{7.100000\du}}
\pgfpathcurveto{\pgfpoint{20.511980\du}{7.100000\du}}{\pgfpoint{20.900000\du}{7.581294\du}}{\pgfpoint{20.900000\du}{8.175000\du}}
\pgfpathcurveto{\pgfpoint{20.900000\du}{8.768706\du}}{\pgfpoint{20.511980\du}{9.250000\du}}{\pgfpoint{20.033333\du}{9.250000\du}}
\pgfpathlineto{\pgfpoint{16.566667\du}{9.250000\du}}
\pgfpathcurveto{\pgfpoint{16.088020\du}{9.250000\du}}{\pgfpoint{15.700000\du}{8.768706\du}}{\pgfpoint{15.700000\du}{8.175000\du}}
\pgfpathcurveto{\pgfpoint{15.700000\du}{7.581294\du}}{\pgfpoint{16.088020\du}{7.100000\du}}{\pgfpoint{16.566667\du}{7.100000\du}}
\pgfusepath{stroke}
% setfont left to latex
\definecolor{dialinecolor}{rgb}{0.000000, 0.000000, 0.000000}
\pgfsetstrokecolor{dialinecolor}
\node at (18.300000\du,8.375000\du){15,17};
\pgfsetlinewidth{0.100000\du}
\pgfsetdash{}{0pt}
\pgfsetdash{}{0pt}
\pgfsetbuttcap
{
\definecolor{dialinecolor}{rgb}{0.000000, 0.000000, 0.000000}
\pgfsetfillcolor{dialinecolor}
% was here!!!
\pgfsetarrowsend{stealth}
\definecolor{dialinecolor}{rgb}{0.000000, 0.000000, 0.000000}
\pgfsetstrokecolor{dialinecolor}
\draw (15.208929\du,4.799475\du)--(17.271301\du,7.051636\du);
}
\pgfsetlinewidth{0.100000\du}
\pgfsetdash{}{0pt}
\pgfsetdash{}{0pt}
\pgfsetbuttcap
{
\definecolor{dialinecolor}{rgb}{0.000000, 0.000000, 0.000000}
\pgfsetfillcolor{dialinecolor}
% was here!!!
\pgfsetarrowsend{stealth}
\definecolor{dialinecolor}{rgb}{0.000000, 0.000000, 0.000000}
\pgfsetstrokecolor{dialinecolor}
\draw (13.021901\du,4.800233\du)--(10.336801\du,7.199960\du);
}
\definecolor{dialinecolor}{rgb}{1.000000, 1.000000, 1.000000}
\pgfsetfillcolor{dialinecolor}
\fill (2.450000\du,10.700000\du)--(2.450000\du,12.600000\du)--(4.450000\du,12.600000\du)--(4.450000\du,10.700000\du)--cycle;
\pgfsetlinewidth{0.100000\du}
\pgfsetdash{}{0pt}
\pgfsetdash{}{0pt}
\pgfsetmiterjoin
\definecolor{dialinecolor}{rgb}{0.000000, 0.000000, 0.000000}
\pgfsetstrokecolor{dialinecolor}
\draw (2.450000\du,10.700000\du)--(2.450000\du,12.600000\du)--(4.450000\du,12.600000\du)--(4.450000\du,10.700000\du)--cycle;
% setfont left to latex
\definecolor{dialinecolor}{rgb}{0.000000, 0.000000, 0.000000}
\pgfsetstrokecolor{dialinecolor}
\node at (3.450000\du,11.845000\du){2};
\definecolor{dialinecolor}{rgb}{1.000000, 1.000000, 1.000000}
\pgfsetfillcolor{dialinecolor}
\fill (5.300000\du,10.800000\du)--(5.300000\du,12.700000\du)--(7.300000\du,12.700000\du)--(7.300000\du,10.800000\du)--cycle;
\pgfsetlinewidth{0.100000\du}
\pgfsetdash{}{0pt}
\pgfsetdash{}{0pt}
\pgfsetmiterjoin
\definecolor{dialinecolor}{rgb}{0.000000, 0.000000, 0.000000}
\pgfsetstrokecolor{dialinecolor}
\draw (5.300000\du,10.800000\du)--(5.300000\du,12.700000\du)--(7.300000\du,12.700000\du)--(7.300000\du,10.800000\du)--cycle;
% setfont left to latex
\definecolor{dialinecolor}{rgb}{0.000000, 0.000000, 0.000000}
\pgfsetstrokecolor{dialinecolor}
\node at (6.300000\du,11.945000\du){5};
\definecolor{dialinecolor}{rgb}{1.000000, 1.000000, 1.000000}
\pgfsetfillcolor{dialinecolor}
\fill (7.996250\du,10.850000\du)--(7.996250\du,12.750000\du)--(9.503750\du,12.750000\du)--(9.503750\du,10.850000\du)--cycle;
\pgfsetlinewidth{0.100000\du}
\pgfsetdash{}{0pt}
\pgfsetdash{}{0pt}
\pgfsetmiterjoin
\definecolor{dialinecolor}{rgb}{0.000000, 0.000000, 0.000000}
\pgfsetstrokecolor{dialinecolor}
\draw (7.996250\du,10.850000\du)--(7.996250\du,12.750000\du)--(9.503750\du,12.750000\du)--(9.503750\du,10.850000\du)--cycle;
% setfont left to latex
\definecolor{dialinecolor}{rgb}{0.000000, 0.000000, 0.000000}
\pgfsetstrokecolor{dialinecolor}
\node at (8.750000\du,11.995000\du){7};
\definecolor{dialinecolor}{rgb}{1.000000, 1.000000, 1.000000}
\pgfsetfillcolor{dialinecolor}
\fill (10.300000\du,10.950000\du)--(10.300000\du,12.850000\du)--(12.300000\du,12.850000\du)--(12.300000\du,10.950000\du)--cycle;
\pgfsetlinewidth{0.100000\du}
\pgfsetdash{}{0pt}
\pgfsetdash{}{0pt}
\pgfsetmiterjoin
\definecolor{dialinecolor}{rgb}{0.000000, 0.000000, 0.000000}
\pgfsetstrokecolor{dialinecolor}
\draw (10.300000\du,10.950000\du)--(10.300000\du,12.850000\du)--(12.300000\du,12.850000\du)--(12.300000\du,10.950000\du)--cycle;
% setfont left to latex
\definecolor{dialinecolor}{rgb}{0.000000, 0.000000, 0.000000}
\pgfsetstrokecolor{dialinecolor}
\node at (11.300000\du,12.095000\du){11};
\definecolor{dialinecolor}{rgb}{1.000000, 1.000000, 1.000000}
\pgfsetfillcolor{dialinecolor}
\fill (15.200000\du,10.950000\du)--(15.200000\du,12.850000\du)--(17.200000\du,12.850000\du)--(17.200000\du,10.950000\du)--cycle;
\pgfsetlinewidth{0.100000\du}
\pgfsetdash{}{0pt}
\pgfsetdash{}{0pt}
\pgfsetmiterjoin
\definecolor{dialinecolor}{rgb}{0.000000, 0.000000, 0.000000}
\pgfsetstrokecolor{dialinecolor}
\draw (15.200000\du,10.950000\du)--(15.200000\du,12.850000\du)--(17.200000\du,12.850000\du)--(17.200000\du,10.950000\du)--cycle;
% setfont left to latex
\definecolor{dialinecolor}{rgb}{0.000000, 0.000000, 0.000000}
\pgfsetstrokecolor{dialinecolor}
\node at (16.200000\du,12.095000\du){15};
\definecolor{dialinecolor}{rgb}{1.000000, 1.000000, 1.000000}
\pgfsetfillcolor{dialinecolor}
\fill (17.800000\du,10.950000\du)--(17.800000\du,12.850000\du)--(19.800000\du,12.850000\du)--(19.800000\du,10.950000\du)--cycle;
\pgfsetlinewidth{0.100000\du}
\pgfsetdash{}{0pt}
\pgfsetdash{}{0pt}
\pgfsetmiterjoin
\definecolor{dialinecolor}{rgb}{0.000000, 0.000000, 0.000000}
\pgfsetstrokecolor{dialinecolor}
\draw (17.800000\du,10.950000\du)--(17.800000\du,12.850000\du)--(19.800000\du,12.850000\du)--(19.800000\du,10.950000\du)--cycle;
% setfont left to latex
\definecolor{dialinecolor}{rgb}{0.000000, 0.000000, 0.000000}
\pgfsetstrokecolor{dialinecolor}
\node at (18.800000\du,12.095000\du){17};
\definecolor{dialinecolor}{rgb}{1.000000, 1.000000, 1.000000}
\pgfsetfillcolor{dialinecolor}
\fill (20.350000\du,10.900000\du)--(20.350000\du,12.800000\du)--(22.350000\du,12.800000\du)--(22.350000\du,10.900000\du)--cycle;
\pgfsetlinewidth{0.100000\du}
\pgfsetdash{}{0pt}
\pgfsetdash{}{0pt}
\pgfsetmiterjoin
\definecolor{dialinecolor}{rgb}{0.000000, 0.000000, 0.000000}
\pgfsetstrokecolor{dialinecolor}
\draw (20.350000\du,10.900000\du)--(20.350000\du,12.800000\du)--(22.350000\du,12.800000\du)--(22.350000\du,10.900000\du)--cycle;
% setfont left to latex
\definecolor{dialinecolor}{rgb}{0.000000, 0.000000, 0.000000}
\pgfsetstrokecolor{dialinecolor}
\node at (21.350000\du,12.045000\du){19};
\pgfsetlinewidth{0.100000\du}
\pgfsetdash{}{0pt}
\pgfsetdash{}{0pt}
\pgfsetbuttcap
{
\definecolor{dialinecolor}{rgb}{0.000000, 0.000000, 0.000000}
\pgfsetfillcolor{dialinecolor}
% was here!!!
\pgfsetarrowsend{stealth}
\definecolor{dialinecolor}{rgb}{0.000000, 0.000000, 0.000000}
\pgfsetstrokecolor{dialinecolor}
\draw (17.448413\du,9.300311\du)--(16.200000\du,10.950000\du);
}
\pgfsetlinewidth{0.100000\du}
\pgfsetdash{}{0pt}
\pgfsetdash{}{0pt}
\pgfsetbuttcap
{
\definecolor{dialinecolor}{rgb}{0.000000, 0.000000, 0.000000}
\pgfsetfillcolor{dialinecolor}
% was here!!!
\pgfsetarrowsend{stealth}
\definecolor{dialinecolor}{rgb}{0.000000, 0.000000, 0.000000}
\pgfsetstrokecolor{dialinecolor}
\draw (18.502759\du,9.300311\du)--(18.800000\du,10.950000\du);
}
\pgfsetlinewidth{0.100000\du}
\pgfsetdash{}{0pt}
\pgfsetdash{}{0pt}
\pgfsetbuttcap
{
\definecolor{dialinecolor}{rgb}{0.000000, 0.000000, 0.000000}
\pgfsetfillcolor{dialinecolor}
% was here!!!
\pgfsetarrowsend{stealth}
\definecolor{dialinecolor}{rgb}{0.000000, 0.000000, 0.000000}
\pgfsetstrokecolor{dialinecolor}
\draw (19.233020\du,9.299213\du)--(20.520111\du,10.850052\du);
}
\pgfsetlinewidth{0.100000\du}
\pgfsetdash{}{0pt}
\pgfsetdash{}{0pt}
\pgfsetbuttcap
{
\definecolor{dialinecolor}{rgb}{0.000000, 0.000000, 0.000000}
\pgfsetfillcolor{dialinecolor}
% was here!!!
\pgfsetarrowsend{stealth}
\definecolor{dialinecolor}{rgb}{0.000000, 0.000000, 0.000000}
\pgfsetstrokecolor{dialinecolor}
\draw (6.769189\du,9.307126\du)--(3.450000\du,10.700000\du);
}
\pgfsetlinewidth{0.100000\du}
\pgfsetdash{}{0pt}
\pgfsetdash{}{0pt}
\pgfsetbuttcap
{
\definecolor{dialinecolor}{rgb}{0.000000, 0.000000, 0.000000}
\pgfsetfillcolor{dialinecolor}
% was here!!!
\pgfsetarrowsend{stealth}
\definecolor{dialinecolor}{rgb}{0.000000, 0.000000, 0.000000}
\pgfsetstrokecolor{dialinecolor}
\draw (8.120129\du,9.499658\du)--(7.108685\du,10.750171\du);
}
\pgfsetlinewidth{0.100000\du}
\pgfsetdash{}{0pt}
\pgfsetdash{}{0pt}
\pgfsetbuttcap
{
\definecolor{dialinecolor}{rgb}{0.000000, 0.000000, 0.000000}
\pgfsetfillcolor{dialinecolor}
% was here!!!
\pgfsetarrowsend{stealth}
\definecolor{dialinecolor}{rgb}{0.000000, 0.000000, 0.000000}
\pgfsetstrokecolor{dialinecolor}
\draw (8.950024\du,9.499719\du)--(8.836865\du,10.801050\du);
}
\pgfsetlinewidth{0.100000\du}
\pgfsetdash{}{0pt}
\pgfsetdash{}{0pt}
\pgfsetbuttcap
{
\definecolor{dialinecolor}{rgb}{0.000000, 0.000000, 0.000000}
\pgfsetfillcolor{dialinecolor}
% was here!!!
\pgfsetarrowsend{stealth}
\definecolor{dialinecolor}{rgb}{0.000000, 0.000000, 0.000000}
\pgfsetstrokecolor{dialinecolor}
\draw (9.775098\du,9.494043\du)--(10.666638\du,10.900696\du);
}
\end{tikzpicture}

            	\caption{(2,4)-Baum}
            \end{figure}
            
        \end{bsp}

        \subsection{Suchen in (2,4)-Bäumen}
            Suchen(k) $\checkmark$ \qquad Suche nach $k$ liefert Blatt $k^{'} = \text{ min }\{x \in S \mid k \leq x \}$ 
            
        \subsection{Einfügen in (2,4)-Bäumen}
            Einfügen(k): Zuerst \textit{Suchen}$(k)$ liefert Blatt $v_i$ mit \textit{Schlüssel} $(v_i) < k < \text{\textit{ Schlüssel}} (v_i) $ \\

        \begin{alltt}
            Sei \( w \) der parent von \( v \) : \\
            while \( w \) hat 5 kinder do Spalte(\( w \)) 
            \( w \) \( \Leftarrow \) parent(\( w \)) 
        \end{alltt}

        Laufzeit: $\LO(1+ \#$ Spaltungen) \\
        
        
        \subsection{Streichen in (2,4)-Bäumen}
            Streiche$(k)$ : Zuerst Suchen$(k)$ $\ra$ Endet in Blatt v mit Schlüssel k. \\
            \begin{enumerate}[1.]
                \item Fall:  $K$ steht auch in parent$(v) = w$ \\
                % Graphic for TeX using PGF
% Title: /home/alex/WualaDrive/myfiles/Uni/4 Semester/Algorithmen/Skript/algorithmen/Diagramm20.dia
% Creator: Dia v0.97.1
% CreationDate: Tue May 18 14:41:29 2010
% For: alex
% \usepackage{tikz}
% The following commands are not supported in PSTricks at present
% We define them conditionally, so when they are implemented,
% this pgf file will use them.
\ifx\du\undefined
  \newlength{\du}
\fi
\setlength{\du}{15\unitlength}
\begin{tikzpicture}
\pgftransformxscale{1.000000}
\pgftransformyscale{-1.000000}
\definecolor{dialinecolor}{rgb}{0.000000, 0.000000, 0.000000}
\pgfsetstrokecolor{dialinecolor}
\definecolor{dialinecolor}{rgb}{1.000000, 1.000000, 1.000000}
\pgfsetfillcolor{dialinecolor}
\pgfsetlinewidth{0.100000\du}
\pgfsetdash{}{0pt}
\pgfsetdash{}{0pt}
\pgfsetbuttcap
\pgfsetmiterjoin
\pgfsetlinewidth{0.100000\du}
\pgfsetbuttcap
\pgfsetmiterjoin
\pgfsetdash{}{0pt}
\definecolor{dialinecolor}{rgb}{1.000000, 1.000000, 1.000000}
\pgfsetfillcolor{dialinecolor}
\pgfpathmoveto{\pgfpoint{8.933333\du}{4.900000\du}}
\pgfpathlineto{\pgfpoint{10.266667\du}{4.900000\du}}
\pgfpathcurveto{\pgfpoint{10.450762\du}{4.900000\du}}{\pgfpoint{10.600000\du}{5.347715\du}}{\pgfpoint{10.600000\du}{5.900000\du}}
\pgfpathcurveto{\pgfpoint{10.600000\du}{6.452285\du}}{\pgfpoint{10.450762\du}{6.900000\du}}{\pgfpoint{10.266667\du}{6.900000\du}}
\pgfpathlineto{\pgfpoint{8.933333\du}{6.900000\du}}
\pgfpathcurveto{\pgfpoint{8.749238\du}{6.900000\du}}{\pgfpoint{8.600000\du}{6.452285\du}}{\pgfpoint{8.600000\du}{5.900000\du}}
\pgfpathcurveto{\pgfpoint{8.600000\du}{5.347715\du}}{\pgfpoint{8.749238\du}{4.900000\du}}{\pgfpoint{8.933333\du}{4.900000\du}}
\pgfusepath{fill}
\definecolor{dialinecolor}{rgb}{0.000000, 0.000000, 0.000000}
\pgfsetstrokecolor{dialinecolor}
\pgfpathmoveto{\pgfpoint{8.933333\du}{4.900000\du}}
\pgfpathlineto{\pgfpoint{10.266667\du}{4.900000\du}}
\pgfpathcurveto{\pgfpoint{10.450762\du}{4.900000\du}}{\pgfpoint{10.600000\du}{5.347715\du}}{\pgfpoint{10.600000\du}{5.900000\du}}
\pgfpathcurveto{\pgfpoint{10.600000\du}{6.452285\du}}{\pgfpoint{10.450762\du}{6.900000\du}}{\pgfpoint{10.266667\du}{6.900000\du}}
\pgfpathlineto{\pgfpoint{8.933333\du}{6.900000\du}}
\pgfpathcurveto{\pgfpoint{8.749238\du}{6.900000\du}}{\pgfpoint{8.600000\du}{6.452285\du}}{\pgfpoint{8.600000\du}{5.900000\du}}
\pgfpathcurveto{\pgfpoint{8.600000\du}{5.347715\du}}{\pgfpoint{8.749238\du}{4.900000\du}}{\pgfpoint{8.933333\du}{4.900000\du}}
\pgfusepath{stroke}
% setfont left to latex
\definecolor{dialinecolor}{rgb}{0.000000, 0.000000, 0.000000}
\pgfsetstrokecolor{dialinecolor}
\node at (9.600000\du,6.100000\du){k};
% setfont left to latex
\definecolor{dialinecolor}{rgb}{0.000000, 0.000000, 0.000000}
\pgfsetstrokecolor{dialinecolor}
\node[anchor=west] at (6.600000\du,5.200000\du){w};
\definecolor{dialinecolor}{rgb}{1.000000, 1.000000, 1.000000}
\pgfsetfillcolor{dialinecolor}
\fill (5.650000\du,9.300000\du)--(5.650000\du,11.200000\du)--(7.650000\du,11.200000\du)--(7.650000\du,9.300000\du)--cycle;
\pgfsetlinewidth{0.100000\du}
\pgfsetdash{}{0pt}
\pgfsetdash{}{0pt}
\pgfsetmiterjoin
\definecolor{dialinecolor}{rgb}{0.000000, 0.000000, 0.000000}
\pgfsetstrokecolor{dialinecolor}
\draw (5.650000\du,9.300000\du)--(5.650000\du,11.200000\du)--(7.650000\du,11.200000\du)--(7.650000\du,9.300000\du)--cycle;
% setfont left to latex
\definecolor{dialinecolor}{rgb}{0.000000, 0.000000, 0.000000}
\pgfsetstrokecolor{dialinecolor}
\node at (6.650000\du,10.445000\du){k};
\pgfsetlinewidth{0.100000\du}
\pgfsetdash{}{0pt}
\pgfsetdash{}{0pt}
\pgfsetbuttcap
{
\definecolor{dialinecolor}{rgb}{0.000000, 0.000000, 0.000000}
\pgfsetfillcolor{dialinecolor}
% was here!!!
\definecolor{dialinecolor}{rgb}{0.000000, 0.000000, 0.000000}
\pgfsetstrokecolor{dialinecolor}
\draw (6.650000\du,9.300000\du)--(8.769952\du,6.856665\du);
}
\pgfsetlinewidth{0.100000\du}
\pgfsetdash{{1.000000\du}{1.000000\du}}{0\du}
\pgfsetdash{{1.000000\du}{1.000000\du}}{0\du}
\pgfsetbuttcap
{
\definecolor{dialinecolor}{rgb}{0.000000, 0.000000, 0.000000}
\pgfsetfillcolor{dialinecolor}
% was here!!!
\definecolor{dialinecolor}{rgb}{0.000000, 0.000000, 0.000000}
\pgfsetstrokecolor{dialinecolor}
\draw (4.250000\du,8.300000\du)--(8.500000\du,6.200000\du);
}
\pgfsetlinewidth{0.100000\du}
\pgfsetdash{{1.000000\du}{1.000000\du}}{0\du}
\pgfsetdash{{1.000000\du}{1.000000\du}}{0\du}
\pgfsetbuttcap
{
\definecolor{dialinecolor}{rgb}{0.000000, 0.000000, 0.000000}
\pgfsetfillcolor{dialinecolor}
% was here!!!
\definecolor{dialinecolor}{rgb}{0.000000, 0.000000, 0.000000}
\pgfsetstrokecolor{dialinecolor}
\draw (10.244019\du,6.949512\du)--(12.300000\du,10.300000\du);
}
% setfont left to latex
\definecolor{dialinecolor}{rgb}{0.000000, 0.000000, 0.000000}
\pgfsetstrokecolor{dialinecolor}
\node[anchor=west] at (2.200000\du,9.150000\du){v};
\pgfsetlinewidth{0.100000\du}
\pgfsetdash{}{0pt}
\pgfsetdash{}{0pt}
\pgfsetbuttcap
{
\definecolor{dialinecolor}{rgb}{0.000000, 0.000000, 0.000000}
\pgfsetfillcolor{dialinecolor}
% was here!!!
\pgfsetarrowsend{stealth}
\definecolor{dialinecolor}{rgb}{0.000000, 0.000000, 0.000000}
\pgfsetstrokecolor{dialinecolor}
\pgfpathmoveto{\pgfpoint{12.999725\du}{11.199797\du}}
\pgfpatharc{127}{53}{5.000313\du and 5.000313\du}
\pgfusepath{stroke}
}
% setfont left to latex
\definecolor{dialinecolor}{rgb}{0.000000, 0.000000, 0.000000}
\pgfsetstrokecolor{dialinecolor}
\node[anchor=west] at (14.000000\du,8.350000\du){streiche k };
% setfont left to latex
\definecolor{dialinecolor}{rgb}{0.000000, 0.000000, 0.000000}
\pgfsetstrokecolor{dialinecolor}
\node[anchor=west] at (14.000000\du,9.150000\du){in beiden};
\pgfsetlinewidth{0.100000\du}
\pgfsetdash{}{0pt}
\pgfsetdash{}{0pt}
\pgfsetbuttcap
\pgfsetmiterjoin
\pgfsetlinewidth{0.100000\du}
\pgfsetbuttcap
\pgfsetmiterjoin
\pgfsetdash{}{0pt}
\definecolor{dialinecolor}{rgb}{1.000000, 1.000000, 1.000000}
\pgfsetfillcolor{dialinecolor}
\pgfpathmoveto{\pgfpoint{23.450000\du}{6.250000\du}}
\pgfpathlineto{\pgfpoint{26.050000\du}{6.250000\du}}
\pgfpathcurveto{\pgfpoint{26.408985\du}{6.250000\du}}{\pgfpoint{26.700000\du}{6.585786\du}}{\pgfpoint{26.700000\du}{7.000000\du}}
\pgfpathcurveto{\pgfpoint{26.700000\du}{7.414214\du}}{\pgfpoint{26.408985\du}{7.750000\du}}{\pgfpoint{26.050000\du}{7.750000\du}}
\pgfpathlineto{\pgfpoint{23.450000\du}{7.750000\du}}
\pgfpathcurveto{\pgfpoint{23.091015\du}{7.750000\du}}{\pgfpoint{22.800000\du}{7.414214\du}}{\pgfpoint{22.800000\du}{7.000000\du}}
\pgfpathcurveto{\pgfpoint{22.800000\du}{6.585786\du}}{\pgfpoint{23.091015\du}{6.250000\du}}{\pgfpoint{23.450000\du}{6.250000\du}}
\pgfusepath{fill}
\definecolor{dialinecolor}{rgb}{0.000000, 0.000000, 0.000000}
\pgfsetstrokecolor{dialinecolor}
\pgfpathmoveto{\pgfpoint{23.450000\du}{6.250000\du}}
\pgfpathlineto{\pgfpoint{26.050000\du}{6.250000\du}}
\pgfpathcurveto{\pgfpoint{26.408985\du}{6.250000\du}}{\pgfpoint{26.700000\du}{6.585786\du}}{\pgfpoint{26.700000\du}{7.000000\du}}
\pgfpathcurveto{\pgfpoint{26.700000\du}{7.414214\du}}{\pgfpoint{26.408985\du}{7.750000\du}}{\pgfpoint{26.050000\du}{7.750000\du}}
\pgfpathlineto{\pgfpoint{23.450000\du}{7.750000\du}}
\pgfpathcurveto{\pgfpoint{23.091015\du}{7.750000\du}}{\pgfpoint{22.800000\du}{7.414214\du}}{\pgfpoint{22.800000\du}{7.000000\du}}
\pgfpathcurveto{\pgfpoint{22.800000\du}{6.585786\du}}{\pgfpoint{23.091015\du}{6.250000\du}}{\pgfpoint{23.450000\du}{6.250000\du}}
\pgfusepath{stroke}
% setfont left to latex
\definecolor{dialinecolor}{rgb}{0.000000, 0.000000, 0.000000}
\pgfsetstrokecolor{dialinecolor}
\node at (24.750000\du,7.200000\du){......};
% setfont left to latex
\definecolor{dialinecolor}{rgb}{0.000000, 0.000000, 0.000000}
\pgfsetstrokecolor{dialinecolor}
\node[anchor=west] at (21.000000\du,6.500000\du){w};
\end{tikzpicture}

                \item
                $v$ rechtestes Kind $w$ sei Vorgänger von $v$ und $v^{'}$ der linke Nachbar mit Schlüssel $k^{'}$.\\
                Streiche $v$ und Ersetzte Vorkommen von $k$ durch $k^{'}$. Streiche $k^{'}$ im $parent(w)$, da nun höchster Schlüssel. 
                (vgl. \autoref{diag21:rechtestes_Kind_streichen})
                \begin{figure}
                	% Graphic for TeX using PGF
% Title: /home/alex/WualaDrive/myfiles/Uni/4 Semester/Algorithmen/Skript/algorithmen/Diagramm21.dia
% Creator: Dia v0.97.1
% CreationDate: Tue May 18 14:45:57 2010
% For: alex
% \usepackage{tikz}
% The following commands are not supported in PSTricks at present
% We define them conditionally, so when they are implemented,
% this pgf file will use them.
\ifx\du\undefined
  \newlength{\du}
\fi
\setlength{\du}{15\unitlength}
\begin{tikzpicture}
\pgftransformxscale{1.000000}
\pgftransformyscale{-1.000000}
\definecolor{dialinecolor}{rgb}{0.000000, 0.000000, 0.000000}
\pgfsetstrokecolor{dialinecolor}
\definecolor{dialinecolor}{rgb}{1.000000, 1.000000, 1.000000}
\pgfsetfillcolor{dialinecolor}
\pgfsetlinewidth{0.100000\du}
\pgfsetdash{}{0pt}
\pgfsetdash{}{0pt}
\pgfsetbuttcap
\pgfsetmiterjoin
\pgfsetlinewidth{0.100000\du}
\pgfsetbuttcap
\pgfsetmiterjoin
\pgfsetdash{}{0pt}
\definecolor{dialinecolor}{rgb}{1.000000, 1.000000, 1.000000}
\pgfsetfillcolor{dialinecolor}
\pgfpathmoveto{\pgfpoint{16.383333\du}{6.350000\du}}
\pgfpathlineto{\pgfpoint{17.716667\du}{6.350000\du}}
\pgfpathcurveto{\pgfpoint{17.900762\du}{6.350000\du}}{\pgfpoint{18.050000\du}{6.797715\du}}{\pgfpoint{18.050000\du}{7.350000\du}}
\pgfpathcurveto{\pgfpoint{18.050000\du}{7.902285\du}}{\pgfpoint{17.900762\du}{8.350000\du}}{\pgfpoint{17.716667\du}{8.350000\du}}
\pgfpathlineto{\pgfpoint{16.383333\du}{8.350000\du}}
\pgfpathcurveto{\pgfpoint{16.199238\du}{8.350000\du}}{\pgfpoint{16.050000\du}{7.902285\du}}{\pgfpoint{16.050000\du}{7.350000\du}}
\pgfpathcurveto{\pgfpoint{16.050000\du}{6.797715\du}}{\pgfpoint{16.199238\du}{6.350000\du}}{\pgfpoint{16.383333\du}{6.350000\du}}
\pgfusepath{fill}
\definecolor{dialinecolor}{rgb}{0.000000, 0.000000, 0.000000}
\pgfsetstrokecolor{dialinecolor}
\pgfpathmoveto{\pgfpoint{16.383333\du}{6.350000\du}}
\pgfpathlineto{\pgfpoint{17.716667\du}{6.350000\du}}
\pgfpathcurveto{\pgfpoint{17.900762\du}{6.350000\du}}{\pgfpoint{18.050000\du}{6.797715\du}}{\pgfpoint{18.050000\du}{7.350000\du}}
\pgfpathcurveto{\pgfpoint{18.050000\du}{7.902285\du}}{\pgfpoint{17.900762\du}{8.350000\du}}{\pgfpoint{17.716667\du}{8.350000\du}}
\pgfpathlineto{\pgfpoint{16.383333\du}{8.350000\du}}
\pgfpathcurveto{\pgfpoint{16.199238\du}{8.350000\du}}{\pgfpoint{16.050000\du}{7.902285\du}}{\pgfpoint{16.050000\du}{7.350000\du}}
\pgfpathcurveto{\pgfpoint{16.050000\du}{6.797715\du}}{\pgfpoint{16.199238\du}{6.350000\du}}{\pgfpoint{16.383333\du}{6.350000\du}}
\pgfusepath{stroke}
% setfont left to latex
\definecolor{dialinecolor}{rgb}{0.000000, 0.000000, 0.000000}
\pgfsetstrokecolor{dialinecolor}
\node at (17.050000\du,7.550000\du){k'};
% setfont left to latex
\definecolor{dialinecolor}{rgb}{0.000000, 0.000000, 0.000000}
\pgfsetstrokecolor{dialinecolor}
\node[anchor=west] at (14.400000\du,5.950000\du){w};
\pgfsetlinewidth{0.100000\du}
\pgfsetdash{}{0pt}
\pgfsetdash{}{0pt}
\pgfsetbuttcap
\pgfsetmiterjoin
\pgfsetlinewidth{0.100000\du}
\pgfsetbuttcap
\pgfsetmiterjoin
\pgfsetdash{}{0pt}
\definecolor{dialinecolor}{rgb}{1.000000, 1.000000, 1.000000}
\pgfsetfillcolor{dialinecolor}
\pgfpathmoveto{\pgfpoint{19.533333\du}{2.600000\du}}
\pgfpathlineto{\pgfpoint{20.866667\du}{2.600000\du}}
\pgfpathcurveto{\pgfpoint{21.050762\du}{2.600000\du}}{\pgfpoint{21.200000\du}{3.047715\du}}{\pgfpoint{21.200000\du}{3.600000\du}}
\pgfpathcurveto{\pgfpoint{21.200000\du}{4.152285\du}}{\pgfpoint{21.050762\du}{4.600000\du}}{\pgfpoint{20.866667\du}{4.600000\du}}
\pgfpathlineto{\pgfpoint{19.533333\du}{4.600000\du}}
\pgfpathcurveto{\pgfpoint{19.349238\du}{4.600000\du}}{\pgfpoint{19.200000\du}{4.152285\du}}{\pgfpoint{19.200000\du}{3.600000\du}}
\pgfpathcurveto{\pgfpoint{19.200000\du}{3.047715\du}}{\pgfpoint{19.349238\du}{2.600000\du}}{\pgfpoint{19.533333\du}{2.600000\du}}
\pgfusepath{fill}
\definecolor{dialinecolor}{rgb}{0.000000, 0.000000, 0.000000}
\pgfsetstrokecolor{dialinecolor}
\pgfpathmoveto{\pgfpoint{19.533333\du}{2.600000\du}}
\pgfpathlineto{\pgfpoint{20.866667\du}{2.600000\du}}
\pgfpathcurveto{\pgfpoint{21.050762\du}{2.600000\du}}{\pgfpoint{21.200000\du}{3.047715\du}}{\pgfpoint{21.200000\du}{3.600000\du}}
\pgfpathcurveto{\pgfpoint{21.200000\du}{4.152285\du}}{\pgfpoint{21.050762\du}{4.600000\du}}{\pgfpoint{20.866667\du}{4.600000\du}}
\pgfpathlineto{\pgfpoint{19.533333\du}{4.600000\du}}
\pgfpathcurveto{\pgfpoint{19.349238\du}{4.600000\du}}{\pgfpoint{19.200000\du}{4.152285\du}}{\pgfpoint{19.200000\du}{3.600000\du}}
\pgfpathcurveto{\pgfpoint{19.200000\du}{3.047715\du}}{\pgfpoint{19.349238\du}{2.600000\du}}{\pgfpoint{19.533333\du}{2.600000\du}}
\pgfusepath{stroke}
% setfont left to latex
\definecolor{dialinecolor}{rgb}{0.000000, 0.000000, 0.000000}
\pgfsetstrokecolor{dialinecolor}
\node at (20.200000\du,3.800000\du){k};
\pgfsetlinewidth{0.100000\du}
\pgfsetdash{}{0pt}
\pgfsetdash{}{0pt}
\pgfsetbuttcap
{
\definecolor{dialinecolor}{rgb}{0.000000, 0.000000, 0.000000}
\pgfsetfillcolor{dialinecolor}
% was here!!!
\definecolor{dialinecolor}{rgb}{0.000000, 0.000000, 0.000000}
\pgfsetstrokecolor{dialinecolor}
\draw (19.312915\du,4.466089\du)--(17.383333\du,6.350000\du);
}
\pgfsetlinewidth{0.100000\du}
\pgfsetdash{}{0pt}
\pgfsetdash{}{0pt}
\pgfsetbuttcap
{
\definecolor{dialinecolor}{rgb}{0.000000, 0.000000, 0.000000}
\pgfsetfillcolor{dialinecolor}
% was here!!!
\definecolor{dialinecolor}{rgb}{0.000000, 0.000000, 0.000000}
\pgfsetstrokecolor{dialinecolor}
\draw (20.930107\du,4.625626\du)--(22.300000\du,6.550000\du);
}
\definecolor{dialinecolor}{rgb}{1.000000, 1.000000, 1.000000}
\pgfsetfillcolor{dialinecolor}
\fill (15.950000\du,10.850000\du)--(15.950000\du,12.750000\du)--(17.950000\du,12.750000\du)--(17.950000\du,10.850000\du)--cycle;
\pgfsetlinewidth{0.100000\du}
\pgfsetdash{}{0pt}
\pgfsetdash{}{0pt}
\pgfsetmiterjoin
\definecolor{dialinecolor}{rgb}{0.000000, 0.000000, 0.000000}
\pgfsetstrokecolor{dialinecolor}
\draw (15.950000\du,10.850000\du)--(15.950000\du,12.750000\du)--(17.950000\du,12.750000\du)--(17.950000\du,10.850000\du)--cycle;
% setfont left to latex
\definecolor{dialinecolor}{rgb}{0.000000, 0.000000, 0.000000}
\pgfsetstrokecolor{dialinecolor}
\node at (16.950000\du,11.995000\du){k'};
% setfont left to latex
\definecolor{dialinecolor}{rgb}{0.000000, 0.000000, 0.000000}
\pgfsetstrokecolor{dialinecolor}
\node[anchor=west] at (18.150000\du,10.550000\du){v'};
\pgfsetlinewidth{0.100000\du}
\pgfsetdash{}{0pt}
\pgfsetdash{}{0pt}
\pgfsetbuttcap
{
\definecolor{dialinecolor}{rgb}{0.000000, 0.000000, 0.000000}
\pgfsetfillcolor{dialinecolor}
% was here!!!
\definecolor{dialinecolor}{rgb}{0.000000, 0.000000, 0.000000}
\pgfsetstrokecolor{dialinecolor}
\draw (16.972473\du,10.799945\du)--(17.026404\du,8.400031\du);
}
\pgfsetlinewidth{0.100000\du}
\pgfsetdash{}{0pt}
\pgfsetdash{}{0pt}
\pgfsetbuttcap
{
\definecolor{dialinecolor}{rgb}{0.000000, 0.000000, 0.000000}
\pgfsetfillcolor{dialinecolor}
% was here!!!
\definecolor{dialinecolor}{rgb}{0.000000, 0.000000, 0.000000}
\pgfsetstrokecolor{dialinecolor}
\draw (16.050000\du,7.350000\du)--(11.550000\du,10.350000\du);
}
\pgfsetlinewidth{0.100000\du}
\pgfsetdash{}{0pt}
\pgfsetdash{}{0pt}
\pgfsetbuttcap
{
\definecolor{dialinecolor}{rgb}{0.000000, 0.000000, 0.000000}
\pgfsetfillcolor{dialinecolor}
% was here!!!
\definecolor{dialinecolor}{rgb}{0.000000, 0.000000, 0.000000}
\pgfsetstrokecolor{dialinecolor}
\draw (16.186719\du,8.225977\du)--(13.650000\du,10.800000\du);
}
\pgfsetlinewidth{0.100000\du}
\pgfsetdash{}{0pt}
\pgfsetdash{}{0pt}
\pgfsetbuttcap
{
\definecolor{dialinecolor}{rgb}{0.000000, 0.000000, 0.000000}
\pgfsetfillcolor{dialinecolor}
% was here!!!
\definecolor{dialinecolor}{rgb}{0.000000, 0.000000, 0.000000}
\pgfsetstrokecolor{dialinecolor}
\draw (17.902814\du,8.276514\du)--(21.100000\du,11.750000\du);
}
\definecolor{dialinecolor}{rgb}{1.000000, 1.000000, 1.000000}
\pgfsetfillcolor{dialinecolor}
\fill (21.015000\du,11.500000\du)--(21.015000\du,13.400000\du)--(22.485000\du,13.400000\du)--(22.485000\du,11.500000\du)--cycle;
\pgfsetlinewidth{0.100000\du}
\pgfsetdash{}{0pt}
\pgfsetdash{}{0pt}
\pgfsetmiterjoin
\definecolor{dialinecolor}{rgb}{0.000000, 0.000000, 0.000000}
\pgfsetstrokecolor{dialinecolor}
\draw (21.015000\du,11.500000\du)--(21.015000\du,13.400000\du)--(22.485000\du,13.400000\du)--(22.485000\du,11.500000\du)--cycle;
% setfont left to latex
\definecolor{dialinecolor}{rgb}{0.000000, 0.000000, 0.000000}
\pgfsetstrokecolor{dialinecolor}
\node at (21.750000\du,12.645000\du){k};
% setfont left to latex
\definecolor{dialinecolor}{rgb}{0.000000, 0.000000, 0.000000}
\pgfsetstrokecolor{dialinecolor}
\node[anchor=west] at (23.050000\du,11.550000\du){v};
\pgfsetlinewidth{0.100000\du}
\pgfsetdash{}{0pt}
\pgfsetdash{}{0pt}
\pgfsetbuttcap
{
\definecolor{dialinecolor}{rgb}{0.000000, 0.000000, 0.000000}
\pgfsetfillcolor{dialinecolor}
% was here!!!
\definecolor{dialinecolor}{rgb}{0.000000, 0.000000, 0.000000}
\pgfsetstrokecolor{dialinecolor}
\draw (20.200000\du,14.300000\du)--(23.950000\du,10.400000\du);
}
\pgfsetlinewidth{0.100000\du}
\pgfsetdash{}{0pt}
\pgfsetdash{}{0pt}
\pgfsetbuttcap
{
\definecolor{dialinecolor}{rgb}{0.000000, 0.000000, 0.000000}
\pgfsetfillcolor{dialinecolor}
% was here!!!
\definecolor{dialinecolor}{rgb}{0.000000, 0.000000, 0.000000}
\pgfsetstrokecolor{dialinecolor}
\draw (19.350000\du,10.950000\du)--(24.800000\du,14.800000\du);
}
\pgfsetlinewidth{0.100000\du}
\pgfsetdash{}{0pt}
\pgfsetdash{}{0pt}
\pgfsetbuttcap
{
\definecolor{dialinecolor}{rgb}{0.000000, 0.000000, 0.000000}
\pgfsetfillcolor{dialinecolor}
% was here!!!
\pgfsetarrowsend{stealth}
\definecolor{dialinecolor}{rgb}{0.000000, 0.000000, 0.000000}
\pgfsetstrokecolor{dialinecolor}
\pgfpathmoveto{\pgfpoint{24.749848\du}{9.049853\du}}
\pgfpatharc{135}{42}{3.207812\du and 3.207812\du}
\pgfusepath{stroke}
}
\pgfsetlinewidth{0.100000\du}
\pgfsetdash{}{0pt}
\pgfsetdash{}{0pt}
\pgfsetbuttcap
\pgfsetmiterjoin
\pgfsetlinewidth{0.100000\du}
\pgfsetbuttcap
\pgfsetmiterjoin
\pgfsetdash{}{0pt}
\definecolor{dialinecolor}{rgb}{1.000000, 1.000000, 1.000000}
\pgfsetfillcolor{dialinecolor}
\pgfpathmoveto{\pgfpoint{32.283333\du}{6.900000\du}}
\pgfpathlineto{\pgfpoint{33.616667\du}{6.900000\du}}
\pgfpathcurveto{\pgfpoint{33.800762\du}{6.900000\du}}{\pgfpoint{33.950000\du}{7.347715\du}}{\pgfpoint{33.950000\du}{7.900000\du}}
\pgfpathcurveto{\pgfpoint{33.950000\du}{8.452285\du}}{\pgfpoint{33.800762\du}{8.900000\du}}{\pgfpoint{33.616667\du}{8.900000\du}}
\pgfpathlineto{\pgfpoint{32.283333\du}{8.900000\du}}
\pgfpathcurveto{\pgfpoint{32.099238\du}{8.900000\du}}{\pgfpoint{31.950000\du}{8.452285\du}}{\pgfpoint{31.950000\du}{7.900000\du}}
\pgfpathcurveto{\pgfpoint{31.950000\du}{7.347715\du}}{\pgfpoint{32.099238\du}{6.900000\du}}{\pgfpoint{32.283333\du}{6.900000\du}}
\pgfusepath{fill}
\definecolor{dialinecolor}{rgb}{0.000000, 0.000000, 0.000000}
\pgfsetstrokecolor{dialinecolor}
\pgfpathmoveto{\pgfpoint{32.283333\du}{6.900000\du}}
\pgfpathlineto{\pgfpoint{33.616667\du}{6.900000\du}}
\pgfpathcurveto{\pgfpoint{33.800762\du}{6.900000\du}}{\pgfpoint{33.950000\du}{7.347715\du}}{\pgfpoint{33.950000\du}{7.900000\du}}
\pgfpathcurveto{\pgfpoint{33.950000\du}{8.452285\du}}{\pgfpoint{33.800762\du}{8.900000\du}}{\pgfpoint{33.616667\du}{8.900000\du}}
\pgfpathlineto{\pgfpoint{32.283333\du}{8.900000\du}}
\pgfpathcurveto{\pgfpoint{32.099238\du}{8.900000\du}}{\pgfpoint{31.950000\du}{8.452285\du}}{\pgfpoint{31.950000\du}{7.900000\du}}
\pgfpathcurveto{\pgfpoint{31.950000\du}{7.347715\du}}{\pgfpoint{32.099238\du}{6.900000\du}}{\pgfpoint{32.283333\du}{6.900000\du}}
\pgfusepath{stroke}
% setfont left to latex
\definecolor{dialinecolor}{rgb}{0.000000, 0.000000, 0.000000}
\pgfsetstrokecolor{dialinecolor}
\node at (32.950000\du,8.100000\du){};
\pgfsetlinewidth{0.100000\du}
\pgfsetdash{}{0pt}
\pgfsetdash{}{0pt}
\pgfsetbuttcap
\pgfsetmiterjoin
\pgfsetlinewidth{0.100000\du}
\pgfsetbuttcap
\pgfsetmiterjoin
\pgfsetdash{}{0pt}
\definecolor{dialinecolor}{rgb}{1.000000, 1.000000, 1.000000}
\pgfsetfillcolor{dialinecolor}
\pgfpathmoveto{\pgfpoint{36.602500\du}{3.600000\du}}
\pgfpathlineto{\pgfpoint{37.447500\du}{3.600000\du}}
\pgfpathcurveto{\pgfpoint{37.564170\du}{3.600000\du}}{\pgfpoint{37.658750\du}{3.846243\du}}{\pgfpoint{37.658750\du}{4.150000\du}}
\pgfpathcurveto{\pgfpoint{37.658750\du}{4.453757\du}}{\pgfpoint{37.564170\du}{4.700000\du}}{\pgfpoint{37.447500\du}{4.700000\du}}
\pgfpathlineto{\pgfpoint{36.602500\du}{4.700000\du}}
\pgfpathcurveto{\pgfpoint{36.485830\du}{4.700000\du}}{\pgfpoint{36.391250\du}{4.453757\du}}{\pgfpoint{36.391250\du}{4.150000\du}}
\pgfpathcurveto{\pgfpoint{36.391250\du}{3.846243\du}}{\pgfpoint{36.485830\du}{3.600000\du}}{\pgfpoint{36.602500\du}{3.600000\du}}
\pgfusepath{fill}
\definecolor{dialinecolor}{rgb}{0.000000, 0.000000, 0.000000}
\pgfsetstrokecolor{dialinecolor}
\pgfpathmoveto{\pgfpoint{36.602500\du}{3.600000\du}}
\pgfpathlineto{\pgfpoint{37.447500\du}{3.600000\du}}
\pgfpathcurveto{\pgfpoint{37.564170\du}{3.600000\du}}{\pgfpoint{37.658750\du}{3.846243\du}}{\pgfpoint{37.658750\du}{4.150000\du}}
\pgfpathcurveto{\pgfpoint{37.658750\du}{4.453757\du}}{\pgfpoint{37.564170\du}{4.700000\du}}{\pgfpoint{37.447500\du}{4.700000\du}}
\pgfpathlineto{\pgfpoint{36.602500\du}{4.700000\du}}
\pgfpathcurveto{\pgfpoint{36.485830\du}{4.700000\du}}{\pgfpoint{36.391250\du}{4.453757\du}}{\pgfpoint{36.391250\du}{4.150000\du}}
\pgfpathcurveto{\pgfpoint{36.391250\du}{3.846243\du}}{\pgfpoint{36.485830\du}{3.600000\du}}{\pgfpoint{36.602500\du}{3.600000\du}}
\pgfusepath{stroke}
% setfont left to latex
\definecolor{dialinecolor}{rgb}{0.000000, 0.000000, 0.000000}
\pgfsetstrokecolor{dialinecolor}
\node at (37.025000\du,4.350000\du){k'};
\pgfsetlinewidth{0.100000\du}
\pgfsetdash{}{0pt}
\pgfsetdash{}{0pt}
\pgfsetbuttcap
{
\definecolor{dialinecolor}{rgb}{0.000000, 0.000000, 0.000000}
\pgfsetfillcolor{dialinecolor}
% was here!!!
\definecolor{dialinecolor}{rgb}{0.000000, 0.000000, 0.000000}
\pgfsetstrokecolor{dialinecolor}
\draw (33.836851\du,7.083879\du)--(36.470688\du,4.660103\du);
}
\pgfsetlinewidth{0.100000\du}
\pgfsetdash{}{0pt}
\pgfsetdash{}{0pt}
\pgfsetbuttcap
{
\definecolor{dialinecolor}{rgb}{0.000000, 0.000000, 0.000000}
\pgfsetfillcolor{dialinecolor}
% was here!!!
\definecolor{dialinecolor}{rgb}{0.000000, 0.000000, 0.000000}
\pgfsetstrokecolor{dialinecolor}
\draw (32.283333\du,8.900000\du)--(30.900000\du,11.500000\du);
}
\pgfsetlinewidth{0.100000\du}
\pgfsetdash{}{0pt}
\pgfsetdash{}{0pt}
\pgfsetbuttcap
{
\definecolor{dialinecolor}{rgb}{0.000000, 0.000000, 0.000000}
\pgfsetfillcolor{dialinecolor}
% was here!!!
\definecolor{dialinecolor}{rgb}{0.000000, 0.000000, 0.000000}
\pgfsetstrokecolor{dialinecolor}
\draw (32.950000\du,8.900000\du)--(33.600000\du,11.900000\du);
}
\pgfsetlinewidth{0.100000\du}
\pgfsetdash{}{0pt}
\pgfsetdash{}{0pt}
\pgfsetbuttcap
{
\definecolor{dialinecolor}{rgb}{0.000000, 0.000000, 0.000000}
\pgfsetfillcolor{dialinecolor}
% was here!!!
\definecolor{dialinecolor}{rgb}{0.000000, 0.000000, 0.000000}
\pgfsetstrokecolor{dialinecolor}
\draw (33.283333\du,8.900000\du)--(35.450000\du,11.850000\du);
}
\definecolor{dialinecolor}{rgb}{1.000000, 1.000000, 1.000000}
\pgfsetfillcolor{dialinecolor}
\fill (35.500000\du,11.500000\du)--(35.500000\du,13.400000\du)--(37.500000\du,13.400000\du)--(37.500000\du,11.500000\du)--cycle;
\pgfsetlinewidth{0.100000\du}
\pgfsetdash{}{0pt}
\pgfsetdash{}{0pt}
\pgfsetmiterjoin
\definecolor{dialinecolor}{rgb}{0.000000, 0.000000, 0.000000}
\pgfsetstrokecolor{dialinecolor}
\draw (35.500000\du,11.500000\du)--(35.500000\du,13.400000\du)--(37.500000\du,13.400000\du)--(37.500000\du,11.500000\du)--cycle;
% setfont left to latex
\definecolor{dialinecolor}{rgb}{0.000000, 0.000000, 0.000000}
\pgfsetstrokecolor{dialinecolor}
\node at (36.500000\du,12.645000\du){k'};
%% setfont left to latex
%\definecolor{dialinecolor}{rgb}{0.000000, 0.000000, 0.000000}
%\pgfsetstrokecolor{dialinecolor}
%\node[anchor=west] at (39.000000\du,7.550000\du){streiche v};
%% setfont left to latex
%\definecolor{dialinecolor}{rgb}{0.000000, 0.000000, 0.000000}
%\pgfsetstrokecolor{dialinecolor}
%\node[anchor=west] at (39.000000\du,8.350000\du){ersetze Vorkommen };
%% setfont left to latex
%\definecolor{dialinecolor}{rgb}{0.000000, 0.000000, 0.000000}
%\pgfsetstrokecolor{dialinecolor}
%\node[anchor=west] at (39.000000\du,9.150000\du){von k durch k'};
\end{tikzpicture}

                	\caption{Bsp. $v$ rechtestes Kind von $parent(v)$ streichen (blattorientiert)}
                	\label{diag21:rechtestes_Kind_streichen}
                \end{figure}
                \begin{alltt}
                while w hat ein Kind
                do Verschmelze oder Stehlen od
                \end{alltt}
                Verschmelzen: \\
                % Graphic for TeX using PGF
% Title: /home/alex/WualaDrive/myfiles/Uni/4 Semester/Algorithmen/Skript/algorithmen/Diagramm22.dia
% Creator: Dia v0.97.1
% CreationDate: Tue May 18 14:50:43 2010
% For: alex
% \usepackage{tikz}
% The following commands are not supported in PSTricks at present
% We define them conditionally, so when they are implemented,
% this pgf file will use them.
\ifx\du\undefined
  \newlength{\du}
\fi
\setlength{\du}{15\unitlength}
\begin{tikzpicture}
\pgftransformxscale{1.000000}
\pgftransformyscale{-1.000000}
\definecolor{dialinecolor}{rgb}{0.000000, 0.000000, 0.000000}
\pgfsetstrokecolor{dialinecolor}
\definecolor{dialinecolor}{rgb}{1.000000, 1.000000, 1.000000}
\pgfsetfillcolor{dialinecolor}
\pgfsetlinewidth{0.100000\du}
\pgfsetdash{}{0pt}
\pgfsetdash{}{0pt}
\pgfsetbuttcap
\pgfsetmiterjoin
\pgfsetlinewidth{0.100000\du}
\pgfsetbuttcap
\pgfsetmiterjoin
\pgfsetdash{}{0pt}
\definecolor{dialinecolor}{rgb}{1.000000, 1.000000, 1.000000}
\pgfsetfillcolor{dialinecolor}
\pgfpathmoveto{\pgfpoint{8.483333\du}{2.450000\du}}
\pgfpathlineto{\pgfpoint{9.816667\du}{2.450000\du}}
\pgfpathcurveto{\pgfpoint{10.000762\du}{2.450000\du}}{\pgfpoint{10.150000\du}{2.897715\du}}{\pgfpoint{10.150000\du}{3.450000\du}}
\pgfpathcurveto{\pgfpoint{10.150000\du}{4.002285\du}}{\pgfpoint{10.000762\du}{4.450000\du}}{\pgfpoint{9.816667\du}{4.450000\du}}
\pgfpathlineto{\pgfpoint{8.483333\du}{4.450000\du}}
\pgfpathcurveto{\pgfpoint{8.299238\du}{4.450000\du}}{\pgfpoint{8.150000\du}{4.002285\du}}{\pgfpoint{8.150000\du}{3.450000\du}}
\pgfpathcurveto{\pgfpoint{8.150000\du}{2.897715\du}}{\pgfpoint{8.299238\du}{2.450000\du}}{\pgfpoint{8.483333\du}{2.450000\du}}
\pgfusepath{fill}
\definecolor{dialinecolor}{rgb}{0.000000, 0.000000, 0.000000}
\pgfsetstrokecolor{dialinecolor}
\pgfpathmoveto{\pgfpoint{8.483333\du}{2.450000\du}}
\pgfpathlineto{\pgfpoint{9.816667\du}{2.450000\du}}
\pgfpathcurveto{\pgfpoint{10.000762\du}{2.450000\du}}{\pgfpoint{10.150000\du}{2.897715\du}}{\pgfpoint{10.150000\du}{3.450000\du}}
\pgfpathcurveto{\pgfpoint{10.150000\du}{4.002285\du}}{\pgfpoint{10.000762\du}{4.450000\du}}{\pgfpoint{9.816667\du}{4.450000\du}}
\pgfpathlineto{\pgfpoint{8.483333\du}{4.450000\du}}
\pgfpathcurveto{\pgfpoint{8.299238\du}{4.450000\du}}{\pgfpoint{8.150000\du}{4.002285\du}}{\pgfpoint{8.150000\du}{3.450000\du}}
\pgfpathcurveto{\pgfpoint{8.150000\du}{2.897715\du}}{\pgfpoint{8.299238\du}{2.450000\du}}{\pgfpoint{8.483333\du}{2.450000\du}}
\pgfusepath{stroke}
% setfont left to latex
\definecolor{dialinecolor}{rgb}{0.000000, 0.000000, 0.000000}
\pgfsetstrokecolor{dialinecolor}
\node at (9.150000\du,3.650000\du){};
\pgfsetlinewidth{0.100000\du}
\pgfsetdash{}{0pt}
\pgfsetdash{}{0pt}
\pgfsetbuttcap
\pgfsetmiterjoin
\pgfsetlinewidth{0.100000\du}
\pgfsetbuttcap
\pgfsetmiterjoin
\pgfsetdash{}{0pt}
\definecolor{dialinecolor}{rgb}{1.000000, 1.000000, 1.000000}
\pgfsetfillcolor{dialinecolor}
\pgfpathmoveto{\pgfpoint{5.533333\du}{7.250000\du}}
\pgfpathlineto{\pgfpoint{6.866667\du}{7.250000\du}}
\pgfpathcurveto{\pgfpoint{7.050762\du}{7.250000\du}}{\pgfpoint{7.200000\du}{7.697715\du}}{\pgfpoint{7.200000\du}{8.250000\du}}
\pgfpathcurveto{\pgfpoint{7.200000\du}{8.802285\du}}{\pgfpoint{7.050762\du}{9.250000\du}}{\pgfpoint{6.866667\du}{9.250000\du}}
\pgfpathlineto{\pgfpoint{5.533333\du}{9.250000\du}}
\pgfpathcurveto{\pgfpoint{5.349238\du}{9.250000\du}}{\pgfpoint{5.200000\du}{8.802285\du}}{\pgfpoint{5.200000\du}{8.250000\du}}
\pgfpathcurveto{\pgfpoint{5.200000\du}{7.697715\du}}{\pgfpoint{5.349238\du}{7.250000\du}}{\pgfpoint{5.533333\du}{7.250000\du}}
\pgfusepath{fill}
\definecolor{dialinecolor}{rgb}{0.000000, 0.000000, 0.000000}
\pgfsetstrokecolor{dialinecolor}
\pgfpathmoveto{\pgfpoint{5.533333\du}{7.250000\du}}
\pgfpathlineto{\pgfpoint{6.866667\du}{7.250000\du}}
\pgfpathcurveto{\pgfpoint{7.050762\du}{7.250000\du}}{\pgfpoint{7.200000\du}{7.697715\du}}{\pgfpoint{7.200000\du}{8.250000\du}}
\pgfpathcurveto{\pgfpoint{7.200000\du}{8.802285\du}}{\pgfpoint{7.050762\du}{9.250000\du}}{\pgfpoint{6.866667\du}{9.250000\du}}
\pgfpathlineto{\pgfpoint{5.533333\du}{9.250000\du}}
\pgfpathcurveto{\pgfpoint{5.349238\du}{9.250000\du}}{\pgfpoint{5.200000\du}{8.802285\du}}{\pgfpoint{5.200000\du}{8.250000\du}}
\pgfpathcurveto{\pgfpoint{5.200000\du}{7.697715\du}}{\pgfpoint{5.349238\du}{7.250000\du}}{\pgfpoint{5.533333\du}{7.250000\du}}
\pgfusepath{stroke}
% setfont left to latex
\definecolor{dialinecolor}{rgb}{0.000000, 0.000000, 0.000000}
\pgfsetstrokecolor{dialinecolor}
\node at (6.200000\du,8.450000\du){};
% setfont left to latex
\definecolor{dialinecolor}{rgb}{0.000000, 0.000000, 0.000000}
\pgfsetstrokecolor{dialinecolor}
\node[anchor=west] at (3.600000\du,7.250000\du){w};
\pgfsetlinewidth{0.100000\du}
\pgfsetdash{}{0pt}
\pgfsetdash{}{0pt}
\pgfsetbuttcap
\pgfsetmiterjoin
\pgfsetlinewidth{0.100000\du}
\pgfsetbuttcap
\pgfsetmiterjoin
\pgfsetdash{}{0pt}
\definecolor{dialinecolor}{rgb}{1.000000, 1.000000, 1.000000}
\pgfsetfillcolor{dialinecolor}
\pgfpathmoveto{\pgfpoint{9.783333\du}{7.250000\du}}
\pgfpathlineto{\pgfpoint{11.116667\du}{7.250000\du}}
\pgfpathcurveto{\pgfpoint{11.300762\du}{7.250000\du}}{\pgfpoint{11.450000\du}{7.697715\du}}{\pgfpoint{11.450000\du}{8.250000\du}}
\pgfpathcurveto{\pgfpoint{11.450000\du}{8.802285\du}}{\pgfpoint{11.300762\du}{9.250000\du}}{\pgfpoint{11.116667\du}{9.250000\du}}
\pgfpathlineto{\pgfpoint{9.783333\du}{9.250000\du}}
\pgfpathcurveto{\pgfpoint{9.599238\du}{9.250000\du}}{\pgfpoint{9.450000\du}{8.802285\du}}{\pgfpoint{9.450000\du}{8.250000\du}}
\pgfpathcurveto{\pgfpoint{9.450000\du}{7.697715\du}}{\pgfpoint{9.599238\du}{7.250000\du}}{\pgfpoint{9.783333\du}{7.250000\du}}
\pgfusepath{fill}
\definecolor{dialinecolor}{rgb}{0.000000, 0.000000, 0.000000}
\pgfsetstrokecolor{dialinecolor}
\pgfpathmoveto{\pgfpoint{9.783333\du}{7.250000\du}}
\pgfpathlineto{\pgfpoint{11.116667\du}{7.250000\du}}
\pgfpathcurveto{\pgfpoint{11.300762\du}{7.250000\du}}{\pgfpoint{11.450000\du}{7.697715\du}}{\pgfpoint{11.450000\du}{8.250000\du}}
\pgfpathcurveto{\pgfpoint{11.450000\du}{8.802285\du}}{\pgfpoint{11.300762\du}{9.250000\du}}{\pgfpoint{11.116667\du}{9.250000\du}}
\pgfpathlineto{\pgfpoint{9.783333\du}{9.250000\du}}
\pgfpathcurveto{\pgfpoint{9.599238\du}{9.250000\du}}{\pgfpoint{9.450000\du}{8.802285\du}}{\pgfpoint{9.450000\du}{8.250000\du}}
\pgfpathcurveto{\pgfpoint{9.450000\du}{7.697715\du}}{\pgfpoint{9.599238\du}{7.250000\du}}{\pgfpoint{9.783333\du}{7.250000\du}}
\pgfusepath{stroke}
% setfont left to latex
\definecolor{dialinecolor}{rgb}{0.000000, 0.000000, 0.000000}
\pgfsetstrokecolor{dialinecolor}
\node at (10.450000\du,8.450000\du){};
\definecolor{dialinecolor}{rgb}{1.000000, 1.000000, 1.000000}
\pgfsetfillcolor{dialinecolor}
\fill (4.900000\du,10.650000\du)--(4.900000\du,12.550000\du)--(6.900000\du,12.550000\du)--(6.900000\du,10.650000\du)--cycle;
\pgfsetlinewidth{0.100000\du}
\pgfsetdash{}{0pt}
\pgfsetdash{}{0pt}
\pgfsetmiterjoin
\definecolor{dialinecolor}{rgb}{0.000000, 0.000000, 0.000000}
\pgfsetstrokecolor{dialinecolor}
\draw (4.900000\du,10.650000\du)--(4.900000\du,12.550000\du)--(6.900000\du,12.550000\du)--(6.900000\du,10.650000\du)--cycle;
% setfont left to latex
\definecolor{dialinecolor}{rgb}{0.000000, 0.000000, 0.000000}
\pgfsetstrokecolor{dialinecolor}
\node at (5.900000\du,11.795000\du){k'};
\pgfsetlinewidth{0.100000\du}
\pgfsetdash{}{0pt}
\pgfsetdash{}{0pt}
\pgfsetbuttcap
{
\definecolor{dialinecolor}{rgb}{0.000000, 0.000000, 0.000000}
\pgfsetfillcolor{dialinecolor}
% was here!!!
\definecolor{dialinecolor}{rgb}{0.000000, 0.000000, 0.000000}
\pgfsetstrokecolor{dialinecolor}
\draw (6.845493\du,7.199707\du)--(8.504507\du,4.500293\du);
}
\pgfsetlinewidth{0.100000\du}
\pgfsetdash{}{0pt}
\pgfsetdash{}{0pt}
\pgfsetbuttcap
{
\definecolor{dialinecolor}{rgb}{0.000000, 0.000000, 0.000000}
\pgfsetfillcolor{dialinecolor}
% was here!!!
\definecolor{dialinecolor}{rgb}{0.000000, 0.000000, 0.000000}
\pgfsetstrokecolor{dialinecolor}
\draw (5.989539\du,10.600153\du)--(6.106104\du,9.298511\du);
}
\pgfsetlinewidth{0.100000\du}
\pgfsetdash{}{0pt}
\pgfsetdash{}{0pt}
\pgfsetbuttcap
{
\definecolor{dialinecolor}{rgb}{0.000000, 0.000000, 0.000000}
\pgfsetfillcolor{dialinecolor}
% was here!!!
\definecolor{dialinecolor}{rgb}{0.000000, 0.000000, 0.000000}
\pgfsetstrokecolor{dialinecolor}
\draw (10.165546\du,7.199707\du)--(9.434454\du,4.500293\du);
}
\pgfsetlinewidth{0.100000\du}
\pgfsetdash{}{0pt}
\pgfsetdash{}{0pt}
\pgfsetbuttcap
{
\definecolor{dialinecolor}{rgb}{0.000000, 0.000000, 0.000000}
\pgfsetfillcolor{dialinecolor}
% was here!!!
\definecolor{dialinecolor}{rgb}{0.000000, 0.000000, 0.000000}
\pgfsetstrokecolor{dialinecolor}
\draw (10.116667\du,9.250000\du)--(9.050000\du,11.700000\du);
}
\pgfsetlinewidth{0.100000\du}
\pgfsetdash{}{0pt}
\pgfsetdash{}{0pt}
\pgfsetbuttcap
{
\definecolor{dialinecolor}{rgb}{0.000000, 0.000000, 0.000000}
\pgfsetfillcolor{dialinecolor}
% was here!!!
\definecolor{dialinecolor}{rgb}{0.000000, 0.000000, 0.000000}
\pgfsetstrokecolor{dialinecolor}
\draw (10.982562\du,9.299908\du)--(12.200000\du,11.700000\du);
}
\pgfsetlinewidth{0.100000\du}
\pgfsetdash{}{0pt}
\pgfsetdash{}{0pt}
\pgfsetbuttcap
{
\definecolor{dialinecolor}{rgb}{0.000000, 0.000000, 0.000000}
\pgfsetfillcolor{dialinecolor}
% was here!!!
\pgfsetarrowsend{stealth}
\definecolor{dialinecolor}{rgb}{0.000000, 0.000000, 0.000000}
\pgfsetstrokecolor{dialinecolor}
\pgfpathmoveto{\pgfpoint{12.849923\du}{7.749885\du}}
\pgfpatharc{147}{33}{2.211563\du and 2.211563\du}
\pgfusepath{stroke}
}
\pgfsetlinewidth{0.100000\du}
\pgfsetdash{}{0pt}
\pgfsetdash{}{0pt}
\pgfsetbuttcap
\pgfsetmiterjoin
\pgfsetlinewidth{0.100000\du}
\pgfsetbuttcap
\pgfsetmiterjoin
\pgfsetdash{}{0pt}
\definecolor{dialinecolor}{rgb}{1.000000, 1.000000, 1.000000}
\pgfsetfillcolor{dialinecolor}
\pgfpathmoveto{\pgfpoint{20.833333\du}{2.950000\du}}
\pgfpathlineto{\pgfpoint{22.166667\du}{2.950000\du}}
\pgfpathcurveto{\pgfpoint{22.350762\du}{2.950000\du}}{\pgfpoint{22.500000\du}{3.397715\du}}{\pgfpoint{22.500000\du}{3.950000\du}}
\pgfpathcurveto{\pgfpoint{22.500000\du}{4.502285\du}}{\pgfpoint{22.350762\du}{4.950000\du}}{\pgfpoint{22.166667\du}{4.950000\du}}
\pgfpathlineto{\pgfpoint{20.833333\du}{4.950000\du}}
\pgfpathcurveto{\pgfpoint{20.649238\du}{4.950000\du}}{\pgfpoint{20.500000\du}{4.502285\du}}{\pgfpoint{20.500000\du}{3.950000\du}}
\pgfpathcurveto{\pgfpoint{20.500000\du}{3.397715\du}}{\pgfpoint{20.649238\du}{2.950000\du}}{\pgfpoint{20.833333\du}{2.950000\du}}
\pgfusepath{fill}
\definecolor{dialinecolor}{rgb}{0.000000, 0.000000, 0.000000}
\pgfsetstrokecolor{dialinecolor}
\pgfpathmoveto{\pgfpoint{20.833333\du}{2.950000\du}}
\pgfpathlineto{\pgfpoint{22.166667\du}{2.950000\du}}
\pgfpathcurveto{\pgfpoint{22.350762\du}{2.950000\du}}{\pgfpoint{22.500000\du}{3.397715\du}}{\pgfpoint{22.500000\du}{3.950000\du}}
\pgfpathcurveto{\pgfpoint{22.500000\du}{4.502285\du}}{\pgfpoint{22.350762\du}{4.950000\du}}{\pgfpoint{22.166667\du}{4.950000\du}}
\pgfpathlineto{\pgfpoint{20.833333\du}{4.950000\du}}
\pgfpathcurveto{\pgfpoint{20.649238\du}{4.950000\du}}{\pgfpoint{20.500000\du}{4.502285\du}}{\pgfpoint{20.500000\du}{3.950000\du}}
\pgfpathcurveto{\pgfpoint{20.500000\du}{3.397715\du}}{\pgfpoint{20.649238\du}{2.950000\du}}{\pgfpoint{20.833333\du}{2.950000\du}}
\pgfusepath{stroke}
% setfont left to latex
\definecolor{dialinecolor}{rgb}{0.000000, 0.000000, 0.000000}
\pgfsetstrokecolor{dialinecolor}
\node at (21.500000\du,4.150000\du){};
\pgfsetlinewidth{0.100000\du}
\pgfsetdash{}{0pt}
\pgfsetdash{}{0pt}
\pgfsetbuttcap
\pgfsetmiterjoin
\pgfsetlinewidth{0.100000\du}
\pgfsetbuttcap
\pgfsetmiterjoin
\pgfsetdash{}{0pt}
\definecolor{dialinecolor}{rgb}{1.000000, 1.000000, 1.000000}
\pgfsetfillcolor{dialinecolor}
\pgfpathmoveto{\pgfpoint{20.491667\du}{6.500000\du}}
\pgfpathlineto{\pgfpoint{22.258333\du}{6.500000\du}}
\pgfpathcurveto{\pgfpoint{22.502259\du}{6.500000\du}}{\pgfpoint{22.700000\du}{6.869365\du}}{\pgfpoint{22.700000\du}{7.325000\du}}
\pgfpathcurveto{\pgfpoint{22.700000\du}{7.780635\du}}{\pgfpoint{22.502259\du}{8.150000\du}}{\pgfpoint{22.258333\du}{8.150000\du}}
\pgfpathlineto{\pgfpoint{20.491667\du}{8.150000\du}}
\pgfpathcurveto{\pgfpoint{20.247741\du}{8.150000\du}}{\pgfpoint{20.050000\du}{7.780635\du}}{\pgfpoint{20.050000\du}{7.325000\du}}
\pgfpathcurveto{\pgfpoint{20.050000\du}{6.869365\du}}{\pgfpoint{20.247741\du}{6.500000\du}}{\pgfpoint{20.491667\du}{6.500000\du}}
\pgfusepath{fill}
\definecolor{dialinecolor}{rgb}{0.000000, 0.000000, 0.000000}
\pgfsetstrokecolor{dialinecolor}
\pgfpathmoveto{\pgfpoint{20.491667\du}{6.500000\du}}
\pgfpathlineto{\pgfpoint{22.258333\du}{6.500000\du}}
\pgfpathcurveto{\pgfpoint{22.502259\du}{6.500000\du}}{\pgfpoint{22.700000\du}{6.869365\du}}{\pgfpoint{22.700000\du}{7.325000\du}}
\pgfpathcurveto{\pgfpoint{22.700000\du}{7.780635\du}}{\pgfpoint{22.502259\du}{8.150000\du}}{\pgfpoint{22.258333\du}{8.150000\du}}
\pgfpathlineto{\pgfpoint{20.491667\du}{8.150000\du}}
\pgfpathcurveto{\pgfpoint{20.247741\du}{8.150000\du}}{\pgfpoint{20.050000\du}{7.780635\du}}{\pgfpoint{20.050000\du}{7.325000\du}}
\pgfpathcurveto{\pgfpoint{20.050000\du}{6.869365\du}}{\pgfpoint{20.247741\du}{6.500000\du}}{\pgfpoint{20.491667\du}{6.500000\du}}
\pgfusepath{stroke}
% setfont left to latex
\definecolor{dialinecolor}{rgb}{0.000000, 0.000000, 0.000000}
\pgfsetstrokecolor{dialinecolor}
\node at (21.375000\du,7.525000\du){};
\definecolor{dialinecolor}{rgb}{1.000000, 1.000000, 1.000000}
\pgfsetfillcolor{dialinecolor}
\fill (18.750000\du,9.300000\du)--(18.750000\du,11.200000\du)--(20.750000\du,11.200000\du)--(20.750000\du,9.300000\du)--cycle;
\pgfsetlinewidth{0.100000\du}
\pgfsetdash{}{0pt}
\pgfsetdash{}{0pt}
\pgfsetmiterjoin
\definecolor{dialinecolor}{rgb}{0.000000, 0.000000, 0.000000}
\pgfsetstrokecolor{dialinecolor}
\draw (18.750000\du,9.300000\du)--(18.750000\du,11.200000\du)--(20.750000\du,11.200000\du)--(20.750000\du,9.300000\du)--cycle;
% setfont left to latex
\definecolor{dialinecolor}{rgb}{0.000000, 0.000000, 0.000000}
\pgfsetstrokecolor{dialinecolor}
\node at (19.750000\du,10.445000\du){k'};
\pgfsetlinewidth{0.100000\du}
\pgfsetdash{}{0pt}
\pgfsetdash{}{0pt}
\pgfsetbuttcap
{
\definecolor{dialinecolor}{rgb}{0.000000, 0.000000, 0.000000}
\pgfsetfillcolor{dialinecolor}
% was here!!!
\definecolor{dialinecolor}{rgb}{0.000000, 0.000000, 0.000000}
\pgfsetstrokecolor{dialinecolor}
\draw (21.461105\du,5.000156\du)--(21.407379\du,6.450763\du);
}
\pgfsetlinewidth{0.100000\du}
\pgfsetdash{}{0pt}
\pgfsetdash{}{0pt}
\pgfsetbuttcap
{
\definecolor{dialinecolor}{rgb}{0.000000, 0.000000, 0.000000}
\pgfsetfillcolor{dialinecolor}
% was here!!!
\definecolor{dialinecolor}{rgb}{0.000000, 0.000000, 0.000000}
\pgfsetstrokecolor{dialinecolor}
\draw (20.889008\du,8.199786\du)--(20.305618\du,9.249887\du);
}
\pgfsetlinewidth{0.100000\du}
\pgfsetdash{}{0pt}
\pgfsetdash{}{0pt}
\pgfsetbuttcap
{
\definecolor{dialinecolor}{rgb}{0.000000, 0.000000, 0.000000}
\pgfsetfillcolor{dialinecolor}
% was here!!!
\definecolor{dialinecolor}{rgb}{0.000000, 0.000000, 0.000000}
\pgfsetstrokecolor{dialinecolor}
\draw (21.494733\du,8.199969\du)--(21.700000\du,9.700000\du);
}
\pgfsetlinewidth{0.100000\du}
\pgfsetdash{}{0pt}
\pgfsetdash{}{0pt}
\pgfsetbuttcap
{
\definecolor{dialinecolor}{rgb}{0.000000, 0.000000, 0.000000}
\pgfsetfillcolor{dialinecolor}
% was here!!!
\definecolor{dialinecolor}{rgb}{0.000000, 0.000000, 0.000000}
\pgfsetstrokecolor{dialinecolor}
\draw (22.076038\du,8.199133\du)--(23.400000\du,9.850000\du);
}
% setfont left to latex
\definecolor{dialinecolor}{rgb}{0.000000, 0.000000, 0.000000}
\pgfsetstrokecolor{dialinecolor}
\node[anchor=west] at (24.450000\du,3.100000\du){Falls Geschwister-};
% setfont left to latex
\definecolor{dialinecolor}{rgb}{0.000000, 0.000000, 0.000000}
\pgfsetstrokecolor{dialinecolor}
\node[anchor=west] at (24.450000\du,3.900000\du){knoten Grad 2 hat.};
% setfont left to latex
\definecolor{dialinecolor}{rgb}{0.000000, 0.000000, 0.000000}
\pgfsetstrokecolor{dialinecolor}
\node[anchor=west] at (24.450000\du,4.700000\du){};
% setfont left to latex
\definecolor{dialinecolor}{rgb}{0.000000, 0.000000, 0.000000}
\pgfsetstrokecolor{dialinecolor}
\node[anchor=west] at (24.450000\du,5.500000\du){verschmelze Geschw.-};
% setfont left to latex
\definecolor{dialinecolor}{rgb}{0.000000, 0.000000, 0.000000}
\pgfsetstrokecolor{dialinecolor}
\node[anchor=west] at (24.450000\du,6.300000\du){knoten zu einem };
% setfont left to latex
\definecolor{dialinecolor}{rgb}{0.000000, 0.000000, 0.000000}
\pgfsetstrokecolor{dialinecolor}
\node[anchor=west] at (24.450000\du,7.100000\du){Knoten Grad 3. };
% setfont left to latex
\definecolor{dialinecolor}{rgb}{0.000000, 0.000000, 0.000000}
\pgfsetstrokecolor{dialinecolor}
\node[anchor=west] at (24.450000\du,7.900000\du){};
% setfont left to latex
\definecolor{dialinecolor}{rgb}{0.000000, 0.000000, 0.000000}
\pgfsetstrokecolor{dialinecolor}
\node[anchor=west] at (24.450000\du,8.700000\du){Setze fort!};
\end{tikzpicture}
\\
                Stehlen: Geschwisterknoten von $w$ hat 3 oder 4 Kinder:  Gibt uns an w ab. \\
                % Graphic for TeX using PGF
% Title: /home/alex/WualaDrive/myfiles/Uni/4 Semester/Algorithmen/Skript/algorithmen/Diagramm23.dia
% Creator: Dia v0.97.1
% CreationDate: Tue May 18 14:53:22 2010
% For: alex
% \usepackage{tikz}
% The following commands are not supported in PSTricks at present
% We define them conditionally, so when they are implemented,
% this pgf file will use them.
\ifx\du\undefined
  \newlength{\du}
\fi
\setlength{\du}{15\unitlength}
\begin{tikzpicture}
\pgftransformxscale{1.000000}
\pgftransformyscale{-1.000000}
\definecolor{dialinecolor}{rgb}{0.000000, 0.000000, 0.000000}
\pgfsetstrokecolor{dialinecolor}
\definecolor{dialinecolor}{rgb}{1.000000, 1.000000, 1.000000}
\pgfsetfillcolor{dialinecolor}
\pgfsetlinewidth{0.100000\du}
\pgfsetdash{}{0pt}
\pgfsetdash{}{0pt}
\pgfsetbuttcap
\pgfsetmiterjoin
\pgfsetlinewidth{0.100000\du}
\pgfsetbuttcap
\pgfsetmiterjoin
\pgfsetdash{}{0pt}
\definecolor{dialinecolor}{rgb}{1.000000, 1.000000, 1.000000}
\pgfsetfillcolor{dialinecolor}
\pgfpathmoveto{\pgfpoint{7.433333\du}{3.200000\du}}
\pgfpathlineto{\pgfpoint{8.766667\du}{3.200000\du}}
\pgfpathcurveto{\pgfpoint{8.950762\du}{3.200000\du}}{\pgfpoint{9.100000\du}{3.647715\du}}{\pgfpoint{9.100000\du}{4.200000\du}}
\pgfpathcurveto{\pgfpoint{9.100000\du}{4.752285\du}}{\pgfpoint{8.950762\du}{5.200000\du}}{\pgfpoint{8.766667\du}{5.200000\du}}
\pgfpathlineto{\pgfpoint{7.433333\du}{5.200000\du}}
\pgfpathcurveto{\pgfpoint{7.249238\du}{5.200000\du}}{\pgfpoint{7.100000\du}{4.752285\du}}{\pgfpoint{7.100000\du}{4.200000\du}}
\pgfpathcurveto{\pgfpoint{7.100000\du}{3.647715\du}}{\pgfpoint{7.249238\du}{3.200000\du}}{\pgfpoint{7.433333\du}{3.200000\du}}
\pgfusepath{fill}
\definecolor{dialinecolor}{rgb}{0.000000, 0.000000, 0.000000}
\pgfsetstrokecolor{dialinecolor}
\pgfpathmoveto{\pgfpoint{7.433333\du}{3.200000\du}}
\pgfpathlineto{\pgfpoint{8.766667\du}{3.200000\du}}
\pgfpathcurveto{\pgfpoint{8.950762\du}{3.200000\du}}{\pgfpoint{9.100000\du}{3.647715\du}}{\pgfpoint{9.100000\du}{4.200000\du}}
\pgfpathcurveto{\pgfpoint{9.100000\du}{4.752285\du}}{\pgfpoint{8.950762\du}{5.200000\du}}{\pgfpoint{8.766667\du}{5.200000\du}}
\pgfpathlineto{\pgfpoint{7.433333\du}{5.200000\du}}
\pgfpathcurveto{\pgfpoint{7.249238\du}{5.200000\du}}{\pgfpoint{7.100000\du}{4.752285\du}}{\pgfpoint{7.100000\du}{4.200000\du}}
\pgfpathcurveto{\pgfpoint{7.100000\du}{3.647715\du}}{\pgfpoint{7.249238\du}{3.200000\du}}{\pgfpoint{7.433333\du}{3.200000\du}}
\pgfusepath{stroke}
% setfont left to latex
\definecolor{dialinecolor}{rgb}{0.000000, 0.000000, 0.000000}
\pgfsetstrokecolor{dialinecolor}
\node at (8.100000\du,4.400000\du){};
\pgfsetlinewidth{0.100000\du}
\pgfsetdash{}{0pt}
\pgfsetdash{}{0pt}
\pgfsetbuttcap
\pgfsetmiterjoin
\pgfsetlinewidth{0.100000\du}
\pgfsetbuttcap
\pgfsetmiterjoin
\pgfsetdash{}{0pt}
\definecolor{dialinecolor}{rgb}{0.000000, 0.000000, 0.000000}
\pgfsetfillcolor{dialinecolor}
\pgfpathmoveto{\pgfpoint{5.083333\du}{7.450000\du}}
\pgfpathlineto{\pgfpoint{6.416667\du}{7.450000\du}}
\pgfpathcurveto{\pgfpoint{6.600762\du}{7.450000\du}}{\pgfpoint{6.750000\du}{7.897715\du}}{\pgfpoint{6.750000\du}{8.450000\du}}
\pgfpathcurveto{\pgfpoint{6.750000\du}{9.002285\du}}{\pgfpoint{6.600762\du}{9.450000\du}}{\pgfpoint{6.416667\du}{9.450000\du}}
\pgfpathlineto{\pgfpoint{5.083333\du}{9.450000\du}}
\pgfpathcurveto{\pgfpoint{4.899238\du}{9.450000\du}}{\pgfpoint{4.750000\du}{9.002285\du}}{\pgfpoint{4.750000\du}{8.450000\du}}
\pgfpathcurveto{\pgfpoint{4.750000\du}{7.897715\du}}{\pgfpoint{4.899238\du}{7.450000\du}}{\pgfpoint{5.083333\du}{7.450000\du}}
\pgfusepath{fill}
\definecolor{dialinecolor}{rgb}{0.000000, 0.000000, 0.000000}
\pgfsetstrokecolor{dialinecolor}
\pgfpathmoveto{\pgfpoint{5.083333\du}{7.450000\du}}
\pgfpathlineto{\pgfpoint{6.416667\du}{7.450000\du}}
\pgfpathcurveto{\pgfpoint{6.600762\du}{7.450000\du}}{\pgfpoint{6.750000\du}{7.897715\du}}{\pgfpoint{6.750000\du}{8.450000\du}}
\pgfpathcurveto{\pgfpoint{6.750000\du}{9.002285\du}}{\pgfpoint{6.600762\du}{9.450000\du}}{\pgfpoint{6.416667\du}{9.450000\du}}
\pgfpathlineto{\pgfpoint{5.083333\du}{9.450000\du}}
\pgfpathcurveto{\pgfpoint{4.899238\du}{9.450000\du}}{\pgfpoint{4.750000\du}{9.002285\du}}{\pgfpoint{4.750000\du}{8.450000\du}}
\pgfpathcurveto{\pgfpoint{4.750000\du}{7.897715\du}}{\pgfpoint{4.899238\du}{7.450000\du}}{\pgfpoint{5.083333\du}{7.450000\du}}
\pgfusepath{stroke}
% setfont left to latex
\definecolor{dialinecolor}{rgb}{0.000000, 0.000000, 0.000000}
\pgfsetstrokecolor{dialinecolor}
\node at (5.750000\du,8.650000\du){};
\pgfsetlinewidth{0.100000\du}
\pgfsetdash{}{0pt}
\pgfsetdash{}{0pt}
\pgfsetbuttcap
\pgfsetmiterjoin
\pgfsetlinewidth{0.100000\du}
\pgfsetbuttcap
\pgfsetmiterjoin
\pgfsetdash{}{0pt}
\definecolor{dialinecolor}{rgb}{1.000000, 1.000000, 1.000000}
\pgfsetfillcolor{dialinecolor}
\pgfpathmoveto{\pgfpoint{9.083333\du}{7.350000\du}}
\pgfpathlineto{\pgfpoint{10.416667\du}{7.350000\du}}
\pgfpathcurveto{\pgfpoint{10.600762\du}{7.350000\du}}{\pgfpoint{10.750000\du}{7.797715\du}}{\pgfpoint{10.750000\du}{8.350000\du}}
\pgfpathcurveto{\pgfpoint{10.750000\du}{8.902285\du}}{\pgfpoint{10.600762\du}{9.350000\du}}{\pgfpoint{10.416667\du}{9.350000\du}}
\pgfpathlineto{\pgfpoint{9.083333\du}{9.350000\du}}
\pgfpathcurveto{\pgfpoint{8.899238\du}{9.350000\du}}{\pgfpoint{8.750000\du}{8.902285\du}}{\pgfpoint{8.750000\du}{8.350000\du}}
\pgfpathcurveto{\pgfpoint{8.750000\du}{7.797715\du}}{\pgfpoint{8.899238\du}{7.350000\du}}{\pgfpoint{9.083333\du}{7.350000\du}}
\pgfusepath{fill}
\definecolor{dialinecolor}{rgb}{0.000000, 0.000000, 0.000000}
\pgfsetstrokecolor{dialinecolor}
\pgfpathmoveto{\pgfpoint{9.083333\du}{7.350000\du}}
\pgfpathlineto{\pgfpoint{10.416667\du}{7.350000\du}}
\pgfpathcurveto{\pgfpoint{10.600762\du}{7.350000\du}}{\pgfpoint{10.750000\du}{7.797715\du}}{\pgfpoint{10.750000\du}{8.350000\du}}
\pgfpathcurveto{\pgfpoint{10.750000\du}{8.902285\du}}{\pgfpoint{10.600762\du}{9.350000\du}}{\pgfpoint{10.416667\du}{9.350000\du}}
\pgfpathlineto{\pgfpoint{9.083333\du}{9.350000\du}}
\pgfpathcurveto{\pgfpoint{8.899238\du}{9.350000\du}}{\pgfpoint{8.750000\du}{8.902285\du}}{\pgfpoint{8.750000\du}{8.350000\du}}
\pgfpathcurveto{\pgfpoint{8.750000\du}{7.797715\du}}{\pgfpoint{8.899238\du}{7.350000\du}}{\pgfpoint{9.083333\du}{7.350000\du}}
\pgfusepath{stroke}
% setfont left to latex
\definecolor{dialinecolor}{rgb}{0.000000, 0.000000, 0.000000}
\pgfsetstrokecolor{dialinecolor}
\node at (9.750000\du,8.550000\du){};
\definecolor{dialinecolor}{rgb}{1.000000, 1.000000, 1.000000}
\pgfsetfillcolor{dialinecolor}
\fill (4.800000\du,11.750000\du)--(4.800000\du,13.650000\du)--(6.800000\du,13.650000\du)--(6.800000\du,11.750000\du)--cycle;
\pgfsetlinewidth{0.100000\du}
\pgfsetdash{}{0pt}
\pgfsetdash{}{0pt}
\pgfsetmiterjoin
\definecolor{dialinecolor}{rgb}{0.000000, 0.000000, 0.000000}
\pgfsetstrokecolor{dialinecolor}
\draw (4.800000\du,11.750000\du)--(4.800000\du,13.650000\du)--(6.800000\du,13.650000\du)--(6.800000\du,11.750000\du)--cycle;
% setfont left to latex
\definecolor{dialinecolor}{rgb}{0.000000, 0.000000, 0.000000}
\pgfsetstrokecolor{dialinecolor}
\node at (5.800000\du,12.895000\du){};
\pgfsetlinewidth{0.100000\du}
\pgfsetdash{}{0pt}
\pgfsetdash{}{0pt}
\pgfsetbuttcap
{
\definecolor{dialinecolor}{rgb}{0.000000, 0.000000, 0.000000}
\pgfsetfillcolor{dialinecolor}
% was here!!!
\pgfsetarrowsend{stealth}
\definecolor{dialinecolor}{rgb}{0.000000, 0.000000, 0.000000}
\pgfsetstrokecolor{dialinecolor}
\draw (7.520532\du,5.247974\du)--(6.329468\du,7.402026\du);
}
\pgfsetlinewidth{0.100000\du}
\pgfsetdash{}{0pt}
\pgfsetdash{}{0pt}
\pgfsetbuttcap
{
\definecolor{dialinecolor}{rgb}{0.000000, 0.000000, 0.000000}
\pgfsetfillcolor{dialinecolor}
% was here!!!
\pgfsetarrowsend{stealth}
\definecolor{dialinecolor}{rgb}{0.000000, 0.000000, 0.000000}
\pgfsetstrokecolor{dialinecolor}
\draw (8.517535\du,5.250165\du)--(9.332465\du,7.299835\du);
}
\pgfsetlinewidth{0.100000\du}
\pgfsetdash{}{0pt}
\pgfsetdash{}{0pt}
\pgfsetbuttcap
{
\definecolor{dialinecolor}{rgb}{0.000000, 0.000000, 0.000000}
\pgfsetfillcolor{dialinecolor}
% was here!!!
\pgfsetarrowsend{stealth}
\definecolor{dialinecolor}{rgb}{0.000000, 0.000000, 0.000000}
\pgfsetstrokecolor{dialinecolor}
\draw (5.762329\du,9.497974\du)--(5.788257\du,11.701831\du);
}
% setfont left to latex
\definecolor{dialinecolor}{rgb}{0.000000, 0.000000, 0.000000}
\pgfsetstrokecolor{dialinecolor}
\node[anchor=west] at (3.750000\du,7.450000\du){w};
\pgfsetlinewidth{0.100000\du}
\pgfsetdash{}{0pt}
\pgfsetdash{}{0pt}
\pgfsetbuttcap
{
\definecolor{dialinecolor}{rgb}{0.000000, 0.000000, 0.000000}
\pgfsetfillcolor{dialinecolor}
% was here!!!
\pgfsetarrowsend{stealth}
\definecolor{dialinecolor}{rgb}{0.000000, 0.000000, 0.000000}
\pgfsetstrokecolor{dialinecolor}
\draw (9.416667\du,9.350000\du)--(8.700000\du,11.700000\du);
}
\pgfsetlinewidth{0.100000\du}
\pgfsetdash{}{0pt}
\pgfsetdash{}{0pt}
\pgfsetbuttcap
{
\definecolor{dialinecolor}{rgb}{0.000000, 0.000000, 0.000000}
\pgfsetfillcolor{dialinecolor}
% was here!!!
\pgfsetarrowsend{stealth}
\definecolor{dialinecolor}{rgb}{0.000000, 0.000000, 0.000000}
\pgfsetstrokecolor{dialinecolor}
\draw (9.750000\du,9.350000\du)--(10.000000\du,11.750000\du);
}
\pgfsetlinewidth{0.100000\du}
\pgfsetdash{}{0pt}
\pgfsetdash{}{0pt}
\pgfsetbuttcap
{
\definecolor{dialinecolor}{rgb}{0.000000, 0.000000, 0.000000}
\pgfsetfillcolor{dialinecolor}
% was here!!!
\pgfsetarrowsend{stealth}
\definecolor{dialinecolor}{rgb}{0.000000, 0.000000, 0.000000}
\pgfsetstrokecolor{dialinecolor}
\draw (10.199890\du,9.399744\du)--(11.250000\du,11.850000\du);
}
\pgfsetlinewidth{0.100000\du}
\pgfsetdash{}{0pt}
\pgfsetdash{}{0pt}
\pgfsetbuttcap
{
\definecolor{dialinecolor}{rgb}{0.000000, 0.000000, 0.000000}
\pgfsetfillcolor{dialinecolor}
% was here!!!
\pgfsetarrowsend{stealth}
\definecolor{dialinecolor}{rgb}{0.000000, 0.000000, 0.000000}
\pgfsetstrokecolor{dialinecolor}
\pgfpathmoveto{\pgfpoint{12.049850\du}{7.599847\du}}
\pgfpatharc{136}{44}{3.261563\du and 3.261563\du}
\pgfusepath{stroke}
}
\pgfsetlinewidth{0.100000\du}
\pgfsetdash{}{0pt}
\pgfsetdash{}{0pt}
\pgfsetbuttcap
\pgfsetmiterjoin
\pgfsetlinewidth{0.100000\du}
\pgfsetbuttcap
\pgfsetmiterjoin
\pgfsetdash{}{0pt}
\definecolor{dialinecolor}{rgb}{1.000000, 1.000000, 1.000000}
\pgfsetfillcolor{dialinecolor}
\pgfpathmoveto{\pgfpoint{19.933333\du}{3.600000\du}}
\pgfpathlineto{\pgfpoint{21.266667\du}{3.600000\du}}
\pgfpathcurveto{\pgfpoint{21.450762\du}{3.600000\du}}{\pgfpoint{21.600000\du}{4.047715\du}}{\pgfpoint{21.600000\du}{4.600000\du}}
\pgfpathcurveto{\pgfpoint{21.600000\du}{5.152285\du}}{\pgfpoint{21.450762\du}{5.600000\du}}{\pgfpoint{21.266667\du}{5.600000\du}}
\pgfpathlineto{\pgfpoint{19.933333\du}{5.600000\du}}
\pgfpathcurveto{\pgfpoint{19.749238\du}{5.600000\du}}{\pgfpoint{19.600000\du}{5.152285\du}}{\pgfpoint{19.600000\du}{4.600000\du}}
\pgfpathcurveto{\pgfpoint{19.600000\du}{4.047715\du}}{\pgfpoint{19.749238\du}{3.600000\du}}{\pgfpoint{19.933333\du}{3.600000\du}}
\pgfusepath{fill}
\definecolor{dialinecolor}{rgb}{0.000000, 0.000000, 0.000000}
\pgfsetstrokecolor{dialinecolor}
\pgfpathmoveto{\pgfpoint{19.933333\du}{3.600000\du}}
\pgfpathlineto{\pgfpoint{21.266667\du}{3.600000\du}}
\pgfpathcurveto{\pgfpoint{21.450762\du}{3.600000\du}}{\pgfpoint{21.600000\du}{4.047715\du}}{\pgfpoint{21.600000\du}{4.600000\du}}
\pgfpathcurveto{\pgfpoint{21.600000\du}{5.152285\du}}{\pgfpoint{21.450762\du}{5.600000\du}}{\pgfpoint{21.266667\du}{5.600000\du}}
\pgfpathlineto{\pgfpoint{19.933333\du}{5.600000\du}}
\pgfpathcurveto{\pgfpoint{19.749238\du}{5.600000\du}}{\pgfpoint{19.600000\du}{5.152285\du}}{\pgfpoint{19.600000\du}{4.600000\du}}
\pgfpathcurveto{\pgfpoint{19.600000\du}{4.047715\du}}{\pgfpoint{19.749238\du}{3.600000\du}}{\pgfpoint{19.933333\du}{3.600000\du}}
\pgfusepath{stroke}
% setfont left to latex
\definecolor{dialinecolor}{rgb}{0.000000, 0.000000, 0.000000}
\pgfsetstrokecolor{dialinecolor}
\node at (20.600000\du,4.800000\du){};
\pgfsetlinewidth{0.100000\du}
\pgfsetdash{}{0pt}
\pgfsetdash{}{0pt}
\pgfsetbuttcap
\pgfsetmiterjoin
\pgfsetlinewidth{0.100000\du}
\pgfsetbuttcap
\pgfsetmiterjoin
\pgfsetdash{}{0pt}
\definecolor{dialinecolor}{rgb}{0.000000, 0.000000, 0.000000}
\pgfsetfillcolor{dialinecolor}
\pgfpathmoveto{\pgfpoint{18.483333\du}{7.650000\du}}
\pgfpathlineto{\pgfpoint{19.816667\du}{7.650000\du}}
\pgfpathcurveto{\pgfpoint{20.000762\du}{7.650000\du}}{\pgfpoint{20.150000\du}{8.097715\du}}{\pgfpoint{20.150000\du}{8.650000\du}}
\pgfpathcurveto{\pgfpoint{20.150000\du}{9.202285\du}}{\pgfpoint{20.000762\du}{9.650000\du}}{\pgfpoint{19.816667\du}{9.650000\du}}
\pgfpathlineto{\pgfpoint{18.483333\du}{9.650000\du}}
\pgfpathcurveto{\pgfpoint{18.299238\du}{9.650000\du}}{\pgfpoint{18.150000\du}{9.202285\du}}{\pgfpoint{18.150000\du}{8.650000\du}}
\pgfpathcurveto{\pgfpoint{18.150000\du}{8.097715\du}}{\pgfpoint{18.299238\du}{7.650000\du}}{\pgfpoint{18.483333\du}{7.650000\du}}
\pgfusepath{fill}
\definecolor{dialinecolor}{rgb}{0.000000, 0.000000, 0.000000}
\pgfsetstrokecolor{dialinecolor}
\pgfpathmoveto{\pgfpoint{18.483333\du}{7.650000\du}}
\pgfpathlineto{\pgfpoint{19.816667\du}{7.650000\du}}
\pgfpathcurveto{\pgfpoint{20.000762\du}{7.650000\du}}{\pgfpoint{20.150000\du}{8.097715\du}}{\pgfpoint{20.150000\du}{8.650000\du}}
\pgfpathcurveto{\pgfpoint{20.150000\du}{9.202285\du}}{\pgfpoint{20.000762\du}{9.650000\du}}{\pgfpoint{19.816667\du}{9.650000\du}}
\pgfpathlineto{\pgfpoint{18.483333\du}{9.650000\du}}
\pgfpathcurveto{\pgfpoint{18.299238\du}{9.650000\du}}{\pgfpoint{18.150000\du}{9.202285\du}}{\pgfpoint{18.150000\du}{8.650000\du}}
\pgfpathcurveto{\pgfpoint{18.150000\du}{8.097715\du}}{\pgfpoint{18.299238\du}{7.650000\du}}{\pgfpoint{18.483333\du}{7.650000\du}}
\pgfusepath{stroke}
% setfont left to latex
\definecolor{dialinecolor}{rgb}{0.000000, 0.000000, 0.000000}
\pgfsetstrokecolor{dialinecolor}
\node at (19.150000\du,8.850000\du){};
\pgfsetlinewidth{0.100000\du}
\pgfsetdash{}{0pt}
\pgfsetdash{}{0pt}
\pgfsetbuttcap
\pgfsetmiterjoin
\pgfsetlinewidth{0.100000\du}
\pgfsetbuttcap
\pgfsetmiterjoin
\pgfsetdash{}{0pt}
\definecolor{dialinecolor}{rgb}{0.000000, 0.000000, 0.000000}
\pgfsetfillcolor{dialinecolor}
\pgfpathmoveto{\pgfpoint{21.541667\du}{7.600000\du}}
\pgfpathlineto{\pgfpoint{23.308333\du}{7.600000\du}}
\pgfpathcurveto{\pgfpoint{23.552259\du}{7.600000\du}}{\pgfpoint{23.750000\du}{8.047715\du}}{\pgfpoint{23.750000\du}{8.600000\du}}
\pgfpathcurveto{\pgfpoint{23.750000\du}{9.152285\du}}{\pgfpoint{23.552259\du}{9.600000\du}}{\pgfpoint{23.308333\du}{9.600000\du}}
\pgfpathlineto{\pgfpoint{21.541667\du}{9.600000\du}}
\pgfpathcurveto{\pgfpoint{21.297741\du}{9.600000\du}}{\pgfpoint{21.100000\du}{9.152285\du}}{\pgfpoint{21.100000\du}{8.600000\du}}
\pgfpathcurveto{\pgfpoint{21.100000\du}{8.047715\du}}{\pgfpoint{21.297741\du}{7.600000\du}}{\pgfpoint{21.541667\du}{7.600000\du}}
\pgfusepath{fill}
\definecolor{dialinecolor}{rgb}{0.000000, 0.000000, 0.000000}
\pgfsetstrokecolor{dialinecolor}
\pgfpathmoveto{\pgfpoint{21.541667\du}{7.600000\du}}
\pgfpathlineto{\pgfpoint{23.308333\du}{7.600000\du}}
\pgfpathcurveto{\pgfpoint{23.552259\du}{7.600000\du}}{\pgfpoint{23.750000\du}{8.047715\du}}{\pgfpoint{23.750000\du}{8.600000\du}}
\pgfpathcurveto{\pgfpoint{23.750000\du}{9.152285\du}}{\pgfpoint{23.552259\du}{9.600000\du}}{\pgfpoint{23.308333\du}{9.600000\du}}
\pgfpathlineto{\pgfpoint{21.541667\du}{9.600000\du}}
\pgfpathcurveto{\pgfpoint{21.297741\du}{9.600000\du}}{\pgfpoint{21.100000\du}{9.152285\du}}{\pgfpoint{21.100000\du}{8.600000\du}}
\pgfpathcurveto{\pgfpoint{21.100000\du}{8.047715\du}}{\pgfpoint{21.297741\du}{7.600000\du}}{\pgfpoint{21.541667\du}{7.600000\du}}
\pgfusepath{stroke}
% setfont left to latex
\definecolor{dialinecolor}{rgb}{0.000000, 0.000000, 0.000000}
\pgfsetstrokecolor{dialinecolor}
\node at (22.425000\du,8.800000\du){};
\pgfsetlinewidth{0.100000\du}
\pgfsetdash{}{0pt}
\pgfsetdash{}{0pt}
\pgfsetbuttcap
{
\definecolor{dialinecolor}{rgb}{0.000000, 0.000000, 0.000000}
\pgfsetfillcolor{dialinecolor}
% was here!!!
\pgfsetarrowsend{stealth}
\definecolor{dialinecolor}{rgb}{0.000000, 0.000000, 0.000000}
\pgfsetstrokecolor{dialinecolor}
\draw (18.718420\du,9.700177\du)--(17.650000\du,12.300000\du);
}
\pgfsetlinewidth{0.100000\du}
\pgfsetdash{}{0pt}
\pgfsetdash{}{0pt}
\pgfsetbuttcap
{
\definecolor{dialinecolor}{rgb}{0.000000, 0.000000, 0.000000}
\pgfsetfillcolor{dialinecolor}
% was here!!!
\pgfsetarrowsend{stealth}
\definecolor{dialinecolor}{rgb}{0.000000, 0.000000, 0.000000}
\pgfsetstrokecolor{dialinecolor}
\draw (19.150000\du,9.650000\du)--(19.850000\du,12.500000\du);
}
\pgfsetlinewidth{0.100000\du}
\pgfsetdash{}{0pt}
\pgfsetdash{}{0pt}
\pgfsetbuttcap
{
\definecolor{dialinecolor}{rgb}{0.000000, 0.000000, 0.000000}
\pgfsetfillcolor{dialinecolor}
% was here!!!
\pgfsetarrowsend{stealth}
\definecolor{dialinecolor}{rgb}{0.000000, 0.000000, 0.000000}
\pgfsetstrokecolor{dialinecolor}
\draw (21.983333\du,9.600000\du)--(21.400000\du,12.150000\du);
}
\pgfsetlinewidth{0.100000\du}
\pgfsetdash{}{0pt}
\pgfsetdash{}{0pt}
\pgfsetbuttcap
{
\definecolor{dialinecolor}{rgb}{0.000000, 0.000000, 0.000000}
\pgfsetfillcolor{dialinecolor}
% was here!!!
\pgfsetarrowsend{stealth}
\definecolor{dialinecolor}{rgb}{0.000000, 0.000000, 0.000000}
\pgfsetstrokecolor{dialinecolor}
\draw (22.727795\du,9.648706\du)--(23.450000\du,12.150000\du);
}
\pgfsetlinewidth{0.100000\du}
\pgfsetdash{}{0pt}
\pgfsetdash{}{0pt}
\pgfsetbuttcap
{
\definecolor{dialinecolor}{rgb}{0.000000, 0.000000, 0.000000}
\pgfsetfillcolor{dialinecolor}
% was here!!!
\pgfsetarrowsend{stealth}
\definecolor{dialinecolor}{rgb}{0.000000, 0.000000, 0.000000}
\pgfsetstrokecolor{dialinecolor}
\draw (20.224402\du,5.649084\du)--(19.525598\du,7.600916\du);
}
\pgfsetlinewidth{0.100000\du}
\pgfsetdash{}{0pt}
\pgfsetdash{}{0pt}
\pgfsetbuttcap
{
\definecolor{dialinecolor}{rgb}{0.000000, 0.000000, 0.000000}
\pgfsetfillcolor{dialinecolor}
% was here!!!
\pgfsetarrowsend{stealth}
\definecolor{dialinecolor}{rgb}{0.000000, 0.000000, 0.000000}
\pgfsetstrokecolor{dialinecolor}
\draw (20.933333\du,5.600000\du)--(21.983333\du,7.600000\du);
}
% setfont left to latex
\definecolor{dialinecolor}{rgb}{0.000000, 0.000000, 0.000000}
\pgfsetstrokecolor{dialinecolor}
\node[anchor=west] at (24.100000\du,5.600000\du){keine };
% setfont left to latex
\definecolor{dialinecolor}{rgb}{0.000000, 0.000000, 0.000000}
\pgfsetstrokecolor{dialinecolor}
\node[anchor=west] at (24.100000\du,6.400000\du){Fortsetzung!};
\end{tikzpicture}

            \end{enumerate}

        \subsection{Laufzeitanalyse}
            Nach Spaltung hat der Baum eine bessere Struktur, deshalb:
            \emph{Amortisierte Analyse:}\\
            (= Kosten pro Operation gemittelt über n Operationen)

            \begin{lemma}
                In einem (2,4) Baum sind die amortisierten Kosten der Operationen Einfügen/Streichen $\LO(1)$.
            \end{lemma}
            \begin{proof}
                Beschreibung des Zustands des Baums T: \\
                $\textit{pot} (T)$
                \begin{eqnarray*}
                    &=& 2 \cdot \# \text{ Knoten von Grad } 1 \\
                    &+& 1 \cdot \# \text{ Knoten von Grad } 2 \\
                    &+& 0 \cdot \# \text{ Knoten von Grad } 3 \\
                    &+& 2 \cdot \# \text{ Knoten von Grad } 4 \\
                    &+& 4 \cdot \# \text{ Knoten von Grad } 5
                \end{eqnarray*}
                \emph{Invariante}: \\
                \begin{itemize}
                    \item $pot(T) \geq 0$ 
                    \item bei Spalten/Verschmelzen/Stehlen ist nur ein Knoten nicht auf Grad 2,3,4
                    \item Vor Einfügen /Streichen haben alle Knoten den Grad 2,3,4.
                \end{itemize}
                Einzeloperationen haben Kosten von $\LO(1)$.\\
                \emph{Behauptung:} Spalten / Verschmelzen verringern Potential. Stehlen erhöht es nicht. \\
                \emph{Beweis:} \\
                Stehlen. Knoten $w$ trug 2 Einheiten zum Potential bei und sein Nachbarknoten $p$. Danach trägt $w$ 1 Einheit bei, 
                Nachbar $\leq p +1$.
            \end{proof}

            \emph{Spalten}: $w$ trägt 4 bei zum Potential, der parent trägt p bei. Danach haben die beiden neuen Knoten $0$ und $1$, der
            parent hat $\leq  p +2 $ $\checkmark$.\\\

            \emph{Verschmelzen}: $w$ trägt 2 bei, Geschwisterknoten von $w$ trägt 1 bei, der parent trägt p bei \\
            $\Ra$ Danach 0 und $\leq p+1$ .\\\

            Amortisierte Laufzeit für Einfügen: Tatsächliche Kosten + Potentialerhöhung. \\
            Tatsächliche Kosten 1 + Potentialerhöhung $\leq 2$ \\
            Folge von $f$ Spaltungen: tatsächliche Kosten $f$, Potentialerhöhung $\leq f$. \\
            $\Ra$ Amortisierte Kosten von Einfügen: $\leq 3$ \\\\
            Streichen analog\\

        \subsection{Anwendung: Sortieren} 
            durch Einfügen: $\LO(n \log n)$ \
            vorsortierter Folgen: Mache Suchen billiger! \\
            Sei $x_1, \ldots, x_2$ Folge reeller Zahlen. \\
            $$F= \btl \{ (i,j) \mid i < j \text{ und } x_i > x_j \} \btr$$
            (Zahl der Inversionen). Es gilt $O \leq F \leq \frac{n^2}{2}$

            \begin{satz}
            $x_1, \ldots x_n$ kann in Zeit $\LO(n \log \frac{F}{n})$ sortiert werden. 
            \end{satz}

            \begin{proof}
                Sei $f_i = |\{ j \mid i < j , x_i > x_j \}|$ Es gilt $F= \sum \limits_{i} f_i$ \\
                Starte mit leerem (2,4) Baum. Füge Folge in umgekehrter Reihenfolge ein. \\
                % Graphic for TeX using PGF
% Title: /home/alex/WualaDrive/myfiles/Uni/4 Semester/Algorithmen/Skript/algorithmen/Diagramm24.dia
% Creator: Dia v0.97.1
% CreationDate: Tue May 18 15:50:34 2010
% For: alex
% \usepackage{tikz}
% The following commands are not supported in PSTricks at present
% We define them conditionally, so when they are implemented,
% this pgf file will use them.
\ifx\du\undefined
  \newlength{\du}
\fi
\setlength{\du}{15\unitlength}
\begin{tikzpicture}
\pgftransformxscale{1.000000}
\pgftransformyscale{-1.000000}
\definecolor{dialinecolor}{rgb}{0.000000, 0.000000, 0.000000}
\pgfsetstrokecolor{dialinecolor}
\definecolor{dialinecolor}{rgb}{1.000000, 1.000000, 1.000000}
\pgfsetfillcolor{dialinecolor}
\pgfsetlinewidth{0.100000\du}
\pgfsetdash{}{0pt}
\pgfsetdash{}{0pt}
\pgfsetbuttcap
\pgfsetmiterjoin
\pgfsetlinewidth{0.100000\du}
\pgfsetbuttcap
\pgfsetmiterjoin
\pgfsetdash{}{0pt}
\definecolor{dialinecolor}{rgb}{1.000000, 1.000000, 1.000000}
\pgfsetfillcolor{dialinecolor}
\fill (7.700000\du,10.100000\du)--(14.750000\du,10.100000\du)--(11.225000\du,2.650000\du)--cycle;
\definecolor{dialinecolor}{rgb}{0.000000, 0.000000, 0.000000}
\pgfsetstrokecolor{dialinecolor}
\draw (7.700000\du,10.100000\du)--(14.750000\du,10.100000\du)--(11.225000\du,2.650000\du)--cycle;
% setfont left to latex
\definecolor{dialinecolor}{rgb}{0.000000, 0.000000, 0.000000}
\pgfsetstrokecolor{dialinecolor}
\node at (11.225000\du,8.437500\du){};
% setfont left to latex
\definecolor{dialinecolor}{rgb}{0.000000, 0.000000, 0.000000}
\pgfsetstrokecolor{dialinecolor}
\node[anchor=west] at (6.800000\du,10.900000\du){x\_\{i+1\}};
% setfont left to latex
\definecolor{dialinecolor}{rgb}{0.000000, 0.000000, 0.000000}
\pgfsetstrokecolor{dialinecolor}
\node[anchor=west] at (13.650000\du,10.900000\du){x\_n};
\pgfsetlinewidth{0.100000\du}
\pgfsetdash{}{0pt}
\pgfsetdash{}{0pt}
\pgfsetbuttcap
{
\definecolor{dialinecolor}{rgb}{0.000000, 0.000000, 0.000000}
\pgfsetfillcolor{dialinecolor}
% was here!!!
\pgfsetarrowsend{stealth}
\definecolor{dialinecolor}{rgb}{0.000000, 0.000000, 0.000000}
\pgfsetstrokecolor{dialinecolor}
\draw (9.850000\du,12.250000\du)--(9.900000\du,10.900000\du);
}
\end{tikzpicture}

                Füge $x_n, \ldots, x_1$ ein. Einfügen $(x_i)$: Starte am linkesten Blatt; laufe hoch und drehe an der richtigen Stelle um,
                laufe runter und füge ein. Ist $x_i$ klein, so laufe nicht sehr hoch. \\\
                Es gilt (amortisiert): Kosten für Einfügen $= \LO(1 + \log f_i)$ \\
                Suchen$(x_i) = \LO(\log f_i)$ \\
                Einfügen: = $\LO(1)$. \\\\
                Laufe hoch bis Höhe $h$ (Umkehrpunkt): \\
                $$2^{h-1} \leq f_i \Ra h \leq 1 + \log f_i$$
                $\Ra$ Gesamtkosten: 
                \begin{align*}
                		\sum \limits_{i} \LO(1+ \log f_i) 	&= \LO(n + \sum \limits_{i=1}^n \log f_i) \\
                									&= \LO(n+\log \prod \limits_{i=1}^n f_i\\
                									&= \LO(n+\log \prod \limits_{i=1}^n \frac{F}{n})	\\
                									&\leq \LO(n + n \log \frac{F}{n})
                \end{align*}
                $x_i$ wird nach dem $f_i$-ten Element in die bisherige Liste eingefügt.
            \end{proof}
        
        
        
    \section{B-Bäume (knotenorientiert)}
        \subsection*{Anmerkung}
            In allen Skizzen müssten die Pointer jeweils vor und nach einem Element stehen. Dies ist wegen des erhöhten Zeichenaufwandes vernachlässigt worden.
            
        \subsection{Idee und Definition}            
            Idee: Mehr Daten in einen Knoten 
            \begin{definition}
                B-Baum der Ordnung $k$. $k \geq 2$ ist ein Baum dessen: \\
                \begin{enumerate}[1]
                    \item Blätter alle gleiche Tiefe haben 
                    \item Wurzel $\geq 2$ Kinder und dessen andere inneren Knoten $\geq k$ Kinder Haben.
                    \item innere Knoten $\leq 2k-1$ Kinder haben. \\
                \end{enumerate}
            \end{definition}

            Höhe eines B-Baums: \\
            \begin{lemma}
                Sei $T$ ein Baum der Ordnung $k$ mit Höhe $h$ und $n$ Blättern. \\
                Dann gilt: \\
                \begin{center}
                    $2 \cdot k^{h-1} \leq n \leq (2k-1)^h$ 
                \end{center}
                \begin{proof}
                    Zahl der Blätter ist minimal, wenn jeweils die Minimal-Anzahl von Kindern vorkommen und maximal, wenn jeweils die Maximalzahl 
                    vorkommt.
                    $$2 \cdot \underbrace{k \cdot k \ldots \cdot k}_{h-1} \text{ mal} = n_{\min} \leq n \leq n_{\max} = (2k-1)^h$$
                \end{proof}

                Es gilt somit $\log_{(2k-1)} n \leq h \leq 1+ \log_k (\frac{n}{2})$ (logarithmieren!)\\
                Operationen: \textit{Suchen}(x), \textit{Einfügen}(x), \textit{Streichen}(x) \\
                Sei $u$ Knoten mit $k$ Kindern und Schlüsseln 
                $$s_1 < s_2 < \ldots < s_{l-1}$$ mit Unterbäumen $$T_1, \ldots , T_l$$
                       \begin{center}
            \begin{tikzpicture}[scale=1]
                \tikzstyle{bplus}=[rectangle split, rectangle split horizontal,
                    rectangle split ignore empty parts,rectangle split parts=5,draw,style={transform shape}]
                \tikzstyle{every node}=[bplus]
                \tikzstyle{level 1}=[sibling distance=20mm]
                \tikzstyle{level 2}=[sibling distance=15mm]
                \node {$s_ 1$ \nodepart{two} $s_2$ \nodepart{three} ... \nodepart{four} $s_{i-1}$ \nodepart{five} $s_i$} [->]
                    child {node {$T_1$}}
                    child {node {$T_2$}}
                    child {node {$...$}}
                    child {node {$...$}}
                    child {node {$T_{i-1}$}}
                    child {node {$T_i$}}
                ;
            \end{tikzpicture}
        \end{center}          
                Es gilt für alle $v \in T_i$ und Schlüssel $s$ in $v$: \\
                \begin{center}
                    $s \leq s_i$ falls $i=1$ \\
                    $s_{i-1} < s \leq s_i$ falls $1 < i < l$ \\
                    $s_{i-1} < s$ falls $i=l$ 
                \end{center}
            \end{lemma}

            Algorithmus: \\
            \begin{algorithm}
				\begin{algorithmic}
				\Function{Suchen}{x} \Comment{Starte in w= Wurzel}
	                \State suche kleinstes \( s_i \) in w mit \( x \leq s_i \) 
	                \If{\( s_i \) existiert}
				        \If{\( x= s_i \)}
							\State gefunden
		                \Else
							\State w \( \leftarrow \) i-tes Kind von w
							\State \Call{Suchen}{w}
						\EndIf
	                \Else
		                \State w \( \leftarrow \) rechtestes Kind von w
						\State \Call{Suchen}{w}
					\EndIf
				\EndFunction
				\end{algorithmic}
            \end{algorithm}
            Laufzeit: $\LO(\log_k n \cdot k)$ (Höhe mal Suche innerhalb eines Knotens)\\

        \subsection{Einfügen(x)}
            Suchen(x) endet in Blatt b mit parent v, der l Kinder hat. Nun müssen 2 Fälle unterschieden werden.
            \begin{enumerate}
                \item In Blatt b ist noch genügend Platz für eine Einfügung vorhanden. Dann füge x in b ein.
                \item Nach Einfügung von x hätte Blatt b $2k-1$ Elemente: Teile auf.
            \end{enumerate}
            
            \emph{Aufteilen bei Einfügung:} Sollte bei einer Einfügung der 2. Fall aufgetreten sein, muss der Blattknoten geteilt werden. Sortiere dazu die Blattelemente aus b und entferne den Median $m$. Bilde aus den Elementen $e$ mit $e < m$ das Blatt $b$ und aus den verbleibenden Elementen das Blatt b'. Füge dann $m$ als neuen Seperator in den Vaterknoten $v$ ein. Die Blätter $b$ und $b$' stehen dann jeweils links und rechts von $m$. Ein Beispiel mit konkreten Werten und Erklärungen findet sich im nächsten Kapitel.
            
            \emph{Beachte:} Dabei kann der Vaterknoten $v$ ebenfalls mehr als $2k-2$ Elemente enthalten. Die Einfügung muss also rekursiv nach oben fortgesetzt werden. Wird dabei die Wurzel aufgespalten, so ergibt sich eine neue Wurzel mit 2 Kindern und Tiefe wächst um 1. Dies ist die einzige Möglichkeit, mit der die Tiefe eines B-Baums wachsen kann.
            
            
        \subsection{Konkrete Beispiele für die Operationen}
            \subsubsection{Einfügung mit Split}
                Es wird zu beginn vom folgenden B-Baum ausgegangen:
                \begin{center}
                        
        \begin{center}
            \begin{tikzpicture}
                \tikzstyle{bplus}=[rectangle split, rectangle split horizontal,
                    rectangle split ignore empty parts,draw]
                \tikzstyle{every node}=[bplus]
                \tikzstyle{level 1}=[sibling distance=60mm]
                \tikzstyle{level 2}=[sibling distance=15mm]
                \node {B \nodepart{two} D} [->]
                    child {node {A \nodepart{two} $\square$}}
                    child {node {C \nodepart{two} $\square$}}
                    child {node {H \nodepart{two} K}}
                ;
            \end{tikzpicture}
        \end{center} 
                \end{center}
                Es soll nun der Wert I eingefügt werden. Dies führt zu mehreren Splits und dem B-Baum
                \begin{center}
                    \begin{center}
            \begin{tikzpicture}
                \tikzstyle{bplus}=[rectangle split, rectangle split horizontal,
                    rectangle split ignore empty parts,draw]
                \tikzstyle{every node}=[bplus]
                \tikzstyle{level 1}=[sibling distance=60mm]
                \tikzstyle{level 2}=[sibling distance=15mm]
                \node {D \nodepart{two} $\square$} [->]
                    child {node {B \nodepart{two} $\square$}
                        child {node {A \nodepart{two} $\square$}}
                        child {node {C \nodepart{two} $\square$}}                    
                    }
                    child {node {I \nodepart{two} $\square$}
                        child {node {H \nodepart{two} $\square$}}
                        child {node {K \nodepart{two} $\square$}}                    
                    }                
                ;
            \end{tikzpicture}
        \end{center} 
                \end{center}
                Nachdem $I$ eingefügt wurde, war der Knoten $H,I,K$ überbelegt. Das mittlere Element wird nach oben geschoben. Auch dort ist der Knoten nun überbelegt und wird gesplittet. Dabei wächst die Tiefe des Baums um 1. 
                
            \subsubsection{Löschen mit Merge}
                Es wird zu beginn vom folgenden B-Baum ausgegangen:
                \begin{center}
                    \begin{center}
            \begin{tikzpicture}
                \tikzstyle{bplus}=[rectangle split, rectangle split horizontal,
                    rectangle split ignore empty parts,draw]
                \tikzstyle{every node}=[bplus]
                \tikzstyle{level 1}=[sibling distance=60mm]
                \tikzstyle{level 2}=[sibling distance=15mm]
                \node {B \nodepart{two} D} [->]
                    child {node {A \nodepart{two} $\square$}}
                    child {node {C \nodepart{two} $\square$}}
                    child {node {I \nodepart{two} K}}
                ;
            \end{tikzpicture}
        \end{center} 
                \end{center}
                Es soll nun der Wert B gelöscht werden
                \begin{center}
                           \begin{center}
            \begin{tikzpicture}
                \tikzstyle{bplus}=[rectangle split, rectangle split horizontal,
                    rectangle split ignore empty parts,draw]
                \tikzstyle{every node}=[bplus]
                \tikzstyle{level 1}=[sibling distance=60mm]
                \tikzstyle{level 2}=[sibling distance=15mm]
                \node {D \nodepart{two} $\square$} [->]
                    child {node {A \nodepart{two} C}}
                    child {node {I \nodepart{two} K}}
                ;
            \end{tikzpicture}
        \end{center} 
                \end{center}
                Nachdem B gelöscht wird, müssen Kindknoten gemerged werden.
                

    \section{Randomisierte Suchbäume}
        \begin{itemize}
            \item Treaps(Aragorn/Seidel)
            \item Skiplists (Pugh)
        \end{itemize}
        
        \subsection{Skiplists} 
            Gegeben: $S = \{ x_1, \ldots, x_n \}$ , $p=(\frac{1}{2})$ \\            
            Definiere Datenstruktur bezüglich Leveling aus Listen: \\
            Füge $\pm \infty$ als Spezialelemente in alle Listen ein. Elemente der Liste $L_i$ werden von links nach rechts verbunden. Verbinde zusätzlich alle Elemente $x$ vom höchsten Level bis Level 1. Element $x$ bildet also einen Turm mit $l(x)$ Knoten. Wir haben nun eine Baumstruktur mit Intervallen, die auf Level $i+1$ mehrere Intervalle auf Level $i$ umfassen. Dabei sei $c(I)$ die Zahl der Kinder von Intervall $I$           
            
            $S= L_1 \supseteq L_2 \supseteq \ldots \supseteq L_r=\emptyset$
            
            \subsubsection{Zufälliges Leveling:}
            
            
%            % 2013 anders, s.u.
%            \begin{alltt}
%            Starte mit \( L_1 = S \)  \\
%            Gehe zu \( L_2 \). Mache Münzwurf für jedes Element aus dem Level darunter
%            und kopiere es hoch, mit Wahrscheinlichkeit \( p \). \\
%            Level \( t \) sei das höchste mit  \( x \). Die Wahrscheinlichkeit dass Schlüssel \( x \)\\
%            auf höchstem Level \( k \) ist, ist 
%            $$p^{k-1} (1-p) $$
%            \end{alltt}
            
            $L_{i+1}$ ergibt sich aus $L_i$ indem jedes Element $x\in L_i$ mit Wahrscheinlichkeit $p=\frac{1}{2}$ an $L_{i+1}$ weitergereicht wird.
            $l(x)$ für $x\in S$ sind unabhängige Zufallsvariable nach einer Geometrischen Verteilung mit $p=\frac{1}{2}$.
            
            \emph{Geometische Verteilung:}\\
            Wirf Münze, bis Zahl fällt. X ist Anzahl der benötigten Würfe.
            
            $E(X)=\frac{1}{p}$, $Var(X)=\frac{1-p}{p^2}$\\\\
            
            Erwarteter Platzbedarf für zufällige Skiplist mit $|S|=n$ ist also $2n=\LO(n)$
            
            \begin{lemma}
            	Zahl der Level r bei zufälligem Leveling hat Erwartungswert $E(r)=\LO(\log n)$. Es gilt $r=\LO(\log n)$ mit hoher Wahrscheinlichkeit.
            \end{lemma}
            \begin{proof}
            	Sei $r=\max\{l(x)|x\in S\}$ und $l(x)$ sind geometrisch verteilte Zufallsvariablen mit $p=\frac{1}{2}$.
            	Sei $X_i$ die Zufallsvariable für $x_i\in S$.\\
            	Es gilt:\\
            	$P( X_i>t)<(1-p)^t$ und $P(\max\limits_{i} X_i>t)<n\cdot (1-p)^t=\frac{n}{2^t}$ für $p=\frac{1}{2}$\\
            	Mit $t=\alpha\log n$ und $r=1+\max\limits_i X_i \Ra P(r>\alpha\log n)\le \frac{1}{n^{\alpha\log n -1}}$ 
            \end{proof}


%		%quite strange...
%            Erwartungswert: 
%            $$\sum \limits_{k=1}^\infty k \cdot p^{k-1} (1-p) = (1-p) \sum \limits_{k=1}^\infty k \cdot p^{k-1} = (1-p) \cdot \frac{1}{(1-p)^2} = \frac{1}{1-p}$$
%
%            \underline{höchster Level:}(erwartet) \\
%            Sei $v = \max \{L(t) \mid t \in S \}$ \\
%            $L(x)$ hat den Erwartungswert $\frac{1}{1-p}$ \\
%            Wahrscheinlichkeit, dass $L(t) > t$ ist, ist $< p^t$ \\
%            Levelzuweisung der Elemente ist unabhängig, d.h. die Wahrscheinlichkeit dass ein Element (aus $n$ vielen) Level $2t$ hat, ist $< n \cdot p^t$ \\\
%            Setze $t = c \cdot \log n$, $p= \frac{1}{2}$ \\
%            $\Ra \textit{prob}(v > c \cdot \log n) \leq \frac{n}{2^{c \log n}} < \frac{1}{n}$ für $c > 1 $ \\
%            Höhe des Skipbits ist demnach $O (\log n)$ \\\\
%
%			%2013 anders
%            \emph{Suchen(x)} \\
%
%            Starte im Level $v$ mit Header. Laufe nach rechts in $L_v$, bis Element $> x$ gefunden wird. Im Element zuvor gehe nach unten. Iteriere. \\
%            Suche durchläuft Skipbits von rechts oder hin zu $x$. Strategie: Gehe möglichst nach rechtst, wenns nicht mehr geht, gehe nach unten. \\

	   \subsubsection{Suchen(x)}
	   	  laufe Intervalle levelweise (r$\ra$ 0) ab:\\
	   	  Sei $I_j(x)$ das Intervall auf Level j, das x enthält.\\
	   	  Liegt x auf dem Rand von zwei Intervallen, weise x dem linken Intervall zu.\\
	   	  Folge $I_r(x)\supseteq I_{r-i}(X)\supseteq\dots\supseteq I_1(x)$ lässt sich als Pfad von Wurzel zum Blatt ansehen.

        \subsubsection{Laufzeit}
        	    $$
        	        \LO\left(\sum\limits_{j=0}^r 1 + c\left(I_j(y)\right)\right)
        	    $$
            Anzahl Level: $\LO(\log n)$ , Anzahl Schritte nach unten. \\
            Der Suchpfad von $x$ zurück zu Level $v$ wird beschrieben durch: \\
            Gehe nach oben, wenn möglich; wenn nicht gehe nach links. \\
            Schritt nach links gibt es mit Wahrscheinlichkeit $\frac{1}{2}$ jeweils $\Ra$ Erwartete Anzahl der Schritte nach links ist
            genauso groß, wie erwartete Anzahl der Schritte nach oben. $\Ra$ Laufzeit $\LO(\log n)$ 
        
        \subsubsection{Einfügen(x)}
            Suche zuerst nach $x$. Füge es in $L_1$ ein und hänge zwei Zeiger um $\Ra \LO(\log n)$\\
            Laufe hoch und mache Zufallsexperiment, füge nacheinander eventuell Kopien ein.\\
            Ist $l(x)>r$, kreiere neues Level, $l(x)\gets r+1$, $r\gets r+1$. 
            Bestimme Suchpfade durch Intervalle von $l(x)$ aus und spalte durch Einfügen von x.
            %Gemeint ist, dass x bisher bestehende Intervalle teilt und so neue verknüpfungen nötig werden. Bessere Formulierung?
            Erwartete Zahl von Kopie ist 2 $\Ra \LO(1)$ erwartet. \\
            $\Ra $gesamt $ \LO(\log n)$
        
        \subsubsection{Streichen(x)}
            Während dem Suchlauf nach $x$ lösche von oben nach unten alle Vorkommen von $x$ durch Zeigerumhängen. Laufzeit: $\LO(\log n)$.\\
            
            \begin{satz}
        	        In zufälliger Skiplist für S der Größe n können Suchen/ Einfügen/ Streichen in erwarteter Zeit von $\LO(\log n)$ ausgeführt werden.
            \end{satz}
        
        \subsubsection{Weitere Operationen:}
        	    \begin{itemize}
        	  	    \item[join($S_1,x,S_2$):] ersetzte $S_1,S_2$ durch $S_1\cup\{x\}\cup S_2$ und $\forall y\in S_1, z\in S_2 : y<x<z$\\
        	      		Idee: hänge $L_i(S_1),L_i(S_2)$ aneinander $\forall i$ und füge dann x ein
        	      	\item[paste($S_1,S_2$):] analog
        	      	\item[split($x,S$):] Umkehrung von join()\\
        	      		Idee: streiche x, verweise auf $\pm\infty$ statt nächstes Element
        	    \end{itemize}

		    Alles in $\LO(\log n)$!
		    
		\subsubsection{Beispiel}
		    Es sollen beispielhaft die Elemente
            $$
                23, 16, 18, 35, 11, 28, 56, 13, 14
            $$
            mit Hilfe der Zufallsbitfolge
            $$
                1010011001010001101    
            $$
            in eine Skiplist eingefügt werden. Dies ergibt die folgende Skiplist.
            \begin{center}
                    \begin{tikzpicture}
        [->,>=stealth',shorten >=1pt,
        auto,minimum size=1.2cm,node distance=2cm,
        main node/.style={circle,draw},
        every node/.style={transform shape},
        scale=0.7]

        \node[main node] (ninfty_0) {$-\infty$};
            \node[main node] (ninfty_1)[above of=ninfty_0]  {$-\infty$};
            \node[main node] (ninfty_2)[above of=ninfty_1]  {$-\infty$};
            \node[main node] (ninfty_3)[above of=ninfty_2]  {$-\infty$};
        \node[main node] (11_0)[right of=ninfty_0] {$11$};
            \node[main node] (11_1)[above of=11_0] {$11$};
            \node[main node] (11_2)[above of=11_1] {$11$};
        \node[main node] (13_0)[right of=11_0] {$13$};
        \node[main node] (14_0)[right of=13_0] {$14$};
            \node[main node] (14_1)[above of=14_0] {$14$};
        \node[main node] (16_0)[right of=14_0] {$16$};
            \node[main node] (16_1)[above of=16_0] {$16$};
        \node[main node] (18_0)[right of=16_0] {$18$};
            \node[main node] (18_1)[above of=18_0] {$18$};
            \node[main node] (18_2)[above of=18_1] {$18$};
        \node[main node] (23_0)[right of=18_0] {$23$};
        \node[main node] (28_0)[right of=23_0] {$28$};
            \node[main node] (28_1)[above of=28_0] {$28$};
        \node[main node] (35_0)[right of=28_0] {$35$};
        \node[main node] (56_0)[right of=35_0] {$56$};
            \node[main node] (56_1)[above of=56_0] {$56$};
            \node[main node] (56_2)[above of=56_1] {$56$};
            \node[main node] (56_3)[above of=56_2] {$56$};
        \node[main node] (infty_0)[right of=56_0] {$\infty$};
            \node[main node] (infty_1)[above of=infty_0]  {$\infty$};
            \node[main node] (infty_2)[above of=infty_1]  {$\infty$};
            \node[main node] (infty_3)[above of=infty_2]  {$\infty$};

        % Vertikal
        
        %infty
        \path (infty_3) edge (infty_2)
        (infty_2) edge (infty_1)
        (infty_1) edge (infty_0);
        
        %-infty
        \path (ninfty_3) edge (ninfty_2)
        (ninfty_2) edge (ninfty_1)
        (ninfty_1) edge (ninfty_0);

        % 11
        \path (11_2) edge (11_1)
              (11_1) edge (11_0);

        % 14
        \path (14_1) edge (14_0);

        % 16
        \path (16_1) edge (16_0);

        % 18
        \path (18_2) edge (18_1)
              (18_1) edge (18_0);

        % 28
        \path (28_1) edge (28_0);

        % 56
        \path (56_3) edge (56_2)
              (56_2) edge (56_1)
              (56_1) edge (56_0);

        % Horizontal        
        %base
        \path
        (ninfty_0) edge (11_0)
        (11_0) edge (13_0)
        (13_0) edge (14_0)
        (14_0) edge (16_0)
        (16_0) edge (18_0)
        (18_0) edge (23_0)
        (23_0) edge (28_0)
        (28_0) edge (35_0)
        (35_0) edge (56_0)
        (56_0) edge (infty_0)
        ;
        % level 1
        \path
            (ninfty_1) edge (11_1)
            (11_1) edge (14_1)
            (14_1) edge (16_1)
            (16_1) edge (18_1)
            (18_1) edge (28_1)
            (28_1) edge (56_1)
            (56_1) edge (infty_1)
            ;

        % level 2
        \path
            (ninfty_2) edge (11_2)
            (11_2) edge (18_2)
            (18_2) edge (56_2)
            (56_2) edge (infty_2)
            ;

         % level 3
        \path
            (ninfty_3) edge (56_3)
            (56_3) edge (infty_3)
            ;
        
    \end{tikzpicture}
            \end{center}

% Copyright \copyright\ 2013  Simon Kalt, Jan-Peter Hohloch, Tobias Fabritz
% Es wird die Erlaubnis gegeben, dieses Dokument unter den Bedingungen der von der Free
% Software Foundation veröffentlichen  GNU Free
% Documentation License (Version 1.2 oder neuer) zu kopieren, verteilen und/oder
% zu verändern. Eine Kopie dieser Lizenz ist unter
% http://www.gnu.org/copyleft/fdl.txt erhältlich.
%
% Zusätzlich muss jede Kopie/Aktualisierung wieder über die Seite
% der Fachschaft Informatik der Uni Tübingen
% den Studenten zur Verfügung gestellt werden
% http://www.fsi.uni-tuebingen.de/

\chapter{Hashing} 
    \section{Definition}
        Sei $U$ ein Universum und $S \subset U$ eine Schlüsselmenge mit $|S|<<|U|$. Elemente werden in eine Hashtafel $T$ gespeichert. 
        Wir wollen nun die Operationen 
        \begin{itemize}
            \item Zugriff(a,S)
            \item Einfügen(a,S)
            \item Streichen(a,S)
        \end{itemize}
        unterstützen. Dabei definieren wir
        \begin{itemize}
            \item Hashtafel $T[0,1,...,m-1]$
            \item Universum $U=[0,1,...,N-1]$
            \item Hashfunktion $h: U \rightarrow [0,1,...,m-1]$ als 
                $$
                    a \mapsto h(a)
                $$
                wobei Element $a$ an Stelle $T[h(a)]$ liegt
        \end{itemize}
        
        \subsection{Beispiel}
            Seien $N=50$, $m=3$, $S= \{2,21\}$ sowie 
            $$
                h(x) = x \mod 3
            $$
            (Divisionsmethode). Dann ergibt sich die Hashtafel
            
            \begin{math}
            	\begin{array}{c|c|c}
            		0&1&2\\\hline
            		2&21&-           	
            	\end{array}
            \end{math}
            
    \section{Hashing mit Verkettung}
        Es handelt sich um eine der beiden Ideen. Die andere wäre Hashing mit offener Adressierung (s. Kapitel~\ref{sec:offeneAdressierung}). 
        
        \subsection{Idee}
        		Jeder Tafeleintrag ist eine Liste. Die $i$-te Liste enthält alle $x\in S$ mit $h(x)=i$.
        
        \subsection{Laufzeit}
        		WorstCase: $\LO(|S|)=\LO(n)$ im Mittel aber \textit{wesentlich} besser.
        
            \subsubsection{Annahmen}
            		\begin{itemize}
            		\item $h(x)$ kann ich $\LO(1)$ ausgerechnet werden.
            		\item $\left|h^{-1}(i)\right|=\frac{N}{m}$,  $\forall i\in \{0,\dots , m-1\}$
            		\item Für eine Folge von Operationen gilt: Wahrscheinlichkeit, dass das $j$-te Element in der Folge ein festes $x\in U$ ist,
            			beträgt $\frac{1}{N}\Ra$ Operationen sind unabhängig und gleichverteilt\\
            			Ist $x_k$ Argument der $k$-ten Operation, so gilt: $P(h(x_k)=i)=\frac{1}{m}$ (also auch Hashfuktionswerte gleichverteilt)
            		\end{itemize}
            
            \subsubsection{Beweis für mittlere Kosten}
            	\begin{definition}
            		$$\delta_h(x,y)=\begin{cases} 	1 & x\not=y\wedge h(x)=h(y)\\
            								0 & \text{sonst}
            					\end{cases}$$
            		
            		$\delta_h(x,S)=\sum\limits_{y\in S} \delta_h(x,y)$ (Zahl der Kollisionen mit x)\\
            		Kosten von Operationen (Zugriff/Streichen/Einfügen) $XYZ(x)=1+\delta_h(x,S)$      	
            	\end{definition}
            	\begin{satz}
            		Die mittleren Kosten von $XYZ(x)$ sind $1+\beta=1+\frac{n}{m}=1+\frac{|S|}{|T|}$
            	\end{satz}
            	\begin{proof}
            		Sei $h(x)=i$ und $p_{ik}$ sei die Wahrscheinlichkeit, dass Liste $i$ genau $k$ Elemenete enthält. Dann ist:
            		$$
            			p_{ik}={n\choose k}\left(\frac{1}{m}\right)^k\left(1-\frac{1}{m}\right)^{n-k}
            		$$
            		Die mittleren Kosten sind:
            		\begin{align*}
            			E(\text{Listenlänge}) 	&= \sum\limits_{k=0}^n p_{ik}(1+k)\\
            								&= \sum\limits_{k=0}^n p_{ik} 
            									+\sum\limits_{k=0}^n k{n\choose k}\left(\frac{1}{m}\right)^k\left(1-\frac{1}{m}\right)^{n-k}\\
            			\left[\text{da } k{n\choose k}=n{n-1\choose k-1}\right]
            								&= 1+\frac{n}{m}\sum\limits_{k=1}^n {n-1\choose k-1}
            										\left(\frac{1}{m}\right)^{k-1}\left(1-\frac{1}{m}\right)^{n-k}\\
            								&=1+\frac{n}{m}\left(\frac{1}{m}+\left(1-\frac{1}{m}\right)\right)^{n-1}\\
            								&=1+\frac{n}{m} = 1+ \beta
            		\end{align*}
            	\end{proof}
            	
            
        \subsection{Größe von $\beta$}
            Für $\beta \leq 1$ ist $\LO(1)$ erwartet.\\
            Für $\beta \geq \frac{1}{4}$ ist Platzausnutzung für Tafel gut.\\
            Wobei folgende Laufzeiten gelten: Für die Tafel gilt $\LO(m)$, für die Listen gilt $\LO(n)$, 
            also insgesamt $\LO(n+m) \sim \LO(n)$ für $m = \frac{n}{\beta}, \beta \geq \frac{1}{4}$. 
            Jedoch können Einfügen, Striechen $\beta$ schnell schlecht werden lassen. Wird also $\beta$ zu groß oder zu klein, dann rehashen wir.
            
            \subsubsection{Rehashen}
                Benutze Folge von Hashtafeln $T_0,T_1,...$ der Größe $m,2m,4m,...$\\
                In $T_i$ der Größe $2^i \cdot m$ sei nun $\beta=1$.
                Dann wird umgespeichert auf $T_{i+1}$, was zu $\beta = \frac{1}{2}$ führt. 
                Bei $\beta = \frac{1}{4}$ gehe von $T_i$ zu $T_{i-1}$ was zu $\beta = \frac{1}{2}$ führt.
                Mit amort. Analyse kann man nun zeigen, dass mittlere Laufzeit immernoch $\LO(n)$ ist. Auch mit Rehashing.
                
                
        \subsection{Länge der längsten Liste}  
            Sei $S$ zufällig aus $U$ gewählt. Also gilt
            $$
                \text{prob}(h(x_k)=i) = \frac{1}{m}
            $$            
            für $k \in S, i = [0,1,...,m-1]$. $L$ sei die Länge der längsten Liste und $l(i)$ sei die Länge der Liste $i$. Es gilt nun
            $$
                \text{prob}(l(i) \geq j) \leq \binom{n}{j} \cdot \left( \frac{1}{m} \right)^j
            $$
            und es gilt weiter
            \begin{align*}
                \text{prob}(\max\{ l(i) \geq j\}) &\leq \sum_{i=0}^{m}{\text{prob}(l(i) \geq j)}   \\
                &\leq m \cdot \binom{n}{j} \cdot \left( \frac{1}{m} \right)^j \\
                &=m \cdot \frac{n!}{j!(n-j)!m^j} \\
                &\leq n \cdot \left( \frac{n}{m} \right)^{j-1} \cdot \frac{1}{j!}        
            \end{align*}
            Also ist der Erwartungswert
            \begin{align*}
                E(L) &= \sum_{j\geq 1} \text{prob}( \max \{l(i) \geq j\} \\
                    &\leq \sum_{j \geq 1} \min \left(1, n \cdot \left( \frac{n}{m} \right)^{j-1} \cdot \frac{1}{j!}\right)
            \end{align*}
            Sei 
            $$
                j_0 = \min \left\{ j , n \cdot \left( \frac{n}{m} \right)^{j-1} \cdot \frac{1}{j!}  \right\} \leq \min \left\{ j , n \leq j! \right\}
            $$
            da $\frac{n}{m} \leq 1$. Und mit $j! \geq \left( \frac{j}{2} \right)^{\frac{j}{2}}$ folgt nun
            $$
                j_0 = \LO \left( \frac{\log n}{\log \log n} \right)
            $$
            Und damit folgt nun
            $$
                E(L) = \sum_{j=1}^{j_0} 1 + \sum_{j > j_0}{\frac{1}{j_0^{j-j_0}}} = \LO \left( \frac{\log n}{\log \log n} \right)
            $$
            Im Mittel sind also Längen $\LO(1)$ aber es gibt auch Listen der Länge $\LO \left( \frac{\log n}{\log \log n} \right)$
            
                    
        
    \section{Hashing mit offener Adressierung}
    \label{sec:offeneAdressierung}
        Die Idee ist, dass nur ein Element pro Tafeleintrag gespeichert wird.
        Es wird eine Folge von Hashfunktionen $h_1, h_2, ...$ benutzt. Ist der Eintrag $T[h_1(x)]$ belegt, probiere Funktion $h_2$.
        
        \subsection{Beispiel}
            Es sei 
            $$
                h_i(x) = (h(x) + i) \mod m
            $$ 
            Damit wird versucht Elemente in benachbarte Felder zu schieben, so lange bis man ein freies Feld findet. (Linear probing)
            
            Alternativ sei
            $$
                h_i(x) = (h(x) + c_1\cdot i + c_2 \cdot i^2) \mod m
            $$
            (quadratic probing). Hier wird weiter gesprungen; man verhindert so, in einem ''Block'' stecken zu bleiben.
            
            
    \section{Perfektes Hashing}
        \subsection{Idee}
            Die Hashfunktion sollte injektiv sein, dann entstehen keine Kollisionen
            
        \subsection{Theorie}
            S sei fest. Dann geht man in 2 Stufen vor
            \begin{enumerate}
                \item Stufe: Hashing mit Verkettung. Ergibt Listen wie in vorherigem Kapitel beschrieben
                \item Stufe: für jede Liste einzelne eigene injektive Hashfunktion. Jede Liste wird also zu einer Hashtafel.
            \end{enumerate}
            
            \subsubsection{Wahl einer injektiven Hashfunktion}
                Sei $U= \{ 0,...,N-1 \}$, $h_k = \{0,...,N-1 \} \rightarrow \{ 0,...,m-1 \}$ sowie $k \in \{1,...,N-1 \}$ mit
                $$
                    h_k(x) = ((k \cdot x) \mod N) \mod m
                $$
                Wähle $h_k$ injektiv, wobei Injektivität wie folgt gemessen wird
                $$
                    b_{ik} = \abs{\{ x \in S | h_k(x) = i \} } 
                $$
                für $1 \leq k \leq N-1, 0 \leq i \leq m-1$. Dann ist 
                $$
                    b_{ik}(b_{ik}-1) = \abs{ \{ \underbrace{(x,y)}_{\text{Paare in Konflikt}} \in S^2 | x \neq y, h_k(x) = h_k(y) = i \} }
                $$
                und die Zahl der Paare, die für $h_k$ insgesamt in Konflikt stehen ist gegeben durch
                $$
                    B_k = \sum_{i=0}^{m-1}{b_{ik}(b_{ik}-1)}
                $$
                Dann gilt
                $$
                    B_k < 2 \Leftrightarrow h_k|s \text{ injektiv}
                $$
                Als Abschätzung sei $\abs{S} = n)$ und $b_{ik} = \frac{n}{m}$ für alle i ($S$ gleichmäßig verteilt), dann folgt
                \begin{align*}
                    B_k = \sum_{i=0}^{m-1}{\frac{n}{m} \cdot \left(\frac{n}{m}-1 \right)} = n \cdot \left( \frac{n}{m}-1 \right)
                \end{align*}
                Falls dann gilt $m \geq n^2$, dann folgt $B_k < 2$
                
                \begin{lemma}
                    Mit vorherigen Voraussetzungen und $N$ Primzahl gilt
                    $$
                        \sum_{k=1}^{N-1}{\sum_{i=0}^{m-1}{b_{ik}(b_{ik}-1)}} \leq 2 \cdot \frac{n\cdot (n-1)}{m}\cdot (N-1)
                    $$
                \end{lemma}
                Beweis in Mehlkorn I
                \begin{bemerkung}
                    Die Durchschnittliche Anzahl der Konflikte ist $< \frac{2n(n-1)}{m}$. Also folgt, dass für $m > n(n-1)$ die durchschnittliche Anzahl kleiner als 2 ist. Es existiert also ein $k$, so dass die Funktion injektiv wird.
                \end{bemerkung}
                Nun folgt
                \begin{korollar}
                    Mit obigen Voraussetzungen gilt
                    \begin{enumerate}
                        \item $\exists k \in \{1,...,N-1 \} : B_k \leq 2\cdot \frac{n(n-1)}{m}$
                        \item Sei $A= \{k | B_k > \frac{4(n(n-1))}{m} \}$ und sei $\abs{A} > \frac{N-1}{2}$, dann gilt
                            $$
                                \sum_{k}{B_k} \geq \sum_{A}{B_k} \geq \frac{N-1}{2} \cdot \frac{4(n(n-1))}{m} = 2\cdot \frac{n(n-1)}{m} \cdot (N-1)
                            $$
                            Dies ist ein Widerspruch. Also sind mindestens $\frac{N-1}{2}$ aller $h_k$ haben $B_k \leq \frac{4(n(n-1))}{m}$
                    \end{enumerate}
                \end{korollar}
                Es folgt ebenfalls
                \begin{korollar}
                    Mit obigen Voraussetzungen gilt
                    \begin{enumerate}
                        \item Ein $k \in \{1,...,N-1 \}$ mit $B_k \leq  2\cdot \frac{n(n-1)}{m}$ kann in Zeit $\LO(m+N\cdot n)$ berechnet werden. Teste dafür für alle $k$ wie viele Konflikte es gibt. Das geht jeweils in $\LO(n)$. Der Aufbau des Feldes kostet $\LO(m)$
                        \item Sei $m = n(n-1)+1$. Dann gibt es ein k mit $h_k|S$ injektiv. Wobei $k$ in Zeit $\LO(n^2+N\cdot n)$ gefunden werden kann.
                        \item Sei $m= 2n(n-1)+1$. Dann ist die Hälfte der $h_k|S$ injektiv. Ein injektives $h_k|S$ kann randomisiert in Zeit $\LO(n^2)$ gefunden werden.
                        \item Sei $m=n$. Ein $k$ mit $B_k \leq 2(n-1)$ kann in Zeit $\LO(n\cdot N)$ gefunden werden.
                        \item Sei $m=n$. Randomisiert kann ein $k$ mit $B_k \leq 4(n-1)$ in Zeit $\LO(n)$ gefunden werden.
                    \end{enumerate}
                \end{korollar}
                
            \subsection{Realisierung}
                \subsubsection{1. Stufe}
                    Bei $m=n$ gibt es $h_k$, so dass Länge der Listen $\LO(\sqrt{n})$ ist, da $B_k \leq \LO(n)$
                    
                \subsubsection{2. Stufe}
                    Wende Hashing auf jede Liste getrennt an. Sei $\abs{S} = n = m$. Dann
                    \begin{enumerate}
                        \item Wähle $k$ mit $B_k \leq 4(n-1) < 4n$. Sei 
                            $$
                                h_k(x) = (kx \mod N) \mod m
                            $$
                        \item Sei $w_i = \{ x \in S | h_k(x) = i \}$, $b_i = \abs{w_i}$ und sei $m_i = 2 \cdot b_i \cdot (b_i -1) +1 $. Wähle $k_i$ mit 
                            $$
                                h_{k_i} = (k_ix \mod N) \mod m_i
                            $$
                            so dass $h_{k_i}|w_i$ injektiv.
                        \item Sei $s_i = \sum_{j<i}{m_j}$. Speichere $x$ in $T[s_i + j]$ wobei 
                            $$
                                i = (kx \mod N) \mod m, j = (k_ix \mod N) \mod m
                            $$
                    \end{enumerate}
                    \subsubsection{Beispiel}
                    	$S = \{ 5, 6, 10, 12, 17, 21, 23, 42, 59\},\ m = |S|,\ k = 42$\\\\
					\begin{math}
						m=10:
						\begin{array}{c|c|c|c|c|c|c|c|c|c}
							0 & 1 & 2 & 3 & 4 & 5 & 6 & 7 & 8 & 9\\\hline
							21&   & 1 & 5 &   & 6 &   & 23&   &10\\
							  &   &   &12 &   &59 &   &   &   &17\\
							  &   &   &42 &   &   &   &   &   &
						\end{array}
						\Rightarrow \text{ Kollision in 3,5 und 9}
					\end{math}\\
					$b_3=3 \Rightarrow m_3=2\cdot 3\cdot 2 +1 = 13 \Rightarrow h_{k_3}= (k_3 x \mod 127)\mod 13$\\
					$b_5=2 \Rightarrow m_5=2\cdot 2\cdot1 +1 =5 \Rightarrow h_{k_5}= (k_5 x \mod 127)\mod 5$\\
					$b_9=2 \Rightarrow m_9=5 \Rightarrow h_{k_9}= (k_9 x \mod 127)\mod 5$\\\\
					Wähle $k_3=k_5=k_9=1$, da dann alle Abb. injektiv:\\\\
					\begin{math}
						\begin{array}{c|c|c|c|c|c|c|c|c|c}
							0 & 1 & 2 & 3 & 4 & 5 & 6 & 7 & 8 & 9\\\hline
							21&   & 1 & 
							\begin{array}{|c|c|c|}
								3 & 5 & 12\\\hline
								42& 5 & 12
							\end{array}
									 &   & 
									\begin{array}{|c|c|}
										 1&4\\\hline
										 6&59				 
									\end{array}&   & 23&   &
									\begin{array}{|c|c|}
										0&2\\\hline
										10&17				
									\end{array}
						\end{array}
					\end{math}
                    
                \subsubsection{Platzbedarf}
                    Es gilt 
                    $$
                        m = \sum_{i=0}^{n-1}{m_i} = \sum_{i=0}^{n-1}{2b_i(b_i-1)+1}= n + 2 \underbrace{B_k}_{\leq 4n} \leq 9n = \LO(n)
                    $$
                    
                \subsubsection{Laufzeit}
                    \begin{enumerate}
                        \item $\LO(n)$
                        \item $3\sum_i{\LO(w_i)} = \LO(n)$
                    \end{enumerate}
                    
                \subsubsection{Zusammenfassung}
                    \begin{satz}
                        Mit obigen Voraussetzungen kann in linearer Zeit/Platz eine perfekte Hashtafel gebaut werden. Wobei die Hashfunktionen eine Zugriffszeit von $\LO(1)$ haben. (randomisiert)
                    \end{satz}
                    
            \subsection{Dynamischer Fall}
                Starte mit leerer Hashtafel und füge $xs_1,xs_2,...$ ein. Bei i-ter Iteration liegen $xs_1,xs_2,...,xs_i$ vor. Es können folgende Probleme auftreten
                \begin{itemize}
                    \item Auf Stufe 1 zu viele Konflikte. Daraus folgt, dass $B_k$ zu groß wird.
                    \item Auf Stufe 2 kann es passieren, dass die Hahsfunktion nicht mehr injektiv ist oder die Tafeln zu klein sind.
                \end{itemize}
                Der Tafelaufbau ist in \autoref{alg:perfect-hashing-init} beschrieben.
                \begin{algorithm}
                    \capstart
                    \caption{Aufbau von Hashtafel(n) mit $n$ Werten aus $xs$ für gegebene $N, h_k, b_{i,k}$}
                    \label{alg:perfect-hashing-init}
                    \begin{algorithmic}
                        \State $B \gets 0$
                        \State $w \gets \{w_0, \dots, w_{n-1}\}$
                        \State $k \gets \{k_0, \dots, k_{N-1}\}$
                        \State $xs \gets \{xs_0, \dots, xs_{n-1}\}$
                        \Function{tafelaufbau}{}
                            \For{$i \gets 0 \malgto n - 1$}
                                \State $T_i \gets $ \Call{newTable}{{}}
                                \State $k_i \gets $ \Call{random}{$0, N-1$}
                            \EndFor
                            \State $m \gets n$
                            \For{$i \gets 0 \malgto n - 1$}
                                \State \Call{insert}{$xs_i, k_i, n, N$}
                            \EndFor
                        \EndFunction
                        \Statex
                        \Function{insert}{$x, n$}
                            \State $i \gets (k_x \bmod N) \bmod n$
                            \State $j \gets (k_i\cdot x \bmod N) \bmod m_i$ % von h_k: {1, ..., N-1} -> {1, ..., ,-1}
                            \State $w_i \gets w_i \cup \{x\}$
                            \State $B \gets B + 2 b_i$
                            \State $b_i \gets b_i + 1$
                            \If{$B > 4(n-1)$}
                                 \State \Call{tafelaufbau}{{}}
                            \Else
                                \If{\Call{free}{$T[s_i + j]$}}
                                    \State \Call{hashInsert}{$T[s_i + j], x$}
                                \Else \Comment{$h_k$ nicht injektiv}
                                    \If{$m_i < 2b_i \cdot (b_i - 1) + 1$}
                                        \State $\var{repeat} \gets \True$
                                        \While{repeat}
                                            \State \Call{enlargeArea}{$w_i$}
                                            \State finde neue Funktion für $w_i$ % ??
                                            \State $k_i \gets $ \Call{random}{$0, N-1$}
                                            \State \Call{deleteArea}{$T[s_i, s_i + m_i]$}
                                            \For{$x \in w_i$}
                                                \State $j \gets (k_i \cdot x \bmod N) \bmod m_i$
                                                \If{\Call{free}{$T[s_i + j]$}}
                                                    \State \Call{hashInsert}{$T[s_i + j], x$}
                                                    \State $\var{repeat} \gets \False$
                                                \Else
                                                    \State $\var{repeat} \gets \True$
                                                    \State \textbf{break}
                                                \EndIf
                                            \EndFor
                                        \EndWhile
                                    \EndIf
                                \EndIf
                            \EndIf
                        \EndFunction
                        \Statex
                        \Function{enlargeArea}{$w_i$}
                            \State $m_i \gets 4b_i(b_i-1)+1$
                            \State $s_i \gets m$
                            \State $m \gets m + m_i$
                            \State $T[s_i,s_i + m_i]$ ist reserviert für $w_i$
                        \EndFunction
                    \end{algorithmic}
                \end{algorithm}

                    % \begin{verbatim}
                    %     Tafelaufbau()
                    %     for i <- 0 to n-1 do
                    %         Initialisierung der Tafeln und wähle k zufällig
                    %     od
                    %     m <- n, B <- 0
                    %     for l = 0 to n - 1 do
                    %         Einfügen(x_l)
                    %     od
                    % \end{verbatim}
                    % Wobei
                    % \begin{verbatim}
                    %     Einfügen(x)
                    %     i <- (kx mod N) mod n, j <-  (k_i * x mod N) mod m_i
                    %     w_i <- w_i UNION {x}, B <- B + 2 b_i, b_i <- b_i +1
                        
                    %     if B > 4(n-1) 
                    %     then Tafelaufbau()
                    %     else    if T[s_i + j] frei
                    %             then speichere x in T[s_i + j]
                    %             else (h_k hier nicht injektiv)
                    %                 if m_i < 2b_i(b_i-1)+1 
                    %                 then vergrößere Bereich für w_i  
                    %                      versuche nun wieder eine neue Funktion für w_i zu finden
                    %                      Wähle k_i zufällig
                    %                      Lösche Bereich T[s_i, s_i + m_i]
                    %                      for all x \in w_i do
                    %                          j <- ((k_i x) mod N) mod m_i
                    %                          if T[s_i + j] frei
                    %                          then speichere x dort
                    %                          else repeat
                    %                      od
                    % \end{verbatim}
                    % und
                    % \begin{verbatim}
                    %     Vergrößere Bereich:
                    %     m_i <- 4b_i(b_i-1)+1
                    %     s_i <- m
                    %     m <- m + m_i
                    %     T[s_i,s_i + m_i] ist reserviert für w_i
                    % \end{verbatim}

                    
                    
                \subsubsection{Laufzeit}        
                    Tafelaufbau passiert wenn $B < 4(n-1)$. Für zufälliges $k$ gilt nach Lemma
                    $$
                        B_k = \sum{b_{ik}(b_{ik}-1)} \leq 4(n-1)
                    $$
                    mit Wahrscheinlichkeit $k \geq 0.5$. Um ein geeignetes k zu finden brauchen wir also $\leq 2$ Versuche im Mittel. Damit ergibt sich insgesamt eine Laufzeit von
                    $$
                        \LO(n) + \LO(\sum{(b_{ik})^2}) = \LO(n) + \LO(B_k) = \LO(n)
                    $$
                    
                \subsubsection{Platzbedarf}
                    $m_i$ wächst immer. Werte sind $m_{i_1},m_{i_2},...$. Dabei gilt $m_{i_{p+1}} \geq 2 \cdot m_{i_p} + 1$. Nun folgt
                    $$
                        \sum_{p \geq 1}{m_{i_p}} \leq \sum_{p \geq 0}{\frac{m_{ik}}{2^p}} \leq m_{ik}\cdot \left(1+\frac{1}{2}+\frac{1}{4}+...\right) \leq 2m_{ik}
                    $$
                    Das ergibt einen Gesamtplatz von
                    $$
                        \sum_{i}{2m_{ik}} \leq \sum_{i}{2(4b_{ik}(b_{ik}-1)+1)} \leq 8 \cdot 4(n-1) + 2n
 \leq 34n
                    $$
                    
                    \begin{satz}
                        Sei N Primzahl, $S \subseteq U$, $\abs{S}=n$, $\abs{U}=N$. Eine perfekte Hashtafel der Größe $\LO(n)$ kann online in $\LO(n)$ erwarteter Zeit berechnet werden mit Zugriffszeit $\LO(1)$
                    \end{satz}
                    
                    
                    
                    
            
            
        
    

% Copyright \copyright\ 2013  Simon Kalt, Jan-Peter Hohloch, Tobias Fabritz
% Es wird die Erlaubnis gegeben, dieses Dokument unter den Bedingungen der von der Free
% Software Foundation veröffentlichen  GNU Free
% Documentation License (Version 1.2 oder neuer) zu kopieren, verteilen und/oder
% zu verändern. Eine Kopie dieser Lizenz ist unter
% http://www.gnu.org/copyleft/fdl.txt erhältlich.
%
% Zusätzlich muss jede Kopie/Aktualisierung wieder über die Seite
% der Fachschaft Informatik der Uni Tübingen
% den Studenten zur Verfügung gestellt werden
% http://www.fsi.uni-tuebingen.de/

\chapter{Graphen}
    \section{Definition}
        Ein Graph sei definiert durch $G=(V,E)$ wobei $V$ (Vertices) eine endliche Menge von Knoten sei und $E \subseteq V \times V$ (Edges) eine endliche Menge von Kanten. Dabei heißt $e=(v,w) \in E$ Kante von $v$ nach $w$ und $w$ heißt Nachbarknoten von $v$ (adjazent). Ein \emph{Pfad} in $G$ ist eine Folge $(v_0,v_1,...,v_k)$ von Kanten mit $k \geq 0$ und $(v_i,v_{i+1} \in E \ \forall \ 0 \leq i \leq k-1$ (Pfad von $v_0$ nach $v_k$). Falls $v_0 = v_k$ und $k \geq 1$ heißt der Pfad Zykel. Falls $v_i \neq v_j$ für $i \neq j$ heißt der Pfad einfach. Wir schreiben
     
        $$
            v \overset{*}{\rightarrow_{G}} w 
        $$
        für einen Pfad von $v$ nach $w$. Der Graph heißt \emph{zyklisch} falls er Zykel enthält, sonst \emph{azyklisch}. $G$ heißt \emph{Baum}, falls
        \begin{enumerate}[a)]
            \item $V$ enthält genau ein $v_0$ mit $\text{inadj}(v_0) = 0$
            \item $\forall v \in V \backslash \{v_0\} : \text{inadj(v)}=1$
            \item G ist azyklisch
        \end{enumerate}
        
        \section{Darstellung von Graphen}            
            \subsection{Darstellung im Computer}
                Wir nehmen an, dass $V=\{1,2,...,n\}$. Dann ergeben sich die Darstellungen
                \begin{enumerate}
                    \item Darstellung: \emph{Adjazenzmatrix A}. Dabei gilt in der Matrix
                        $$
                            a_{i,j} \begin{cases}
                                        1 & \text{falls} (i,j) \in E \\
                                        0 & \text{sonst}
                                    \end{cases}
                        $$
                        wobei diese Matrix in einem ungerichteten Graphen symmetrisch ist. Dabei verbraucht die Matrix einen Platz von $\LO(n^2)$. Dies ist gut, falls $\abs{E} = m$ ungefähr so groß ist wie $n^2$.
                    \item Darstellung: \emph{Adjazenzlisten}. Dabei wird für jedes $v$ die Knotenmenge gespeichert: Dabei ist
                        $$
                            \text{Out: adj}(v) = \{ w \in V | (v,w) \in E \}
                        $$
                        und
                        $$
                            \text{Inadj}(v) = \{ w \in V | (w,v) \in E \}
                        $$
                        Für ungerichtete Graphen ergibt sich
                        $$
                            \text{Outadj}(v) = \{ w \in V | (v,w) \in E \}
                        $$     
                        Dabei haben wir einen Platzverbrauch von $\LO(n+m)$. Als Nachteil ergibt sich beim Zugrif auf Kante $(v,w)$ kostet $\LO(addj(v))$ wobei
                        $$
                            \text{addj}(v) = \abs{\{w \in V | (v,w) \in E \}}
                        $$        
                \end{enumerate}
                
            \subsubsection{Beispiel}
            %%%%%%%%%%%%%%%%%
            % Hier Skizze und Liste noch einfügen, evtl noch baum skizze
            %%%%%%%%%%%%%%%%%
                Sei $G = (V, E)$ mit $V= \{1,2,3,4,5\}$ und $E=\{(1,2),(1,3),(1,4),(4,5),(5,1),(3,5)\}$. Für $G$ ergibt sich die folgende Adjazenzmatrix:
                $$
                    \begin{array}{c|ccccc}
                    - & 1 & 2 & 3 & 4 & 5 \\ 
                    \hline
                    1 & 0 & 1 & 1 & 1 & 0 \\ 
                    2 & 0 & 0 & 0 & 0 & 0 \\ 
                    3 & 0 & 0 & 0 & 0 & 1 \\ 
                    4 & 0 & 0 & 0 & 0 & 1 \\ 
                    5 & 1 & 0 & 0 & 0 & 0
                    \end{array} 
                $$
                Der zugehörige Graph ist in \autoref{fig:directed-graph} dargestellt.
                \begin{figure}[htp]
                    \capstart
                    \centering
                    \begin{tikzpicture}[->,every node/.style={shape=circle, draw, minimum size=8mm},node distance=2cm]
    \node (1) {1};
    \node (2) [above of=1] {2};
    \node (3) [left of=1] {3};
    \node (4) [right of=1] {4};
    \node (5) [below  of=1] {5};

    \path (1) edge (2)
          (1) edge (3)
          (1) edge (4)
          (3) edge (5)
          (4) edge (5)
          (5) edge (1);
\end{tikzpicture}
                    \caption{Gerichteter Graph $G = (V, E)$}
                    \label{fig:directed-graph}
                \end{figure}      

            \section{Topologisches Sortieren}
                Sei $G=(V,E)$ ein gerichteter Graph (Digraph). Abbildung 
                $$
                    \text{num: } V \rightarrow \{1,2,...,n \}
                $$
                und $n= \abs{V}$ heißt topologische Sortierung falls für alle $(v,w) \in E$ gilt
                $$
                    \text{num}(v) < \text{num}(w)
                $$
                \begin{satz}
                    Die Abbildung \emph{num} existiert genau für azyklische Graphen
                \end{satz}
                
                \begin{proof}
                    \begin{itemize}
                        \item $\Rightarrow$: Annahme $G$ zyklisch. Sei $(V_0,...,v_k = v_0)$ ein Zykel. Es muss gelten: 
                        $$
                            \text{num}(v_0) < \text{num}(v_1) < ... < \text{num}(v_k) = \text{num}(v_0)
                        $$
                        Dies ist ein Widerspruch.
                        \item $\Leftarrow$: Sei $G$ azyklisch. Behauptung: $G$ enthält Knoten mit $\text{indeg} = 0$ (Anzahl eingehender Kanten)
                            \begin{proof}
                                Induktion über Größe von $V$. Für $\abs{V} = 1$ trivial. Für $\abs{V} > 1$: Entferne beliebigen Knoten $v \in V$. Dann erhalte ich damit $G' = (V,E')$ mit $V' = V \backslash \{v\}$ und $E' = E \cap (V' \times V')$. Nach Induktionsannahme ist $G'$ azyklisch und enthält Knoten $v'$ mit $\text{indeg}(v') = 0$. Entferne $v'$ aus $G'$ und erhalte $G''$. $G''$ enthält auch wieder ein $v''$ mit $\text{indeg}(v'')=0$. Lese den Beweis nochmal, wobei als $v$ nun $v''$ gewählt ist. Entweder ist $\text{indeg}(v')=0$ in $G$ oder $\text{indeg}(v'') = 0$ in $G$, denn es können nicht beide Knoten $(v',v'')$ und $(v'',v')$ in $E$ existieren wegen Zykelfreiheit.
                            \end{proof}
                    \end{itemize}
                \end{proof}
                
                \subsection{Algorithmus}
                    Beschreibung (\autoref{alg:topologischesSortieren}):
                    \begin{itemize}
                        \item $\abs{V}=1$ trivial
                        \item $\abs{V} > 1$: Wähle v mit $\text{indeg}(v)=0$, entferne dann $v$ und erstelle rekursiv $G'$
                            Sei $\text{num': } V' \rightarrow \{1,...., \abs{V'}\}$ Dann ist num für G:
                            $$
                                \text{num}(w) = \begin{cases}
                                                    \text{num}'(w) + 1 & \text{falls } w \neq v \\
                                                    1 & \text{sonst}
                                                \end{cases}
                            $$                          
                    \end{itemize}
                    
                    \begin{algorithm}
                    	\caption{Topologisches Sortieren}
                    	\label{alg:topologischesSortieren}
                    	\begin{algorithmic}
                    		\State count $\gets$ 0
                    		\While{$\exists v\in V:indeg(v)=0$}
                    			\State count $\gets$ count$ +1$
                    			\State num(v) $\gets$ count
                    			\State streiche v und ausgehende Kanten
                    		\EndWhile
                    		\If{count$ < |V|$}
                    			G zyklisch
                    		\EndIf
                    	\end{algorithmic}              
                    \end{algorithm}
                    
                    
%                    \begin{verbatim}
%                        count <- 0
%                        while exists v aus V mit indeg(v) = 0 do
%                            count++
%                            num(v) <- count
%                            streiche v und ausgeh. Kanten
%                        od
%                        if count < |V| 
%                        then G zyklisch
%                    \end{verbatim}


                    Dabei ergeben sich als Kosten $\LO(n+m)$ für die Zeit und den Platzverbrauch, wobei $\abs{V} = n, \abs{E}=m$. Dabei werden Adjazenzlisten benutzt, da sonst der Platz quadratisch wäre. Außerdem brauchen wir ein Array \emph{indeg}, welches die Zahl der eingehenden Kanten zählt.
                    (\autoref{fig:topSort})
                    
										
					\begin{figure}[htp]
						\capstart
						\centering
						
\begin{tikzpicture}[-, >=stealth', auto, node distance=1cm]
   %Outadjazenzliste:
	 \node[rectangle,draw] (out1) {1};
		 \node[circle,draw] (out12) [right of=out1] {2};
		 \node[circle,draw] (out13) [right of=out12] {3};
	 	 \node[circle,draw] (out14) [right of=out13] {4};
		 \node[circle,draw] (out17) [right of=out14] {7};
		
	 \node[rectangle,draw] (out2) [below of=out1] {2};
	     \node[circle,draw] (out21) [right of=out2] {6};
		\node[circle,draw] (out26) [right of=out21] {7};
			
	\node[rectangle,draw] (out3) [below of=out2] {3};
	    \node[circle,draw] (out34) [right of=out3] {4};
		
	 \node[rectangle,draw](out4)[below of=out3]{4};
	    \node[circle,draw](out45)[right of=out4]{5};
			
	 \node[rectangle,draw](out5)[below of=out4]{5};
		\node[circle,draw](out56)[right of=out5]{6};
		\node[circle,draw](out57)[right of=out56]{7};
    
	 \node[rectangle,draw](out6)[below of=out5]{6};
	
	 \node[rectangle,draw](out7)[below of=out6]{7};
	 
	 
	 \node (map) [right of=out17] {$\mapsto$};
%Indegree:
   
	 \node[rectangle,draw] (in1) [right of=map] {1};
	 	\node (ind1) [right of=in1] {0};
		
	 \node[rectangle,draw] (in2) [below of=in1] {2};
	 	\node (ind2) [right of=in2] {I};
			
	\node[rectangle,draw] (in3) [below of=in2] {3};
	 	\node (ind3) [right of=in3] {I};
		
	 \node[rectangle,draw](in4)[below of=in3]{4};
	 	\node (ind4) [right of=in4] {II};
			
	 \node[rectangle,draw](in5)[below of=in4]{5};
	 	\node (ind5) [right of=in5] {I};
    
	 \node[rectangle,draw](in6)[below of=in5]{6};
	 	\node (ind6) [right of=in6] {II};
	
	 \node[rectangle,draw](in7)[below of=in6]{7};
	 	\node (ind7) [right of=in7] {III};
	 	
	 	
	 %Text:
	 \node (Outadj) [above of=out1] {Outadj:};
	 \node (Indeg) [above of=in1] {Indeg:};
	
	\path[draw]
		 (out1) edge (out2)
		 (out2) edge (out3)
		 (out3) edge (out4)
		 (out4) edge (out5)
		 (out5) edge (out6)
		 (out6) edge (out7)
		
		 (out1) edge (out12)
		 (out12) edge (out13)
		 (out13) edge (out14)
		 (out14) edge (out17)
		
		 (out2) edge (out21)
		 (out21) edge (out26)
		
		 (out3) edge (out34)
		
		 (out4) edge (out45)
		
		 (out5) edge (out56)
		 (out56) edge (out57);
	
	\path[draw]
		 (in1) edge (in2)
		 (in2) edge (in3)
		 (in3) edge (in4)
		 (in4) edge (in5)
		 (in5) edge (in6)
		 (in6) edge (in7);
		 
\end{tikzpicture}

						\label{fig:topSort}
						\caption{Beispiel zum topologischen Sortieren}
					\end{figure}
                    
                    
            \section{Billigste Wege}
                Gegeben sei ein Netzwerk $(V,E,c)$ mit $G=(V,E)$ und $c: E \rightarrow \R$
                
								\begin{figure}
									\centering
									\begin{tikzpicture}[->,node distance=3cm]
  \node[state] (1) {1};
	\node[state] (2) [above right of=1] {2};
	\node[state] (3) [right of=1] {3};
	\node[state] (4) [right of=2] {4};
	\node[state] (5) [right of=3] {5};
	
	\path
			(1) edge node[above left]{2} (2)
			(1) edge[bend right] node[below]{3} (3)
			
			(2) edge node[right]{-3} (3)
			(2) edge node[above]{-1} (4)
			
			(3) edge[bend right] node[above]{-1} (1)
			(3) edge node[below]{1} (5)
			(3) edge node[right]{2} (4)
			
			(4) edge node[right]{3} (5);
\end{tikzpicture}
									\caption{Graph mit Pfadkosten (negativer Zykel)}
								\end{figure}
                Ziel ist es den Pfad mit den geringsten Kosten zwischen zwei gegebenen Knoten zu finden. Dabei sind die Kosten eines Pfades $p= v_0,...,v_k$  gegeben durch 
                $$
                    c(p) = \sum_{i=0}^{k-1}{c((v_i,v_{i+1}))}
                $$
                Dabei unterscheiden wir folgende Probleme
                \begin{enumerate}
                    \item Single source - single sink shortest path
                    \item single source shortest path
                    \item all pairs shortest path
                \end{enumerate}
                
           \subsection{Single source shortest path}
                    Für $u \in V$ sei $P(s,u)$ die Menge aller Pfade von $s$ nach $u$. Wir definieren als Kosten
                    $$
                        \delta(u) = \begin{cases}
                                            \infty & \text{falls } P(s,u) = \emptyset \\
                                            \inf\{c(p) | p \in P(s,u)\} & \text{sonst}
                                        \end{cases}
                    $$ 
                    Ein negativer Zykel $p$ hat dabei $c(p) < 0$.
                    \begin{lemma}
                        Sei $u \in V$ Dann
                        \begin{enumerate}[i)]
                            \item $\delta(u) = -\infty$ gdw. u erreichbar über negativen Zykel, der von $s$ aus erreichbar ist.
                            \item $\delta(v) \in \R$: Es existiert ein billigster Weg von $s$ nach $u$ mit Kosten $\delta(u)$
                        \end{enumerate}
                    \end{lemma}
                    \begin{proof}
                        \begin{enumerate}[i)]
                            \item   \begin{itemize}
                                        \item $\Rightarrow$: trivial
                                        \item $\Leftarrow$: Sei $C= \sum_{i \in E}{\abs{c(e)}}$ und sei $p$ ein billigster Weg von $s$ nach $n$ mit $c(p) < -C$. Es folgt, dass $p$ einen negativen Zykel enthalten muss.
                                    \end{itemize}   
                            \item $\delta(u) < \infty$, d.h. es gibt Pfad von $s$ nach $u$. Es wird nun Behauptet, dass 
                            $$
                                \delta(u) = \min\{c(p) | p \in P(s,u) \text{ und p einfach}\}
                            $$
                                \begin{proof}
                                    Sei $p'$ ein billigster, einfacher Weg von $s$ nach $u$. Ist die Behauptung falsch, so gibt es einen nicht einfachen Weg $q$ mit $c(q) < c(p')$. Da es keine negativen Zykel gibt gilt: Durch Wegnahme des Zykels aus $q$ entsteht ein billigerer Weg $q'$ mit $c(q') \leq c(q) < c(p')$. Dies ist ein Widerspruch.
                                \end{proof}                                                             
                        \end{enumerate}
                    \end{proof}
                    
                \paragraph{Beispiel}
								    %TODO : sinnvolle Aufteilung der Fälle
                    \emph{1. Fall}: $G$ ist azyklisch. Wir nehmen an, dass $G$ topologisch sortiert ist.  Dann ergibt sich als Algorithmus
                    
                    \begin{algorithm}
                    	\caption{Single Source Shortest Path --- Graph azyklisch}
                    	\label{alg:sssp1}
                    	\begin{algorithmic}
                    		\State d(s) $\gets$ 0, Pfad(s) $\gets$ s
                    		\State d(v) $\gets \infty \forall v\in V\backslash \set{s}$
                    		\For{$i\gets s+1$ to $n$}
                    			\State d(v) $\gets \min_u\set{d(n)+c((u,v)) | (u,v)\in E}$
                    			\State Pfad(v) $\gets$ Pfad(u*) + v \Comment Konkatenation
                    		\EndFor
                    	\end{algorithmic}
                    \end{algorithm}
                    
%                    \begin{verbatim}
%d(s) <- 0, Pfad(s) <- s
%d(v) <- infty für alle v \in V \ {s}
%for i <- s+1 to n do
%    d(v) <- min_u{d(n) + c((u,v)) | (u,v) \in E }
%    Pfad(v) <- Pfad(u*) + v // konkateniere
%od
%                    \end{verbatim}

                % Vorlesung 18. Juni
            \subsection{Dijkstra's Algorithmus}
						%TODO : s.o.
						\emph{2. Fall}: Kanten in G haben nur positive Kosten, Zykel erlaubt\\
            Gegeben ein Graph $G = (V, E)$ und eine Kostenfunktion $c: E \to \R^+$.
            Weiterhin zwei Mengen $S, S'$ mit Source $s \in S$. Hierbei ist $S$ die Menge der Knoten $u \in V$ mit bekanntem $\delta(u)$. $S'$ ist die Menge der Knoten aus $V \setminus S$, die Nachbarn in $S$ haben.

            \begin{algorithm}
            	\capstart
            	\caption{Dijkstra's Algorithmus}
            	\begin{algorithmic}
            		\State $S\gets \{s\}$
            		\State $d(s)\gets 0$
            		\State $S'\gets$ \Call{Out}{S} \Comment Knoten welche von S aus direkt erreichbar sind
            		\State $d'(u)\gets c((s,u)) \forall u\in S'$
            		\State $d'(u)\gets \infty \forall u\in V\backslash \left(S\cup S'\right)$
            		\While{$S\not= V$}
            			\State wähle $w\in S'$ mit geringstem $d'(w)$
            			\State $d(w)\gets d'(w)$
            			\State $S\gets S\cup \{w\}$
            			\ForAll{$u\in$ \Call{OUT}{w}}
            				\If{$u\not\in S$}
            					\State $S'\gets S'\cup \{u\}$\Comment falls bereits zuvor $u\in S'$ kein Unterschied
            					\State $d'(u)\gets \min\{d'(u),d(w)+c(w,u)\}$
            				\EndIf
            			\EndFor
            		\EndWhile
            	\end{algorithmic}
            	\label{alg:Dijkstras}
            \end{algorithm}
            
%            \begin{verbatim}
%S <- {s}, d(s) <- 0
%S' <- Out(s) // (Knoten welche von S aus direkt erreichbar sind)
%d'(u) < c((s,u)) für alle u \in S'
%d'(u) <- \infty für alle u \in V\{S \cup S'}
%while(S \neq V) do
%    wähle geeignetes w \in S
%    d(w) <- d'(w)
%    S <- S \cup {w}
%    S' <- S' \ {w}
%    forall u \in Out(w) do
%        if u \notin S then
%            S' <- S' \cup {u}
%            d'(u) <- min{d'(u),d(w) + c(w,u)}
%        fi
%    od
%od
%\end{verbatim}


            \begin{lemma}
                Sei $w \in S'$, sodass $d'(w)$ minimal, dann ist $d'(w) = \delta(u)$
                \begin{proof}
                    Sei $p$ billigster Weg von $s$ nach $w$ mit allen Knoten (bis auf $w$) in S.
                    Annahme: es gibt billigeren Weg $q$ von $s$ nach $w$. $q$ muss einen ersten Knoten $v$ in $V \setminus S$ haben.
                    Nach Wahl von $w$ ist $d'(v) > d'(w)$.
                    Da alle Kanten nicht negativ, gilt $c(q) \ge d'(v) \ge d'(u) = c(p)$.
                    Folglich ist $c(q) \ge c(p)$, wodurch die Annahme widerlegt ist.
                \end{proof}
            \end{lemma}

            Es reicht also, sich bei der Wahl von $w$ auf $S'$ zu beschränken.

            \paragraph{Laufzeit}
            Sei $n = |V|$, $m = |E|$. Implementieren $S, S'$ als Bitvektor, $d, d'$ als Arrays.
            \begin{itemize}
                \item Schleife über alle $u \in \textsc{out}(w)$ in $o(|\textsc{out}(w)|)$. Insgesamt also:
                    $$
                    \sum_{w} |\textsc{out}(w)| = \LO(n + m)
                    $$
                \item $n$-mal Wahl des Minimums aus $S'$: $\LO(n^2)$
                \item Gesamtlaufzeit: $\LO(n^2 + m)$
            \end{itemize}

            Alternativ: Speichere $S'$ mit $d'$ Werten als Heap, geordnet nach $d'$. Hiermit ergibt sich ein Laufzeit von $\LO((n + m) \log n)$. Gut für dimere Graphen. Mit einem Fibonacci-Heap ist eine Laufzeit von $\LO(n \log n + m)$ möglich.

        %TODO : s.o.
        \emph{3.Fall}: Negative Kanten, aber keine negativen Zykel\\
        \subsection{Bellman-Ford Algorithmus}
        Neues Szenario: Erlaube negative Kanten (keine negativen Zykel).
        \begin{algorithmic}
            \Function{relax}{$v, w$}
                \State $d(w) \gets \min(d(w), d(v) + c((v, w)))$
            \EndFunction
        \end{algorithmic}

        Beobachtung: Relax-Operation erhöht keine $d$-Werte.

        Algorithmus:
        Sei 
        $$
        d(v) = \begin{cases} 0 & \text{für } v = s \\ \infty & \text{sonst} \end{cases}
        $$

        Iteriere Relax-Operation.
        Idee: $d$-Werte werden immer kleiner, aber höchstens bis $\delta$-Wert
        Frage: Reihenfolge der Relax
        \begin{lemma}
            Sei $w \in V$ und $\delta(w) < \infty$ und sei $(v, w)$ die letzte Kante
            auf billigstem Weg zu $w$. Ist $d(v) = \delta(v)$ und wird $\textsc{relax}(v,w)$
            durchgeführt, so ist danach auch $d(w) = \delta(w)$.
        \end{lemma}

Algorithmus
\begin{verbatim}
    for i <- 1 to n - 1 do
        forall $(v, w) \in E$ do
            relax(v, w)
        od
    od
\end{verbatim}

        \begin{lemma}
            Für $i=0,...,n-1$ gilt: Nach Phase i ist:\\
            $d(w)=\delta (w)$ für alle $w\in V$, für die es einen billigsten Pfad der Länge i von s nach w gibt.
        \end{lemma}
        \begin{proof}[Beweis durch vollständige Induktion]
            \begin{enumerate}
                \item $i = 0$: $d(s) = \delta(s)$
                \item $i \to i + 1$:
                Sei $w$ Knoten mit billigstem Weg der Länge $i+1$ von $s$ nach $w$. Dessen letzte Kanten sei $(v, w)$. Also gibt es einen
                billigsten Weg der Länge $i$ von $s$ nach $v$und nach Ind. Annahme ist nach Phase $i$ $\delta(v) = d(v)$. In Phase $i+1$ wird $\textsc{relax}(v, w)$ aufgerufen und $d(w) \gets \delta(v) + c(v, w) = \delta(v)$.
                $\implies$ Nach Phase $n - 1$ ist $d(v) = \delta(v)$ für alle $v$.
                $\implies$ Laufzeit: $\LO(n \cdot m)$.
            \end{enumerate}
        \end{proof}
				
				%TODO : s.o.
				\emph{4. Fall}: negative Kosten und negative Zykel erlaubt\\
				%TODO : fill with sense :D
				\begin{itemize}
					\item[1.] $n-1$ Phasen nach Bellman-Ford, alte d-Werte
					\item[2.] nochmal $n-1$ Phasen ausführen, neue d-Werte
					\item[3.] vergleiche alte mit neuen d-Werten
				\end{itemize}

        \subsection{All pairs shortest paths}
            Annahme: Keine negativen Zykel. Sei $V = \set{1, \dots, n}$
            Für $i, j \in V$ definiere:
            $$
                \delta_k(i, j) = \text{Kosten des billigsten Weges von $i$ nach $j$ dessen innere Knoten $\le k$ sind}
            $$

            Für den Graphen in \autoref{fig:all-pairs-shortest-paths} ergeben sich folgende Werte für $\delta$:
            \begin{align*}
                \delta_0(1,4) &= \infty \\
                \delta_1(1,4) &= \infty \\
                \delta_2(1,4) &= 4 + 2 = 6 \\
                \delta_3(1,4) &= -3 + 2 + 2 = 1  \\
                \delta_4(1,4) &= 1 \\
            \end{align*}

            \begin{figure}[htp]
                \centering
                \capstart
                \begin{tikzpicture}[->, label/.style={draw=none}, every node/.style={shape=circle, draw, minimum size=8mm},node distance=2cm]
    \node (1) {1};
    \node (2) [above right of=1] {2};
    \node (3) [below right of=1] {3};
    \node (4) [below right of=2] {4};

    \path (1) edge node[above left, label] {4} (2) % 4
          (2) edge node[above right, label] {2} (4) % 2
          (1) edge node[below left, label] {-3} (3) % -3
          (3) edge node[left, label] {2} (2) % 2
          (3) edge node[below right, label] {5} (4); % 5
\end{tikzpicture}
                \caption{Graph mit Kantenkosten/gewichteter Graph}
                \label{fig:all-pairs-shortest-paths}
            \end{figure}

            \begin{align*}
                \delta_0(i,j) &= \begin{cases}
                    c(i, j) & \text{falls } (i, j) \in E \\
                    0 & \text{falls } i = j \\
                    \infty & \text{sonst}
                \end{cases} \\
                \delta_n(i,j) &= \delta(i,j) = \text{ Kosten des kürzesten Wehes von $i$ nach $j$}
            \end{align*}

            Frage: Wie berechnet sich $\delta_k$ aus $\delta_{k-1}$? Es gibt zwei Möglichkeiten:
            \begin{enumerate}
                \item Es kann kein neuer Knoten verwendet werden, dann: $\delta_k(i,j) = \delta_{k-1}(i, j)$
                \item Es kann ein neuer Knoten verwendet werden, dann: $\delta_k(i,j) = \delta_{k-1}(i, k) + \delta_{k-1}(k, j)$
            \end{enumerate}
            \begin{tikzpicture}[->,>=stealth',node distance=2cm, graph node/.style={circle,draw}]
            	\node[graph node] (i) {i};
            	\node[graph node] (k) [above right of=i] {k};
            	\node[graph node] (j) [below right of=k] {j};
            	
            	\path[draw, dashed]
            		(i) edge node[above left]{$\delta_{k-1} (i,k)$} (k)
            		(k) edge node[above right]{$\delta_{k-1} (k,j)$} (j);
            		
            	\path[draw]
            		(i) edge node[below]{$\delta_{k-1} (i,j)$} (j);
            \end{tikzpicture}\\
            Also:
            $$
                \delta_k(i, j) = \min\set{\delta_{k-1}(i,j), \delta_{k-1}(i, k) + \delta_{k-1}(k, j)}
            $$

            Algorithmus:
            \begin{algorithmic}
            	\For{$k\gets 1$ to $n$}
            		\ForAll{$i,j\in V$}
            			\State $\delta_k(i,j) = \min\set{\delta_{k-1}(i,j), \delta_{k-1}(i, k) + \delta_{k-1}(k, j)}$
            		\EndFor
            	\EndFor
            \end{algorithmic}
%\begin{verbatim}
%    Berechne \delta_0 nach Definition
%    for k <- 1 to n do
%        forall $i, j \in V$
%            \delta_k(i,j) = \min\set{\delta_{k-1}(i,j), \delta_{k-1}(i, k) + \delta_{k-1}(k, j)}
%    od
%\end{verbatim}


            \paragraph{Alternativ}
            $n$ mal Bellman-Ford für jeden Knoten als source $s$ durchführen.
            Laufzeit: $\LO(n \cdot n m)$.
            Besser: Übertrage Kantengewichte auf nicht negative Werte. Dann $n$ mal Dijkstra anwenden.
            Dies ergibt eine Laufzeit von $\LO(n \log n + m)$. Zur Neugewichtung der Kanten soll folgendes gelten:
            \begin{lemma}
                $\forall u, v \in V$ sei $p$ der billigste Pfad $u \to v$ mit Kantengewicht $c \in \R$.
                $\iff \forall u, v \in V$ sei $p$ der billigste Pfad $u \to v$ mit Kantengewicht  $c' \ge c$. Wähle $c'(u, v) + h(u) - h(v)$ für Funktion $h: V \to \R$.
                Dann gilt für Pfad $p = (v_0, \dots, v_k)$:
                \begin{align*}
                    c'(p) &= \sum_{i=0}^{k-1} c'(v_i, v_{i+1}) \\
                          &= \sum_{i=0}^{k-1} c(v_i, v_{i+1}) + h(v_i) - h(v_{i+1}) \\
                          &= \dots \\
                          &= c(p) + h(v_0) - h(v_k)
                \end{align*}
                Somit ergibt sich:
                \begin{align*}
                    c(p) &= \delta(v_0, v_k) \\
                    \iff c'(p) &= \delta'(v_0, v_k) &&\text{(wie $\delta$ nur mit $c'$ statt $c$)}
                \end{align*}
            \end{lemma}
            Falls $p$ Zykel $(v_0 = v_k) \implies c(p) = c'(p)$.

            Wähle $h$ so, dass alle $c' \ge 0$. Erweitere $G$ um einen Knoten $s \not\in V$ und
            die Kanten $(s, v), \forall v \in V$ mit $c(s, v) = 0$.
            Wähle $h(v) = \delta(s, v)$. Dann gilt:
            \begin{align*}
                h(v) &\le h(u) + c(u, v) \text{ für alle } (u, v) \in E \\
                \intertext{und somit:}
                0 &\le c(u, v) + h(u) - h(v) = c'(u, v)
            \end{align*}
            h wird durch einmaligen Bellman-Ford von $s$ aus in $\LO(n \cdot m)$ berechnet.




        \section{Durchmusterung von Graphen}
        Es gibt DFS (Depth-First-Search) und BFS (Breadth-First-Search).
        Bestimmung aller Knoten, die von vorgegebenem $v \in V$ erreichbar sind.
        \begin{algorithmic}
        		\State $S\gets \set{n}$
        		\State markiere alle Kanten als unbenutzt
        		\While{$\exists e\gets (u,v)\in E,\ u\in S\text{ und } (u,v)$ unbenutzt}
        			\State markiere $e$ als benutzt
        			\State $S\gets S\cup \set{v}$
        		\EndWhile
        \end{algorithmic}
%\begin{verbatim}
%    S <- {n}
%    markiere alle Kanten als unbenutzt.
%    while $\exists e = (u,v) \in E$,  $u \in S$ und $(u, v)$ unbenutzt do
%        Markiere $e$ als benutzt
%        S <- S \cup {v}
%    od
%\end{verbatim}
        Probleme:
        \begin{enumerate}
            \item Realisierung benutzt $\leftrightarrow$ unbenutzt
            \item Finden geeigneter Kanten
            \item Realisierung von $S$
        \end{enumerate}

        Lösungen:
        \begin{enumerate}
            \item Verwende Adjazenzlisten, markiere mit Trennzeiger in Liste. % Diagramma
                  Alle Knoten links davon sind benutzt, alle rechts davon unbenutzt.
            \item $\tilde S \subset S$. In \~S befinden sich alle Knoten, für die noch nicht alle Kanten gesehen wurden. (Trennzeiger noch nicht ganz rechts)
            \item Operationen: \textsc{init, insert}, $w \in S$. Lege $S$ als boolesches Feld an. Dann alle Operationen in $\LO(1)$.
                Operationen auf \~S: \textsc{init, insert}, $w \in \tilde S$, wähle $w \in \tilde S$, streichen, $\tilde S = \emptyset$?. Verwende Stack (dann ergibt sich DFS) oder Queue (dann ergibt sich BFS).
        \end{enumerate}
        
        \begin{algorithmic}
        	\Function{ExploreFrom}{Knoten s}
        		\State $S\gets \{s\}$
        		\State $\tilde S \gets \{s\}$
        		\ForAll{$v\in V$}
        			\State $p(v)\gets adjHead(v)$
        		\EndFor
        		\While{$\tilde S \not= \emptyset$}
        			\State $w\gets p(v)$, verschiebe $p(v)$
        			\If{$w\not\in S$}
        				\State\Call{Insert}{w,S}, \Call{Insert}{w,$\tilde S$}
        			\Else
        				\State \Call{Delete}{w,$\tilde S$}
        			\EndIf
        		\EndWhile
        	\EndFunction
        \end{algorithmic}

%\begin{verbatim}
%    ExploreFrom(s)
%        S <- {s}
%        S~ <- {s}
%        forall $v \in V$ do
%            p(v) <- adjHead(v);
%        od
%        while s~ != {} do
%            Sei $v \in \~S$ bel.
%            if p(v) != NIL then
%               w\gets p(v), verschiebe p(v)
%               if w\not\in S then
%               einfügen(w,S), einfügen(w,\~S)
%               else streiche (w,\~S) fi
%            fi
%        od
%\end{verbatim}

					\underline{Laufzeit}\\
					$\LO (n_s m_s)$ mit $n_s=|V_s|,\ m_s=|E_s|$, $V_s=\{v\in V|s\overset{*}{\rightarrow} v\}$\\
					(Indizierter Teilgraph $(V_s,E_s)\subseteq (V,E)$, falls $V_s\subseteq V$
					
					\begin{satz}
						Zusammenhangskomponenten können bei ungerichtetem Graphen in $\LO (n+n)$ berechnet werden.
					\end{satz}
					
%
%\begin{verbatim}
%	forall s\in V do
%	  if s\not\in S then
%	     ExploreFrom(s)
%	  fi
%\end{verbatim}					
					\begin{algorithmic}
						\ForAll{$s\in V$}
							\If{$s\not\in S$}
								\State \Call{ExploreFrom}{s}
							\EndIf
						\EndFor				
					\end{algorithmic}
					Für jede Zusammehangskomponente wird ExploreFrom 1-mal aufgerufen
										
					\subsection{Depth-First-Search (Tiefensuche)}
						bei $\tilde S$ = Keller (stack)\\
						\begin{figure}
							\centering
							\begin{tikzpicture}[-, node distance=2cm, every node/.style={circle,draw}]
	\node (b) {b};
	\node (d) [above of=b] {d};
	\node (a) [right of=b] {a};
	\node (e) [right of=d] {e};
	\node (c) [above right of=a] {c};
	
	\path[draw]
		(b) edge (d)
		(b) edge (a)
		(b) edge (e)
		
		(d) edge (e)
		(e) edge (c)
		(c) edge (a);

\end{tikzpicture}
							\label{fig:GraphFuerSuche}
							\caption{Beispielgraph für Tiefen- und Breitensuche}
						\end{figure}
						DFS startet in a:\\
						S: a,b,d,e,c\\
						\hspace*{2cm}$\tilde S:\begin{array}{|c|}
							c\\e\\d\\b\\a\\\hline
						\end{array}$
					
					\subsection{Breadth-First-Search (Breitensuche)}
						bei $\tilde S$ = Schlange (queue)\\
						\autoref{fig:GraphFuerSuche}\\
						BFS startet in a:\\
						S: a,b,c,d,e\\
						$\tilde S$: c b a $\rightarrow$ e d c b $\rightarrow$ ...
					
					
					\subsection{DFS rekursiv}
					    $\tilde S$ als Stack.
					    \begin{algorithm}
					    		\caption{DFS}
					    		\begin{algorithmic}
					    			\Function{dfs}{Knoten v}
					    				\ForAll{$(v,w)\in E$}
					    					\If{$w\not\in S$}
					    						\State $S\gets S\cup \{w\}$
					    						\State $dfsnum(w)\gets count1,\ coun1++$
					    						\State \Call{Insert}{(v,w), T} \Comment{$(v,w)$ ist Baumkante}
					    						\State \Call{dfs}{w}
					    						\State $compnum(w)\gets count2,\ count2++$					    						
					    					\Else
					    						\If{v$\overset{*}{\underset{T}{\rightarrow}}$w}
					    							\State\Call{Insert}{(v,w),F} \Comment{$(v,w)$ ist Vorwärtskante}
					    						\Else
					    							\If{w$\overset{*}{\underset{T}{\rightarrow}}$v}
					    								\State\Call{Insert}{(v,w),B} \Comment{$(v,w)$ ist Rückwärtskante}
					    							\Else
					    								\State\Call{Insert}{(v,w),C} \Comment{$(v,w)$ ist Querkante}
					    							\EndIf
					    						\EndIf
					    					\EndIf
					    				\EndFor
					    			\EndFunction
					    		\end{algorithmic}
					    \end{algorithm}

%\begin{verbatim}
%	dfs(v)
%	for all (v,w) \in E do
%	    if w \not\in S 
%	    then 
%	        S <- S \cup {w}
%	        dfsnum(w) <- count1, count1++
%	        dfs(w)
%	        compnum(w) <- count2, count2++
%	        füge (v,w) in T hinzu
%	    else
%	        if v ->_T^* w
%	        then
%	            füge (v,w) zu F hinzu 
%	        else
%	            if w ->_T^* v
%	            then 
%	                füge (v,w) zu B hinzu
%	            else 
%	                füge (v,w) zu C hinzu
%                fi
%            fi
%        fi
%    od  
%\end{verbatim}	
                        wobei dfsnum zählt als wievielter Knoten w besucht wird und compnum sagt als wievielter Knoten w abgeschlossen wird. T ist Baumkanten und F ist Vorwärtskanten, B Rückwärtskanten, C Querkante, S die schon gesehenen Knoten. Die Notation $v \rightarrow_T^* w$ bedeutet, dass $w$ von $v$ aus über Baumkanten erreichbar ist.
                        
                        \begin{figure}[htp]
                            \centering
                            \capstart
                            \begin{tikzpicture}[->,>=stealth', every node/.style={draw, ellipse}, node distance=2cm]
	\node (a) {$a|1|6$};
	\node (b) [above of=a] {$b|2|5$};
	\node (c) [above of=b] {$c|3|1$};
	\node (d) [right of=b] {$d|4|4$};
	\node (e) [above of=d] {$e|5|3$};
	\node (f) [right of=e] {$f|6|2$};
	\node (g) [below right of=d] {$g|7|7$};
	
	\path[draw] 
		(a) edge (b)
		(b) edge (c)
		(b) edge (d)
		(d) edge[bend left] (e)
		(e) edge (f);
		
	\path[draw, dotted]
		(d) edge (f);
	
	\path[draw, densely dashed]
		(e) edge[bend left] (d);
		
	\path[draw, loosely dashed]
		(e) edge (c)
		(f) edge[bend right] (c)
		(g) edge (d);
\end{tikzpicture}
                            \caption{Rekursive Tiefensuche}
                            \label{fig:dfsrecursive}
                        \end{figure}
                        
                        \begin{lemma}
                            Sie $G=(V,E)$ ein gerichteter Graph. Es gelten
                            \begin{enumerate}
                                \item DFS auf $G$ hat lineare Laufzeit $\LO(n+m)$
                                \item $T,B,F,C$ ist Partition von $E$
                                \item $T$ entspricht dem Aufrufbaum der Rekursion
                                \item $v \rightarrow_T^* w \iff \text{dfsnum}(v) \leq \text{dfsnum}(w) \land \text{compnum}(w) < \text{compnum}(v)$
                                \item $\forall (v,w) \in E$ gilt
                                    \begin{enumerate}[a)]
                                        \item $(v,w) \in T \cup F \iff \text{dfsnum}(v) \leq \text{dfsnum}(w)$
                                        \item $(v,w) \in B \iff \text{dfsnum}(w) < \text{dfsnum}(v) \land \text{compnum}(v) < \text{compnum}(w)$
                                        \item $(v,w) \in C \iff \text{dfsnum}(w) < \text{dfsnum}(v) \land \text{compnum}(w) < \text{compnum}(v)$
                                    \end{enumerate}
                            \end{enumerate}
                        \end{lemma}
                        Aus $5.$ folgt, dass Berechnung der Partition in $T,F,B,C$ aus dfsnum/compnum folgt. Zykelfrei bedeutet, dass es keine Rückwärtskanten in $G$ gibt, d.h. $\forall (v,w) \in E : \text{compnum}(v) > \text{compnum}(w)$. Also ist $\text{num}(v) = n+1-\text{compnum}(v)$ topologische Sortierung.
                               
                \subsection{Starke Zusammenhangskomponente (SZKs)}
                    Gegeben sei ein gerichteter Graph $G = (V,E)$. Dieser heißt stark zusammenhängend, wenn es für alle $v,w \in V$ ein $v \rightarrow_E^* w$ gibt. Eine SZK von $G$ ist ein maximal stark zusammenhängender Teilgraph von $G$.
                    
                    \begin{definition}
                        Sei $(V',E')$ eine SZK. Knoten $v \in V'$ heißt Wurzel der SZK wenn 
                        $$
                            \text{dfsnum}(v) = \min\{\text{dfsnum}(w)\ : w \in V'\}
                        $$
                    \end{definition}
                    \emph{Idee:} Sei $G_{\text{aktuell}} = (V_{\text{akt}}, E_{\text{akt}})$ der von den  schon gesehenen Knoten aufgespannt wird. Verwalten die SZKs von $G_{\text{akt}}$. Am Anfang ist $V_{\text{akt}} = 1$ und $E_{\text{akt}} = \emptyset$. Betrachte Kante $(v,w)$:
                    \begin{enumerate}
                        \item $(v,w) \in T$: w kommt zu $V_{\text{akt}}$ hinzu. Bildet also eine eigene SZK
                        \item $(v,w) \not \in T$: mische eventuell mehrere SZKs zu einer
                    \end{enumerate}


%
%   Vorlesung 27.06.
%
        \subsection{Durchmustern} DFS, Berechnung von SZKs, \textsc{dfsnum}, \textsc{compnum}.

        \define{Wurzeln}{Folge von Wurzeln von nicht abg. Komp. % abgeschlossenen Komponenten?
        in aufsteigender Reihenfolge von \textsc{dfsnum}.}

        \define{Unfertig}{Folge von Knoten $v$, für die $\textsc{dfs}(v)$ aufgerufen wurde, aber SZK nicht abgeschlossen wurde. Aufsteigend nach \textsc{dfsnum}.}

        %TODO : Beispielgraph

        \paragraph{Fälle für Knoten g}
        \begin{enumerate}
            \item Kante $(g, d)$: nichts, da $d$ Abgeschlossen.
            \emph{Invariante:} Es gibt keine Kanten, die in abgschlossener Komponente starten und in nichtabgeschlossener enden.
            \item Kante $(g, h)$: $h$ ist neuer Knoten. $(g, h)$ ist Baumkante. $h$ ist neue SZK. Füge $h$ zu unfertig und Wurzeln hinzu.
            \item Kante $(g, c)$: Vereinige 3 SZKs mit Wurzeln $g, e, b$. Lösche $g, e$ aus Wurzeln.
        \end{enumerate}

        \paragraph{Algorithmus} 
        %TODO : In Algorithmusumgebung umschreiben
\begin{verbatim}
    Unfertig <- {}
    Wurzeln <- {}
    for all $v \in V$ do
        InUnfertig <- false // Boolesches Feld das Vorhandensein in Unfertig anzeigt
    od
    in dfs(v):
        push(v, Unfertig)
        InUnfertig(v) <- true
        count1 <- count1 + 1
        dfsnum(v) <- count1
        S <- S \cup {v}
        push(v, Wurzeln)
        for all (v, w) \in E do
            if w \not\in S then
                dfs(w)
            else
                if InUnfertig(w) then
                    while dfsnum(w) < dfsnum(top(Wurzeln)) do
                        pop(Wurzeln)
                    od
                fi
            fi
        od
        count2 <- count2 + 1
        compnum(v) <- count2
        if v = top(Wurzeln) then
            repeat
                w <- top(Unfertig)
                InUnfertig(w) <- false
                pop(Unfertig)
            until w = v
            pop(Wurzeln)
        fi
\end{verbatim}
        \paragraph{Laufzeit}
            Normales DFS in $\LO(n + m)$ zusätzlich wird jeder Knoten (jeweils?) einmal in \textsc{InUnfertig}, \textsc{Unfertig}, \textsc{Wurzeln} aufgenommen und gelöscht.
            $$
                \implies \LO(n + m)
            $$

    \section{Minmal aufspannende Bäume (MSTs)}
        Sei $G$ ein zusammenhängender ungerichteter Graph. Sei $c: E \to \R^+$ eine Kostenfunktion.
        %TODO : input diagram (a)
        Wollen minimal aufspannenden Teilgraph $G' = (V, E_T), E_T \subseteq E$. $G'$ zusammenhängend und $c(E_T) = \sum_{e \in E_T} c(e)$ minimal.
        \emph{Beob.:} $G'$ ist azyklisch. (Beweis durch Widerspruch). Daher: Minimal aufspannender Baum (minimum spanning tree, MST).
        %TODO : input diagram (b)

        \subsection{Kruskal-Algorithmus (Greedy)}
        \begin{enumerate}
            \item Sortiere Kanten, sodass $c(e_1) < c(e_2) < \dots < c(e_n)$.
            \item $E_T = \emptyset$, createsets(u), $V_1 \leftarrow \set{1}, V_2 \leftarrow \set{2}$
            \item
            	\begin{algorithmic}
            		\For{$i\gets 1$ to $m$}
            			\If{$(V,E_T\cup \{e_i\})$ azyklisch}
            				\State $E_T\gets E_T\cup \{e_i\}$
            			\EndIf
            		\EndFor
            	\end{algorithmic}
%\begin{verbatim}
%for i <- 1 to m do
%    if (V, E_T) \cup {e_i} azyklisch then
%        E_T <- E_T \cup {e_i}
%    fi
%od
%\end{verbatim}
        \end{enumerate}

        \begin{satz}
            Der Kruskal-Algorithmus ist korrekt.
            \begin{proof} % WAS PASSIERT HIER??
                Nenne Kantenmenge $E' \subseteq E_T$ \q{gut}, falls sie zu einem MST erweiterbar ist.
                \begin{enumerate}
                    \item \emph{Behauptung:} Sei $E' \subseteq E_T$ gut und sei $e \in E \setminus E'$ die billigste Kante, sodass $(V, E' \cup \set{e})$ azyklisch ist. Dann ist auch $E' \cup \set{e}$ gut. $\to$ Indirektion
                    \begin{proof}%[Beweis durch Widerspruch]
                        Sei $T_1 = (V, E_1)$ ein MST mit $E' \subseteq E_1$. $T_1$ existiert, da $E'$ gut. Ist $e \in E_1$, dann fertig. Falls nicht, betrachte Graph $H = (V, E_1 \cup \set{e})$. $H$ enthält einen Zykel, auf dem $e$ liegt. Da $(V, E' \cup \set{e})$ azyklisch, enthält der Zykel auch Kante aus $E_1 \setminus (E' \cup \set{e})$. Sei $e_1$ eine solche Kante. Betrachte $T_2 = (V, (E_1 \setminus \set{e_1}) \cup{e})$. $T_2$ ist aufspannend und $c(T_2) = c(T_1) + c(e) - c(e_1)$. Da $e$ billigste Kante, gilt $c(e) \le c(e_1)$ und damit $c(T_2) \le c(T_1)$. Da $T_1$ MST nach Voraussetzung, ist auch $T_2$ MST. $\implies E' \cup \set{e}$ ist gut, denn kann zu $T_2$ erweitert werden.
                    \end{proof}
                \end{enumerate}
            \end{proof}
        \end{satz}
        
        Wir ersetzen nun den Test im Algorithmus. Dies ist in \autoref{alg:kruskalUnion} beschrieben.
    	\begin{algorithm}
        		\caption{Kruskals Algorithmus mit Union-Find}
        		\label{alg:kruskalUnion}
        		\begin{algorithmic}[1]
        			\Function{Kruskal}{$G$}
        			    \For{$i \gets 1$ to $m$}
        				    \State $e_i \gets (u,v)$
        				    \State $a \gets $ \Call{Find}{$u$}
        				    \State $a \gets $ \Call{Find}{$v$}
        				    \If{$a \neq b$}
        				        \State \Call{Union}{$a,b$}
        				        \State $E_T \gets E_T \cup \set{e_i}$
        				    \EndIf
        				\EndFor
        			\EndFunction
        		\end{algorithmic}
        	\end{algorithm}
%\begin{verbatim}
%for i <- 1 to m do
%    sei e_i <- (u,v)
%    a <- Find(u)
%    b <- Find(v)
%    if a != b then
%        Union(a,b)
%        E_T <- E_T \cup {e_i}
%    fi
%od
%\end{verbatim}
        Find(u) liefert Namen der Menge in der u ist. Es gibt $2n$ Finds und $n-1$ Unions.
        
        \subsubsection{Union-Find Datenstruktur}
            \begin{enumerate}
                \item Namensfeld R[1,..,n] -> [1,..,n] wobei $R(x)$ der Name der Menge, zu der $x$ gehört. Da ergibt etwa \\
                Find(x): return R(x) \\
                Union(a,b): \\
                %TODO algorithmic
\begin{verbatim}
for i <- 1 to n do
    if R(i) = a then R(i) <- b
od
\end{verbatim}
                Damit erhält man eine Laufzeit von Find in $\LO(1)$ und Union in $\LO(n)$. Gesamt also $\LO(m+n^2üm \log m)$
            \end{enumerate}   
            Wir wollen nun den Union von $\LO(n)$ so verbessern, dass er nicht alle Elemente besucht.  Merke für jede Menge die Elemente, die dort enthalten sind. Behalte bei Union immer den Namen der größeren Menge. Benötigte Funktionen sind nun in \autoref{alg:unionFind} dargestellt. Worst case Union: $\LO(n), \abs{a} = \frac{n}{2}, \abs{b} = \frac{n}{2}$. 
    	\begin{algorithm}
        		\caption{Funktionen für Union-Find}
        		\label{alg:unionFind}
        		\begin{algorithmic}[1]
        			\Function{createSet}{}
        			    \For{$i \gets 1$ to $n$}
        				    \State \Call{R}{x} $ \gets x$ \Comment{Speichert Wurzel des Baumes}
                            \State \Call{Elem}{x} $ \gets x$ \Comment{Speichert Elemente des Baums}
                            \State \Call{Size}{x} $ \gets x$
                        \EndFor
        			\EndFunction
        			\Statex
        			\Function{Find}{$x$}
        			    \State \Return \Call{R}{$x$}
        			\EndFunction
        			\Statex
        			\Function{Union}{$a,b$}
        			    \If{\Call{size}{$a$} $<$ \Call{size}{$b$}} \Comment{$a$ muss größer/gleich $b$ sein}
        			        \State \Call{Swap}{$a,b$}
        			    \EndIf
        			    \ForAll{$x \in $ \Call{Elem}{$b$}}
        			        \State \Call{R}{x} $ \gets a$  
        			        \State $e_a \gets $ \Call{Elem}{$a$} 
        			        \State \Call{Insert}{$x,e_a$}
        			    \EndFor
        			\EndFunction  
        		\end{algorithmic}
        	\end{algorithm}
%\begin{verbatim}
%createSet:
%for x <- 1 to n do
%    R(x) <- x
%    elem(x) <- x
%    size(x) <- 1
%    
%Find(x): 
%return r(x)
%
%Union(a,b)
%if size(a) < size(b) then vertausche(a,b)
%for all x \in Elem(b)
%    R(x) <- a
%    insert(x, Elem(a))
%od
%\end{verbatim}                       
            \begin{satz}
                Mit der beschriebenen Struktur können createset, $n-1$ Unions und $2n$ Finds in Zeit $\LO(n \log n + m)$ ausgeführt werden.
            \end{satz}        
            
            \begin{proof}
                Createset und $2m$ Finds gehen in $\LO(n+m)$. Wir Behaupten nun, dass $n-1$ Unions $\LO(n \log n)$ kosten. Ein Union(a,b) vereinigt zwei Mengen mit $n_a, n_b$ Elementen. Sei $n_b < n_a$, dann folgt eine Laufzeit von $\LO(1 + n_b)$. Damit haben wir eine Gesamtzeit von $\LO(\sum_{i=1}^{n-1}{n_i+1})$ mit $n_i$ Größe der kleineren Menge beim $i$-ten Union. Wechselt ein Element die Menge, so trängt es $1$ zu $n_i$ bei. Element $j$ wechselt $r_j$ die Menge. Es gilt also
                $$
                    \sum_{i=1}^{n-1}{n_i}  = \sum_{j=1}^{n-1}{r_j}
                $$ 
                Behauptung: $r_j \leq \log n$ für alle Knoten $j$
                \begin{proof}
                    Wenn $j$ in Menge mit $l$ Elementen ist und wechselt, dann kommt $j$ in eine Menge der Größe $\geq 2l$. Das heißt beim $k$-ten Wechsel ist $j$ in Menge mit mindestens $\geq 2^k$ Elementen. Da es nur $n$ Elemente gibt gilt $2^k \leq n$ und daraus folgt $k \leq \log n$
                \end{proof}
                Insgesamt folgt nun
                $$
                    \sum_{i=1}^{n-1}{n_i}  = \sum_{j=1}^{n-1}{r_j} = n \log n    
                $$
                Damit ist die Laufzeit für Kruskals Algorithmen $\LO(m \log n)$
            \end{proof}
            
    \subsection{Union-Find}
        Halte für jede Menge Baum. Damit geht Union in $\LO(1)$. Für Find läuft man im Baum hoch zur Wurzel. Das geht in $\LO(\text{Tiefe}(\text{Baum}))$. Für weitere Verbesserungen führe folgende Optimierungen ein
        \begin{enumerate}
            \item Gewichtete Verinigungsregel: Hänge kleinen an großen Baum
            \item Pfadkomprimierung: Laufe Pfad hoch. Hänge alle Zeiger auf Wurzel
        \end{enumerate}   
        \begin{satz}
            n Unions + m Finds gehen in Zeit $\LO(n + m \cdot \alpha (m+n,n))$ wobei $\alpha$ invers zur Ackermannfunktion ist. Das ist kleiner als 5 für alle realistischen Werte.
        \end{satz}          
            
            
    \subsection{PRIMs Algorthmus}
        Beschrieben in \autoref{alg:prim}. Dabei liegen in $E_T$ die Kanten des Spannbaums. Korrektheitsbeweis analog zu vorher: Induktiv über gute Erweiterungen.
    	\begin{algorithm}
        		\caption{Prims Algorithmus}
        		\label{alg:prim}
        		\begin{algorithmic}[1]
        			\Function{Prim}{$G$}
        			    \State $T \gets \set{v}$
        				\While{$T \neq \set{V}$}
        				    \State $e \gets (u,w) \in E \text{ mit } u \in T \land w \not \in T \text{ und } c(e) \text{ minimal}$
        				    \State $E_T \gets E_T \cup \set{e}$
        				    \State $T \gets T \cup \set{w}$
        				\EndWhile
        			\EndFunction
        		\end{algorithmic}
        	\end{algorithm}
%\begin{verbatim}
%t <- {v}
%while T != {V} do
%    sei e = (u,w) \in E mit u \in T und w \not \in T 
%        mit c(e) minimal
%    E_T <- E_t \cup {e}
%    T <- T \cup {w}
%od
%\end{verbatim}
        
        
        \subsubsection{Laufzeit}
            Priority Queue mit Schlüsseln $\set{c(w) | w \not \in T \land c(w) = \min\set{c(u,w),u \in T}}$. Suche $w$ mit kleinstem $c(w)$, entspricht deleteMin bei Dijkstra. Wird $w$ in $T$ eingefügt, so tue für alle $(w,x) \in E$ mit $x \not \in T$ und $c(x) \leftarrow \min \set{c(x), c(w,x)}$. Das entspricht DecreasyKey bei Dijkstra. 
            
        \subsubsection{Implementierung}
            Priority Queue als primären Heap. Dabei gehen Operationen in $\LO(\log n)$. Insgesamt also $\LO(m \log n)$. Statt bin. Heap verwende (a,b)-Baum mit $ a = \max(2,\frac{m}{n})$ und $b=2a$. Damit ergibt sich eine Laufzeit von $\LO(m \frac{\log n}{\log \frac{m}{n}})$ Das ist gut, wenn $m$ etwa $n^2$ ist.
            
    
        
     \section{Zweifach Zusammenhangs Komponente von ungerichteten Graphen}
        Graph heißt 2-fach zusammenhängend, falls $G-\{v\}$\footnote{$G'=\left(V\backslash\{v\},\left(E\backslash\{(v,x)\forall x\in V\}
        \right)\backslash\{(x,v)\forall x\in V\}\right)$} zusammenhängend für alle $v\in V$\\
        Eine 2ZK ist maximal 2-fach zusammenhängender Teilgraph. $v\in V$ heißt Artikulationspunkt, wenn $G-\{v\}$ nicht zusammenhängend\\
        
        %TODO Zeichnung

        2ZK's: $\{1,2\},\{6,7\},\{2,3,4\},\{4,5,6\}$\\
        Artikulationspunkte: $2,4,6$
        
        \subsection{DFS für ungerichtete Graphen}
        	Es gibt Baumkanten T und Rückwärtskanten B, jedoch keine Vorwärts- und Querkanten.\\\\
        	\emph{Idee:}\\
        		v ist kein Artikulationspunkt, falls ein Baumnachfolger w von v eine Rückwärtskante ''vor'' v hat\\
        		x ''vor'' v $\Leftrightarrow dfsnum(x)<dfsnum(v)$\\
        		Falls v Wurzel des dfs-Baumes und mehrere Zweige, so ist v Artikulationspunkt.
        	\begin{definition}
        		$low(u)\gets \min\{dfsnum(u);\min\{dfsnum(v)$ mit Pfad $u\overset{*}{\underset{T}{\rightarrow}}Z\rightarrow v\}\}$
        	\end{definition}
        	\begin{algorithm}
        		\caption{Tiefensuche auf ungerichteten Graphen}
        		\begin{algorithmic}[1]
        			\Function{dfs}{Knoten v}
        				\ForAll{$(v,w)\in E$)}
        					\If{w unbesucht}
        						\State $dfsnum(w)\gets count$; $count++$
        						\State $low(w)\gets dfsnum(w)$
        						\If{$low(w)<low(v)$}
        							\State $low(v)\gets low(w)$
        						\EndIf
        						\If{$low(w)>dfsnum(v)$}
        							\State $artpunkt(v)\gets TRUE$
        						\EndIf
        					\Else
        						\If{$dfsnum(w)<low(v)$}
        							\State $low(v)\gets dfsnum(w)$
        						\EndIf
        					\EndIf
        				\EndFor
        			\EndFunction
        		\end{algorithmic}
        	\end{algorithm}
        	In (7) wird low-Wert von w an low(v) übergeben, falls er kleiner ist\\
        	In (14) wird Rückwärtskante von v nach w berücksichtigt\\
        	In (10) wird erkannt, ob der Zweig, der aus v in Richtung w startet keine Rückwärtskante ''vor'' v enthält $\Rightarrow$ v ist Artikulationspunkt\\\\
        	\emph{Laufzeit:} $\mathcal{O}(n+m)$ (Verfeinerung von DFS)\\\\
        	\emph{Spezialfall:} v Wurzel
        	\begin{algorithmic}
	        	\If{$dfsnum(v)==1 \&\& \exists w_1\not= w_2\text{ mit }(v,w_1),(v,w_2)\in T$}
	        		\State $artpunkt(v)\gets TRUE$
	        	\EndIf
        	\end{algorithmic}
        	\emph{Was fehlt:} Berechnung der 2ZK wenn Artikulationspunkte gegeben ($\rightarrow$ Übung)
        	
    \section{EXKURS: Stable Marriage}
		Vollständiger bipartiter Graph aus n Männern und n Frauen.\\
		Jede Person hat Rangliste $\{0,...,n-1\}$ für das andere Geschlecht:
		\begin{itemize}
			\item $w(x,Y)$ Gewicht von Mann aus gesehen
			\item $w(Y,x)$ Gewicht von Frau aus gesehen
		\end{itemize}
		%TODO Beispielgraph
		(a,A),(b,B),(c,C) nicht stabil:\\
		c findet B attraktiver als C und B findet c attraktiver als b\\\\
		
		Suche perfektes Matching M, sodass es kein $(x,Y),(Y,x)$ mit $w(x,Y)>w(x,X)$ und $w(Y,x)>w(Y,y)$
		
		\subsection{Algorithmus}
			Alle Männer sind frei.\\
			Wähle einen freien Mann, der wählt seine beste Frau, die ihn noch nicht abgelehnt hat. Diese akzeptiert oder lehnt ab, je nachdem ob das Angebot ihren Status verbessert.\\
			Evtl. wird ihr bisheriger Mann frei.\\
			Es gilt: Verheiratete Frauen bleiben verheiraten, aber nicht unbedingt mit demselben Mann
			\subsubsection{Termination}
				Alle Frauen bekommen irgendwann ein Angebot, bleiben ab dann verheiratet.\\
				\emph{Männersicht:}\\
					Macht Mann n Angebote haben vorher $n-1$ abgelehnt. Diese letzte muss frei sein, diese akzeptiert.
			\subsubsection{Korrektheit}
				Es gibt \emph{kein} (x,X)(y,Y), wo x Y attraktiver findet als X (x hat Y vor X gefragt) \emph{und gleichzeitig} Y x attraktiver findet als y (denn Y hatte x abgelehnt)
			\subsubsection{Laufzeit}
				$\LO\left(\sum\limits_{i=1}^n i\right)=\LO(n^2)$\footnote{$\rightarrow$ Buch: Kleinberg/Tardos}
				\paragraph{Frage:}
					Erwartete Laufzeit, wenn Referenzen der Männer zufällig sind, die der Frauen beliebig aber fest.\\
					\emph{Idee:}\\
					Jeder Mann wählt seine Präferenz nach und nach (wenn er sie braucht).
					x macht immer einer zufälligen Frau ein Angebot, der er noch kein Angebot gemacht hat oder einfach: Er wählt \emph{immer eine} zufällige Frau.\\
					Wann haben alle Frauen mind. 1 Angebot?\\
					$\rightarrow$ ''Coupon Collector Problem''
		\subsection{Coupon Collector Problem - Laufzeit}
			Wie viele Bälle müssen auf n Körbe geworfen werden, bis jeder Korb $\geq 1\times$ getroffen ist?\\
			
			Ergebnis: $\Theta(n\log n)$\\
			
			$P(\text{Korb i nach k Würfen nicht getroffen})=\left( 1-\frac{1}{n}\right)^k\approx e^{-\frac{k}{n}}$\\
			Mit $k=c\cdot n$: $P(\text{Korb i nach k Würfen nicht getroffen})=e^{-c}$\\
			Mit $k=c\cdot n\log n$: $P=e^{-c\cdot \log n}=\left(\frac{1}{n}\right)^c$\\\\
			$\Rightarrow$ Mit hoher Wahrscheinlichkeit ist nach $(c-1)\cdot n\log n$ Versuchen kein Korb leer. Wähle $c>1$.
			
        	

% Copyright \copyright\ 2013  Simon Kalt, Jan-Peter Hohloch, Tobias Fabritz
% Es wird die Erlaubnis gegeben, dieses Dokument unter den Bedingungen der von der Free
% Software Foundation veröffentlichen  GNU Free
% Documentation License (Version 1.2 oder neuer) zu kopieren, verteilen und/oder
% zu verändern. Eine Kopie dieser Lizenz ist unter
% http://www.gnu.org/copyleft/fdl.txt erhältlich.
%
% Zusätzlich muss jede Kopie/Aktualisierung wieder über die Seite
% der Fachschaft Informatik der Uni Tübingen
% den Studenten zur Verfügung gestellt werden
% http://www.fsi.uni-tuebingen.de/

\chapter{Patternmatching} 			
    \section{Patternmatching auf Strings}
        Sei $T$ ein String über dem Alphabet $\Sigma$. Dabei sei $T[1..n], 1 \leq i \leq j \leq n$ ein Teilstring von Stelle $i$ bis Stelle $j$. Gegeben sei nun ein String $T[1..n]$ und ein Pattern $P[1..m]$ mit $m << n$. Suche nun $P$ in $T$.
        
        \subsection{Naive Lösung}
            Man schiebt das Pattern durch den String und vergleicht jeweils alle Stellen.
\begin{verbatim}
for s <- 1 to n-m+1 do
    Teste ob T[s..s+m-1] = P
od
\end{verbatim}
            Dies findet alle Vorkommen und hat eine Laufzeit von $\LO(n \cdot m )$
            
        \subsection{Algorithmus von Knuth/Morris/Pratt}
            \subsubsection{Idee}
                Verschiebe $P$ nicht nur um 1. Betrachte P. Berechne für alle q
                $$
                    \Pi(q) = \max \set{k | P[1..k] \text{ ist echtes Suffix von } P[1..q]}
                $$
                Verschiebe P so, dass $P[\Pi(q-1)+1]$ auf $T[j]$. Wir suchen das kleinste $s' > s$ so dass $P[1..k]-T[s'+1,..,s'+k]$ mit $s'+k = s + q$.
\begin{verbatim}
P        ababaa
Index    123456
PI(i)    001231
\end{verbatim}
            \subsubsection{Algorithmus für Präfixfunktion}
\begin{verbatim}
Präfixfunktion(P)

PI(1) <- 0, k <- 0
for q<- 2 to m do
    while k>0 and P[k+1] != P[q] do
        k <- PI(k)
    od
    if P[k+1]=P[q] then 
        k <- k+1
    fi
    PI(q) <- k
od
\end{verbatim}

            \subsubsection{Laufzeit}
                In der while-Schleife befinden wir uns nur wenn $k>0$ und innerhalb der while-Schleife wird $k$ immer erniedrigt. In jeder for-Iteration wir $k$ einmal erhöht, d.h. es steigt nur $n$ mal. Es kann also auch nur maximal m Erniedrigungen geben. Damit haben wir Laufzeit $\LO(m)$
                
            \subsubsection{Algorithmus}
\begin{verbatim}
Gesamt
Sei n,m,T,P,PI gegeben

q <- 0
for i <- 1 to n do
    while q > 0 and P[q+1] != T[i] do
        q <- PI(q)
    od
    
    if P[q+1] = T[i] then
        q <- q+1
    fi
    
    if q=m then
        // P kommt ab (i-m+1)-ter Stelle vor
        q <- PI(q)
    fi
od
\end{verbatim}
            \subsubsection{Laufzeit}
                Amortisierungsargument wie bei Präfix, also Laufzeit $\LO(n)$
                
            \subsubsection{Beispiel}
                Siehe Wikipedia\footnote{\url{http://de.wikipedia.org/wiki/Knuth-Morris-Pratt-Algorithmus}}
                
    \subsection{Algorithmus von Boyer/Moore}
\begin{verbatim}
Naiv:

s<-0
while s <= n-m do
    j <- m
    while j > 0 and P[j] = T[s+j] do
        j <- j-1
    od
    
    if j = 0 then
        //P kommt ab Stelle s+1 vor
        s<- s+1    // ersetze durch s <- s+ gamma[0]
    else
        s <- s+ 1  //ersetze durch s <- s + max{gamme[j], j-lambda[T[s+j]]}
                   //hierzu brauchen wir "gutes Suffix" und schlechter Buchstabe
    fi
od      
\end{verbatim}
        Berechne $\lambda$ als letztes Vorkommen$(P,m,\Sigma)$ und $\gamma$ als gutes Suffix$(P,m)$.
        
        \subsubsection{Schlechter Buchstabe}
            sei $P[i] \neq T[s+j]$ ein gefundenes Mismatch. Das ist gut wenn $T[s+j]$ gar nicht vorkommt, sonst verschiebe bis zum Vorkommen. Dann sei
            $$
                k = \max \set{i | T[s+j] = P[i]} \text{ falls T[s+j] in P, sonst } 0
            $$
            \begin{lemma}
                s kann um $j-k$ verschoben werden.
            \end{lemma}
            \begin{proof}
                \begin{enumerate}
                    \item $k=0$: $T[s+j]$ kommt nicht vor, ist also schlechter Buchstabe. Wir verschieben um $j$ bis hinter diesen Buchstaben
                    \item $k<j$: $T[s+j]$ kommt in P links von Stelle j vor und $j-k>0$. Verschiebe P um $j-k$.
                \end{enumerate}
            \end{proof}
            Brauchen für jedes $x \in \Sigma$ die am weitesten rechts stehende Position von $a$ in $P$ also $a = P[\lambda(a)]$.
\begin{verbatim}
Berechne Lambda:

for all a \in Sigma do
    lambda(a) <- 0
od
for j <- 1 to m do
    \lambda[P[j]] <- j
od
\end{verbatim}

            \subsubsection{Gutes Suffix}
                \emph{Idee:} Falls Mismatch $P[j] \neq T[s+j]$ verschiebe P um 
                $$
                    \gamma[j] \rightarrow m - \max \set{k | P[1..k] \text{ ist Suffix von }P[j+1,..,m]}
                $$
                Ähnlich wie Präfixfunktion
                
        \subsection{Random Ansatz: Fingerabdruck (Karp/Rabin)}
            Sei $\Sigma = \set{0,...,9}$, $x$, $y$ sind dann natürliche Zahlen. $F_p: \mathbb{Z} \rightarrow \mathbb{Z}_p$ mit $F_P(z) = z \mod p$ und $p$ Primzahl.
            
            \subsubsection{Algorithmus}
\begin{verbatim}
p <- zufällige Primzahl zwischen 2 und mn^2log(mn^2)
match <- false
i <- 1

while not match and 1 <= i <= n-m+1 do
    if F_p(X(i)) = F_p(Y) then
        match <- true
    else
        i <- i+1
        berechne F_p(X(i))
    fi
od
\end{verbatim}
        \subsubsection{Beispiel}
        Sei X: 2 3 7 5 8 6 1 2 3 5 und m=4
        $$
            F_p(X(i+1)) = (10(F_p(X(i)) - 10^{m-1}x_i)+x_{i+m}) \mod p
        $$
        Es ist nötig $10^{m-1}$ vorzuberechnen, ansonsten geht es in $\LO(1)$
        
        \begin{satz}
            Algorithmus von Karp/Rabin läuft in $\LO(n+m)$ und hat Fehlerwahrschienlichkeit von $\LO\left(\frac{1}{n}\right)$
        \end{satz}
        
        \begin{proof}
            Was ist $\text{prob}(F_p(Y)=F_p(X(i)) | Y \neq X(i))$. Sei $a = Y$, $b = X(i)$. $a \mod p = b \mod p \Leftrightarrow p | (a-b)$ Es gilt $\abs{a-b} \leq 10^m \leq 2^{4m}$. Damit $\abs{a-b} < 2^{4m}$ hat $\leq(4m)$ verschiedene Primfaktoren. Haben p aus dem Bereich $[2, mn^2\log(mn^2)]$. In diesem bereich gibt es
            $$
                \LO\left( \frac{mn^2\log(mn^2)}{\log(mn^2\log(mn^2)}\right) = \LO(mn^2)
            $$
            Primzahlen. Für festes $i$ gilt also
            $$
                \text{prob}(F_p(Y) = F_p(X(i)) | Y \neq X(i)) = \LO \left(\frac{4m}{mn^2}\right) = \LO \left(\frac{1}{n^2} \right)
            $$
            Gesamt also
            $$
                \LO((n-m+1) \frac{1}{n^2}) = \LO(\frac{1}{n})
            $$
        \end{proof}
        
        \subsubsection{Von Monte Carlo nach Las Vegas}
        \begin{enumerate}
            \item Falls $F_p(X(i)) = F_p(Y)$ teste in $\LO(m)$ ob $X(i) = y$. Falls nicht starte naiven Algorithmus. Das ergibt eine erwartete Laufzeit von 
            $$
                \LO((n+m)(1-\frac{1}{n})+(n\cdot m) \frac{1}{n}) = \LO(n+m)
            $$
            \item Falls $F_p(X(i)) = F_p(Y)$ teste in $\LO(m)$ ob $X(i) = y$. Falls nicht starte von dort mit neuem zufälligem $p$. Das ergibt eine neue Fehlerwahrscheinlichkeit von $\text{prob(t. Iteration des Algorithmus)}=\LO(\frac{1}{n^t})$       
        \end{enumerate}
        
        
    \subsection{Preprocessing von Suffixbäumen}
        Sei S ein String der Länge $n$ und $P$ ein Pattern der Länge $m$. Preprocessing des Patterns geht in $\LO(m)$. Jetzt machen wir ein Preprocessing des Strings in $\LO(n)$. Damit erhalten wir insgesamt eine Laufzeit von $\LO(m)$ anstatt $\LO(n+m)$.
        
        \subsubsection{Beispiel zu Suffixbäumen}
            String sei a a b b a b b und einem Sonderzeichen am Ende.
            
            % Hier Baum einfügen
            
            Patternsuche geht einfach: Ablesen des Patterns im Baum. Ablesen der Stellenzahlen unter den gefundenen Patterns.
            
        \subsubsection{Definition}
            Baum $T$ für String $S$ hat n Blätter, nummeriert von 1 bis $n$
            \begin{itemize}
                \item Jeder innere Knoten hat $\geq 2$ Kinder
                \item Jede Kante ist mit Teilwort aus $S$ markiert
                \item Kanten aus demselben Knoten haben Markierungen mit versch. ersten Zeichen
                \item Konkatentationen der Kantenmarkierungen von Wurzel zum Blatt $i$ ergibt den Suffix der bei $i$ beginnt.
            \end{itemize}
            Laufzeit für Patternsuche ist $\LO(m+k)$ wenn es $k$ Vorkommen des Pattern gibt. Wenn ein Baum bereits aufgebaut ist geht die Suche in $\LO(\abs{P})$ wobei $\abs{P}$ die Länge des Patterns ist.
            
        \subsubsection{Aufbau des Baumes}
            \begin{enumerate}
                \item Starte mit Einzelkante für $S[1...n]$
                \item \emph{Iteriere:} Sei $T_i$ Baum der Suffixe $S[1...n],...,S[i...n]$. Starte in Wurzel. Suche einen möglichst Langen Pfad für Suffix $S[i+1...n]$ in $T_i$, der mit Präfix von $S[i+1...n]$ übereinstimmt. 
                \begin{enumerate}
                    \item Pfad endet auf Kante $(u,v)$. Dort endet Übereinstimmung. Teile $(u,v)$ in $(u,w)$ und $(w,v)$ und markiere entsprechend. Füge neue Kante an $(w, \text{Blatt}(i+1)$ ein mit Markierung restl. Suffix von $S[i+1...n]$
                    \item Pfad endet auf Knoten $v$. Füge neue Kante $(w, \text{Blatt}(i+1)$ ein
                \end{enumerate}
            \end{enumerate}
            Laufzeit ist $\LO(\sum_{i=1}^{n}{i}) = \LO(n^2)$
            
            
        \subsubsection{Verallgemeinerung}
            Es gebe nun mehrere Strings $S_1,S_2,...,S_q$, die mit $\$_1,\$_2,...$ enden. Wir konkatenieren nun die Strings und bauen Suffixbaum für den Superstring auf in $\LO(\abs{S_1} + \abs{S_2} +...)$. Da bei diesem Vorgehen über Stringgrenzen hinweg Suffixe entstehen, muss man bei einem Endzeichen ein Blatt entstehen lassen.
            
            %Hier fehlt ein Bsp mit ab$_1aa$_2
            
            Gegeben sei nun $S_1,S_2$ und wir suchen den größten gemeinsamen Teilstring in den beiden Strings.
            \begin{enumerate}
                \item Konstruiere Suffixbaum für $S_1$ und $S_2$
                \item Innere Knoten werden mit 1/2 markiert falls sie zum Blatt mit Suffixen aus $S_1$ bzw $S_2$ führen.
                \item Gemeinsame Teilstrings werden repräsentiert durch Pfade zu doppelt beschrifteten Knoten
                \item Tiefensuche von Wurzel über solche 1/2 Knoten liefert größten gemeinsamen Teilstring
            \end{enumerate}
            Laufzeit ist $\LO(\abs{S_1}+\abs{S_2})$
            
        \subsubsection{Korollare}
            \begin{enumerate}
                \item Gegeben sei ein String S und Pattern $P_1,P_2,...$. Finde alle Vorkommen von $P_1,P_2,...,P_k$. Dies geht in Zeit $\LO(\sum_{i=1}^{k}{\abs{P_i}+z})$ wenn $z$ die Anzahl der Vorkommen ist.   
                \item Gegeben seien $S_1,S_2$. Suche alle maximalen Wiederholungen.
\begin{verbatim}
S_1:  abbacbaa
S_2:  abbcbbbb
\end{verbatim}
                Dies geht analog zum größten gemeinsamen Teilstring
                \item Suche den am häufigsten Vorkommenden Teilstring. Das ist der Teilstring bis zum Knoten mit den meisten Kindern, da jedes Kind für ein Vorkommen des Knotens steht. Am einfachsten geht das, wenn man sich bei jedem Knoten die Anzahl der Kinder merkt
                \item Suche die kürzesten Pattern aus $\Sigma*$, die nicht in $S_i$ vorkommen. Laufe BFS zur ersten Kante, die Mehrfachbeschriftungen hat, oder wo eine Kante aus $\Sigma$ fehlt. $\LO(\Sigma \abs{S})$
                \item Suche das längste Palindrom in String $S$. Wir konkatenieren dazu $S$ mit $\overline{S}$ und suchen den längsten Teilstring.    
            \end{enumerate}
            
    \subsection{Anwendung: Datenkompression}
        Das Zvi-Lempel Verfahren.
        \begin{definition}
            Gegeben sei String $S$ der Länge $n$. $\text{Prior}_i$ sei das längste Präfix von $S[i..n]$, welches zugleich Teilstring aus $S[1... i-1]$ ist. Sei $l_i \leftarrow \abs{\text{Prior}_i}$ und für $l_i > 0$ sei $s_i$ die Anfangsposition des ersten Vorkommens von $\text{Prior}_i$ in $S[i... i-1]$.
        \end{definition}
\begin{verbatim}
Bsp: S = abaxcabaxabz
Prior_7 = bax
l_7 = 3
s_7 = 2
\end{verbatim}
        \emph{Idee:} Falls $S[1.. i-1]$ schon dargestellt und $l_i > 0$, so stelle die nächsten $l_i$ Stellen nicht explizit dar, sondern durch das Paar $(s_i,l_i)$           
\begin{verbatim}
Bsp: S = abacabaxabz
-> ab(1,1)c(1,3)x(1,2)z
S = (ab)^16
-> ab(1,2)(1,4)(1,8)

Algorithmus:
i <- 1
repeat berechne l_i, s_i
    if l_i >0 then
        return (s_i,l_i)
        i<- i+l_i
    else
        return S[i]
        i <- i+1
until i>n
\end{verbatim}
            

% Copyright \copyright\ 2015 Lennard Boden, David-Elias Künstle
% Es wird die Erlaubnis gegeben, dieses Dokument unter den Bedingungen der von der Free
% Software Foundation veröffentlichen  GNU Free
% Documentation License (Version 1.2 oder neuer) zu kopieren, verteilen und/oder
% zu verändern. Eine Kopie dieser Lizenz ist unter
% http://www.gnu.org/copyleft/fdl.txt erhältlich.
%
% Zusätzlich muss jede Kopie/Aktualisierung wieder über die Seite
% der Fachschaft Informatik der Uni Tübingen
% den Studenten zur Verfügung gestellt werden
% http://www.fsi.uni-tuebingen.de/

\chapter{Dynamisches Programmieren} 			

\section{Idee:} optimale Lösung setzt sich aus optimalen Lösungen für kleinere Teilprobleme zusammen löse rekursiv und setze zusammen.\\
\ \\
Beispiel: Schnittproblem für Stab der Länge n.\\
$\rightarrow$ zerschneide Stab in kleine Stücke der Länge\\
$\rightarrow$ Teil der Länge i kostet pi\\
$\rightarrow$ maximiere Gesamterlös $r_n$.\\
1. Lass Stab ganz  	   $p_n + r_n$\\
2. 1 Stück mit Länge 1  $p_1 + r_{n-1}$\\
3. 1 Stück mit Länge 2  $p_2 + r_{n-2}$\\

\section{Schnittproblem}
\textbf{Algorithmus:}

\begin{algorithmic}[1]
\Function{Schnitt}{n,p}
\Comment{Input: n, Preisliste p}
\State q := $- \infty$;
\If{n = 0} 
\State{return 0;}
\EndIf
\For{i $\in [1, n]$} 
	\State{q := max(q, p[i] + Schnitt(n-1);}
\EndFor
\Return q;
\EndFunction 
\end{algorithmic}

Analyse:
$T(n) = \text{Aufruf von } Schnitt()$
$T(0) = 1$
$T(n) = 1 + \Sigma_j = 0n-1(T(j))$
\\
\textbf{dynamische Programmierung:} berechne Teillösungen!\\
\begin{algorithmic}
\Function{Schnitt}{n,p}
\State r[0] = 0;
\For{j $\in [1, n]$}
	\State{
	\State $q := \infty;$\\
	\For{i $\in [1, j]$}
		\State{q := max(q, p[i] + r[j-i])}
	\EndFor
	r[j] := q;}
\EndFor
 \Return r[n];
\EndFunction
\end{algorithmic}
Laufzeit:\\
$T(n) = \sum\nolimits_{j = 1}^n \dfrac{n(n+1)}{2} = \mathcal O(n^2)$\\
\textbf{Beachte}: Größe $j$ wird erst betrachtet, wenn die kleineren Größen alle feststehen.\\
Rekonstruktion der optimalen Lösung:\\

Merke für r[j] den 1. Schritt; also ersetze q = max(q, ...) durch\\ 
\begin{algorithmic}
\If{q < p[i] + r[j-1]}
	\State{
	q := p[i] + r[j-1];\\
	s[j] := i;}
\EndIf
\end{algorithmic}

Das Stück s[j] := i heißt, das Stück am Stab der Länge 1.\\

\section{Matrizenmultiplikation}

Matrizenmultiplikation $A \cdot B$\\
Voraussetzung: Anzahl Spalten von A = Anzahl Zeilen von B.\\
\begin{algorithmic}
\For{i $\in [1, A.Zeilen()]$}
 \State{
	\For{j $\in [1, B.Spalten()]$}  \State{
		$c_{i, j} := 0$\\
		\For{k $\in [1, A.Spalten()]$}  \State{
		$c_{i, j} := c_{i, j} + a_{i, k} \cdot b_{k, j}$}
		\EndFor
	}
	\EndFor
}
\EndFor
\Return C;
\end{algorithmic}
Berechne $A \cdot B \cdot C \cdot D \cdot E \cdot F$\\
Bsp.: \\
A := (4 x 2)\\
A $\cdot$ B [:= (2 x 3)] $\Rightarrow$ ergibt [4 x 3] $4 \cdot 3 \cdot 2 = 24$ Operationen\\
B $\cdot$ C [:= (3 x 1)] $\Rightarrow$ ergibt [4 x 1] $4 \cdot 1 \cdot 3 = 12$ Operationen\\
C $\cdot$ D [:= (1 x 2)] $\Rightarrow$ ergibt [4 x 2] $4 \cdot 2 \cdot 1 = 8$ Operationen\\
D $\cdot$ E [:= (2 x 2)] $\Rightarrow$ ergibt [4 x 2] $4 \cdot 2 \cdot 2 = 16$ Operationen\\
E $\cdot$ F [:= (2 x 3)] $\Rightarrow$ ergibt [4 x 3] $4 \cdot 3 \cdot 2 = 24$ Operationen\\	
Reihenfolge spielt große Rolle: \\
$(((((A \cdot B) \cdot C) \cdot D) \cdot E) \cdot F) \rightarrow 84$ Operationen\\
$(A \cdot (B \cdot (C \cdot (D \cdot (E \cdot F))))) \rightarrow 69$ Operationen\\
\ \\
$\rightarrow$ brauche \textbf{Optimale Klammerung}\\
Teilproblem: $A_i, ..., A_j$ wird geklammert $(A_i... A_k)(A_{k+1}...A_j)$\\
d. h. $\forall i \leq k \leq j$\\
berechne $(A_i... A_k)$ und $(A_{k+1}...A_j)$; multipliziere $A_{i,k}$ mit $A_{k+1, j}$. Wähle dazu bestes $k$.\\
\textbf{Beachte:} Innerhalb der Teilprobleme muss Klammerung optimal sein.\\
Also berechne nun Kosten für alle Teilprobleme $A_i \cdot \cdot \cdot A_j$ für alle $1 \leq i \leq j \leq n$.
$\Rightarrow$ MIN[1, n] ist Minimum aller Multiplikationen für $A_1 \cdot A_2 \cdot ... \cdot A_n$.\\
Sei $A_i$ Matrix mit Dimension $p_{i_1} \times p_i$\\
rekursiv $\Rightarrow$ MIN[i, j] = min$\lbrace MIN[i, k] + MIN[k+1, j] + p_{i-1} \cdot p_k \cdot p_j\rbrace$\\
MIN[i, j] = 0 für $i = j$\\
Speichern noch $s[i, j]$ ab mit $k$ als den Wert, der MIN[$i, j$] bestimmt.\\
\begin{algorithmic}
\State n := Anzahl Matrizen;
\For{i $\in [1, n]$}  \State{
	MIN[i, i] := 0;}
\EndFor
\For{l $\in [2, n]$}  \State{
	\For{i $\in [1, n-l+1]$} \State{
		j := i + l -1;\\
		MIN[i, j] := $\infty$;\\
		\For{k $\in [i, j-1]$}  \State{
			q := MIN[i, k] + MIN[k+1, j] + $p_{i-1} \cdot p_k \cdot p_j$
			\If{q < Min[i, j]} \State{
				MIN[i, j] := q;\\
				s[i, j] := k;
			}
			\EndIf
		}
		\EndFor
	}
	\EndFor
}
\EndFor
\Return (MIN[1, n], s);
\end{algorithmic}
\textbf{Laufzeit:} 3 geschachtelte Vorschleifen mit Länge $\leq n$\\
innerer Schleifenrumpf: $\mathcal O(1)$\\
$\Rightarrow \mathcal O(n \cdot n \cdot n) = \mathcal O(n^3)$\\
genauer: l = 2, ..., n-1\\
$\sum\nolimits_{l = 2}^{n-1}(n - l)\cdot(l-1) = \sum\nolimits_{l = 2}^{n-1}(n-l-1) \cdot l = \sum\nolimits_{l = 2}^{n-1} nl - l^2 - l$
\ \\
\ \\
\ \\
\section{Greedy-Algorithmen}
\ \\
Optimierungsproblem: lokale Entscheidungen\\
$\rightarrow$ globales Optimum\\
\ \\

\subsection{Beispiel: Jobauswahl}
$n$ Jobs $a_1, ..., a_n$ mit Startzeit $s_i$, Endzeit $e_i$. $a_i$ und $a_j$ kompatibel, falls Intervalle nicht überlappen\\
Aufgabe: Maximale Anzahl kompatibler Jobs.\\
Bsp:\\
\begin{tabular}{c c c c c c c c c c c c}
$a_i$ & 1 & 2 & 3 & 4 & 5 & 6 & 7 & 8 & 9 & 10 & 11\\
$s_i$ & 1 & 3 & 0 & 5 & 3 & 5 & 6 & 8 & 8 & 2 & 12\\
$e_i$ & 4 & 5 & 6 & 7 & 9 & 9 & 10 & 11 & 12 & 14 & 16\\
\end{tabular}

besser: $\lbrace a_1, a_4, a_9, a_11 \rbrace$\\
Geht es besser?\\
\ \\
\subsection{dynamisches Programmieren}
optimale Lösung besteht aus optimalen Teillösungen\\
$\rightarrow$ Wähle die beste Kombination aus optimalen Teillösungen.\\
\underline{Greedy:} Betrachte nur \emph{eine} Kombination, nicht alle.\\
Seien $S_{i, j}$ die Jobs, die starten, nachdem $a_i$ endet, und die enden, bevor $a_j$ startet.\\
Suche maximale Anzahl kompatibler Jobs in $S_{i, j}$.\\
Sei $A_{i, j}$ so eine Menge und sei $a_k \in A_{i, j}$. Ist $a_k \in$ OPT, so ergeben sich Teilprobleme $S_{i, k}$ und $S_{k, j}$ mit \\
$A_{i, k} = A_{i, j} \cap S_{i, k}$ und $A_{k, j} = A_{i, j} \cap S_{k, j}$ und $A_{i, j} = A_{i, k} \cup \lbrace a_k \rbrace \cup A_{k, j}$\\
Optimale Menge $A_{i, j}$ setzt sich zusammen aus $a_k$ und sich ergebenen optimalen Mengen $S_{i, k}$ und $S_{k, j}$.\\
$\rightarrow$ \emph{Dynamisches Programmieren:} $c[i, j]$ optimale Anzahl von Jobs für $S_{i, j}$.\\
$c[i, j] = 0$ falls $S_{i, j} = \emptyset$\\
$c[i, j] = max\lbrace c[i, k] + c[k, j] +1 \rbrace$ ($a_k \in S_{i, j}$\\
$\rightarrow$ Laufzeit: $\mathcal O(n^3)$\\
\ \\
\textbf{Name:} Max. unabhängige Menge in Intervallgraphen\\
\underline{Greedy:} \\
\emph{Idee: } Wähle den Job als ersten, der als erstes endet.\\
$\rightarrow a_1$: lasse alle weg, die mit $a_1$ nicht kompatibel sind.\\
$\rightarrow$ Iteriere.\\
Einfach, wenn Jobs sortiert nach Endzeitpunkten.\\
Zeige durch Widerspruch, dass allgemein immer eine optimale Lösung existiert, wo der Job mit dem frühesten Ende drin ist.\\
In optimaler Lösung: Kann den 1. Job, falls er nicht am frühesten endet, durch einen anderen Job ersetzen, der früher endet (Der Rest der Jobs davon nicht betroffen).\\

\textbf{Algorithmus:}\\
Felder $s, e$ halten Index $k$ ($k$ letzter gefundener Job)\\
Jobs nach Endzeit sortiert.\\
Aufruf Jobauswahl(0) liefert $a_o$ fiktiv mit $c_0 = 0$\\
\begin{algorithmic}
\Function{Jobauswahl}{k}
$i := k+1$;\\
\While{$i \leq n \wedge s[i] < e[k]$} \State{
	$i++;$}
\EndWhile
\If{$i \leq n$} \State{
	\textbf{return} $\lbrace a_i \rbrace$ und Jobauswahl($i$)}
\Else \State{\textbf{return} $\emptyset$}
\EndIf
\EndFunction
\end{algorithmic}
Laufzeit: $\mathcal O(n)$ jeder Job nur einmal betrachtet\\
\textbf{iterativ:}

\begin{algorithmic}
\State A := $\lbrace A_1 \rbrace$; 
\State $ k := 1$;
\For{$i \in [2, n]$}  \State{
	\If{$s[1 \geq e[k]$} \State{
		A := A $\cup \lbrace a_i \rbrace$;
		$k$ := 1;}
	\EndIf}
\EndFor
\Return A;
\end{algorithmic}\ \\

Allgemein:\\
\begin{itemize}
	\item[-] bestimme Teilstruktur des Problems
	\item[-] rekursive Lösung
	\item[-] Greedy-Variante: nur 1 Teilproblem
	\item[-] zeige, dass Greedy zum Optimum führt
	\item[-] rekursiver Algorithmus
	\item[-] iterativer Algorithmus
\end{itemize}

\emph{Geht das immer?}\\
Beispiel: Gewichtetes Rucksackproblem\\
$n$ Gegenstände $g_1, ..., g_n$; $g_i$ hat Wert $v_i$ und Gewicht $w_i$. Träger kann $\leq W$ Kilogramm tragen.\\
Maximiere Gesamtwert der Ladung.\\

\textbf{dynamische Programmierung:}\\
Wertvollste Ladung mit $\leq W$ Kg Gewicht besteht aus einem $g_j$ mit Wert $v_j$ und dem Rest mit Gewicht $\leq W - w_j$.\\
\underline{Greedy:}\\
Strategie: 
\begin{itemize}
	\item[-] $g_j$ mit größtem Wert
 	\item[-] $g_j$ mit kleinstem Gewicht
 	\item[-] $g_j$ mit größtem Wert pro Kilogramm
\end{itemize}
Beispiel:\\
kleinstes Gewicht; w = 50\\
\begin{tabular}{c c c}
$w_1 = 10$ & $v_1 = 280$ & $\rightarrow w_1, w_2 \rightarrow 380$ \\
$w_2 = 20$ & $v_2 = 100$ & ""\\
$w_3 = 30$ & $v_3 = 120$ & besser ist $w_1, w_3$\\
\end{tabular}
\\
\\
\\
größter Wert pro Kilo;\\
\begin{tabular}{c c c}
$w_1 = 10$ & $v_1 = 60$ & $\rightarrow (greedy) w_1, w_2 \rightarrow 380$ \\
$w_2 = 20$ & $v_2 = 100$ & ""\\
$w_3 = 30$ & $v_3 = 120$ & besser ist $w_2, w_3$\\
\end{tabular}
\\
\\
\\
\textbf{dynamische Programmierung:} $\mathcal O(n \cdot W)$\\
------------------------------------------------------------------\\
1 | 2 | 3 |.......... |17| ...... |Q|.............................|W|\\
------------------------------------------------------------------\\
 Wertvollste Ladung mit Gewicht = 23
(//sollte überarbeitet werden)\\
\\
\emph{Variante:} Gegenstände sind teilbar.\\
\underline{Greedy:} (funktioniert hier) \\

\begin{algorithmic}
\State{Ordne Gegenstände nach Wert pro Kilogramm;}
\State{Lege $g_1$ in den Rucksack;}
\State $i := 2$
\While{W nicht erreicht}
  \State{
	Stecke $g_i$ in den Rucksack;
	i++;}
\EndWhile
\Comment{Teile $g_i$ auf und fülle Rucksack mit Teil von $g_i$;}
\end{algorithmic}
Laufzeit: $\mathcal O (n)$


\listoffigures
\addcontentsline{toc}{chapter}{\listfigurename}
\listofalgorithms
\addcontentsline{toc}{chapter}{\listalgorithmname}

\printindex
\end{document}
