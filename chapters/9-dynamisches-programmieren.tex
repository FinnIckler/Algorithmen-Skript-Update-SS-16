% Copyright \copyright\ 2015 Lennard Boden, David-Elias Künstle
% Es wird die Erlaubnis gegeben, dieses Dokument unter den Bedingungen der von der Free
% Software Foundation veröffentlichen  GNU Free
% Documentation License (Version 1.2 oder neuer) zu kopieren, verteilen und/oder
% zu verändern. Eine Kopie dieser Lizenz ist unter
% http://www.gnu.org/copyleft/fdl.txt erhältlich.
%
% Zusätzlich muss jede Kopie/Aktualisierung wieder über die Seite
% der Fachschaft Informatik der Uni Tübingen
% den Studenten zur Verfügung gestellt werden
% http://www.fsi.uni-tuebingen.de/

\chapter{Dynamisches Programmieren} 			

\section{Idee:} optimale Lösung setzt sich aus optimalen Lösungen für kleinere Teilprobleme zusammen löse rekursiv und setze zusammen.\\
\ \\
Beispiel: Schnittproblem für Stab der Länge n.\\
$\rightarrow$ zerschneide Stab in kleine Stücke der Länge\\
$\rightarrow$ Teil der Länge i kostet pi\\
$\rightarrow$ maximiere Gesamterlös $r_n$.\\
1. Lass Stab ganz  	   $p_n + r_n$\\
2. 1 Stück mit Länge 1  $p_1 + r_{n-1}$\\
3. 1 Stück mit Länge 2  $p_2 + r_{n-2}$\\

\section{Schnittproblem}
\textbf{Algorithmus:}

\begin{algorithmic}[1]
\Function{Schnitt}{n,p}
\Comment{Input: n, Preisliste p}
\State q := $- \infty$;
\If{n = 0} 
\State{return 0;}
\EndIf
\For{i $\in [1, n]$} 
	\State{q := max(q, p[i] + Schnitt(n-1);}
\EndFor
\Return q;
\EndFunction 
\end{algorithmic}

Analyse:
$T(n) = \text{Aufruf von } Schnitt()$
$T(0) = 1$
$T(n) = 1 + \Sigma_j = 0n-1(T(j))$
\\
\textbf{dynamische Programmierung:} berechne Teillösungen!\\
\begin{algorithmic}
\Function{Schnitt}{n,p}
\State r[0] = 0;
\For{j $\in [1, n]$}
	\State{
	\State $q := \infty;$\\
	\For{i $\in [1, j]$}
		\State{q := max(q, p[i] + r[j-i])}
	\EndFor
	r[j] := q;}
\EndFor
 \Return r[n];
\EndFunction
\end{algorithmic}
Laufzeit:\\
$T(n) = \sum\nolimits_{j = 1}^n \dfrac{n(n+1)}{2} = \mathcal O(n^2)$\\
\textbf{Beachte}: Größe $j$ wird erst betrachtet, wenn die kleineren Größen alle feststehen.\\
Rekonstruktion der optimalen Lösung:\\

Merke für r[j] den 1. Schritt; also ersetze q = max(q, ...) durch\\ 
\begin{algorithmic}
\If{q < p[i] + r[j-1]}
	\State{
	q := p[i] + r[j-1];\\
	s[j] := i;}
\EndIf
\end{algorithmic}

Das Stück s[j] := i heißt, das Stück am Stab der Länge 1.\\

\section{Matrizenmultiplikation}

Matrizenmultiplikation $A \cdot B$\\
Voraussetzung: Anzahl Spalten von A = Anzahl Zeilen von B.\\
\begin{algorithmic}
\For{i $\in [1, A.Zeilen()]$}
 \State{
	\For{j $\in [1, B.Spalten()]$}  \State{
		$c_{i, j} := 0$\\
		\For{k $\in [1, A.Spalten()]$}  \State{
		$c_{i, j} := c_{i, j} + a_{i, k} \cdot b_{k, j}$}
		\EndFor
	}
	\EndFor
}
\EndFor
\Return C;
\end{algorithmic}
Berechne $A \cdot B \cdot C \cdot D \cdot E \cdot F$\\
Bsp.: \\
A := (4 x 2)\\
A $\cdot$ B [:= (2 x 3)] $\Rightarrow$ ergibt [4 x 3] $4 \cdot 3 \cdot 2 = 24$ Operationen\\
B $\cdot$ C [:= (3 x 1)] $\Rightarrow$ ergibt [4 x 1] $4 \cdot 1 \cdot 3 = 12$ Operationen\\
C $\cdot$ D [:= (1 x 2)] $\Rightarrow$ ergibt [4 x 2] $4 \cdot 2 \cdot 1 = 8$ Operationen\\
D $\cdot$ E [:= (2 x 2)] $\Rightarrow$ ergibt [4 x 2] $4 \cdot 2 \cdot 2 = 16$ Operationen\\
E $\cdot$ F [:= (2 x 3)] $\Rightarrow$ ergibt [4 x 3] $4 \cdot 3 \cdot 2 = 24$ Operationen\\	
Reihenfolge spielt große Rolle: \\
$(((((A \cdot B) \cdot C) \cdot D) \cdot E) \cdot F) \rightarrow 84$ Operationen\\
$(A \cdot (B \cdot (C \cdot (D \cdot (E \cdot F))))) \rightarrow 69$ Operationen\\
\ \\
$\rightarrow$ brauche \textbf{Optimale Klammerung}\\
Teilproblem: $A_i, ..., A_j$ wird geklammert $(A_i... A_k)(A_{k+1}...A_j)$\\
d. h. $\forall i \leq k \leq j$\\
berechne $(A_i... A_k)$ und $(A_{k+1}...A_j)$; multipliziere $A_{i,k}$ mit $A_{k+1, j}$. Wähle dazu bestes $k$.\\
\textbf{Beachte:} Innerhalb der Teilprobleme muss Klammerung optimal sein.\\
Also berechne nun Kosten für alle Teilprobleme $A_i \cdot \cdot \cdot A_j$ für alle $1 \leq i \leq j \leq n$.
$\Rightarrow$ MIN[1, n] ist Minimum aller Multiplikationen für $A_1 \cdot A_2 \cdot ... \cdot A_n$.\\
Sei $A_i$ Matrix mit Dimension $p_{i_1} \times p_i$\\
rekursiv $\Rightarrow$ MIN[i, j] = min$\lbrace MIN[i, k] + MIN[k+1, j] + p_{i-1} \cdot p_k \cdot p_j\rbrace$\\
MIN[i, j] = 0 für $i = j$\\
Speichern noch $s[i, j]$ ab mit $k$ als den Wert, der MIN[$i, j$] bestimmt.\\
\begin{algorithmic}
\State n := Anzahl Matrizen;
\For{i $\in [1, n]$}  \State{
	MIN[i, i] := 0;}
\EndFor
\For{l $\in [2, n]$}  \State{
	\For{i $\in [1, n-l+1]$} \State{
		j := i + l -1;\\
		MIN[i, j] := $\infty$;\\
		\For{k $\in [i, j-1]$}  \State{
			q := MIN[i, k] + MIN[k+1, j] + $p_{i-1} \cdot p_k \cdot p_j$
			\If{q < Min[i, j]} \State{
				MIN[i, j] := q;\\
				s[i, j] := k;
			}
			\EndIf
		}
		\EndFor
	}
	\EndFor
}
\EndFor
\Return (MIN[1, n], s);
\end{algorithmic}
\textbf{Laufzeit:} 3 geschachtelte Vorschleifen mit Länge $\leq n$\\
innerer Schleifenrumpf: $\mathcal O(1)$\\
$\Rightarrow \mathcal O(n \cdot n \cdot n) = \mathcal O(n^3)$\\
genauer: l = 2, ..., n-1\\
$\sum\nolimits_{l = 2}^{n-1}(n - l)\cdot(l-1) = \sum\nolimits_{l = 2}^{n-1}(n-l-1) \cdot l = \sum\nolimits_{l = 2}^{n-1} nl - l^2 - l$
\ \\
\ \\
\ \\
\section{Greedy-Algorithmen}
\ \\
Optimierungsproblem: lokale Entscheidungen\\
$\rightarrow$ globales Optimum\\
\ \\

\subsection{Beispiel: Jobauswahl}
$n$ Jobs $a_1, ..., a_n$ mit Startzeit $s_i$, Endzeit $e_i$. $a_i$ und $a_j$ kompatibel, falls Intervalle nicht überlappen\\
Aufgabe: Maximale Anzahl kompatibler Jobs.\\
Bsp:\\
\begin{tabular}{c c c c c c c c c c c c}
$a_i$ & 1 & 2 & 3 & 4 & 5 & 6 & 7 & 8 & 9 & 10 & 11\\
$s_i$ & 1 & 3 & 0 & 5 & 3 & 5 & 6 & 8 & 8 & 2 & 12\\
$e_i$ & 4 & 5 & 6 & 7 & 9 & 9 & 10 & 11 & 12 & 14 & 16\\
\end{tabular}

besser: $\lbrace a_1, a_4, a_9, a_11 \rbrace$\\
Geht es besser?\\
\ \\
\subsection{dynamisches Programmieren}
optimale Lösung besteht aus optimalen Teillösungen\\
$\rightarrow$ Wähle die beste Kombination aus optimalen Teillösungen.\\
\underline{Greedy:} Betrachte nur \emph{eine} Kombination, nicht alle.\\
Seien $S_{i, j}$ die Jobs, die starten, nachdem $a_i$ endet, und die enden, bevor $a_j$ startet.\\
Suche maximale Anzahl kompatibler Jobs in $S_{i, j}$.\\
Sei $A_{i, j}$ so eine Menge und sei $a_k \in A_{i, j}$. Ist $a_k \in$ OPT, so ergeben sich Teilprobleme $S_{i, k}$ und $S_{k, j}$ mit \\
$A_{i, k} = A_{i, j} \cap S_{i, k}$ und $A_{k, j} = A_{i, j} \cap S_{k, j}$ und $A_{i, j} = A_{i, k} \cup \lbrace a_k \rbrace \cup A_{k, j}$\\
Optimale Menge $A_{i, j}$ setzt sich zusammen aus $a_k$ und sich ergebenen optimalen Mengen $S_{i, k}$ und $S_{k, j}$.\\
$\rightarrow$ \emph{Dynamisches Programmieren:} $c[i, j]$ optimale Anzahl von Jobs für $S_{i, j}$.\\
$c[i, j] = 0$ falls $S_{i, j} = \emptyset$\\
$c[i, j] = max\lbrace c[i, k] + c[k, j] +1 \rbrace$ ($a_k \in S_{i, j}$\\
$\rightarrow$ Laufzeit: $\mathcal O(n^3)$\\
\ \\
\textbf{Name:} Max. unabhängige Menge in Intervallgraphen\\
\underline{Greedy:} \\
\emph{Idee: } Wähle den Job als ersten, der als erstes endet.\\
$\rightarrow a_1$: lasse alle weg, die mit $a_1$ nicht kompatibel sind.\\
$\rightarrow$ Iteriere.\\
Einfach, wenn Jobs sortiert nach Endzeitpunkten.\\
Zeige durch Widerspruch, dass allgemein immer eine optimale Lösung existiert, wo der Job mit dem frühesten Ende drin ist.\\
In optimaler Lösung: Kann den 1. Job, falls er nicht am frühesten endet, durch einen anderen Job ersetzen, der früher endet (Der Rest der Jobs davon nicht betroffen).\\

\textbf{Algorithmus:}\\
Felder $s, e$ halten Index $k$ ($k$ letzter gefundener Job)\\
Jobs nach Endzeit sortiert.\\
Aufruf Jobauswahl(0) liefert $a_o$ fiktiv mit $c_0 = 0$\\
\begin{algorithmic}
\Function{Jobauswahl}{k}
$i := k+1$;\\
\While{$i \leq n \wedge s[i] < e[k]$} \State{
	$i++;$}
\EndWhile
\If{$i \leq n$} \State{
	\textbf{return} $\lbrace a_i \rbrace$ und Jobauswahl($i$)}
\Else \State{\textbf{return} $\emptyset$}
\EndIf
\EndFunction
\end{algorithmic}
Laufzeit: $\mathcal O(n)$ jeder Job nur einmal betrachtet\\
\textbf{iterativ:}

\begin{algorithmic}
\State A := $\lbrace A_1 \rbrace$; 
\State $ k := 1$;
\For{$i \in [2, n]$}  \State{
	\If{$s[1 \geq e[k]$} \State{
		A := A $\cup \lbrace a_i \rbrace$;
		$k$ := 1;}
	\EndIf}
\EndFor
\Return A;
\end{algorithmic}\ \\

Allgemein:\\
\begin{itemize}
	\item[-] bestimme Teilstruktur des Problems
	\item[-] rekursive Lösung
	\item[-] Greedy-Variante: nur 1 Teilproblem
	\item[-] zeige, dass Greedy zum Optimum führt
	\item[-] rekursiver Algorithmus
	\item[-] iterativer Algorithmus
\end{itemize}

\emph{Geht das immer?}\\
Beispiel: Gewichtetes Rucksackproblem\\
$n$ Gegenstände $g_1, ..., g_n$; $g_i$ hat Wert $v_i$ und Gewicht $w_i$. Träger kann $\leq W$ Kilogramm tragen.\\
Maximiere Gesamtwert der Ladung.\\

\textbf{dynamische Programmierung:}\\
Wertvollste Ladung mit $\leq W$ Kg Gewicht besteht aus einem $g_j$ mit Wert $v_j$ und dem Rest mit Gewicht $\leq W - w_j$.\\
\underline{Greedy:}\\
Strategie: 
\begin{itemize}
	\item[-] $g_j$ mit größtem Wert
 	\item[-] $g_j$ mit kleinstem Gewicht
 	\item[-] $g_j$ mit größtem Wert pro Kilogramm
\end{itemize}
Beispiel:\\
kleinstes Gewicht; w = 50\\
\begin{tabular}{c c c}
$w_1 = 10$ & $v_1 = 280$ & $\rightarrow w_1, w_2 \rightarrow 380$ \\
$w_2 = 20$ & $v_2 = 100$ & ""\\
$w_3 = 30$ & $v_3 = 120$ & besser ist $w_1, w_3$\\
\end{tabular}
\\
\\
\\
größter Wert pro Kilo;\\
\begin{tabular}{c c c}
$w_1 = 10$ & $v_1 = 60$ & $\rightarrow (greedy) w_1, w_2 \rightarrow 380$ \\
$w_2 = 20$ & $v_2 = 100$ & ""\\
$w_3 = 30$ & $v_3 = 120$ & besser ist $w_2, w_3$\\
\end{tabular}
\\
\\
\\
\textbf{dynamische Programmierung:} $\mathcal O(n \cdot W)$\\
------------------------------------------------------------------\\
1 | 2 | 3 |.......... |17| ...... |Q|.............................|W|\\
------------------------------------------------------------------\\
 Wertvollste Ladung mit Gewicht = 23
(//sollte überarbeitet werden)\\
\\
\emph{Variante:} Gegenstände sind teilbar.\\
\underline{Greedy:} (funktioniert hier) \\

\begin{algorithmic}
\State{Ordne Gegenstände nach Wert pro Kilogramm;}
\State{Lege $g_1$ in den Rucksack;}
\State $i := 2$
\While{W nicht erreicht}
  \State{
	Stecke $g_i$ in den Rucksack;
	i++;}
\EndWhile
\Comment{Teile $g_i$ auf und fülle Rucksack mit Teil von $g_i$;}
\end{algorithmic}
Laufzeit: $\mathcal O (n)$
