% Copyright \copyright\ 2010  Volodymyr Piven und Alexander Peltzer.
% Es wird die Erlaubnis gegeben, dieses Dokument unter den Bedingungen der von der Free
% Software Foundation veröffentlichen  GNU Free
% Documentation License (Version 1.2 oder neuer) zu kopieren, verteilen und/oder
% zu verändern. Eine Kopie dieser Lizenz ist unter
% http://www.gnu.org/copyleft/fdl.txt erhältlich.
%
% Copyright der Aktualisierung und Überarbeitung \copyright\ 2013  Simon Kalt, Jan-Peter Hohloch, Tobias Fabritz
% Es wird die Erlaubnis gegeben, dieses Dokument unter den Bedingungen der von der Free
% Software Foundation veröffentlichen  GNU Free
% Documentation License (Version 1.2 oder neuer) zu kopieren, verteilen und/oder
% zu verändern. Eine Kopie dieser Lizenz ist unter
% http://www.gnu.org/copyleft/fdl.txt erhältlich.
%
% Zusätzlich muss jede Kopie/Aktualisierung wieder über die Seite
% der Fachschaft Informatik der Uni Tübingen
% den Studenten zur Verfügung gestellt werden
% http://www.fsi.uni-tuebingen.de/

\chapter{Komplexität}
    \section{$\LO$-Notation}
    $f(n) = \LO(g(n))$ (obere Schranke), wenn 
    $$
        \exists \ c ,\ n_0 > 0 \ \forall \ n \geq n_0: 0 \leq f(n) \leq c \cdot g(n)
    $$\\
    $f(n) = \Omega(g(n))$ (untere Schranke), wenn
    $$
    		\exists \ c ,\ n_0 > 0 \ \forall \ n \geq n_0: 0\leq c\cdot g(n)\leq f(n)
    $$\\
    $f(n) = \Theta(g(n))$ (scharfe Schranke), wenn
    $$
    		\exists \ c_1,c_2 ,\ n_0 > 0 \ \forall \ n \geq n_0: 0\leq c_1\cdot g(n)\leq f(n)\leq c_2\cdot g(n)
    $$
    \textit{Anmerkung:} \\
 	 Mit $f(n) = \LO(g(n))$ ist gemeint $f(n) \in \LO(g(n))$ \\
     $\Theta(g(n))=\LO(g(n))\cap \Omega(g(n))$
    
    \section{Nützliche Rechenregeln}
        Logarithmen:
        
        \begin{align*}
            \log(a \cdot b) &= \log a + \log b \\
            \log_a b &= \frac{\log b}{\log a} \\
            \log a^b &= b \cdot \log a \\
            b^{\log_b a} &= a 
        \end{align*}

        Andere nützliche Regeln:
        \begin{itemize}
            \item Stirling-Approximation:
            $$
                n! = \sqrt{2 \pi n} \cdot \left(\frac{n}{e}\right)^n \left(1+\LO\left(\frac{1}{n}\right)\right)
            $$
            Eine einfachere Abschätzung ist:
            $$
                \left(\frac{n}{2}\right)^{\frac{n}{2}} \leq n! \leq n^n
            $$
            \item Gauß'sche Summenformel:
            $$
                \sum \limits_{k=1}^n k = \frac{n(n+1)}{2}
            $$
            \item geometrische Reihe:
            \begin{align*}
                &\sum \limits_{k=0}^n q^k = \frac{q^{n+1}-1}{q-1} \\
                &\sum \limits_{k=0}^\infty q^k = \frac{1}{1-q} \ \ \text{  für } \abs{q} < 1
            \end{align*}
            Also zB für $q= \frac{1}{2}: 1 + \frac{1}{2} + \frac{1}{4} + \frac{1}{8} + \ldots = 2$ 
            \item Ebenfalls gilt (Ableitung):
            $$
                \sum \limits_{k=0}^\infty k \cdot q^{k-1} = \frac{1}{(1-q)^2} \ \ \text{  für } \abs{q} < 1
            $$ 
            \item n-te harmonische Zahl:
            $$
                H_n = 1 + \frac{1}{2} + \frac{1}{3} + \frac{1}{4} + \ldots + \frac{1}{n} = \sum \limits_{k=1}^n \frac{1}{k} = \ln n + \LO(1)
            $$
        \end{itemize}
