% Copyright \copyright\ 2010  Volodymyr Piven und Alexander Peltzer.
% Es wird die Erlaubnis gegeben, dieses Dokument unter den Bedingungen der von der Free
% Software Foundation veröffentlichen  GNU Free
% Documentation License (Version 1.2 oder neuer) zu kopieren, verteilen und/oder
% zu verändern. Eine Kopie dieser Lizenz ist unter
% http://www.gnu.org/copyleft/fdl.txt erhältlich.
%
% Copyright der Aktualisierung und Überarbeitung \copyright\ 2013  Simon Kalt, Jan-Peter Hohloch, Tobias Fabritz
% Es wird die Erlaubnis gegeben, dieses Dokument unter den Bedingungen der von der Free
% Software Foundation veröffentlichen  GNU Free
% Documentation License (Version 1.2 oder neuer) zu kopieren, verteilen und/oder
% zu verändern. Eine Kopie dieser Lizenz ist unter
% http://www.gnu.org/copyleft/fdl.txt erhältlich.
%
% Zusätzlich muss jede Kopie/Aktualisierung wieder über die Seite
% der Fachschaft Informatik der Uni Tübingen
% den Studenten zur Verfügung gestellt werden
% http://www.fsi.uni-tuebingen.de/
\chapter{Rekursionen}
\begin{itemize}
\item
Merge-Sort: 
$$
T(n) = \underbrace{2 \cdot T(\frac{n}{2})}_{\text{2 Teilprobleme}} + \underbrace{n}_{\text{Aufwand des Aufteilens}} \text{ und } T(1)=1
$$
\item 
Schaltkreise: $C(n) \leq C(\frac{2}{3}n) + \frac{n}{3}$ 
\item
parallele Algorithmen: $T(n) = T(\frac{9}{10}) + \LO(\log n)$ 
\item
Selection: $T(n) \leq \frac{2}{n} \cdot \sum \limits_{k=\frac{n}{2}}^{n-1} T(k) + \LO(n)$ 
\end{itemize}

\section{3 Methoden zur Berechnung}

\subsection{Schätzen und Beweisen}
\begin{itemize}
\item
$T(n) = 2 \cdot T\left( \frac{n}{2}\right) + n$ \\
Schätzen: $T(n) = \LO(n \log n)$ \\
Beweisen: Behauptung $T(n) \leq c \cdot n \log n$ für c groß genug. \\ %Warum +1?
Induktion: \\
Induktionsanfang: 
$$
T(1) = 1\ \checkmark
$$ 
Induktionsschritt: \\
Es gilt 
$$\begin{aligned}
T(n) &\leq 2 \cdot c \cdot \frac{n}{2} \cdot \log \left(\frac{n}{2}\right) + n \\
&= 2 \cdot c \cdot \frac{n}{2} \cdot (\log n - \log 2) + n \\
&= c \cdot n \cdot ((\log n) -1) + n \leq c \cdot n \log n 
\end{aligned} $$
für $c \geq 1$
\item
$C(n) \leq C ( \frac{2}{3} n) + \frac{n}{3}$ \\
Schätzen: $C(n) = \LO(n)$
\begin{proof}
$C(n) \leq d \cdot n$ für konstante d 
\[C(1) = 1 \leq d \text{ für }d \geq 1\]
$$C(n) \leq d \cdot \frac{2}{3} n + \frac{n}{3} \leq d(\frac{2}{3}n + \frac{n}{3})  \text{ für } d \geq 1 =d \cdot n$$ \end{proof}

\item
$A(n) = 2 \cdot A(\frac{n}{2}) + \sqrt{n}$ \\
Schätzen:$A(n) = \LO(n)$ \\
Versuchter Beweis:\\
$A(n) \leq c \cdot n$ für c groß.\\
$A(1) = 1 $ \\
$A(n) = 2 \cdot c \cdot \frac{n}{2} + \sqrt{n} \nleq c \cdot n$ so nicht!
\end{itemize}  

\subsection{Ausrechnen} 
\begin{align*}
A(n) &= 2 \cdot A\left(\frac{n}{2}\right) + \sqrt{n} \\
&= 2\cdot \left( 2 \cdot A\left(\frac{n}{4}\right) + \sqrt{\frac{n}{2}} \right) + \sqrt{n} \\
&= 4 \cdot A\left(\frac{n}{4}\right) + 2 \cdot \sqrt{\frac{n}{2}} + \sqrt{n} \\
&= 8 \cdot A\left(\frac{n}{8}\right) + 4 \sqrt{\frac{n}{4}} + 2 \sqrt{\frac{n}{2}} + \sqrt{n} \\
&\ldots \\
&= 2^i \cdot A \left( \frac{n}{2^i}\right) + \sum \limits_{j=0}^{i-1} 2^j \cdot \sqrt{\frac{n}{2^j}}  \ \tag{i-ter Schritt}\\
&= 2^i \cdot A\left(\frac{n}{2^i}\right) + \sqrt{n} \cdot \sum \limits_{j=0}^{i-1} 2^{\frac{j}{2}} 
\end{align*}
Schritt ist durch Induktion zu zeigen:
\begin{align*}
&= 2^i \cdot \left(2\cdot A\left(\frac{n}{2^{i+1}}\right) +\sqrt{\frac{1}{2^i}}\right) 
+ \sqrt{n} \cdot \sum \limits_{j=0}^{i-1} 2^{\frac{j}{2}}\\
&=2^{i+1}\cdot A\left(\frac{n}{2^{i+1}}\right) + \sqrt{n} \cdot \sum \limits_{j=0}^{i} 2^{\frac{j}{2}} \ \checkmark
\end{align*}

Sei $i = \log n$, da dies der maximale Fall ist, wenn $n$ eine Zweierpotenz: 
\begin{align*}
&2^{\log n} \cdot A(1) + \sqrt{n} \sum \limits_{j=0}^{\log n-1} \sqrt{2}^j \\
&= n + \sqrt{n} \cdot \LO(\sqrt{n}) \\
&= \LO(n)
\end{align*}

\subsection{Master Theorem}
Löse $T(n) = a \cdot T(\frac{n}{b}) + f(n) \ \ (a \geq 1,\ b > 1,\  f(n) \geq 0)$ \\
Beispielsweise für Merge-Sort: $a=b=2, f(n) = n$ \\

\begin{satz}
Sei $a \geq 1, b > 1$ konstant, $f(n) , T(n)$ nicht negativ mit 
$$
T(n) = a \cdot T \left(\frac{n}{b}\right) + f(n)
$$
wobei $\frac{n}{b}$ für $\lceil \frac{n}{b} \rceil$ oder $\lfloor \frac{n}{b} \rfloor$ steht. Dann gelten
\begin{enumerate}[1.]
\item Für $f(n) = O\left(n^{\log_b a - \epsilon}\right)\ , \epsilon > 0 \Rightarrow T(n) = \Theta(n^{\log_b a})$ 
\item Für $f(n) = \Theta \left(n^{\log_b a}\right) \Rightarrow T(n) = \Theta (n^{\log_b a} \cdot \log n)$ 
\item Für $f(n) = \Omega \left(n^{\log_b a + \epsilon}\right)\ , \epsilon > 0$ und $a \cdot f(\frac{n}{b}) \leq c \cdot f(n)$\\
für $c < 1$ und $n$ genügend groß, so ist: $T(n) = \Theta(f(n))$
\end{enumerate}
\end{satz}

Beispiel:
\begin{itemize}
\item
$T(n) = 9 \cdot T(\frac{n}{3}) +n$, $a=9, b=3, f(n) = n$ \\
$\log_3 9 = 2 \Ra $Fall 1 $\Ra T(n) = \Theta (n^2)$
\item
$T(n) = T(\frac{n}{2}) +1$, $a=1, b=2, f(n) = 1$ \\
$\log_b a= \log_2 1  = 0 \Ra $Fall 2 $\Ra T(n) = \Theta(\log n)$
\item $T(n) = 4 \cdot T(\frac{n}{2}) + n^4\\
f(n) = n^4 \geq \Theta(n^{2+2}) \checkmark\\
\frac{n^4}{4} = 4 \cdot (\frac{n}{2})^4 \leq c \cdot n^4 \checkmark$ Fall 3\\
$\Rightarrow T(n) = \Theta(n^4)$
\item
$T(n) = 2 \cdot T(\frac{n}{2}) + n \log n$, $a=2, b=2, f(n) = n \log n $ \\
$log_2 2 = 1$ \\
$n \log n < n^{1+\epsilon} \Ra $ nicht anwendbar! (da \q{zwischen} Fall 2 und 3!)
\end{itemize}
$T(n) = a \cdot T(\frac{n}{b}) + f(n)$, $a\geq 1,\ b > 1,\ f(n) \geq 0$ \\
\emph{Beweis}:\\
Annahme: $n=b^i, i \in \N$ \\
das heißt, Teilprobleme haben Größe $b^i, b^{i-1}, \ldots, b, 1$ \\
\begin{lemma} 
Sei $a,b \geq 1, f(n) \geq 0, n=b^i, i \in \N$ mit 
$$
T(n) = 
\begin{cases} 
\Theta (1) 					& \text{\normalfont für } n=1 \\ 
a \cdot T(\frac{n}{b}) + f(n) 	& \text{\normalfont für } n=b^i 
\end{cases}
$$

Dann gilt $T(n) = \Theta (n^{\log_b a}) + \sum \limits_{j=1}^{(\log_b n)-1} a^j f(\frac{n}{b^j})$

\begin{proof}
\begin{align*} 
T\left(n\right) &= f\left(n\right) + a \cdot T\left(\frac{n}{b}\right) \\
&= f\left(n\right) + a \left(f\left(\frac{n}{b}\right) + a \cdot T\left( \frac{n}{b^2}\right)\right) \\
&= f\left(n\right) + a \cdot f\left(\frac{n}{b}\right) + a^2 \cdot T\left(\frac{n}{b^2}\right) \\
&= f\left(n\right) + a \cdot f\left(\frac{n}{b}\right) 
+ a^2 \cdot T\left(\frac{n}{b^2}\right) + a^3 \cdot T\left(\frac{n}{b^3}\right) \\
&= \dots \\
&= f\left(n\right) + a \cdot f\left(\frac{n}{b}\right) + a^2 \cdot f\left(\frac{n}{b^2}\right) 
+ \ldots + a^i T\left(\frac{n}{b^i}\right) \\
&= f\left(n\right) + a \cdot f\left(\frac{n}{b}\right) + \ldots + \underbrace{a^{\log_b n} 
\cdot T\left(1\right)}_{=  \Theta\left(n^{\log_b a}\right)} \\
&= \Theta\left(n^{\log_b a}\right) 
+ \sum \limits_{j=0}^{\left(\log_b n\right)-1} a^j f\left(\frac{n}{b^j}\right)
\end{align*}
\end{proof}
\end{lemma}
\begin{figure}[htp]
\centering
\begin{tikzpicture}[->,node distance=3cm,tree node/.style={draw=none,fill=none}]
  \node[tree node] (1) {$f(n)$};
  \node[tree node] (2) [below left of=1] {$f(n/b)$};
  \node[tree node] (3) [below right of=1] {$f(n/b)$};
  \node[tree node] (4) [below left of=2] {$f(n/b^2)$};
  \node[tree node] (5) [below right of=3] {$f(n/b^2)$};
  \node[tree node] (6) at ($(2)!0.5!(3)$) {\dots $a$-mal \dots};
  \node[tree node] (7) [right of=5] {$a^2 \cdot f(n/b^2)$};
  \node[tree node] (8) [above of=7, yshift=-0.88cm] {$a \cdot f(n/b)$};
  \node[tree node] (9) [above of=8, yshift=-0.88cm] {$f(n)$};
  
  \path (1) edge node [left] {} (2)
            edge node [right] {} (3);
  \path (2) edge node [left] {} (4);
  \path (3) edge node [right] {} (5);
\end{tikzpicture}

\caption{Rekursionsbaum} % Bessere Caption?
\label{diag1:recursion-tree}
\end{figure}
3 Fälle, je nachdem wie groß $f(n)$ ist. \\
\textit{Anschaulich:}\\
Entweder dominiert die Wurzel (Fall 3) oder das Blattlevel (Fall 1) 
oder es fallen alle Terme gleich ins Gewicht (Fall 2, $\log_b n$ Terme) 
\begin{lemma}
Sei $a, b \geq 1$, $f(n)$ definiert auf Potenzen von b und sei $g(n) = \sum \limits_{j=0}^{\log_b n -1} a^j f\left(\frac{n}{b^j}\right)$
Ist $f(n) = \LO\left(n^{\log_b a - \epsilon}\right)$, $\epsilon \geq 0$, so ist $g(n) = \LO\left(n^{\log_b a}\right)$.

\begin{proof}
Es gilt:
\begin{align*}
f\left(\frac{n}{b^j}\right) &= \LO \left(\left(\frac{n}{b^j}\right)^{\log_b a -\epsilon}\right) \\
\Ra g(n) &= \LO \left(\sum \limits_{j = 0}^{\log_b n-1} a^j \cdot \left(\frac{n}{b^j}\right)^{\log_b a - \epsilon} \right) \\
&= \LO \left( n^{\log_b a - \epsilon} 
		\cdot \sum \limits_{j=0}^{\log_b n-1} \left(\frac{a \cdot b^\epsilon}{b^{\log_b a}}\right)^j \right) \\
&= \LO \left( n^{\log_b a - \epsilon} \cdot \sum \limits_{j=0}^{\log_b n-1} \left(b^\epsilon\right)^j \right) \tag{geom. Reihe}\\
&= \LO \left( n^{\log_b a - \epsilon} \cdot \frac{\left(b^\epsilon\right)^{\log_b n -1}}{b^\epsilon -1}\right)\\
&= \LO \left( n^{\log_b a - \epsilon} \cdot \underbrace{\frac{n^{\epsilon -1}}{b^\epsilon -1}}_{ \in\LO( n^\epsilon) }\right)\\ 
&= \LO \left( n^{\log_b a - \epsilon} \cdot n^\epsilon \right) = \LO \left(n^{\log_b a} \right)
\end{align*}    
\end{proof}
\end{lemma}
