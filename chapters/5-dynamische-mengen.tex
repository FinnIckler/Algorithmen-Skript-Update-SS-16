% Copyright \copyright\ 2010  Volodymyr Piven und Alexander Peltzer.
% Es wird die Erlaubnis gegeben, dieses Dokument unter den Bedingungen der von der Free
% Software Foundation veröffentlichen  GNU Free
% Documentation License (Version 1.2 oder neuer) zu kopieren, verteilen und/oder
% zu verändern. Eine Kopie dieser Lizenz ist unter
% http://www.gnu.org/copyleft/fdl.txt erhältlich.
%
% Copyright der Aktualisierung und Überarbeitung \copyright\ 2013  Simon Kalt, Jan-Peter Hohloch, Tobias Fabritz
% Es wird die Erlaubnis gegeben, dieses Dokument unter den Bedingungen der von der Free
% Software Foundation veröffentlichen  GNU Free
% Documentation License (Version 1.2 oder neuer) zu kopieren, verteilen und/oder
% zu verändern. Eine Kopie dieser Lizenz ist unter
% http://www.gnu.org/copyleft/fdl.txt erhältlich.
%
% Zusätzlich muss jede Kopie/Aktualisierung wieder über die Seite
% der Fachschaft Informatik der Uni Tübingen
% den Studenten zur Verfügung gestellt werden
% http://www.fsi.uni-tuebingen.de/

\chapter{Dynamische Mengen}
    Operationen: Einfügen, Streichen, Vereinigen, Spalten, etc. \\
    
    \section{Suchbäume}
        Beispiel: $\{ 2,3,5,7,11,13,17 \}$ (\autoref{diag4:search-tree},\autoref{diag5:leaf-oriented}).

        Knoten haben: 
        \begin{itemize}
            \item Key-Element
            \item Verweise zu Kindern (lson, rson)
            \item oft Verweis zu parent
        \end{itemize}
        
        
         \subsection{Knotenorientierte Speicherung} 
            \begin{itemize}
                \item 1 Element pro Knoten
                \item Alle Elemente im linken Teilbaum von Knoten v sind kleiner als alle im rechten Teilbaum
                \item keine Elemente in den Blättern
            \end{itemize}
            Dies ist die weiterhin verwendete Speicherform in diesem Skript.
            
            \begin{figure}[htp]
            	\Tree [.5 [.2 $\square$ 3 ]
          [.7 $\square$
              [.13 11 17 ] ] ]

            	\caption{Knotenorientierter Suchbaum}
            	\label{diag4:search-tree}
            \end{figure}


        \subsection{EXKURS: Blattorientierte Speicherung} 
            1 Element pro Blatt, Elemente aus linkem Teilbaum $\leq$ (Hilfs-) Schlüssel an $v$ $\leq$ Element aus rechtem Teilbaum.
            Ein Beispiel für diese Speicherung befindet sich in \autoref{diag5:leaf-oriented}.
            
            \begin{figure}[htp]
                \centering
                
% \begin{tikzpicture}[anchor=mid,>=latex',line join=bevel,]
\begin{tikzpicture}[-,line join=bevel,]
  \pgfsetlinewidth{1bp}
%%
\pgfsetcolor{black}
  % Edge: c -> f
  \draw [->] (339bp,144bp) .. controls (339bp,144bp) and (339bp,125.49bp)  .. (339bp,103bp);
  % Edge: d -> h
  \draw [->] (57bp,77bp) .. controls (57bp,77bp) and (48.079bp,59.752bp)  .. (35.357bp,35.156bp);
  % Edge: c -> g
  \draw [->] (389bp,144bp) .. controls (389bp,144bp) and (396.81bp,129.72bp)  .. (409.14bp,107.19bp);
  % Edge: e -> k
  \draw [->] (228bp,77bp) .. controls (228bp,77bp) and (232.19bp,60.787bp)  .. (238.66bp,35.776bp);
  % Edge: f -> m
  \draw [->] (363bp,77bp) .. controls (363bp,77bp) and (369.92bp,60.275bp)  .. (380.19bp,35.465bp);
  % Edge: f -> l
  \draw [->] (315bp,77bp) .. controls (315bp,77bp) and (315bp,61.286bp)  .. (315bp,36.088bp);
  % Edge: e -> j
  \draw [->] (186bp,77bp) .. controls (186bp,77bp) and (181.81bp,60.787bp)  .. (175.34bp,35.776bp);
  % Edge: a -> b
  \draw [->] (257bp,206bp) .. controls (257bp,206bp) and (214.1bp,184.25bp)  .. (186bp,170bp);
  % Edge: b -> e
  \draw [->] (207bp,144bp) .. controls (207bp,144bp) and (207bp,125.49bp)  .. (207bp,103bp);
  % Edge: a -> c
  \draw [->] (307bp,206bp) .. controls (307bp,206bp) and (339.06bp,185.75bp)  .. (364bp,170bp);
  % Edge: d -> i
  \draw [->] (99bp,77bp) .. controls (99bp,77bp) and (99bp,61.286bp)  .. (99bp,36.088bp);
  % Edge: b -> d
  \draw [->] (154bp,157bp) .. controls (154bp,157bp) and (106.21bp,123.04bp)  .. (78bp,103bp);
  % Node: a
\begin{scope}
  \definecolor{strokecol}{rgb}{0.0,0.0,0.0};
  \pgfsetstrokecolor{strokecol}
  \draw (248bp,207bp) -- (248bp,231bp) -- (317bp,231bp) -- (317bp,207bp) -- cycle;
  \draw (267bp,207bp) -- (267bp,231bp);
  \draw (297bp,207bp) -- (297bp,231bp);
  \draw (257bp,213bp) node { };
  \draw (282bp,213bp) node {10};
  \draw (307bp,213bp) node {};
\end{scope}
  % Node: c
\begin{scope}
  \definecolor{strokecol}{rgb}{0.0,0.0,0.0};
  \pgfsetstrokecolor{strokecol}
  \draw (330bp,145bp) -- (330bp,169bp) -- (399bp,169bp) -- (399bp,145bp) -- cycle;
  \draw (349bp,145bp) -- (349bp,169bp);
  \draw (379bp,145bp) -- (379bp,169bp);
  \draw (339bp,151bp) node { };
  \draw (364bp,151bp) node {15};
  \draw (389bp,151bp) node {};
\end{scope}
  % Node: b
\begin{scope}
  \definecolor{strokecol}{rgb}{0.0,0.0,0.0};
  \pgfsetstrokecolor{strokecol}
  \draw (155bp,145bp) -- (155bp,169bp) -- (217bp,169bp) -- (217bp,145bp) -- cycle;
  \draw (175bp,145bp) -- (175bp,169bp);
  \draw (198bp,145bp) -- (198bp,169bp);
  \draw (165bp,151bp) node { };
  \draw (186bp,151bp) node {3};
  \draw (207bp,151bp) node {};
\end{scope}
  % Node: e
\begin{scope}
  \definecolor{strokecol}{rgb}{0.0,0.0,0.0};
  \pgfsetstrokecolor{strokecol}
  \draw (176bp,78bp) -- (176bp,102bp) -- (238bp,102bp) -- (238bp,78bp) -- cycle;
  \draw (196bp,78bp) -- (196bp,102bp);
  \draw (219bp,78bp) -- (219bp,102bp);
  \draw (186bp,84bp) node { };
  \draw (207bp,84bp) node {5};
  \draw (228bp,84bp) node {};
\end{scope}
  % Node: d
\begin{scope}
  \definecolor{strokecol}{rgb}{0.0,0.0,0.0};
  \pgfsetstrokecolor{strokecol}
  \draw (47bp,78bp) -- (47bp,102bp) -- (109bp,102bp) -- (109bp,78bp) -- cycle;
  \draw (67bp,78bp) -- (67bp,102bp);
  \draw (90bp,78bp) -- (90bp,102bp);
  \draw (57bp,84bp) node { };
  \draw (78bp,84bp) node {2};
  \draw (99bp,84bp) node {};
\end{scope}
  % Node: g
\begin{scope}
  \definecolor{strokecol}{rgb}{0.0,0.0,0.0};
  \pgfsetstrokecolor{strokecol}
  \draw (418bp,90bp) ellipse (27bp and 18bp);
  \draw (418bp,90bp) node {17};
\end{scope}
  % Node: f
\begin{scope}
  \definecolor{strokecol}{rgb}{0.0,0.0,0.0};
  \pgfsetstrokecolor{strokecol}
  \draw (305bp,78bp) -- (305bp,102bp) -- (373bp,102bp) -- (373bp,78bp) -- cycle;
  \draw (324bp,78bp) -- (324bp,102bp);
  \draw (354bp,78bp) -- (354bp,102bp);
  \draw (315bp,84bp) node { };
  \draw (339bp,84bp) node {11};
  \draw (363bp,84bp) node {};
\end{scope}
  % Node: i
\begin{scope}
  \definecolor{strokecol}{rgb}{0.0,0.0,0.0};
  \pgfsetstrokecolor{strokecol}
  \draw (99bp,18bp) ellipse (27bp and 18bp);
  \draw (99bp,18bp) node {3};
\end{scope}
  % Node: h
\begin{scope}
  \definecolor{strokecol}{rgb}{0.0,0.0,0.0};
  \pgfsetstrokecolor{strokecol}
  \draw (27bp,18bp) ellipse (27bp and 18bp);
  \draw (27bp,18bp) node {2};
\end{scope}
  % Node: k
\begin{scope}
  \definecolor{strokecol}{rgb}{0.0,0.0,0.0};
  \pgfsetstrokecolor{strokecol}
  \draw (243bp,18bp) ellipse (27bp and 18bp);
  \draw (243bp,18bp) node {7};
\end{scope}
  % Node: j
\begin{scope}
  \definecolor{strokecol}{rgb}{0.0,0.0,0.0};
  \pgfsetstrokecolor{strokecol}
  \draw (171bp,18bp) ellipse (27bp and 18bp);
  \draw (171bp,18bp) node {5};
\end{scope}
  % Node: m
\begin{scope}
  \definecolor{strokecol}{rgb}{0.0,0.0,0.0};
  \pgfsetstrokecolor{strokecol}
  \draw (387bp,18bp) ellipse (27bp and 18bp);
  \draw (387bp,18bp) node {13};
\end{scope}
  % Node: l
\begin{scope}
  \definecolor{strokecol}{rgb}{0.0,0.0,0.0};
  \pgfsetstrokecolor{strokecol}
  \draw (315bp,18bp) ellipse (27bp and 18bp);
  \draw (315bp,18bp) node {11};
\end{scope}
%
\end{tikzpicture}
\begin{flushleft}
	\textit{Anmerkung:}\\
		Beispielsweise 10 ist \textbf{nicht} Element dieses Baumes. sie wird lediglich als Hilfsschlüssel verwendet.
		11 dagegen ist sowohl Schlüssel, als auch Element.
\end{flushleft}
                \caption{Blattorientierte Speicherung}
                \label{diag5:leaf-oriented}
            \end{figure}
        
       
        
        \subsection{Suchen in Suchbäumen}  
            Beschrieben in \autoref{alg:searchSearchtree}
    	\begin{algorithm}
        		\caption{Suchen in Suchbäumen}
        		\label{alg:searchSearchtree}
        		\begin{algorithmic}[1]
        			\Function{Suchen}{$x$}
        			    \State $u \gets \text{Wurzel}$
        			    \State $\text{found} \gets \text{false}$
        				\While{$u$ exists and !found}
        				    \If{\Call{key}{$u$} = x}
        				        \State $\text{found} \gets \text{true}$
        				    \Else
        				        \If{\Call{key}{$u$} $> x$}
        				            \State $u \gets $ \Call{lson}{$u$}
        				        \Else 
        				            \State $u \gets $ \Call{rson}{$u$}
        				        \EndIf
        				    \EndIf
        				\EndWhile
        				\State \Call{Return}{found}
        			\EndFunction
        		\end{algorithmic}
        	\end{algorithm}

        \subsection{Einfügen in Suchbäumen}
            Einfügen (x) : zuerstSuchen(x) \\
            Sei $u$ der zuletzt besuchte Knoten $u$ hat $\leq 1$ Kind (oder $x \in S$).\\
            Falls $x < $ key($u$), erzeuge neues Kind $v$ von $u$ mit key($v$) = x als lson($u$), andernfalls als rson ($u$). \\
            Beispielhaft (Füge 10 ein): \\
            % Graphic for TeX using PGF
% Title: /home/alex/wuala/WualaDrive/myfiles/Uni/4 Semester/Algorithmen/Skript/algorithmen/Diagramm6.dia
% Creator: Dia v0.97.1
% CreationDate: Tue May  4 14:44:28 2010
% For: alex
% \usepackage{tikz}
% The following commands are not supported in PSTricks at present
% We define them conditionally, so when they are implemented,
% this pgf file will use them.
\ifx\du\undefined
  \newlength{\du}
\fi
\setlength{\du}{15\unitlength}
\begin{tikzpicture}
\pgftransformxscale{1.000000}
\pgftransformyscale{-1.000000}
\definecolor{dialinecolor}{rgb}{0.000000, 0.000000, 0.000000}
\pgfsetstrokecolor{dialinecolor}
\definecolor{dialinecolor}{rgb}{1.000000, 1.000000, 1.000000}
\pgfsetfillcolor{dialinecolor}
\pgfsetlinewidth{0.100000\du}
\pgfsetdash{}{0pt}
\pgfsetdash{}{0pt}
\pgfsetmiterjoin
\definecolor{dialinecolor}{rgb}{1.000000, 1.000000, 1.000000}
\pgfsetfillcolor{dialinecolor}
\fill (12.750000\du,3.900000\du)--(12.750000\du,4.900000\du)--(14.750000\du,4.900000\du)--(14.750000\du,3.900000\du)--cycle;
\definecolor{dialinecolor}{rgb}{0.000000, 0.000000, 0.000000}
\pgfsetstrokecolor{dialinecolor}
\draw (12.750000\du,3.900000\du)--(12.750000\du,4.900000\du)--(14.750000\du,4.900000\du)--(14.750000\du,3.900000\du)--cycle;
% setfont left to latex
\definecolor{dialinecolor}{rgb}{0.000000, 0.000000, 0.000000}
\pgfsetstrokecolor{dialinecolor}
\node[anchor=west] at (13.400000\du,4.600000\du){13};
\pgfsetlinewidth{0.100000\du}
\pgfsetdash{{1.000000\du}{1.000000\du}}{0\du}
\pgfsetdash{{1.000000\du}{1.000000\du}}{0\du}
\pgfsetbuttcap
{
\definecolor{dialinecolor}{rgb}{0.000000, 0.000000, 0.000000}
\pgfsetfillcolor{dialinecolor}
% was here!!!
\definecolor{dialinecolor}{rgb}{0.000000, 0.000000, 0.000000}
\pgfsetstrokecolor{dialinecolor}
\draw (9.800000\du,1.650000\du)--(12.750000\du,3.900000\du);
}
\pgfsetlinewidth{0.100000\du}
\pgfsetdash{}{0pt}
\pgfsetdash{}{0pt}
\pgfsetmiterjoin
\definecolor{dialinecolor}{rgb}{1.000000, 1.000000, 1.000000}
\pgfsetfillcolor{dialinecolor}
\fill (8.950000\du,6.150000\du)--(8.950000\du,8.700000\du)--(12.900000\du,8.700000\du)--(12.900000\du,6.150000\du)--cycle;
\definecolor{dialinecolor}{rgb}{0.000000, 0.000000, 0.000000}
\pgfsetstrokecolor{dialinecolor}
\draw (8.950000\du,6.150000\du)--(8.950000\du,8.700000\du)--(12.900000\du,8.700000\du)--(12.900000\du,6.150000\du)--cycle;
\pgfsetlinewidth{0.100000\du}
\pgfsetdash{}{0pt}
\pgfsetdash{}{0pt}
\pgfsetmiterjoin
\definecolor{dialinecolor}{rgb}{1.000000, 1.000000, 1.000000}
\pgfsetfillcolor{dialinecolor}
\fill (4.950000\du,10.150000\du)--(4.950000\du,13.100000\du)--(9.050000\du,13.100000\du)--(9.050000\du,10.150000\du)--cycle;
\definecolor{dialinecolor}{rgb}{0.000000, 0.000000, 0.000000}
\pgfsetstrokecolor{dialinecolor}
\draw (4.950000\du,10.150000\du)--(4.950000\du,13.100000\du)--(9.050000\du,13.100000\du)--(9.050000\du,10.150000\du)--cycle;
\pgfsetlinewidth{0.100000\du}
\pgfsetdash{}{0pt}
\pgfsetdash{}{0pt}
\pgfsetbuttcap
{
\definecolor{dialinecolor}{rgb}{0.000000, 0.000000, 0.000000}
\pgfsetfillcolor{dialinecolor}
% was here!!!
\pgfsetarrowsend{stealth}
\definecolor{dialinecolor}{rgb}{0.000000, 0.000000, 0.000000}
\pgfsetstrokecolor{dialinecolor}
\draw (12.750000\du,4.900000\du)--(10.925000\du,6.150000\du);
}
\pgfsetlinewidth{0.100000\du}
\pgfsetdash{}{0pt}
\pgfsetdash{}{0pt}
\pgfsetbuttcap
{
\definecolor{dialinecolor}{rgb}{0.000000, 0.000000, 0.000000}
\pgfsetfillcolor{dialinecolor}
% was here!!!
\pgfsetarrowsend{stealth}
\definecolor{dialinecolor}{rgb}{0.000000, 0.000000, 0.000000}
\pgfsetstrokecolor{dialinecolor}
\draw (8.950000\du,8.700000\du)--(7.000000\du,10.150000\du);
}
\pgfsetlinewidth{0.100000\du}
\pgfsetdash{}{0pt}
\pgfsetdash{}{0pt}
\pgfsetbuttcap
{
\definecolor{dialinecolor}{rgb}{0.000000, 0.000000, 0.000000}
\pgfsetfillcolor{dialinecolor}
% was here!!!
\pgfsetarrowsend{stealth}
\definecolor{dialinecolor}{rgb}{0.000000, 0.000000, 0.000000}
\pgfsetstrokecolor{dialinecolor}
\draw (12.900000\du,8.700000\du)--(14.650000\du,10.400000\du);
}
\pgfsetlinewidth{0.100000\du}
\pgfsetdash{}{0pt}
\pgfsetdash{}{0pt}
\pgfsetbuttcap
{
\definecolor{dialinecolor}{rgb}{0.000000, 0.000000, 0.000000}
\pgfsetfillcolor{dialinecolor}
% was here!!!
\definecolor{dialinecolor}{rgb}{0.000000, 0.000000, 0.000000}
\pgfsetstrokecolor{dialinecolor}
\draw (8.800000\du,7.550000\du)--(13.000000\du,7.600000\du);
}
\pgfsetlinewidth{0.100000\du}
\pgfsetdash{}{0pt}
\pgfsetdash{}{0pt}
\pgfsetbuttcap
{
\definecolor{dialinecolor}{rgb}{0.000000, 0.000000, 0.000000}
\pgfsetfillcolor{dialinecolor}
% was here!!!
\definecolor{dialinecolor}{rgb}{0.000000, 0.000000, 0.000000}
\pgfsetstrokecolor{dialinecolor}
\draw (4.950000\du,11.625000\du)--(9.050000\du,11.625000\du);
}
% setfont left to latex
\definecolor{dialinecolor}{rgb}{0.000000, 0.000000, 0.000000}
\pgfsetstrokecolor{dialinecolor}
\node[anchor=west] at (10.675000\du,7.175000\du){11 ($u$)};
% setfont left to latex
\definecolor{dialinecolor}{rgb}{0.000000, 0.000000, 0.000000}
\pgfsetstrokecolor{dialinecolor}
\node[anchor=west] at (6.750000\du,11.125000\du){10 ($v$)};
\pgfsetlinewidth{0.100000\du}
\pgfsetdash{}{0pt}
\pgfsetdash{}{0pt}
\pgfsetbuttcap
{
\definecolor{dialinecolor}{rgb}{0.000000, 0.000000, 0.000000}
\pgfsetfillcolor{dialinecolor}
% was here!!!
\pgfsetarrowsend{stealth}
\definecolor{dialinecolor}{rgb}{0.000000, 0.000000, 0.000000}
\pgfsetstrokecolor{dialinecolor}
\draw (4.950000\du,13.100000\du)--(4.050000\du,13.850000\du);
}
\pgfsetlinewidth{0.100000\du}
\pgfsetdash{}{0pt}
\pgfsetdash{}{0pt}
\pgfsetbuttcap
{
\definecolor{dialinecolor}{rgb}{0.000000, 0.000000, 0.000000}
\pgfsetfillcolor{dialinecolor}
% was here!!!
\pgfsetarrowsend{stealth}
\definecolor{dialinecolor}{rgb}{0.000000, 0.000000, 0.000000}
\pgfsetstrokecolor{dialinecolor}
\draw (9.050000\du,13.100000\du)--(10.050000\du,14.100000\du);
}
\end{tikzpicture}
 \\
        
        \subsection{Löschen in Suchbäumen}
            \begin{alltt}
			Streichen (x); zuerst Suchen(x), 
			Suchen endet in Knoten \( u, x \in S \)
			1. u ist Blatt
			streiche u, rson, lson, Verweis von parent(v) auf undef.
			2. u hat 1 Kind
			Streiche u, setze w als Kind von parent(u) an die Stelle von u.
			3. u hat 2 Kinder
			Suche v mit größtem Info-Element im linken Teilbaum von u, 
			(einmal nach links, dann immer nach rechts! \( \Ra \) v hat keinen rson)
			Ersetze u durch v & Streiche v unten (wie Fall 1 oder 2)
            \end{alltt}
            
            Laufzeit $\LO(h+1)$ , wobei h Höhe des Suchbaums ist.\\

        \subsection{Diskussion}
            Ein Suchbaum kann sich sehr schlecht verhalten. Werden etwa aufsteigende Zahlen eingefügt wie beschrieben,
            so ergibt sich eine entartete Baumform, die stark an eine Liste erinnert. Beispiel: \\
            % Graphic for TeX using PGF
% Title: /home/alex/wuala/WualaDrive/myfiles/Uni/4 Semester/Algorithmen/Skript/algorithmen/Diagramm8.dia
% Creator: Dia v0.97.1
% CreationDate: Tue May  4 15:00:19 2010
% For: alex
% \usepackage{tikz}
% The following commands are not supported in PSTricks at present
% We define them conditionally, so when they are implemented,
% this pgf file will use them.
\ifx\du\undefined
  \newlength{\du}
\fi
\setlength{\du}{15\unitlength}
\begin{tikzpicture}
\pgftransformxscale{1.000000}
\pgftransformyscale{-1.000000}
\definecolor{dialinecolor}{rgb}{0.000000, 0.000000, 0.000000}
\pgfsetstrokecolor{dialinecolor}
\definecolor{dialinecolor}{rgb}{1.000000, 1.000000, 1.000000}
\pgfsetfillcolor{dialinecolor}
\pgfsetlinewidth{0.100000\du}
\pgfsetdash{}{0pt}
\pgfsetdash{}{0pt}
\pgfsetmiterjoin
\definecolor{dialinecolor}{rgb}{1.000000, 1.000000, 1.000000}
\pgfsetfillcolor{dialinecolor}
\fill (6.500000\du,6.100000\du)--(6.500000\du,7.100000\du)--(8.500000\du,7.100000\du)--(8.500000\du,6.100000\du)--cycle;
\definecolor{dialinecolor}{rgb}{0.000000, 0.000000, 0.000000}
\pgfsetstrokecolor{dialinecolor}
\draw (6.500000\du,6.100000\du)--(6.500000\du,7.100000\du)--(8.500000\du,7.100000\du)--(8.500000\du,6.100000\du)--cycle;
\pgfsetlinewidth{0.100000\du}
\pgfsetdash{}{0pt}
\pgfsetdash{}{0pt}
\pgfsetmiterjoin
\definecolor{dialinecolor}{rgb}{1.000000, 1.000000, 1.000000}
\pgfsetfillcolor{dialinecolor}
\fill (10.850000\du,9.750000\du)--(10.850000\du,10.750000\du)--(12.850000\du,10.750000\du)--(12.850000\du,9.750000\du)--cycle;
\definecolor{dialinecolor}{rgb}{0.000000, 0.000000, 0.000000}
\pgfsetstrokecolor{dialinecolor}
\draw (10.850000\du,9.750000\du)--(10.850000\du,10.750000\du)--(12.850000\du,10.750000\du)--(12.850000\du,9.750000\du)--cycle;
\pgfsetlinewidth{0.100000\du}
\pgfsetdash{}{0pt}
\pgfsetdash{}{0pt}
\pgfsetmiterjoin
\definecolor{dialinecolor}{rgb}{1.000000, 1.000000, 1.000000}
\pgfsetfillcolor{dialinecolor}
\fill (15.300000\du,13.550000\du)--(15.300000\du,14.550000\du)--(17.450000\du,14.550000\du)--(17.450000\du,13.550000\du)--cycle;
\definecolor{dialinecolor}{rgb}{0.000000, 0.000000, 0.000000}
\pgfsetstrokecolor{dialinecolor}
\draw (15.300000\du,13.550000\du)--(15.300000\du,14.550000\du)--(17.450000\du,14.550000\du)--(17.450000\du,13.550000\du)--cycle;
\pgfsetlinewidth{0.100000\du}
\pgfsetdash{}{0pt}
\pgfsetdash{}{0pt}
\pgfsetbuttcap
{
\definecolor{dialinecolor}{rgb}{0.000000, 0.000000, 0.000000}
\pgfsetfillcolor{dialinecolor}
% was here!!!
\pgfsetarrowsend{stealth}
\definecolor{dialinecolor}{rgb}{0.000000, 0.000000, 0.000000}
\pgfsetstrokecolor{dialinecolor}
\draw (8.500000\du,7.100000\du)--(10.850000\du,9.750000\du);
}
\pgfsetlinewidth{0.100000\du}
\pgfsetdash{}{0pt}
\pgfsetdash{}{0pt}
\pgfsetbuttcap
{
\definecolor{dialinecolor}{rgb}{0.000000, 0.000000, 0.000000}
\pgfsetfillcolor{dialinecolor}
% was here!!!
\pgfsetarrowsend{stealth}
\definecolor{dialinecolor}{rgb}{0.000000, 0.000000, 0.000000}
\pgfsetstrokecolor{dialinecolor}
\draw (12.850000\du,10.750000\du)--(15.300000\du,13.550000\du);
}
\pgfsetlinewidth{0.100000\du}
\pgfsetdash{}{0pt}
\pgfsetdash{}{0pt}
\pgfsetbuttcap
{
\definecolor{dialinecolor}{rgb}{0.000000, 0.000000, 0.000000}
\pgfsetfillcolor{dialinecolor}
% was here!!!
\pgfsetarrowsend{stealth}
\definecolor{dialinecolor}{rgb}{0.000000, 0.000000, 0.000000}
\pgfsetstrokecolor{dialinecolor}
\draw (6.500000\du,7.100000\du)--(5.600000\du,8.200000\du);
}
\pgfsetlinewidth{0.100000\du}
\pgfsetdash{}{0pt}
\pgfsetdash{}{0pt}
\pgfsetbuttcap
{
\definecolor{dialinecolor}{rgb}{0.000000, 0.000000, 0.000000}
\pgfsetfillcolor{dialinecolor}
% was here!!!
\pgfsetarrowsend{stealth}
\definecolor{dialinecolor}{rgb}{0.000000, 0.000000, 0.000000}
\pgfsetstrokecolor{dialinecolor}
\draw (10.850000\du,10.750000\du)--(9.650000\du,12.350000\du);
}
\pgfsetlinewidth{0.100000\du}
\pgfsetdash{}{0pt}
\pgfsetdash{}{0pt}
\pgfsetbuttcap
{
\definecolor{dialinecolor}{rgb}{0.000000, 0.000000, 0.000000}
\pgfsetfillcolor{dialinecolor}
% was here!!!
\pgfsetarrowsend{stealth}
\definecolor{dialinecolor}{rgb}{0.000000, 0.000000, 0.000000}
\pgfsetstrokecolor{dialinecolor}
\draw (15.300000\du,14.550000\du)--(14.050000\du,16.100000\du);
}
\pgfsetlinewidth{0.100000\du}
\pgfsetdash{}{0pt}
\pgfsetdash{}{0pt}
\pgfsetbuttcap
{
\definecolor{dialinecolor}{rgb}{0.000000, 0.000000, 0.000000}
\pgfsetfillcolor{dialinecolor}
% was here!!!
\pgfsetarrowsend{stealth}
\definecolor{dialinecolor}{rgb}{0.000000, 0.000000, 0.000000}
\pgfsetstrokecolor{dialinecolor}
\draw (16.823709\du,14.600031\du)--(18.700000\du,16.900000\du);
}
% setfont left to latex
\definecolor{dialinecolor}{rgb}{0.000000, 0.000000, 0.000000}
\pgfsetstrokecolor{dialinecolor}
\node[anchor=west] at (19.450000\du,18.500000\du){.....};
\pgfsetlinewidth{0.100000\du}
\pgfsetdash{}{0pt}
\pgfsetdash{}{0pt}
\pgfsetmiterjoin
\pgfsetbuttcap
{
\definecolor{dialinecolor}{rgb}{0.000000, 0.000000, 0.000000}
\pgfsetfillcolor{dialinecolor}
% was here!!!
\pgfsetarrowsend{stealth}
\definecolor{dialinecolor}{rgb}{0.000000, 0.000000, 0.000000}
\pgfsetstrokecolor{dialinecolor}
\pgfpathmoveto{\pgfpoint{8.650000\du}{4.850000\du}}
\pgfpathcurveto{\pgfpoint{10.450000\du}{2.500000\du}}{\pgfpoint{21.200000\du}{15.300000\du}}{\pgfpoint{20.400000\du}{16.500000\du}}
\pgfusepath{stroke}
}
% setfont left to latex
\definecolor{dialinecolor}{rgb}{0.000000, 0.000000, 0.000000}
\pgfsetstrokecolor{dialinecolor}
\node[anchor=west] at (15.950000\du,8.500000\du){n};
\end{tikzpicture}
 \\
            Idee, dies zu lösen: \\
            \begin{enumerate}[1]
                \item Hoffe, dass es nicht vorkommt. (Unwahrscheinlich, dass ein solcher Input erfolgt)
                \item Baue Baum von Zeit zu Zeit neu auf.
                \item Balancierte Bäume.
            \end{enumerate}

    \section{Balancierte Bäume}
        Sei u Knoten im Suchbaum. Die Balance von u sei definiert als 
        $$
            Bal(u) = \text{Höhe}(\text{rson}) - \text{Höhe}(\text{lson})
        $$
        Setze Höhe des undefinierten TBs $= -1$.\\
        Beispiele s. \autoref{diag26:balance}

        
        \begin{figure}[htp]
			\centering
			\begin{tikzpicture}[grow via three points={
                         one child at (-1,-1) and two children at (-1,-1) and (1,-1)}]
\node at (0,0) (u) {u}
     child{node (v) {v}};

\node at (10,0) (x) {x}
	child{node (y) {y}
		child{node (z) {z}}};

\end{tikzpicture}
			\caption{$Bal(u) = 0 -1 = -1$, $Bal(x) = 0-2 = -2$}
			\label{diag26:balance}
        \end{figure}

        \begin{definition}
            Ein binärer Baum $T$ heißt AVL-Baum, falls gilt 
            $$
                \forall u \in T : \abs{Bal(u)} \leq 1 
            $$
        \end{definition}

        \begin{figure}[htp]
            \centering
            \begin{tikzpicture}[xscale=2,yscale=2,thick]
  \draw (0,0) -- (-135:1cm) -- ++(-120:1cm) -- ++(-105:1cm) ++(-0.15,0) node {0};;
  \draw (-135:1cm) -- ++(-60:1cm) ++(0.15,0) node {0};;
  \draw (0,0) -- (-45:1cm) -- ++(-60:1cm) ++(0.15,0) node {0};
  \draw (0,0.2) node {$-1$};
  \draw (-135:1cm) ++(-0.3,0) node {$-1$};
  \draw (-135:1cm) ++(-120:1cm) ++(-0.3,0) node {$-1$};
  \draw (-45:1cm) ++(0.2,0)  node {$1$};
\end{tikzpicture}
            \caption{Ein AVL Baum. An den Knoten ist die Balance des jeweiligen Teilbaums angegeben.}
            \label{diag9:avl-balance}
        \end{figure}

        \begin{definition}
            Fibonacci-Bäume (\autoref{diag10:fibonacci-tree})
            $$T_0, T_1, T_2, \ldots$$ sind definiert durch $T_0 = $ leer, $T_1 = \bullet$, $T_2 = /$, \\
            $T_h$ siehe Abbildung \ref{diag10:fibonacci-tree} \\
            Fibonacci-Zahlen: 
            $$F_0 = 0, F_1 = 1, F_h = F_{h-1} + F_{h-2}$$
        \end{definition}

        \begin{figure}[htp]
            \centering
            \resizebox{0.5\linewidth}{!}{%
\Tree [. \qroof{$T_{h-1}$}. \qroof{$T_{h-2}$}. !{\qbalance} ]
}
            \caption{Fibonacci-Baum}
            \label{diag10:fibonacci-tree}
        \end{figure}
        
        \emph{Behauptung:} $T_h$ enthält genau $F_h$ Blätter (für alle $h \geq 0$) \\
        Beweis per Induktion. (klar)\\
        \emph{Behauptung:} AVL-Bäume der Höhe $h$ haben $\geq F_h$ Blätter. \\
        \begin{proof}     
            \begin{itemize}
                \item $h=0$ \ $\bullet \ \ \text{ gilt für einen Knoten}$ 
                \item $h=1$ \ / oder \verb=\= oder /\verb=\=
                \item $h \geq 2$ : Erhalten den blattminimalen AVl-Baum der Höhe h durch Kombination von blattmin. AVL-Bäumen der Höhe $h-1$ und $h-2$.

                $\Ra \geq F_{h-1} + F_{h-2} = F_h$ Blätter. \\
                $F_h = h$-te Fibonacci -Zahl. 
                $$F_h = \frac{\alpha^h - \beta^h}{\sqrt{5}}$$ mit $\alpha = \frac{1+ \sqrt{5}}{2}, \beta = \frac{1-\sqrt{5}}{2}$.   
            \end{itemize}
        \end{proof}

        \begin{lemma}
            AVL-Bäume mit $n$ Knoten haben Höhe $\LO(\log n)$ 
        \end{lemma}

        \begin{proof}
            Baum hat $\leq n$ Blätter, also $F_h \leq n$, damit gilt 
            $$
                \frac{\alpha^h - \beta^h}{\sqrt{5}} \leq n
            $$ 
            Da $\btl \beta \btr < 1$ gilt: 
            \begin{align*}            
                \frac{\alpha^h - \beta^h}{\sqrt{5}} &\geq \frac{\alpha^h}{2 \sqrt{5}} \\
                \Ra \alpha^h &\leq 2 \sqrt{5} \cdot n. \\
                h \cdot \log \alpha &< \log (2\sqrt{5}) + \log n \\
                h &\leq \frac{\log 2\sqrt{5} + \log n}{\log \alpha} = \LO(\log n)
            \end{align*}
        \end{proof}

        Beim Einfügen/Streichen kann die Balance gestört werden, z.B. auf -2 oder 2 (siehe anschließendes Beispiel). Wir fordern aber für einen AVL Baum, dass ständig gilt 
        $$
            \forall u \in T : \abs{Bal(u)} \leq 1 
        $$
        und wissen von Fibonacci-Bäumen, dass die Höhe $\LO(\log n)$ beträgt. Es müssen also bei Operationen am Baum Korrekturen ausgeführt werden,
        dass die AVL-Eigenschaften erhalten bleiben.
   
        \subsection{Einfügen($w$) in AVL-Bäumen}
            Angenommen es wurde die AVL-Bedingung zerstört und es ist eine Balance von $\pm 2$ entstanden. Dann sei $u$ der tiefste dieser "`unbal."' Knoten. O.B.d.A. sei $Bal(u)=2$ \\
            
            Sei $w$ der neue Knoten. Der wurde rechts eingefügt. Sei $v$ rechtes Kind von $u$\\
            \begin{enumerate}[\text{Fall} 1)]
                \item Bal(v) = 1

                \begin{tabbing}
                    Es gilt: \= 1) Links-Rechts Ordnung wird aufrechterhalten:\\
                    \> $T_{L} \leq u \leq T_{A} \leq v \leq T_{B}$\\
                    \> 2) Nach Rotation haben $u$ und $v$ Balance 0.\\
                    \> Knoten in $T_{A}, T_{B}, T_{L}$ behalten ihre alte Balance.\\
                    \> 3) Höhe($v$) nach Rotation = Höhe($u$) vor Einfügen($w$).
                \end{tabbing}
                \item Bal(v) = -1\\
                Nach Einfügen von $w$ folge dem Pfad von $w$ zur Wurzel.\\
                Berechne auf diesem Pfad alle Balancen neu. \\
                Tritt Bal. -2/2 auf, führe Rotation/Doppelrotation aus.\\
                Laufzeit: $\LO(\log n)$ (Rot./Doppelrot. in $\LO(1)$)
                \item Bal(v) = 0\\
                Kann nicht vorkommen. AVL-Bedingung vorher schon verletzt.
            \end{enumerate}
            
            \subsubsection{Beispiel für Einfügeoperation}
                Es soll in den folgenden Baum das Element 3 eingefügt werden:
                \begin{center}
                    \begin{tikzpicture}[
    edge from parent path=
    {(\tikzparentnode.south) .. controls +(0,-.5) and +(0,.5)
        .. (\tikzchildnode.north)},
    every node/.style={draw,circle},
    label distance=-1mm,
    level/.style={sibling distance = 5cm/#1,
    level distance = 1.5cm}]
    \node [label=330:$1$]{1}
        child {node[label=330:$-1$] {$\square$}}
        child {node[label=330:$0$] {$2$}
            child {node[label=330:$-1$] {$\square$}}
            child {node[label=330:$-1$] {$\square$}}
        };
\end{tikzpicture}
                \end{center}
                Dieser hat jedoch nach einer Einfügung die nachfolgende Gestalt
                \begin{center}
                    \begin{tikzpicture}[
    edge from parent path=
    {(\tikzparentnode.south) .. controls +(0,-.5) and +(0,.5)
        .. (\tikzchildnode.north)},
    every node/.style={draw,circle},
    label distance=-1mm,
    level/.style={sibling distance = 5cm/#1,
    level distance = 1.5cm}]
    \node [label=330:$2$]{1}
        child {node[label=330:$-1$] {$\square$}}
        child {node[label=330:$1$] {2}
            child {node[label=330:$-1$] {$\square$}}
            child {node[label=330:$0$] {3}
                child {node[label=330:$-1$] {$\square$}}
                child {node[label=330:$-1$] {$\square$}}
            }
        };
\end{tikzpicture}
                \end{center}
                Da nun am obersten Knoten die Balance 2 entstanden ist, muss an der Wurzel rotiert werden. Dies führt zu folgendem Baum:
                \begin{center}
                    \begin{tikzpicture}[
    edge from parent path=
    {(\tikzparentnode.south) .. controls +(0,-.5) and +(0,.5)
        .. (\tikzchildnode.north)},
    every node/.style={draw,circle},
    label distance=-1mm,
    level/.style={sibling distance = 5cm/#1,
    level distance = 1.5cm}]
    \node [label=330:$0$]{2}
        child {node[label=330:$0$] {1}
            child {node[label=330:$-1$] {$\square$}}
            child {node[label=330:$-1$] {$\square$}}
        }
        child {node[label=330:$0$] {3}
            child {node[label=330:$-1$] {$\square$}}
            child {node[label=330:$-1$] {$\square$}}
        };
\end{tikzpicture}
                \end{center}
                Es sind dabei an allen Knoten die jeweiligen Balancen angegeben.

        
        \subsection{Streichen($w$) in AVL-Bäumen}
			\begin{algorithmic}
\begin{algorithm}
\Function{Streiche}{AVL}
	\State key $\gets$ \Call{Suche}{x}
	\If{\textsc{isBlatt}{u}}
		\State \Call{streiche}{u}
		\State \Call{rson}{parent(u)} $\gets$ \Call{lson}{parent}{u}
		$\gets$ \textbf{void}
	\EndIf
	\If{\textsc{numberOfChildren}{u} = 1}\Comment{o.B.d.A sei $u = rson(parent(u))$}
		\State \Call{rson}{parent(u)} $\gets$ w
		\State \Call{streiche}{u}
	\EndIf
	\If{\textsc{numberOfChildren}{u} > 1}
		\State $v \gets$  \Call{FindMaxKey}{\textsc{lson}(u)}
		\State \Call{ersetze}{u,v}
		\State \Call{streiche}{v}
	\EndIf
\EndFunction
	
\end{algorithm}
\end{algorithmic}
            Annahme: es gibt Knoten mit Bal $\pm 2$.\\
            Sei $u$ der tiefste solche Knoten.\\
            O.B.d.A. sei $Bal(u) = 2$.\\
            Sei $v$ das rechte Kind von $u$
            \begin{enumerate}[\text{Fall} 1)]
                \item Bal($v$) = 0\\
                Höhe insgesamt hat sich nicht geändert gegenüber vor Streichen. Können hier abbrechen!
                \item Bal($v$) = 1\\
                Beobachte: Der Teilbaum hat jetzt kleinere Höhe als zuvor. Balancen oben drüber ändern sich!\\
                $\Rightarrow$ Iteriere Rebal.-Prozess weiter oben.
            \end{enumerate}
            \begin{satz}
                Balancierte Bäume (AVL) erlauben Suchen/Einfügen/Streichen in Zeit $\LO(\log n)$, wobei $n$ Zahl der Knoten.
            \end{satz}
        
        \subsection{Anwendung (Schnitt von achsenparallelen Liniensegmenten)}
            Ziel: Anzahl der Schnittpunkte von achsenparallelen Liniensegmenten berechnen. Die Lage wird als allgemein angenommen.
            Ein naiver Algorithmus, der paarweise alle Elemente vergleicht hätte eine Laufzeit von $\LO(n^{2})$. 
            Wir wählen also eine bessere Alternative:
            
            \subsubsection{PlaneSweep}
                \emph{x-Struktur:} geordnete Liste von Endpunkten nach x-Koor. statisch, durch Sortieren erzeugt.\\
                \emph{y-Struktur:} repräsentiert einen Zustand der Sweepline $L$ (dynamisch). Speichern in $L$ horiz. Segmente, 
                die im Moment von $L$ gekreuzt werden geordnet nach y-Koordinate.\\
                $\rightarrow$ AVL-Baum unterstützt Einf./Streichen in Zeit $\LO(\log n)$\\
                
                \emph{Vertikale Segmente:} $(x, y_{u}, y_{o})$\\
                Wollen berechnen \# horiz. Segmente mit y-Koord. zwischen $y_{u}$ und $y_{o}$\\
                $\rightarrow$ berechne Rang($y_{u}$), Rang($y_{o}$). (\# Elemente, die kleiner sind)\\
                Rang($y_{o}$) - Rang($y_{u}$) = \# Schnittpunkt auf vert. Segment.\\
                zu tun: Bestimme Rang(x) in AVL-Baum.\\

                Merke in jedem Knoten die Zahl der Knoten im linken Teilbaum (l count).\\
                Suchen(x): Beim Rechtsabbiegen erhöhe Rangzähler um (l count + 1) $\rightarrow$ Rang(x) in $\LO(\log n)$\\
                $\Rightarrow$ Gesamtlaufzeit: $\LO(n \cdot \log n)$ \\
                Wollen wir die Kreuzungen explizit bestimmen, dann brauchen wir eine Laufzeit von $\LO(n \cdot \log n + k)$

    \section{(2,4)-Bäume (blattorientiert)}
        (2,4)-Bäume gehören zu den (a,b)-Bäumen.
        
        \begin{definition}
            Seien $a,b \in \N$ mit $a \geq 2$, $b \geq 2a-1$. T heißt (a,b)-Baum falls gelten:
            \begin{enumerate}[a)]
                \item alle Blätter von T haben gleiche Tiefe
                \item alle Knoten haben $\leq b$ Kinder
                \item alle Knoten (außer Wurzel) haben $\geq a$ Kinder
                \item Wurzel hat $\geq 2$ Kinder
            \end{enumerate}
        \end{definition}
        
        Das Ziel ist nun die Menge $S = \{x_1 < x_2 < ... < x_n\}$ abzuspeichern. Dabei sollen die Schlüssel in den Blättern aufsteigend von links
        nach rechts geordnet abgespeichert werden.
        
        \begin{enumerate}[1.]
            \item Schlüssel in Blättern.
            \item Innere Knoten $v$ mit $d$ Kindern hat Elemente $K_1(v), \ldots, K_{d-1}(v)$ \\
            $k_i(v)$ = Inhalt des rechtesten Blattes im i-ten Unterbaum. 
        \end{enumerate}


        \begin{bsp}
            $$S=\{2,5,7,11,15,17,19 \}$$ 
            Klar: Tiefe ist $\LO(\log n)$
            \begin{figure}
            	% Graphic for TeX using PGF
% Title: /home/alex/WualaDrive/myfiles/Uni/4 Semester/Algorithmen/Skript/algorithmen/Diagramm18.dia
% Creator: Dia v0.97.1
% CreationDate: Tue May 18 14:27:21 2010
% For: alex
% \usepackage{tikz}
% The following commands are not supported in PSTricks at present
% We define them conditionally, so when they are implemented,
% this pgf file will use them.
\ifx\du\undefined
  \newlength{\du}
\fi
\setlength{\du}{15\unitlength}
\begin{tikzpicture}
\pgftransformxscale{1.000000}
\pgftransformyscale{-1.000000}
\definecolor{dialinecolor}{rgb}{0.000000, 0.000000, 0.000000}
\pgfsetstrokecolor{dialinecolor}
\definecolor{dialinecolor}{rgb}{1.000000, 1.000000, 1.000000}
\pgfsetfillcolor{dialinecolor}
\pgfsetlinewidth{0.100000\du}
\pgfsetdash{}{0pt}
\pgfsetdash{}{0pt}
\pgfsetbuttcap
\pgfsetmiterjoin
\pgfsetlinewidth{0.100000\du}
\pgfsetbuttcap
\pgfsetmiterjoin
\pgfsetdash{}{0pt}
\definecolor{dialinecolor}{rgb}{1.000000, 1.000000, 1.000000}
\pgfsetfillcolor{dialinecolor}
\pgfpathmoveto{\pgfpoint{12.508333\du}{2.700000\du}}
\pgfpathlineto{\pgfpoint{15.941667\du}{2.700000\du}}
\pgfpathcurveto{\pgfpoint{16.415711\du}{2.700000\du}}{\pgfpoint{16.800000\du}{3.158908\du}}{\pgfpoint{16.800000\du}{3.725000\du}}
\pgfpathcurveto{\pgfpoint{16.800000\du}{4.291092\du}}{\pgfpoint{16.415711\du}{4.750000\du}}{\pgfpoint{15.941667\du}{4.750000\du}}
\pgfpathlineto{\pgfpoint{12.508333\du}{4.750000\du}}
\pgfpathcurveto{\pgfpoint{12.034289\du}{4.750000\du}}{\pgfpoint{11.650000\du}{4.291092\du}}{\pgfpoint{11.650000\du}{3.725000\du}}
\pgfpathcurveto{\pgfpoint{11.650000\du}{3.158908\du}}{\pgfpoint{12.034289\du}{2.700000\du}}{\pgfpoint{12.508333\du}{2.700000\du}}
\pgfusepath{fill}
\definecolor{dialinecolor}{rgb}{0.000000, 0.000000, 0.000000}
\pgfsetstrokecolor{dialinecolor}
\pgfpathmoveto{\pgfpoint{12.508333\du}{2.700000\du}}
\pgfpathlineto{\pgfpoint{15.941667\du}{2.700000\du}}
\pgfpathcurveto{\pgfpoint{16.415711\du}{2.700000\du}}{\pgfpoint{16.800000\du}{3.158908\du}}{\pgfpoint{16.800000\du}{3.725000\du}}
\pgfpathcurveto{\pgfpoint{16.800000\du}{4.291092\du}}{\pgfpoint{16.415711\du}{4.750000\du}}{\pgfpoint{15.941667\du}{4.750000\du}}
\pgfpathlineto{\pgfpoint{12.508333\du}{4.750000\du}}
\pgfpathcurveto{\pgfpoint{12.034289\du}{4.750000\du}}{\pgfpoint{11.650000\du}{4.291092\du}}{\pgfpoint{11.650000\du}{3.725000\du}}
\pgfpathcurveto{\pgfpoint{11.650000\du}{3.158908\du}}{\pgfpoint{12.034289\du}{2.700000\du}}{\pgfpoint{12.508333\du}{2.700000\du}}
\pgfusepath{stroke}
% setfont left to latex
\definecolor{dialinecolor}{rgb}{0.000000, 0.000000, 0.000000}
\pgfsetstrokecolor{dialinecolor}
\node at (14.225000\du,3.925000\du){11,19};
\pgfsetlinewidth{0.100000\du}
\pgfsetdash{}{0pt}
\pgfsetdash{}{0pt}
\pgfsetbuttcap
\pgfsetmiterjoin
\pgfsetlinewidth{0.100000\du}
\pgfsetbuttcap
\pgfsetmiterjoin
\pgfsetdash{}{0pt}
\definecolor{dialinecolor}{rgb}{1.000000, 1.000000, 1.000000}
\pgfsetfillcolor{dialinecolor}
\pgfpathmoveto{\pgfpoint{7.283333\du}{7.250000\du}}
\pgfpathlineto{\pgfpoint{10.816667\du}{7.250000\du}}
\pgfpathcurveto{\pgfpoint{11.304518\du}{7.250000\du}}{\pgfpoint{11.700000\du}{7.742487\du}}{\pgfpoint{11.700000\du}{8.350000\du}}
\pgfpathcurveto{\pgfpoint{11.700000\du}{8.957514\du}}{\pgfpoint{11.304518\du}{9.450000\du}}{\pgfpoint{10.816667\du}{9.450000\du}}
\pgfpathlineto{\pgfpoint{7.283333\du}{9.450000\du}}
\pgfpathcurveto{\pgfpoint{6.795482\du}{9.450000\du}}{\pgfpoint{6.400000\du}{8.957514\du}}{\pgfpoint{6.400000\du}{8.350000\du}}
\pgfpathcurveto{\pgfpoint{6.400000\du}{7.742487\du}}{\pgfpoint{6.795482\du}{7.250000\du}}{\pgfpoint{7.283333\du}{7.250000\du}}
\pgfusepath{fill}
\definecolor{dialinecolor}{rgb}{0.000000, 0.000000, 0.000000}
\pgfsetstrokecolor{dialinecolor}
\pgfpathmoveto{\pgfpoint{7.283333\du}{7.250000\du}}
\pgfpathlineto{\pgfpoint{10.816667\du}{7.250000\du}}
\pgfpathcurveto{\pgfpoint{11.304518\du}{7.250000\du}}{\pgfpoint{11.700000\du}{7.742487\du}}{\pgfpoint{11.700000\du}{8.350000\du}}
\pgfpathcurveto{\pgfpoint{11.700000\du}{8.957514\du}}{\pgfpoint{11.304518\du}{9.450000\du}}{\pgfpoint{10.816667\du}{9.450000\du}}
\pgfpathlineto{\pgfpoint{7.283333\du}{9.450000\du}}
\pgfpathcurveto{\pgfpoint{6.795482\du}{9.450000\du}}{\pgfpoint{6.400000\du}{8.957514\du}}{\pgfpoint{6.400000\du}{8.350000\du}}
\pgfpathcurveto{\pgfpoint{6.400000\du}{7.742487\du}}{\pgfpoint{6.795482\du}{7.250000\du}}{\pgfpoint{7.283333\du}{7.250000\du}}
\pgfusepath{stroke}
% setfont left to latex
\definecolor{dialinecolor}{rgb}{0.000000, 0.000000, 0.000000}
\pgfsetstrokecolor{dialinecolor}
\node at (9.050000\du,8.550000\du){2,5,7};
\pgfsetlinewidth{0.100000\du}
\pgfsetdash{}{0pt}
\pgfsetdash{}{0pt}
\pgfsetbuttcap
\pgfsetmiterjoin
\pgfsetlinewidth{0.100000\du}
\pgfsetbuttcap
\pgfsetmiterjoin
\pgfsetdash{}{0pt}
\definecolor{dialinecolor}{rgb}{1.000000, 1.000000, 1.000000}
\pgfsetfillcolor{dialinecolor}
\pgfpathmoveto{\pgfpoint{16.566667\du}{7.100000\du}}
\pgfpathlineto{\pgfpoint{20.033333\du}{7.100000\du}}
\pgfpathcurveto{\pgfpoint{20.511980\du}{7.100000\du}}{\pgfpoint{20.900000\du}{7.581294\du}}{\pgfpoint{20.900000\du}{8.175000\du}}
\pgfpathcurveto{\pgfpoint{20.900000\du}{8.768706\du}}{\pgfpoint{20.511980\du}{9.250000\du}}{\pgfpoint{20.033333\du}{9.250000\du}}
\pgfpathlineto{\pgfpoint{16.566667\du}{9.250000\du}}
\pgfpathcurveto{\pgfpoint{16.088020\du}{9.250000\du}}{\pgfpoint{15.700000\du}{8.768706\du}}{\pgfpoint{15.700000\du}{8.175000\du}}
\pgfpathcurveto{\pgfpoint{15.700000\du}{7.581294\du}}{\pgfpoint{16.088020\du}{7.100000\du}}{\pgfpoint{16.566667\du}{7.100000\du}}
\pgfusepath{fill}
\definecolor{dialinecolor}{rgb}{0.000000, 0.000000, 0.000000}
\pgfsetstrokecolor{dialinecolor}
\pgfpathmoveto{\pgfpoint{16.566667\du}{7.100000\du}}
\pgfpathlineto{\pgfpoint{20.033333\du}{7.100000\du}}
\pgfpathcurveto{\pgfpoint{20.511980\du}{7.100000\du}}{\pgfpoint{20.900000\du}{7.581294\du}}{\pgfpoint{20.900000\du}{8.175000\du}}
\pgfpathcurveto{\pgfpoint{20.900000\du}{8.768706\du}}{\pgfpoint{20.511980\du}{9.250000\du}}{\pgfpoint{20.033333\du}{9.250000\du}}
\pgfpathlineto{\pgfpoint{16.566667\du}{9.250000\du}}
\pgfpathcurveto{\pgfpoint{16.088020\du}{9.250000\du}}{\pgfpoint{15.700000\du}{8.768706\du}}{\pgfpoint{15.700000\du}{8.175000\du}}
\pgfpathcurveto{\pgfpoint{15.700000\du}{7.581294\du}}{\pgfpoint{16.088020\du}{7.100000\du}}{\pgfpoint{16.566667\du}{7.100000\du}}
\pgfusepath{stroke}
% setfont left to latex
\definecolor{dialinecolor}{rgb}{0.000000, 0.000000, 0.000000}
\pgfsetstrokecolor{dialinecolor}
\node at (18.300000\du,8.375000\du){15,17};
\pgfsetlinewidth{0.100000\du}
\pgfsetdash{}{0pt}
\pgfsetdash{}{0pt}
\pgfsetbuttcap
{
\definecolor{dialinecolor}{rgb}{0.000000, 0.000000, 0.000000}
\pgfsetfillcolor{dialinecolor}
% was here!!!
\pgfsetarrowsend{stealth}
\definecolor{dialinecolor}{rgb}{0.000000, 0.000000, 0.000000}
\pgfsetstrokecolor{dialinecolor}
\draw (15.208929\du,4.799475\du)--(17.271301\du,7.051636\du);
}
\pgfsetlinewidth{0.100000\du}
\pgfsetdash{}{0pt}
\pgfsetdash{}{0pt}
\pgfsetbuttcap
{
\definecolor{dialinecolor}{rgb}{0.000000, 0.000000, 0.000000}
\pgfsetfillcolor{dialinecolor}
% was here!!!
\pgfsetarrowsend{stealth}
\definecolor{dialinecolor}{rgb}{0.000000, 0.000000, 0.000000}
\pgfsetstrokecolor{dialinecolor}
\draw (13.021901\du,4.800233\du)--(10.336801\du,7.199960\du);
}
\definecolor{dialinecolor}{rgb}{1.000000, 1.000000, 1.000000}
\pgfsetfillcolor{dialinecolor}
\fill (2.450000\du,10.700000\du)--(2.450000\du,12.600000\du)--(4.450000\du,12.600000\du)--(4.450000\du,10.700000\du)--cycle;
\pgfsetlinewidth{0.100000\du}
\pgfsetdash{}{0pt}
\pgfsetdash{}{0pt}
\pgfsetmiterjoin
\definecolor{dialinecolor}{rgb}{0.000000, 0.000000, 0.000000}
\pgfsetstrokecolor{dialinecolor}
\draw (2.450000\du,10.700000\du)--(2.450000\du,12.600000\du)--(4.450000\du,12.600000\du)--(4.450000\du,10.700000\du)--cycle;
% setfont left to latex
\definecolor{dialinecolor}{rgb}{0.000000, 0.000000, 0.000000}
\pgfsetstrokecolor{dialinecolor}
\node at (3.450000\du,11.845000\du){2};
\definecolor{dialinecolor}{rgb}{1.000000, 1.000000, 1.000000}
\pgfsetfillcolor{dialinecolor}
\fill (5.300000\du,10.800000\du)--(5.300000\du,12.700000\du)--(7.300000\du,12.700000\du)--(7.300000\du,10.800000\du)--cycle;
\pgfsetlinewidth{0.100000\du}
\pgfsetdash{}{0pt}
\pgfsetdash{}{0pt}
\pgfsetmiterjoin
\definecolor{dialinecolor}{rgb}{0.000000, 0.000000, 0.000000}
\pgfsetstrokecolor{dialinecolor}
\draw (5.300000\du,10.800000\du)--(5.300000\du,12.700000\du)--(7.300000\du,12.700000\du)--(7.300000\du,10.800000\du)--cycle;
% setfont left to latex
\definecolor{dialinecolor}{rgb}{0.000000, 0.000000, 0.000000}
\pgfsetstrokecolor{dialinecolor}
\node at (6.300000\du,11.945000\du){5};
\definecolor{dialinecolor}{rgb}{1.000000, 1.000000, 1.000000}
\pgfsetfillcolor{dialinecolor}
\fill (7.996250\du,10.850000\du)--(7.996250\du,12.750000\du)--(9.503750\du,12.750000\du)--(9.503750\du,10.850000\du)--cycle;
\pgfsetlinewidth{0.100000\du}
\pgfsetdash{}{0pt}
\pgfsetdash{}{0pt}
\pgfsetmiterjoin
\definecolor{dialinecolor}{rgb}{0.000000, 0.000000, 0.000000}
\pgfsetstrokecolor{dialinecolor}
\draw (7.996250\du,10.850000\du)--(7.996250\du,12.750000\du)--(9.503750\du,12.750000\du)--(9.503750\du,10.850000\du)--cycle;
% setfont left to latex
\definecolor{dialinecolor}{rgb}{0.000000, 0.000000, 0.000000}
\pgfsetstrokecolor{dialinecolor}
\node at (8.750000\du,11.995000\du){7};
\definecolor{dialinecolor}{rgb}{1.000000, 1.000000, 1.000000}
\pgfsetfillcolor{dialinecolor}
\fill (10.300000\du,10.950000\du)--(10.300000\du,12.850000\du)--(12.300000\du,12.850000\du)--(12.300000\du,10.950000\du)--cycle;
\pgfsetlinewidth{0.100000\du}
\pgfsetdash{}{0pt}
\pgfsetdash{}{0pt}
\pgfsetmiterjoin
\definecolor{dialinecolor}{rgb}{0.000000, 0.000000, 0.000000}
\pgfsetstrokecolor{dialinecolor}
\draw (10.300000\du,10.950000\du)--(10.300000\du,12.850000\du)--(12.300000\du,12.850000\du)--(12.300000\du,10.950000\du)--cycle;
% setfont left to latex
\definecolor{dialinecolor}{rgb}{0.000000, 0.000000, 0.000000}
\pgfsetstrokecolor{dialinecolor}
\node at (11.300000\du,12.095000\du){11};
\definecolor{dialinecolor}{rgb}{1.000000, 1.000000, 1.000000}
\pgfsetfillcolor{dialinecolor}
\fill (15.200000\du,10.950000\du)--(15.200000\du,12.850000\du)--(17.200000\du,12.850000\du)--(17.200000\du,10.950000\du)--cycle;
\pgfsetlinewidth{0.100000\du}
\pgfsetdash{}{0pt}
\pgfsetdash{}{0pt}
\pgfsetmiterjoin
\definecolor{dialinecolor}{rgb}{0.000000, 0.000000, 0.000000}
\pgfsetstrokecolor{dialinecolor}
\draw (15.200000\du,10.950000\du)--(15.200000\du,12.850000\du)--(17.200000\du,12.850000\du)--(17.200000\du,10.950000\du)--cycle;
% setfont left to latex
\definecolor{dialinecolor}{rgb}{0.000000, 0.000000, 0.000000}
\pgfsetstrokecolor{dialinecolor}
\node at (16.200000\du,12.095000\du){15};
\definecolor{dialinecolor}{rgb}{1.000000, 1.000000, 1.000000}
\pgfsetfillcolor{dialinecolor}
\fill (17.800000\du,10.950000\du)--(17.800000\du,12.850000\du)--(19.800000\du,12.850000\du)--(19.800000\du,10.950000\du)--cycle;
\pgfsetlinewidth{0.100000\du}
\pgfsetdash{}{0pt}
\pgfsetdash{}{0pt}
\pgfsetmiterjoin
\definecolor{dialinecolor}{rgb}{0.000000, 0.000000, 0.000000}
\pgfsetstrokecolor{dialinecolor}
\draw (17.800000\du,10.950000\du)--(17.800000\du,12.850000\du)--(19.800000\du,12.850000\du)--(19.800000\du,10.950000\du)--cycle;
% setfont left to latex
\definecolor{dialinecolor}{rgb}{0.000000, 0.000000, 0.000000}
\pgfsetstrokecolor{dialinecolor}
\node at (18.800000\du,12.095000\du){17};
\definecolor{dialinecolor}{rgb}{1.000000, 1.000000, 1.000000}
\pgfsetfillcolor{dialinecolor}
\fill (20.350000\du,10.900000\du)--(20.350000\du,12.800000\du)--(22.350000\du,12.800000\du)--(22.350000\du,10.900000\du)--cycle;
\pgfsetlinewidth{0.100000\du}
\pgfsetdash{}{0pt}
\pgfsetdash{}{0pt}
\pgfsetmiterjoin
\definecolor{dialinecolor}{rgb}{0.000000, 0.000000, 0.000000}
\pgfsetstrokecolor{dialinecolor}
\draw (20.350000\du,10.900000\du)--(20.350000\du,12.800000\du)--(22.350000\du,12.800000\du)--(22.350000\du,10.900000\du)--cycle;
% setfont left to latex
\definecolor{dialinecolor}{rgb}{0.000000, 0.000000, 0.000000}
\pgfsetstrokecolor{dialinecolor}
\node at (21.350000\du,12.045000\du){19};
\pgfsetlinewidth{0.100000\du}
\pgfsetdash{}{0pt}
\pgfsetdash{}{0pt}
\pgfsetbuttcap
{
\definecolor{dialinecolor}{rgb}{0.000000, 0.000000, 0.000000}
\pgfsetfillcolor{dialinecolor}
% was here!!!
\pgfsetarrowsend{stealth}
\definecolor{dialinecolor}{rgb}{0.000000, 0.000000, 0.000000}
\pgfsetstrokecolor{dialinecolor}
\draw (17.448413\du,9.300311\du)--(16.200000\du,10.950000\du);
}
\pgfsetlinewidth{0.100000\du}
\pgfsetdash{}{0pt}
\pgfsetdash{}{0pt}
\pgfsetbuttcap
{
\definecolor{dialinecolor}{rgb}{0.000000, 0.000000, 0.000000}
\pgfsetfillcolor{dialinecolor}
% was here!!!
\pgfsetarrowsend{stealth}
\definecolor{dialinecolor}{rgb}{0.000000, 0.000000, 0.000000}
\pgfsetstrokecolor{dialinecolor}
\draw (18.502759\du,9.300311\du)--(18.800000\du,10.950000\du);
}
\pgfsetlinewidth{0.100000\du}
\pgfsetdash{}{0pt}
\pgfsetdash{}{0pt}
\pgfsetbuttcap
{
\definecolor{dialinecolor}{rgb}{0.000000, 0.000000, 0.000000}
\pgfsetfillcolor{dialinecolor}
% was here!!!
\pgfsetarrowsend{stealth}
\definecolor{dialinecolor}{rgb}{0.000000, 0.000000, 0.000000}
\pgfsetstrokecolor{dialinecolor}
\draw (19.233020\du,9.299213\du)--(20.520111\du,10.850052\du);
}
\pgfsetlinewidth{0.100000\du}
\pgfsetdash{}{0pt}
\pgfsetdash{}{0pt}
\pgfsetbuttcap
{
\definecolor{dialinecolor}{rgb}{0.000000, 0.000000, 0.000000}
\pgfsetfillcolor{dialinecolor}
% was here!!!
\pgfsetarrowsend{stealth}
\definecolor{dialinecolor}{rgb}{0.000000, 0.000000, 0.000000}
\pgfsetstrokecolor{dialinecolor}
\draw (6.769189\du,9.307126\du)--(3.450000\du,10.700000\du);
}
\pgfsetlinewidth{0.100000\du}
\pgfsetdash{}{0pt}
\pgfsetdash{}{0pt}
\pgfsetbuttcap
{
\definecolor{dialinecolor}{rgb}{0.000000, 0.000000, 0.000000}
\pgfsetfillcolor{dialinecolor}
% was here!!!
\pgfsetarrowsend{stealth}
\definecolor{dialinecolor}{rgb}{0.000000, 0.000000, 0.000000}
\pgfsetstrokecolor{dialinecolor}
\draw (8.120129\du,9.499658\du)--(7.108685\du,10.750171\du);
}
\pgfsetlinewidth{0.100000\du}
\pgfsetdash{}{0pt}
\pgfsetdash{}{0pt}
\pgfsetbuttcap
{
\definecolor{dialinecolor}{rgb}{0.000000, 0.000000, 0.000000}
\pgfsetfillcolor{dialinecolor}
% was here!!!
\pgfsetarrowsend{stealth}
\definecolor{dialinecolor}{rgb}{0.000000, 0.000000, 0.000000}
\pgfsetstrokecolor{dialinecolor}
\draw (8.950024\du,9.499719\du)--(8.836865\du,10.801050\du);
}
\pgfsetlinewidth{0.100000\du}
\pgfsetdash{}{0pt}
\pgfsetdash{}{0pt}
\pgfsetbuttcap
{
\definecolor{dialinecolor}{rgb}{0.000000, 0.000000, 0.000000}
\pgfsetfillcolor{dialinecolor}
% was here!!!
\pgfsetarrowsend{stealth}
\definecolor{dialinecolor}{rgb}{0.000000, 0.000000, 0.000000}
\pgfsetstrokecolor{dialinecolor}
\draw (9.775098\du,9.494043\du)--(10.666638\du,10.900696\du);
}
\end{tikzpicture}

            	\caption{(2,4)-Baum}
            \end{figure}
            
        \end{bsp}

        \subsection{Suchen in (2,4)-Bäumen}
            Suchen(k) $\checkmark$ \qquad Suche nach $k$ liefert Blatt $k^{'} = \text{ min }\{x \in S \mid k \leq x \}$ 
            
        \subsection{Einfügen in (2,4)-Bäumen}
            Einfügen(k): Zuerst \textit{Suchen}$(k)$ liefert Blatt $v_i$ mit \textit{Schlüssel} $(v_i) < k < \text{\textit{ Schlüssel}} (v_i) $ \\

        \begin{alltt}
            Sei \( w \) der parent von \( v \) : \\
            while \( w \) hat 5 kinder do Spalte(\( w \)) 
            \( w \) \( \Leftarrow \) parent(\( w \)) 
        \end{alltt}

        Laufzeit: $\LO(1+ \#$ Spaltungen) \\
        
        
        \subsection{Streichen in (2,4)-Bäumen}
            Streiche$(k)$ : Zuerst Suchen$(k)$ $\ra$ Endet in Blatt v mit Schlüssel k. \\
            \begin{enumerate}[1.]
                \item Fall:  $K$ steht auch in parent$(v) = w$ \\
                % Graphic for TeX using PGF
% Title: /home/alex/WualaDrive/myfiles/Uni/4 Semester/Algorithmen/Skript/algorithmen/Diagramm20.dia
% Creator: Dia v0.97.1
% CreationDate: Tue May 18 14:41:29 2010
% For: alex
% \usepackage{tikz}
% The following commands are not supported in PSTricks at present
% We define them conditionally, so when they are implemented,
% this pgf file will use them.
\ifx\du\undefined
  \newlength{\du}
\fi
\setlength{\du}{15\unitlength}
\begin{tikzpicture}
\pgftransformxscale{1.000000}
\pgftransformyscale{-1.000000}
\definecolor{dialinecolor}{rgb}{0.000000, 0.000000, 0.000000}
\pgfsetstrokecolor{dialinecolor}
\definecolor{dialinecolor}{rgb}{1.000000, 1.000000, 1.000000}
\pgfsetfillcolor{dialinecolor}
\pgfsetlinewidth{0.100000\du}
\pgfsetdash{}{0pt}
\pgfsetdash{}{0pt}
\pgfsetbuttcap
\pgfsetmiterjoin
\pgfsetlinewidth{0.100000\du}
\pgfsetbuttcap
\pgfsetmiterjoin
\pgfsetdash{}{0pt}
\definecolor{dialinecolor}{rgb}{1.000000, 1.000000, 1.000000}
\pgfsetfillcolor{dialinecolor}
\pgfpathmoveto{\pgfpoint{8.933333\du}{4.900000\du}}
\pgfpathlineto{\pgfpoint{10.266667\du}{4.900000\du}}
\pgfpathcurveto{\pgfpoint{10.450762\du}{4.900000\du}}{\pgfpoint{10.600000\du}{5.347715\du}}{\pgfpoint{10.600000\du}{5.900000\du}}
\pgfpathcurveto{\pgfpoint{10.600000\du}{6.452285\du}}{\pgfpoint{10.450762\du}{6.900000\du}}{\pgfpoint{10.266667\du}{6.900000\du}}
\pgfpathlineto{\pgfpoint{8.933333\du}{6.900000\du}}
\pgfpathcurveto{\pgfpoint{8.749238\du}{6.900000\du}}{\pgfpoint{8.600000\du}{6.452285\du}}{\pgfpoint{8.600000\du}{5.900000\du}}
\pgfpathcurveto{\pgfpoint{8.600000\du}{5.347715\du}}{\pgfpoint{8.749238\du}{4.900000\du}}{\pgfpoint{8.933333\du}{4.900000\du}}
\pgfusepath{fill}
\definecolor{dialinecolor}{rgb}{0.000000, 0.000000, 0.000000}
\pgfsetstrokecolor{dialinecolor}
\pgfpathmoveto{\pgfpoint{8.933333\du}{4.900000\du}}
\pgfpathlineto{\pgfpoint{10.266667\du}{4.900000\du}}
\pgfpathcurveto{\pgfpoint{10.450762\du}{4.900000\du}}{\pgfpoint{10.600000\du}{5.347715\du}}{\pgfpoint{10.600000\du}{5.900000\du}}
\pgfpathcurveto{\pgfpoint{10.600000\du}{6.452285\du}}{\pgfpoint{10.450762\du}{6.900000\du}}{\pgfpoint{10.266667\du}{6.900000\du}}
\pgfpathlineto{\pgfpoint{8.933333\du}{6.900000\du}}
\pgfpathcurveto{\pgfpoint{8.749238\du}{6.900000\du}}{\pgfpoint{8.600000\du}{6.452285\du}}{\pgfpoint{8.600000\du}{5.900000\du}}
\pgfpathcurveto{\pgfpoint{8.600000\du}{5.347715\du}}{\pgfpoint{8.749238\du}{4.900000\du}}{\pgfpoint{8.933333\du}{4.900000\du}}
\pgfusepath{stroke}
% setfont left to latex
\definecolor{dialinecolor}{rgb}{0.000000, 0.000000, 0.000000}
\pgfsetstrokecolor{dialinecolor}
\node at (9.600000\du,6.100000\du){k};
% setfont left to latex
\definecolor{dialinecolor}{rgb}{0.000000, 0.000000, 0.000000}
\pgfsetstrokecolor{dialinecolor}
\node[anchor=west] at (6.600000\du,5.200000\du){w};
\definecolor{dialinecolor}{rgb}{1.000000, 1.000000, 1.000000}
\pgfsetfillcolor{dialinecolor}
\fill (5.650000\du,9.300000\du)--(5.650000\du,11.200000\du)--(7.650000\du,11.200000\du)--(7.650000\du,9.300000\du)--cycle;
\pgfsetlinewidth{0.100000\du}
\pgfsetdash{}{0pt}
\pgfsetdash{}{0pt}
\pgfsetmiterjoin
\definecolor{dialinecolor}{rgb}{0.000000, 0.000000, 0.000000}
\pgfsetstrokecolor{dialinecolor}
\draw (5.650000\du,9.300000\du)--(5.650000\du,11.200000\du)--(7.650000\du,11.200000\du)--(7.650000\du,9.300000\du)--cycle;
% setfont left to latex
\definecolor{dialinecolor}{rgb}{0.000000, 0.000000, 0.000000}
\pgfsetstrokecolor{dialinecolor}
\node at (6.650000\du,10.445000\du){k};
\pgfsetlinewidth{0.100000\du}
\pgfsetdash{}{0pt}
\pgfsetdash{}{0pt}
\pgfsetbuttcap
{
\definecolor{dialinecolor}{rgb}{0.000000, 0.000000, 0.000000}
\pgfsetfillcolor{dialinecolor}
% was here!!!
\definecolor{dialinecolor}{rgb}{0.000000, 0.000000, 0.000000}
\pgfsetstrokecolor{dialinecolor}
\draw (6.650000\du,9.300000\du)--(8.769952\du,6.856665\du);
}
\pgfsetlinewidth{0.100000\du}
\pgfsetdash{{1.000000\du}{1.000000\du}}{0\du}
\pgfsetdash{{1.000000\du}{1.000000\du}}{0\du}
\pgfsetbuttcap
{
\definecolor{dialinecolor}{rgb}{0.000000, 0.000000, 0.000000}
\pgfsetfillcolor{dialinecolor}
% was here!!!
\definecolor{dialinecolor}{rgb}{0.000000, 0.000000, 0.000000}
\pgfsetstrokecolor{dialinecolor}
\draw (4.250000\du,8.300000\du)--(8.500000\du,6.200000\du);
}
\pgfsetlinewidth{0.100000\du}
\pgfsetdash{{1.000000\du}{1.000000\du}}{0\du}
\pgfsetdash{{1.000000\du}{1.000000\du}}{0\du}
\pgfsetbuttcap
{
\definecolor{dialinecolor}{rgb}{0.000000, 0.000000, 0.000000}
\pgfsetfillcolor{dialinecolor}
% was here!!!
\definecolor{dialinecolor}{rgb}{0.000000, 0.000000, 0.000000}
\pgfsetstrokecolor{dialinecolor}
\draw (10.244019\du,6.949512\du)--(12.300000\du,10.300000\du);
}
% setfont left to latex
\definecolor{dialinecolor}{rgb}{0.000000, 0.000000, 0.000000}
\pgfsetstrokecolor{dialinecolor}
\node[anchor=west] at (2.200000\du,9.150000\du){v};
\pgfsetlinewidth{0.100000\du}
\pgfsetdash{}{0pt}
\pgfsetdash{}{0pt}
\pgfsetbuttcap
{
\definecolor{dialinecolor}{rgb}{0.000000, 0.000000, 0.000000}
\pgfsetfillcolor{dialinecolor}
% was here!!!
\pgfsetarrowsend{stealth}
\definecolor{dialinecolor}{rgb}{0.000000, 0.000000, 0.000000}
\pgfsetstrokecolor{dialinecolor}
\pgfpathmoveto{\pgfpoint{12.999725\du}{11.199797\du}}
\pgfpatharc{127}{53}{5.000313\du and 5.000313\du}
\pgfusepath{stroke}
}
% setfont left to latex
\definecolor{dialinecolor}{rgb}{0.000000, 0.000000, 0.000000}
\pgfsetstrokecolor{dialinecolor}
\node[anchor=west] at (14.000000\du,8.350000\du){streiche k };
% setfont left to latex
\definecolor{dialinecolor}{rgb}{0.000000, 0.000000, 0.000000}
\pgfsetstrokecolor{dialinecolor}
\node[anchor=west] at (14.000000\du,9.150000\du){in beiden};
\pgfsetlinewidth{0.100000\du}
\pgfsetdash{}{0pt}
\pgfsetdash{}{0pt}
\pgfsetbuttcap
\pgfsetmiterjoin
\pgfsetlinewidth{0.100000\du}
\pgfsetbuttcap
\pgfsetmiterjoin
\pgfsetdash{}{0pt}
\definecolor{dialinecolor}{rgb}{1.000000, 1.000000, 1.000000}
\pgfsetfillcolor{dialinecolor}
\pgfpathmoveto{\pgfpoint{23.450000\du}{6.250000\du}}
\pgfpathlineto{\pgfpoint{26.050000\du}{6.250000\du}}
\pgfpathcurveto{\pgfpoint{26.408985\du}{6.250000\du}}{\pgfpoint{26.700000\du}{6.585786\du}}{\pgfpoint{26.700000\du}{7.000000\du}}
\pgfpathcurveto{\pgfpoint{26.700000\du}{7.414214\du}}{\pgfpoint{26.408985\du}{7.750000\du}}{\pgfpoint{26.050000\du}{7.750000\du}}
\pgfpathlineto{\pgfpoint{23.450000\du}{7.750000\du}}
\pgfpathcurveto{\pgfpoint{23.091015\du}{7.750000\du}}{\pgfpoint{22.800000\du}{7.414214\du}}{\pgfpoint{22.800000\du}{7.000000\du}}
\pgfpathcurveto{\pgfpoint{22.800000\du}{6.585786\du}}{\pgfpoint{23.091015\du}{6.250000\du}}{\pgfpoint{23.450000\du}{6.250000\du}}
\pgfusepath{fill}
\definecolor{dialinecolor}{rgb}{0.000000, 0.000000, 0.000000}
\pgfsetstrokecolor{dialinecolor}
\pgfpathmoveto{\pgfpoint{23.450000\du}{6.250000\du}}
\pgfpathlineto{\pgfpoint{26.050000\du}{6.250000\du}}
\pgfpathcurveto{\pgfpoint{26.408985\du}{6.250000\du}}{\pgfpoint{26.700000\du}{6.585786\du}}{\pgfpoint{26.700000\du}{7.000000\du}}
\pgfpathcurveto{\pgfpoint{26.700000\du}{7.414214\du}}{\pgfpoint{26.408985\du}{7.750000\du}}{\pgfpoint{26.050000\du}{7.750000\du}}
\pgfpathlineto{\pgfpoint{23.450000\du}{7.750000\du}}
\pgfpathcurveto{\pgfpoint{23.091015\du}{7.750000\du}}{\pgfpoint{22.800000\du}{7.414214\du}}{\pgfpoint{22.800000\du}{7.000000\du}}
\pgfpathcurveto{\pgfpoint{22.800000\du}{6.585786\du}}{\pgfpoint{23.091015\du}{6.250000\du}}{\pgfpoint{23.450000\du}{6.250000\du}}
\pgfusepath{stroke}
% setfont left to latex
\definecolor{dialinecolor}{rgb}{0.000000, 0.000000, 0.000000}
\pgfsetstrokecolor{dialinecolor}
\node at (24.750000\du,7.200000\du){......};
% setfont left to latex
\definecolor{dialinecolor}{rgb}{0.000000, 0.000000, 0.000000}
\pgfsetstrokecolor{dialinecolor}
\node[anchor=west] at (21.000000\du,6.500000\du){w};
\end{tikzpicture}

                \item
                $v$ rechtestes Kind $w$ sei Vorgänger von $v$ und $v^{'}$ der linke Nachbar mit Schlüssel $k^{'}$.\\
                Streiche $v$ und Ersetzte Vorkommen von $k$ durch $k^{'}$. Streiche $k^{'}$ im $parent(w)$, da nun höchster Schlüssel. 
                (vgl. \autoref{diag21:rechtestes_Kind_streichen})
                \begin{figure}
                	% Graphic for TeX using PGF
% Title: /home/alex/WualaDrive/myfiles/Uni/4 Semester/Algorithmen/Skript/algorithmen/Diagramm21.dia
% Creator: Dia v0.97.1
% CreationDate: Tue May 18 14:45:57 2010
% For: alex
% \usepackage{tikz}
% The following commands are not supported in PSTricks at present
% We define them conditionally, so when they are implemented,
% this pgf file will use them.
\ifx\du\undefined
  \newlength{\du}
\fi
\setlength{\du}{15\unitlength}
\begin{tikzpicture}
\pgftransformxscale{1.000000}
\pgftransformyscale{-1.000000}
\definecolor{dialinecolor}{rgb}{0.000000, 0.000000, 0.000000}
\pgfsetstrokecolor{dialinecolor}
\definecolor{dialinecolor}{rgb}{1.000000, 1.000000, 1.000000}
\pgfsetfillcolor{dialinecolor}
\pgfsetlinewidth{0.100000\du}
\pgfsetdash{}{0pt}
\pgfsetdash{}{0pt}
\pgfsetbuttcap
\pgfsetmiterjoin
\pgfsetlinewidth{0.100000\du}
\pgfsetbuttcap
\pgfsetmiterjoin
\pgfsetdash{}{0pt}
\definecolor{dialinecolor}{rgb}{1.000000, 1.000000, 1.000000}
\pgfsetfillcolor{dialinecolor}
\pgfpathmoveto{\pgfpoint{16.383333\du}{6.350000\du}}
\pgfpathlineto{\pgfpoint{17.716667\du}{6.350000\du}}
\pgfpathcurveto{\pgfpoint{17.900762\du}{6.350000\du}}{\pgfpoint{18.050000\du}{6.797715\du}}{\pgfpoint{18.050000\du}{7.350000\du}}
\pgfpathcurveto{\pgfpoint{18.050000\du}{7.902285\du}}{\pgfpoint{17.900762\du}{8.350000\du}}{\pgfpoint{17.716667\du}{8.350000\du}}
\pgfpathlineto{\pgfpoint{16.383333\du}{8.350000\du}}
\pgfpathcurveto{\pgfpoint{16.199238\du}{8.350000\du}}{\pgfpoint{16.050000\du}{7.902285\du}}{\pgfpoint{16.050000\du}{7.350000\du}}
\pgfpathcurveto{\pgfpoint{16.050000\du}{6.797715\du}}{\pgfpoint{16.199238\du}{6.350000\du}}{\pgfpoint{16.383333\du}{6.350000\du}}
\pgfusepath{fill}
\definecolor{dialinecolor}{rgb}{0.000000, 0.000000, 0.000000}
\pgfsetstrokecolor{dialinecolor}
\pgfpathmoveto{\pgfpoint{16.383333\du}{6.350000\du}}
\pgfpathlineto{\pgfpoint{17.716667\du}{6.350000\du}}
\pgfpathcurveto{\pgfpoint{17.900762\du}{6.350000\du}}{\pgfpoint{18.050000\du}{6.797715\du}}{\pgfpoint{18.050000\du}{7.350000\du}}
\pgfpathcurveto{\pgfpoint{18.050000\du}{7.902285\du}}{\pgfpoint{17.900762\du}{8.350000\du}}{\pgfpoint{17.716667\du}{8.350000\du}}
\pgfpathlineto{\pgfpoint{16.383333\du}{8.350000\du}}
\pgfpathcurveto{\pgfpoint{16.199238\du}{8.350000\du}}{\pgfpoint{16.050000\du}{7.902285\du}}{\pgfpoint{16.050000\du}{7.350000\du}}
\pgfpathcurveto{\pgfpoint{16.050000\du}{6.797715\du}}{\pgfpoint{16.199238\du}{6.350000\du}}{\pgfpoint{16.383333\du}{6.350000\du}}
\pgfusepath{stroke}
% setfont left to latex
\definecolor{dialinecolor}{rgb}{0.000000, 0.000000, 0.000000}
\pgfsetstrokecolor{dialinecolor}
\node at (17.050000\du,7.550000\du){k'};
% setfont left to latex
\definecolor{dialinecolor}{rgb}{0.000000, 0.000000, 0.000000}
\pgfsetstrokecolor{dialinecolor}
\node[anchor=west] at (14.400000\du,5.950000\du){w};
\pgfsetlinewidth{0.100000\du}
\pgfsetdash{}{0pt}
\pgfsetdash{}{0pt}
\pgfsetbuttcap
\pgfsetmiterjoin
\pgfsetlinewidth{0.100000\du}
\pgfsetbuttcap
\pgfsetmiterjoin
\pgfsetdash{}{0pt}
\definecolor{dialinecolor}{rgb}{1.000000, 1.000000, 1.000000}
\pgfsetfillcolor{dialinecolor}
\pgfpathmoveto{\pgfpoint{19.533333\du}{2.600000\du}}
\pgfpathlineto{\pgfpoint{20.866667\du}{2.600000\du}}
\pgfpathcurveto{\pgfpoint{21.050762\du}{2.600000\du}}{\pgfpoint{21.200000\du}{3.047715\du}}{\pgfpoint{21.200000\du}{3.600000\du}}
\pgfpathcurveto{\pgfpoint{21.200000\du}{4.152285\du}}{\pgfpoint{21.050762\du}{4.600000\du}}{\pgfpoint{20.866667\du}{4.600000\du}}
\pgfpathlineto{\pgfpoint{19.533333\du}{4.600000\du}}
\pgfpathcurveto{\pgfpoint{19.349238\du}{4.600000\du}}{\pgfpoint{19.200000\du}{4.152285\du}}{\pgfpoint{19.200000\du}{3.600000\du}}
\pgfpathcurveto{\pgfpoint{19.200000\du}{3.047715\du}}{\pgfpoint{19.349238\du}{2.600000\du}}{\pgfpoint{19.533333\du}{2.600000\du}}
\pgfusepath{fill}
\definecolor{dialinecolor}{rgb}{0.000000, 0.000000, 0.000000}
\pgfsetstrokecolor{dialinecolor}
\pgfpathmoveto{\pgfpoint{19.533333\du}{2.600000\du}}
\pgfpathlineto{\pgfpoint{20.866667\du}{2.600000\du}}
\pgfpathcurveto{\pgfpoint{21.050762\du}{2.600000\du}}{\pgfpoint{21.200000\du}{3.047715\du}}{\pgfpoint{21.200000\du}{3.600000\du}}
\pgfpathcurveto{\pgfpoint{21.200000\du}{4.152285\du}}{\pgfpoint{21.050762\du}{4.600000\du}}{\pgfpoint{20.866667\du}{4.600000\du}}
\pgfpathlineto{\pgfpoint{19.533333\du}{4.600000\du}}
\pgfpathcurveto{\pgfpoint{19.349238\du}{4.600000\du}}{\pgfpoint{19.200000\du}{4.152285\du}}{\pgfpoint{19.200000\du}{3.600000\du}}
\pgfpathcurveto{\pgfpoint{19.200000\du}{3.047715\du}}{\pgfpoint{19.349238\du}{2.600000\du}}{\pgfpoint{19.533333\du}{2.600000\du}}
\pgfusepath{stroke}
% setfont left to latex
\definecolor{dialinecolor}{rgb}{0.000000, 0.000000, 0.000000}
\pgfsetstrokecolor{dialinecolor}
\node at (20.200000\du,3.800000\du){k};
\pgfsetlinewidth{0.100000\du}
\pgfsetdash{}{0pt}
\pgfsetdash{}{0pt}
\pgfsetbuttcap
{
\definecolor{dialinecolor}{rgb}{0.000000, 0.000000, 0.000000}
\pgfsetfillcolor{dialinecolor}
% was here!!!
\definecolor{dialinecolor}{rgb}{0.000000, 0.000000, 0.000000}
\pgfsetstrokecolor{dialinecolor}
\draw (19.312915\du,4.466089\du)--(17.383333\du,6.350000\du);
}
\pgfsetlinewidth{0.100000\du}
\pgfsetdash{}{0pt}
\pgfsetdash{}{0pt}
\pgfsetbuttcap
{
\definecolor{dialinecolor}{rgb}{0.000000, 0.000000, 0.000000}
\pgfsetfillcolor{dialinecolor}
% was here!!!
\definecolor{dialinecolor}{rgb}{0.000000, 0.000000, 0.000000}
\pgfsetstrokecolor{dialinecolor}
\draw (20.930107\du,4.625626\du)--(22.300000\du,6.550000\du);
}
\definecolor{dialinecolor}{rgb}{1.000000, 1.000000, 1.000000}
\pgfsetfillcolor{dialinecolor}
\fill (15.950000\du,10.850000\du)--(15.950000\du,12.750000\du)--(17.950000\du,12.750000\du)--(17.950000\du,10.850000\du)--cycle;
\pgfsetlinewidth{0.100000\du}
\pgfsetdash{}{0pt}
\pgfsetdash{}{0pt}
\pgfsetmiterjoin
\definecolor{dialinecolor}{rgb}{0.000000, 0.000000, 0.000000}
\pgfsetstrokecolor{dialinecolor}
\draw (15.950000\du,10.850000\du)--(15.950000\du,12.750000\du)--(17.950000\du,12.750000\du)--(17.950000\du,10.850000\du)--cycle;
% setfont left to latex
\definecolor{dialinecolor}{rgb}{0.000000, 0.000000, 0.000000}
\pgfsetstrokecolor{dialinecolor}
\node at (16.950000\du,11.995000\du){k'};
% setfont left to latex
\definecolor{dialinecolor}{rgb}{0.000000, 0.000000, 0.000000}
\pgfsetstrokecolor{dialinecolor}
\node[anchor=west] at (18.150000\du,10.550000\du){v'};
\pgfsetlinewidth{0.100000\du}
\pgfsetdash{}{0pt}
\pgfsetdash{}{0pt}
\pgfsetbuttcap
{
\definecolor{dialinecolor}{rgb}{0.000000, 0.000000, 0.000000}
\pgfsetfillcolor{dialinecolor}
% was here!!!
\definecolor{dialinecolor}{rgb}{0.000000, 0.000000, 0.000000}
\pgfsetstrokecolor{dialinecolor}
\draw (16.972473\du,10.799945\du)--(17.026404\du,8.400031\du);
}
\pgfsetlinewidth{0.100000\du}
\pgfsetdash{}{0pt}
\pgfsetdash{}{0pt}
\pgfsetbuttcap
{
\definecolor{dialinecolor}{rgb}{0.000000, 0.000000, 0.000000}
\pgfsetfillcolor{dialinecolor}
% was here!!!
\definecolor{dialinecolor}{rgb}{0.000000, 0.000000, 0.000000}
\pgfsetstrokecolor{dialinecolor}
\draw (16.050000\du,7.350000\du)--(11.550000\du,10.350000\du);
}
\pgfsetlinewidth{0.100000\du}
\pgfsetdash{}{0pt}
\pgfsetdash{}{0pt}
\pgfsetbuttcap
{
\definecolor{dialinecolor}{rgb}{0.000000, 0.000000, 0.000000}
\pgfsetfillcolor{dialinecolor}
% was here!!!
\definecolor{dialinecolor}{rgb}{0.000000, 0.000000, 0.000000}
\pgfsetstrokecolor{dialinecolor}
\draw (16.186719\du,8.225977\du)--(13.650000\du,10.800000\du);
}
\pgfsetlinewidth{0.100000\du}
\pgfsetdash{}{0pt}
\pgfsetdash{}{0pt}
\pgfsetbuttcap
{
\definecolor{dialinecolor}{rgb}{0.000000, 0.000000, 0.000000}
\pgfsetfillcolor{dialinecolor}
% was here!!!
\definecolor{dialinecolor}{rgb}{0.000000, 0.000000, 0.000000}
\pgfsetstrokecolor{dialinecolor}
\draw (17.902814\du,8.276514\du)--(21.100000\du,11.750000\du);
}
\definecolor{dialinecolor}{rgb}{1.000000, 1.000000, 1.000000}
\pgfsetfillcolor{dialinecolor}
\fill (21.015000\du,11.500000\du)--(21.015000\du,13.400000\du)--(22.485000\du,13.400000\du)--(22.485000\du,11.500000\du)--cycle;
\pgfsetlinewidth{0.100000\du}
\pgfsetdash{}{0pt}
\pgfsetdash{}{0pt}
\pgfsetmiterjoin
\definecolor{dialinecolor}{rgb}{0.000000, 0.000000, 0.000000}
\pgfsetstrokecolor{dialinecolor}
\draw (21.015000\du,11.500000\du)--(21.015000\du,13.400000\du)--(22.485000\du,13.400000\du)--(22.485000\du,11.500000\du)--cycle;
% setfont left to latex
\definecolor{dialinecolor}{rgb}{0.000000, 0.000000, 0.000000}
\pgfsetstrokecolor{dialinecolor}
\node at (21.750000\du,12.645000\du){k};
% setfont left to latex
\definecolor{dialinecolor}{rgb}{0.000000, 0.000000, 0.000000}
\pgfsetstrokecolor{dialinecolor}
\node[anchor=west] at (23.050000\du,11.550000\du){v};
\pgfsetlinewidth{0.100000\du}
\pgfsetdash{}{0pt}
\pgfsetdash{}{0pt}
\pgfsetbuttcap
{
\definecolor{dialinecolor}{rgb}{0.000000, 0.000000, 0.000000}
\pgfsetfillcolor{dialinecolor}
% was here!!!
\definecolor{dialinecolor}{rgb}{0.000000, 0.000000, 0.000000}
\pgfsetstrokecolor{dialinecolor}
\draw (20.200000\du,14.300000\du)--(23.950000\du,10.400000\du);
}
\pgfsetlinewidth{0.100000\du}
\pgfsetdash{}{0pt}
\pgfsetdash{}{0pt}
\pgfsetbuttcap
{
\definecolor{dialinecolor}{rgb}{0.000000, 0.000000, 0.000000}
\pgfsetfillcolor{dialinecolor}
% was here!!!
\definecolor{dialinecolor}{rgb}{0.000000, 0.000000, 0.000000}
\pgfsetstrokecolor{dialinecolor}
\draw (19.350000\du,10.950000\du)--(24.800000\du,14.800000\du);
}
\pgfsetlinewidth{0.100000\du}
\pgfsetdash{}{0pt}
\pgfsetdash{}{0pt}
\pgfsetbuttcap
{
\definecolor{dialinecolor}{rgb}{0.000000, 0.000000, 0.000000}
\pgfsetfillcolor{dialinecolor}
% was here!!!
\pgfsetarrowsend{stealth}
\definecolor{dialinecolor}{rgb}{0.000000, 0.000000, 0.000000}
\pgfsetstrokecolor{dialinecolor}
\pgfpathmoveto{\pgfpoint{24.749848\du}{9.049853\du}}
\pgfpatharc{135}{42}{3.207812\du and 3.207812\du}
\pgfusepath{stroke}
}
\pgfsetlinewidth{0.100000\du}
\pgfsetdash{}{0pt}
\pgfsetdash{}{0pt}
\pgfsetbuttcap
\pgfsetmiterjoin
\pgfsetlinewidth{0.100000\du}
\pgfsetbuttcap
\pgfsetmiterjoin
\pgfsetdash{}{0pt}
\definecolor{dialinecolor}{rgb}{1.000000, 1.000000, 1.000000}
\pgfsetfillcolor{dialinecolor}
\pgfpathmoveto{\pgfpoint{32.283333\du}{6.900000\du}}
\pgfpathlineto{\pgfpoint{33.616667\du}{6.900000\du}}
\pgfpathcurveto{\pgfpoint{33.800762\du}{6.900000\du}}{\pgfpoint{33.950000\du}{7.347715\du}}{\pgfpoint{33.950000\du}{7.900000\du}}
\pgfpathcurveto{\pgfpoint{33.950000\du}{8.452285\du}}{\pgfpoint{33.800762\du}{8.900000\du}}{\pgfpoint{33.616667\du}{8.900000\du}}
\pgfpathlineto{\pgfpoint{32.283333\du}{8.900000\du}}
\pgfpathcurveto{\pgfpoint{32.099238\du}{8.900000\du}}{\pgfpoint{31.950000\du}{8.452285\du}}{\pgfpoint{31.950000\du}{7.900000\du}}
\pgfpathcurveto{\pgfpoint{31.950000\du}{7.347715\du}}{\pgfpoint{32.099238\du}{6.900000\du}}{\pgfpoint{32.283333\du}{6.900000\du}}
\pgfusepath{fill}
\definecolor{dialinecolor}{rgb}{0.000000, 0.000000, 0.000000}
\pgfsetstrokecolor{dialinecolor}
\pgfpathmoveto{\pgfpoint{32.283333\du}{6.900000\du}}
\pgfpathlineto{\pgfpoint{33.616667\du}{6.900000\du}}
\pgfpathcurveto{\pgfpoint{33.800762\du}{6.900000\du}}{\pgfpoint{33.950000\du}{7.347715\du}}{\pgfpoint{33.950000\du}{7.900000\du}}
\pgfpathcurveto{\pgfpoint{33.950000\du}{8.452285\du}}{\pgfpoint{33.800762\du}{8.900000\du}}{\pgfpoint{33.616667\du}{8.900000\du}}
\pgfpathlineto{\pgfpoint{32.283333\du}{8.900000\du}}
\pgfpathcurveto{\pgfpoint{32.099238\du}{8.900000\du}}{\pgfpoint{31.950000\du}{8.452285\du}}{\pgfpoint{31.950000\du}{7.900000\du}}
\pgfpathcurveto{\pgfpoint{31.950000\du}{7.347715\du}}{\pgfpoint{32.099238\du}{6.900000\du}}{\pgfpoint{32.283333\du}{6.900000\du}}
\pgfusepath{stroke}
% setfont left to latex
\definecolor{dialinecolor}{rgb}{0.000000, 0.000000, 0.000000}
\pgfsetstrokecolor{dialinecolor}
\node at (32.950000\du,8.100000\du){};
\pgfsetlinewidth{0.100000\du}
\pgfsetdash{}{0pt}
\pgfsetdash{}{0pt}
\pgfsetbuttcap
\pgfsetmiterjoin
\pgfsetlinewidth{0.100000\du}
\pgfsetbuttcap
\pgfsetmiterjoin
\pgfsetdash{}{0pt}
\definecolor{dialinecolor}{rgb}{1.000000, 1.000000, 1.000000}
\pgfsetfillcolor{dialinecolor}
\pgfpathmoveto{\pgfpoint{36.602500\du}{3.600000\du}}
\pgfpathlineto{\pgfpoint{37.447500\du}{3.600000\du}}
\pgfpathcurveto{\pgfpoint{37.564170\du}{3.600000\du}}{\pgfpoint{37.658750\du}{3.846243\du}}{\pgfpoint{37.658750\du}{4.150000\du}}
\pgfpathcurveto{\pgfpoint{37.658750\du}{4.453757\du}}{\pgfpoint{37.564170\du}{4.700000\du}}{\pgfpoint{37.447500\du}{4.700000\du}}
\pgfpathlineto{\pgfpoint{36.602500\du}{4.700000\du}}
\pgfpathcurveto{\pgfpoint{36.485830\du}{4.700000\du}}{\pgfpoint{36.391250\du}{4.453757\du}}{\pgfpoint{36.391250\du}{4.150000\du}}
\pgfpathcurveto{\pgfpoint{36.391250\du}{3.846243\du}}{\pgfpoint{36.485830\du}{3.600000\du}}{\pgfpoint{36.602500\du}{3.600000\du}}
\pgfusepath{fill}
\definecolor{dialinecolor}{rgb}{0.000000, 0.000000, 0.000000}
\pgfsetstrokecolor{dialinecolor}
\pgfpathmoveto{\pgfpoint{36.602500\du}{3.600000\du}}
\pgfpathlineto{\pgfpoint{37.447500\du}{3.600000\du}}
\pgfpathcurveto{\pgfpoint{37.564170\du}{3.600000\du}}{\pgfpoint{37.658750\du}{3.846243\du}}{\pgfpoint{37.658750\du}{4.150000\du}}
\pgfpathcurveto{\pgfpoint{37.658750\du}{4.453757\du}}{\pgfpoint{37.564170\du}{4.700000\du}}{\pgfpoint{37.447500\du}{4.700000\du}}
\pgfpathlineto{\pgfpoint{36.602500\du}{4.700000\du}}
\pgfpathcurveto{\pgfpoint{36.485830\du}{4.700000\du}}{\pgfpoint{36.391250\du}{4.453757\du}}{\pgfpoint{36.391250\du}{4.150000\du}}
\pgfpathcurveto{\pgfpoint{36.391250\du}{3.846243\du}}{\pgfpoint{36.485830\du}{3.600000\du}}{\pgfpoint{36.602500\du}{3.600000\du}}
\pgfusepath{stroke}
% setfont left to latex
\definecolor{dialinecolor}{rgb}{0.000000, 0.000000, 0.000000}
\pgfsetstrokecolor{dialinecolor}
\node at (37.025000\du,4.350000\du){k'};
\pgfsetlinewidth{0.100000\du}
\pgfsetdash{}{0pt}
\pgfsetdash{}{0pt}
\pgfsetbuttcap
{
\definecolor{dialinecolor}{rgb}{0.000000, 0.000000, 0.000000}
\pgfsetfillcolor{dialinecolor}
% was here!!!
\definecolor{dialinecolor}{rgb}{0.000000, 0.000000, 0.000000}
\pgfsetstrokecolor{dialinecolor}
\draw (33.836851\du,7.083879\du)--(36.470688\du,4.660103\du);
}
\pgfsetlinewidth{0.100000\du}
\pgfsetdash{}{0pt}
\pgfsetdash{}{0pt}
\pgfsetbuttcap
{
\definecolor{dialinecolor}{rgb}{0.000000, 0.000000, 0.000000}
\pgfsetfillcolor{dialinecolor}
% was here!!!
\definecolor{dialinecolor}{rgb}{0.000000, 0.000000, 0.000000}
\pgfsetstrokecolor{dialinecolor}
\draw (32.283333\du,8.900000\du)--(30.900000\du,11.500000\du);
}
\pgfsetlinewidth{0.100000\du}
\pgfsetdash{}{0pt}
\pgfsetdash{}{0pt}
\pgfsetbuttcap
{
\definecolor{dialinecolor}{rgb}{0.000000, 0.000000, 0.000000}
\pgfsetfillcolor{dialinecolor}
% was here!!!
\definecolor{dialinecolor}{rgb}{0.000000, 0.000000, 0.000000}
\pgfsetstrokecolor{dialinecolor}
\draw (32.950000\du,8.900000\du)--(33.600000\du,11.900000\du);
}
\pgfsetlinewidth{0.100000\du}
\pgfsetdash{}{0pt}
\pgfsetdash{}{0pt}
\pgfsetbuttcap
{
\definecolor{dialinecolor}{rgb}{0.000000, 0.000000, 0.000000}
\pgfsetfillcolor{dialinecolor}
% was here!!!
\definecolor{dialinecolor}{rgb}{0.000000, 0.000000, 0.000000}
\pgfsetstrokecolor{dialinecolor}
\draw (33.283333\du,8.900000\du)--(35.450000\du,11.850000\du);
}
\definecolor{dialinecolor}{rgb}{1.000000, 1.000000, 1.000000}
\pgfsetfillcolor{dialinecolor}
\fill (35.500000\du,11.500000\du)--(35.500000\du,13.400000\du)--(37.500000\du,13.400000\du)--(37.500000\du,11.500000\du)--cycle;
\pgfsetlinewidth{0.100000\du}
\pgfsetdash{}{0pt}
\pgfsetdash{}{0pt}
\pgfsetmiterjoin
\definecolor{dialinecolor}{rgb}{0.000000, 0.000000, 0.000000}
\pgfsetstrokecolor{dialinecolor}
\draw (35.500000\du,11.500000\du)--(35.500000\du,13.400000\du)--(37.500000\du,13.400000\du)--(37.500000\du,11.500000\du)--cycle;
% setfont left to latex
\definecolor{dialinecolor}{rgb}{0.000000, 0.000000, 0.000000}
\pgfsetstrokecolor{dialinecolor}
\node at (36.500000\du,12.645000\du){k'};
%% setfont left to latex
%\definecolor{dialinecolor}{rgb}{0.000000, 0.000000, 0.000000}
%\pgfsetstrokecolor{dialinecolor}
%\node[anchor=west] at (39.000000\du,7.550000\du){streiche v};
%% setfont left to latex
%\definecolor{dialinecolor}{rgb}{0.000000, 0.000000, 0.000000}
%\pgfsetstrokecolor{dialinecolor}
%\node[anchor=west] at (39.000000\du,8.350000\du){ersetze Vorkommen };
%% setfont left to latex
%\definecolor{dialinecolor}{rgb}{0.000000, 0.000000, 0.000000}
%\pgfsetstrokecolor{dialinecolor}
%\node[anchor=west] at (39.000000\du,9.150000\du){von k durch k'};
\end{tikzpicture}

                	\caption{Bsp. $v$ rechtestes Kind von $parent(v)$ streichen (blattorientiert)}
                	\label{diag21:rechtestes_Kind_streichen}
                \end{figure}
                \begin{alltt}
                while w hat ein Kind
                do Verschmelze oder Stehlen od
                \end{alltt}
                Verschmelzen: \\
                % Graphic for TeX using PGF
% Title: /home/alex/WualaDrive/myfiles/Uni/4 Semester/Algorithmen/Skript/algorithmen/Diagramm22.dia
% Creator: Dia v0.97.1
% CreationDate: Tue May 18 14:50:43 2010
% For: alex
% \usepackage{tikz}
% The following commands are not supported in PSTricks at present
% We define them conditionally, so when they are implemented,
% this pgf file will use them.
\ifx\du\undefined
  \newlength{\du}
\fi
\setlength{\du}{15\unitlength}
\begin{tikzpicture}
\pgftransformxscale{1.000000}
\pgftransformyscale{-1.000000}
\definecolor{dialinecolor}{rgb}{0.000000, 0.000000, 0.000000}
\pgfsetstrokecolor{dialinecolor}
\definecolor{dialinecolor}{rgb}{1.000000, 1.000000, 1.000000}
\pgfsetfillcolor{dialinecolor}
\pgfsetlinewidth{0.100000\du}
\pgfsetdash{}{0pt}
\pgfsetdash{}{0pt}
\pgfsetbuttcap
\pgfsetmiterjoin
\pgfsetlinewidth{0.100000\du}
\pgfsetbuttcap
\pgfsetmiterjoin
\pgfsetdash{}{0pt}
\definecolor{dialinecolor}{rgb}{1.000000, 1.000000, 1.000000}
\pgfsetfillcolor{dialinecolor}
\pgfpathmoveto{\pgfpoint{8.483333\du}{2.450000\du}}
\pgfpathlineto{\pgfpoint{9.816667\du}{2.450000\du}}
\pgfpathcurveto{\pgfpoint{10.000762\du}{2.450000\du}}{\pgfpoint{10.150000\du}{2.897715\du}}{\pgfpoint{10.150000\du}{3.450000\du}}
\pgfpathcurveto{\pgfpoint{10.150000\du}{4.002285\du}}{\pgfpoint{10.000762\du}{4.450000\du}}{\pgfpoint{9.816667\du}{4.450000\du}}
\pgfpathlineto{\pgfpoint{8.483333\du}{4.450000\du}}
\pgfpathcurveto{\pgfpoint{8.299238\du}{4.450000\du}}{\pgfpoint{8.150000\du}{4.002285\du}}{\pgfpoint{8.150000\du}{3.450000\du}}
\pgfpathcurveto{\pgfpoint{8.150000\du}{2.897715\du}}{\pgfpoint{8.299238\du}{2.450000\du}}{\pgfpoint{8.483333\du}{2.450000\du}}
\pgfusepath{fill}
\definecolor{dialinecolor}{rgb}{0.000000, 0.000000, 0.000000}
\pgfsetstrokecolor{dialinecolor}
\pgfpathmoveto{\pgfpoint{8.483333\du}{2.450000\du}}
\pgfpathlineto{\pgfpoint{9.816667\du}{2.450000\du}}
\pgfpathcurveto{\pgfpoint{10.000762\du}{2.450000\du}}{\pgfpoint{10.150000\du}{2.897715\du}}{\pgfpoint{10.150000\du}{3.450000\du}}
\pgfpathcurveto{\pgfpoint{10.150000\du}{4.002285\du}}{\pgfpoint{10.000762\du}{4.450000\du}}{\pgfpoint{9.816667\du}{4.450000\du}}
\pgfpathlineto{\pgfpoint{8.483333\du}{4.450000\du}}
\pgfpathcurveto{\pgfpoint{8.299238\du}{4.450000\du}}{\pgfpoint{8.150000\du}{4.002285\du}}{\pgfpoint{8.150000\du}{3.450000\du}}
\pgfpathcurveto{\pgfpoint{8.150000\du}{2.897715\du}}{\pgfpoint{8.299238\du}{2.450000\du}}{\pgfpoint{8.483333\du}{2.450000\du}}
\pgfusepath{stroke}
% setfont left to latex
\definecolor{dialinecolor}{rgb}{0.000000, 0.000000, 0.000000}
\pgfsetstrokecolor{dialinecolor}
\node at (9.150000\du,3.650000\du){};
\pgfsetlinewidth{0.100000\du}
\pgfsetdash{}{0pt}
\pgfsetdash{}{0pt}
\pgfsetbuttcap
\pgfsetmiterjoin
\pgfsetlinewidth{0.100000\du}
\pgfsetbuttcap
\pgfsetmiterjoin
\pgfsetdash{}{0pt}
\definecolor{dialinecolor}{rgb}{1.000000, 1.000000, 1.000000}
\pgfsetfillcolor{dialinecolor}
\pgfpathmoveto{\pgfpoint{5.533333\du}{7.250000\du}}
\pgfpathlineto{\pgfpoint{6.866667\du}{7.250000\du}}
\pgfpathcurveto{\pgfpoint{7.050762\du}{7.250000\du}}{\pgfpoint{7.200000\du}{7.697715\du}}{\pgfpoint{7.200000\du}{8.250000\du}}
\pgfpathcurveto{\pgfpoint{7.200000\du}{8.802285\du}}{\pgfpoint{7.050762\du}{9.250000\du}}{\pgfpoint{6.866667\du}{9.250000\du}}
\pgfpathlineto{\pgfpoint{5.533333\du}{9.250000\du}}
\pgfpathcurveto{\pgfpoint{5.349238\du}{9.250000\du}}{\pgfpoint{5.200000\du}{8.802285\du}}{\pgfpoint{5.200000\du}{8.250000\du}}
\pgfpathcurveto{\pgfpoint{5.200000\du}{7.697715\du}}{\pgfpoint{5.349238\du}{7.250000\du}}{\pgfpoint{5.533333\du}{7.250000\du}}
\pgfusepath{fill}
\definecolor{dialinecolor}{rgb}{0.000000, 0.000000, 0.000000}
\pgfsetstrokecolor{dialinecolor}
\pgfpathmoveto{\pgfpoint{5.533333\du}{7.250000\du}}
\pgfpathlineto{\pgfpoint{6.866667\du}{7.250000\du}}
\pgfpathcurveto{\pgfpoint{7.050762\du}{7.250000\du}}{\pgfpoint{7.200000\du}{7.697715\du}}{\pgfpoint{7.200000\du}{8.250000\du}}
\pgfpathcurveto{\pgfpoint{7.200000\du}{8.802285\du}}{\pgfpoint{7.050762\du}{9.250000\du}}{\pgfpoint{6.866667\du}{9.250000\du}}
\pgfpathlineto{\pgfpoint{5.533333\du}{9.250000\du}}
\pgfpathcurveto{\pgfpoint{5.349238\du}{9.250000\du}}{\pgfpoint{5.200000\du}{8.802285\du}}{\pgfpoint{5.200000\du}{8.250000\du}}
\pgfpathcurveto{\pgfpoint{5.200000\du}{7.697715\du}}{\pgfpoint{5.349238\du}{7.250000\du}}{\pgfpoint{5.533333\du}{7.250000\du}}
\pgfusepath{stroke}
% setfont left to latex
\definecolor{dialinecolor}{rgb}{0.000000, 0.000000, 0.000000}
\pgfsetstrokecolor{dialinecolor}
\node at (6.200000\du,8.450000\du){};
% setfont left to latex
\definecolor{dialinecolor}{rgb}{0.000000, 0.000000, 0.000000}
\pgfsetstrokecolor{dialinecolor}
\node[anchor=west] at (3.600000\du,7.250000\du){w};
\pgfsetlinewidth{0.100000\du}
\pgfsetdash{}{0pt}
\pgfsetdash{}{0pt}
\pgfsetbuttcap
\pgfsetmiterjoin
\pgfsetlinewidth{0.100000\du}
\pgfsetbuttcap
\pgfsetmiterjoin
\pgfsetdash{}{0pt}
\definecolor{dialinecolor}{rgb}{1.000000, 1.000000, 1.000000}
\pgfsetfillcolor{dialinecolor}
\pgfpathmoveto{\pgfpoint{9.783333\du}{7.250000\du}}
\pgfpathlineto{\pgfpoint{11.116667\du}{7.250000\du}}
\pgfpathcurveto{\pgfpoint{11.300762\du}{7.250000\du}}{\pgfpoint{11.450000\du}{7.697715\du}}{\pgfpoint{11.450000\du}{8.250000\du}}
\pgfpathcurveto{\pgfpoint{11.450000\du}{8.802285\du}}{\pgfpoint{11.300762\du}{9.250000\du}}{\pgfpoint{11.116667\du}{9.250000\du}}
\pgfpathlineto{\pgfpoint{9.783333\du}{9.250000\du}}
\pgfpathcurveto{\pgfpoint{9.599238\du}{9.250000\du}}{\pgfpoint{9.450000\du}{8.802285\du}}{\pgfpoint{9.450000\du}{8.250000\du}}
\pgfpathcurveto{\pgfpoint{9.450000\du}{7.697715\du}}{\pgfpoint{9.599238\du}{7.250000\du}}{\pgfpoint{9.783333\du}{7.250000\du}}
\pgfusepath{fill}
\definecolor{dialinecolor}{rgb}{0.000000, 0.000000, 0.000000}
\pgfsetstrokecolor{dialinecolor}
\pgfpathmoveto{\pgfpoint{9.783333\du}{7.250000\du}}
\pgfpathlineto{\pgfpoint{11.116667\du}{7.250000\du}}
\pgfpathcurveto{\pgfpoint{11.300762\du}{7.250000\du}}{\pgfpoint{11.450000\du}{7.697715\du}}{\pgfpoint{11.450000\du}{8.250000\du}}
\pgfpathcurveto{\pgfpoint{11.450000\du}{8.802285\du}}{\pgfpoint{11.300762\du}{9.250000\du}}{\pgfpoint{11.116667\du}{9.250000\du}}
\pgfpathlineto{\pgfpoint{9.783333\du}{9.250000\du}}
\pgfpathcurveto{\pgfpoint{9.599238\du}{9.250000\du}}{\pgfpoint{9.450000\du}{8.802285\du}}{\pgfpoint{9.450000\du}{8.250000\du}}
\pgfpathcurveto{\pgfpoint{9.450000\du}{7.697715\du}}{\pgfpoint{9.599238\du}{7.250000\du}}{\pgfpoint{9.783333\du}{7.250000\du}}
\pgfusepath{stroke}
% setfont left to latex
\definecolor{dialinecolor}{rgb}{0.000000, 0.000000, 0.000000}
\pgfsetstrokecolor{dialinecolor}
\node at (10.450000\du,8.450000\du){};
\definecolor{dialinecolor}{rgb}{1.000000, 1.000000, 1.000000}
\pgfsetfillcolor{dialinecolor}
\fill (4.900000\du,10.650000\du)--(4.900000\du,12.550000\du)--(6.900000\du,12.550000\du)--(6.900000\du,10.650000\du)--cycle;
\pgfsetlinewidth{0.100000\du}
\pgfsetdash{}{0pt}
\pgfsetdash{}{0pt}
\pgfsetmiterjoin
\definecolor{dialinecolor}{rgb}{0.000000, 0.000000, 0.000000}
\pgfsetstrokecolor{dialinecolor}
\draw (4.900000\du,10.650000\du)--(4.900000\du,12.550000\du)--(6.900000\du,12.550000\du)--(6.900000\du,10.650000\du)--cycle;
% setfont left to latex
\definecolor{dialinecolor}{rgb}{0.000000, 0.000000, 0.000000}
\pgfsetstrokecolor{dialinecolor}
\node at (5.900000\du,11.795000\du){k'};
\pgfsetlinewidth{0.100000\du}
\pgfsetdash{}{0pt}
\pgfsetdash{}{0pt}
\pgfsetbuttcap
{
\definecolor{dialinecolor}{rgb}{0.000000, 0.000000, 0.000000}
\pgfsetfillcolor{dialinecolor}
% was here!!!
\definecolor{dialinecolor}{rgb}{0.000000, 0.000000, 0.000000}
\pgfsetstrokecolor{dialinecolor}
\draw (6.845493\du,7.199707\du)--(8.504507\du,4.500293\du);
}
\pgfsetlinewidth{0.100000\du}
\pgfsetdash{}{0pt}
\pgfsetdash{}{0pt}
\pgfsetbuttcap
{
\definecolor{dialinecolor}{rgb}{0.000000, 0.000000, 0.000000}
\pgfsetfillcolor{dialinecolor}
% was here!!!
\definecolor{dialinecolor}{rgb}{0.000000, 0.000000, 0.000000}
\pgfsetstrokecolor{dialinecolor}
\draw (5.989539\du,10.600153\du)--(6.106104\du,9.298511\du);
}
\pgfsetlinewidth{0.100000\du}
\pgfsetdash{}{0pt}
\pgfsetdash{}{0pt}
\pgfsetbuttcap
{
\definecolor{dialinecolor}{rgb}{0.000000, 0.000000, 0.000000}
\pgfsetfillcolor{dialinecolor}
% was here!!!
\definecolor{dialinecolor}{rgb}{0.000000, 0.000000, 0.000000}
\pgfsetstrokecolor{dialinecolor}
\draw (10.165546\du,7.199707\du)--(9.434454\du,4.500293\du);
}
\pgfsetlinewidth{0.100000\du}
\pgfsetdash{}{0pt}
\pgfsetdash{}{0pt}
\pgfsetbuttcap
{
\definecolor{dialinecolor}{rgb}{0.000000, 0.000000, 0.000000}
\pgfsetfillcolor{dialinecolor}
% was here!!!
\definecolor{dialinecolor}{rgb}{0.000000, 0.000000, 0.000000}
\pgfsetstrokecolor{dialinecolor}
\draw (10.116667\du,9.250000\du)--(9.050000\du,11.700000\du);
}
\pgfsetlinewidth{0.100000\du}
\pgfsetdash{}{0pt}
\pgfsetdash{}{0pt}
\pgfsetbuttcap
{
\definecolor{dialinecolor}{rgb}{0.000000, 0.000000, 0.000000}
\pgfsetfillcolor{dialinecolor}
% was here!!!
\definecolor{dialinecolor}{rgb}{0.000000, 0.000000, 0.000000}
\pgfsetstrokecolor{dialinecolor}
\draw (10.982562\du,9.299908\du)--(12.200000\du,11.700000\du);
}
\pgfsetlinewidth{0.100000\du}
\pgfsetdash{}{0pt}
\pgfsetdash{}{0pt}
\pgfsetbuttcap
{
\definecolor{dialinecolor}{rgb}{0.000000, 0.000000, 0.000000}
\pgfsetfillcolor{dialinecolor}
% was here!!!
\pgfsetarrowsend{stealth}
\definecolor{dialinecolor}{rgb}{0.000000, 0.000000, 0.000000}
\pgfsetstrokecolor{dialinecolor}
\pgfpathmoveto{\pgfpoint{12.849923\du}{7.749885\du}}
\pgfpatharc{147}{33}{2.211563\du and 2.211563\du}
\pgfusepath{stroke}
}
\pgfsetlinewidth{0.100000\du}
\pgfsetdash{}{0pt}
\pgfsetdash{}{0pt}
\pgfsetbuttcap
\pgfsetmiterjoin
\pgfsetlinewidth{0.100000\du}
\pgfsetbuttcap
\pgfsetmiterjoin
\pgfsetdash{}{0pt}
\definecolor{dialinecolor}{rgb}{1.000000, 1.000000, 1.000000}
\pgfsetfillcolor{dialinecolor}
\pgfpathmoveto{\pgfpoint{20.833333\du}{2.950000\du}}
\pgfpathlineto{\pgfpoint{22.166667\du}{2.950000\du}}
\pgfpathcurveto{\pgfpoint{22.350762\du}{2.950000\du}}{\pgfpoint{22.500000\du}{3.397715\du}}{\pgfpoint{22.500000\du}{3.950000\du}}
\pgfpathcurveto{\pgfpoint{22.500000\du}{4.502285\du}}{\pgfpoint{22.350762\du}{4.950000\du}}{\pgfpoint{22.166667\du}{4.950000\du}}
\pgfpathlineto{\pgfpoint{20.833333\du}{4.950000\du}}
\pgfpathcurveto{\pgfpoint{20.649238\du}{4.950000\du}}{\pgfpoint{20.500000\du}{4.502285\du}}{\pgfpoint{20.500000\du}{3.950000\du}}
\pgfpathcurveto{\pgfpoint{20.500000\du}{3.397715\du}}{\pgfpoint{20.649238\du}{2.950000\du}}{\pgfpoint{20.833333\du}{2.950000\du}}
\pgfusepath{fill}
\definecolor{dialinecolor}{rgb}{0.000000, 0.000000, 0.000000}
\pgfsetstrokecolor{dialinecolor}
\pgfpathmoveto{\pgfpoint{20.833333\du}{2.950000\du}}
\pgfpathlineto{\pgfpoint{22.166667\du}{2.950000\du}}
\pgfpathcurveto{\pgfpoint{22.350762\du}{2.950000\du}}{\pgfpoint{22.500000\du}{3.397715\du}}{\pgfpoint{22.500000\du}{3.950000\du}}
\pgfpathcurveto{\pgfpoint{22.500000\du}{4.502285\du}}{\pgfpoint{22.350762\du}{4.950000\du}}{\pgfpoint{22.166667\du}{4.950000\du}}
\pgfpathlineto{\pgfpoint{20.833333\du}{4.950000\du}}
\pgfpathcurveto{\pgfpoint{20.649238\du}{4.950000\du}}{\pgfpoint{20.500000\du}{4.502285\du}}{\pgfpoint{20.500000\du}{3.950000\du}}
\pgfpathcurveto{\pgfpoint{20.500000\du}{3.397715\du}}{\pgfpoint{20.649238\du}{2.950000\du}}{\pgfpoint{20.833333\du}{2.950000\du}}
\pgfusepath{stroke}
% setfont left to latex
\definecolor{dialinecolor}{rgb}{0.000000, 0.000000, 0.000000}
\pgfsetstrokecolor{dialinecolor}
\node at (21.500000\du,4.150000\du){};
\pgfsetlinewidth{0.100000\du}
\pgfsetdash{}{0pt}
\pgfsetdash{}{0pt}
\pgfsetbuttcap
\pgfsetmiterjoin
\pgfsetlinewidth{0.100000\du}
\pgfsetbuttcap
\pgfsetmiterjoin
\pgfsetdash{}{0pt}
\definecolor{dialinecolor}{rgb}{1.000000, 1.000000, 1.000000}
\pgfsetfillcolor{dialinecolor}
\pgfpathmoveto{\pgfpoint{20.491667\du}{6.500000\du}}
\pgfpathlineto{\pgfpoint{22.258333\du}{6.500000\du}}
\pgfpathcurveto{\pgfpoint{22.502259\du}{6.500000\du}}{\pgfpoint{22.700000\du}{6.869365\du}}{\pgfpoint{22.700000\du}{7.325000\du}}
\pgfpathcurveto{\pgfpoint{22.700000\du}{7.780635\du}}{\pgfpoint{22.502259\du}{8.150000\du}}{\pgfpoint{22.258333\du}{8.150000\du}}
\pgfpathlineto{\pgfpoint{20.491667\du}{8.150000\du}}
\pgfpathcurveto{\pgfpoint{20.247741\du}{8.150000\du}}{\pgfpoint{20.050000\du}{7.780635\du}}{\pgfpoint{20.050000\du}{7.325000\du}}
\pgfpathcurveto{\pgfpoint{20.050000\du}{6.869365\du}}{\pgfpoint{20.247741\du}{6.500000\du}}{\pgfpoint{20.491667\du}{6.500000\du}}
\pgfusepath{fill}
\definecolor{dialinecolor}{rgb}{0.000000, 0.000000, 0.000000}
\pgfsetstrokecolor{dialinecolor}
\pgfpathmoveto{\pgfpoint{20.491667\du}{6.500000\du}}
\pgfpathlineto{\pgfpoint{22.258333\du}{6.500000\du}}
\pgfpathcurveto{\pgfpoint{22.502259\du}{6.500000\du}}{\pgfpoint{22.700000\du}{6.869365\du}}{\pgfpoint{22.700000\du}{7.325000\du}}
\pgfpathcurveto{\pgfpoint{22.700000\du}{7.780635\du}}{\pgfpoint{22.502259\du}{8.150000\du}}{\pgfpoint{22.258333\du}{8.150000\du}}
\pgfpathlineto{\pgfpoint{20.491667\du}{8.150000\du}}
\pgfpathcurveto{\pgfpoint{20.247741\du}{8.150000\du}}{\pgfpoint{20.050000\du}{7.780635\du}}{\pgfpoint{20.050000\du}{7.325000\du}}
\pgfpathcurveto{\pgfpoint{20.050000\du}{6.869365\du}}{\pgfpoint{20.247741\du}{6.500000\du}}{\pgfpoint{20.491667\du}{6.500000\du}}
\pgfusepath{stroke}
% setfont left to latex
\definecolor{dialinecolor}{rgb}{0.000000, 0.000000, 0.000000}
\pgfsetstrokecolor{dialinecolor}
\node at (21.375000\du,7.525000\du){};
\definecolor{dialinecolor}{rgb}{1.000000, 1.000000, 1.000000}
\pgfsetfillcolor{dialinecolor}
\fill (18.750000\du,9.300000\du)--(18.750000\du,11.200000\du)--(20.750000\du,11.200000\du)--(20.750000\du,9.300000\du)--cycle;
\pgfsetlinewidth{0.100000\du}
\pgfsetdash{}{0pt}
\pgfsetdash{}{0pt}
\pgfsetmiterjoin
\definecolor{dialinecolor}{rgb}{0.000000, 0.000000, 0.000000}
\pgfsetstrokecolor{dialinecolor}
\draw (18.750000\du,9.300000\du)--(18.750000\du,11.200000\du)--(20.750000\du,11.200000\du)--(20.750000\du,9.300000\du)--cycle;
% setfont left to latex
\definecolor{dialinecolor}{rgb}{0.000000, 0.000000, 0.000000}
\pgfsetstrokecolor{dialinecolor}
\node at (19.750000\du,10.445000\du){k'};
\pgfsetlinewidth{0.100000\du}
\pgfsetdash{}{0pt}
\pgfsetdash{}{0pt}
\pgfsetbuttcap
{
\definecolor{dialinecolor}{rgb}{0.000000, 0.000000, 0.000000}
\pgfsetfillcolor{dialinecolor}
% was here!!!
\definecolor{dialinecolor}{rgb}{0.000000, 0.000000, 0.000000}
\pgfsetstrokecolor{dialinecolor}
\draw (21.461105\du,5.000156\du)--(21.407379\du,6.450763\du);
}
\pgfsetlinewidth{0.100000\du}
\pgfsetdash{}{0pt}
\pgfsetdash{}{0pt}
\pgfsetbuttcap
{
\definecolor{dialinecolor}{rgb}{0.000000, 0.000000, 0.000000}
\pgfsetfillcolor{dialinecolor}
% was here!!!
\definecolor{dialinecolor}{rgb}{0.000000, 0.000000, 0.000000}
\pgfsetstrokecolor{dialinecolor}
\draw (20.889008\du,8.199786\du)--(20.305618\du,9.249887\du);
}
\pgfsetlinewidth{0.100000\du}
\pgfsetdash{}{0pt}
\pgfsetdash{}{0pt}
\pgfsetbuttcap
{
\definecolor{dialinecolor}{rgb}{0.000000, 0.000000, 0.000000}
\pgfsetfillcolor{dialinecolor}
% was here!!!
\definecolor{dialinecolor}{rgb}{0.000000, 0.000000, 0.000000}
\pgfsetstrokecolor{dialinecolor}
\draw (21.494733\du,8.199969\du)--(21.700000\du,9.700000\du);
}
\pgfsetlinewidth{0.100000\du}
\pgfsetdash{}{0pt}
\pgfsetdash{}{0pt}
\pgfsetbuttcap
{
\definecolor{dialinecolor}{rgb}{0.000000, 0.000000, 0.000000}
\pgfsetfillcolor{dialinecolor}
% was here!!!
\definecolor{dialinecolor}{rgb}{0.000000, 0.000000, 0.000000}
\pgfsetstrokecolor{dialinecolor}
\draw (22.076038\du,8.199133\du)--(23.400000\du,9.850000\du);
}
% setfont left to latex
\definecolor{dialinecolor}{rgb}{0.000000, 0.000000, 0.000000}
\pgfsetstrokecolor{dialinecolor}
\node[anchor=west] at (24.450000\du,3.100000\du){Falls Geschwister-};
% setfont left to latex
\definecolor{dialinecolor}{rgb}{0.000000, 0.000000, 0.000000}
\pgfsetstrokecolor{dialinecolor}
\node[anchor=west] at (24.450000\du,3.900000\du){knoten Grad 2 hat.};
% setfont left to latex
\definecolor{dialinecolor}{rgb}{0.000000, 0.000000, 0.000000}
\pgfsetstrokecolor{dialinecolor}
\node[anchor=west] at (24.450000\du,4.700000\du){};
% setfont left to latex
\definecolor{dialinecolor}{rgb}{0.000000, 0.000000, 0.000000}
\pgfsetstrokecolor{dialinecolor}
\node[anchor=west] at (24.450000\du,5.500000\du){verschmelze Geschw.-};
% setfont left to latex
\definecolor{dialinecolor}{rgb}{0.000000, 0.000000, 0.000000}
\pgfsetstrokecolor{dialinecolor}
\node[anchor=west] at (24.450000\du,6.300000\du){knoten zu einem };
% setfont left to latex
\definecolor{dialinecolor}{rgb}{0.000000, 0.000000, 0.000000}
\pgfsetstrokecolor{dialinecolor}
\node[anchor=west] at (24.450000\du,7.100000\du){Knoten Grad 3. };
% setfont left to latex
\definecolor{dialinecolor}{rgb}{0.000000, 0.000000, 0.000000}
\pgfsetstrokecolor{dialinecolor}
\node[anchor=west] at (24.450000\du,7.900000\du){};
% setfont left to latex
\definecolor{dialinecolor}{rgb}{0.000000, 0.000000, 0.000000}
\pgfsetstrokecolor{dialinecolor}
\node[anchor=west] at (24.450000\du,8.700000\du){Setze fort!};
\end{tikzpicture}
\\
                Stehlen: Geschwisterknoten von $w$ hat 3 oder 4 Kinder:  Gibt uns an w ab. \\
                % Graphic for TeX using PGF
% Title: /home/alex/WualaDrive/myfiles/Uni/4 Semester/Algorithmen/Skript/algorithmen/Diagramm23.dia
% Creator: Dia v0.97.1
% CreationDate: Tue May 18 14:53:22 2010
% For: alex
% \usepackage{tikz}
% The following commands are not supported in PSTricks at present
% We define them conditionally, so when they are implemented,
% this pgf file will use them.
\ifx\du\undefined
  \newlength{\du}
\fi
\setlength{\du}{15\unitlength}
\begin{tikzpicture}
\pgftransformxscale{1.000000}
\pgftransformyscale{-1.000000}
\definecolor{dialinecolor}{rgb}{0.000000, 0.000000, 0.000000}
\pgfsetstrokecolor{dialinecolor}
\definecolor{dialinecolor}{rgb}{1.000000, 1.000000, 1.000000}
\pgfsetfillcolor{dialinecolor}
\pgfsetlinewidth{0.100000\du}
\pgfsetdash{}{0pt}
\pgfsetdash{}{0pt}
\pgfsetbuttcap
\pgfsetmiterjoin
\pgfsetlinewidth{0.100000\du}
\pgfsetbuttcap
\pgfsetmiterjoin
\pgfsetdash{}{0pt}
\definecolor{dialinecolor}{rgb}{1.000000, 1.000000, 1.000000}
\pgfsetfillcolor{dialinecolor}
\pgfpathmoveto{\pgfpoint{7.433333\du}{3.200000\du}}
\pgfpathlineto{\pgfpoint{8.766667\du}{3.200000\du}}
\pgfpathcurveto{\pgfpoint{8.950762\du}{3.200000\du}}{\pgfpoint{9.100000\du}{3.647715\du}}{\pgfpoint{9.100000\du}{4.200000\du}}
\pgfpathcurveto{\pgfpoint{9.100000\du}{4.752285\du}}{\pgfpoint{8.950762\du}{5.200000\du}}{\pgfpoint{8.766667\du}{5.200000\du}}
\pgfpathlineto{\pgfpoint{7.433333\du}{5.200000\du}}
\pgfpathcurveto{\pgfpoint{7.249238\du}{5.200000\du}}{\pgfpoint{7.100000\du}{4.752285\du}}{\pgfpoint{7.100000\du}{4.200000\du}}
\pgfpathcurveto{\pgfpoint{7.100000\du}{3.647715\du}}{\pgfpoint{7.249238\du}{3.200000\du}}{\pgfpoint{7.433333\du}{3.200000\du}}
\pgfusepath{fill}
\definecolor{dialinecolor}{rgb}{0.000000, 0.000000, 0.000000}
\pgfsetstrokecolor{dialinecolor}
\pgfpathmoveto{\pgfpoint{7.433333\du}{3.200000\du}}
\pgfpathlineto{\pgfpoint{8.766667\du}{3.200000\du}}
\pgfpathcurveto{\pgfpoint{8.950762\du}{3.200000\du}}{\pgfpoint{9.100000\du}{3.647715\du}}{\pgfpoint{9.100000\du}{4.200000\du}}
\pgfpathcurveto{\pgfpoint{9.100000\du}{4.752285\du}}{\pgfpoint{8.950762\du}{5.200000\du}}{\pgfpoint{8.766667\du}{5.200000\du}}
\pgfpathlineto{\pgfpoint{7.433333\du}{5.200000\du}}
\pgfpathcurveto{\pgfpoint{7.249238\du}{5.200000\du}}{\pgfpoint{7.100000\du}{4.752285\du}}{\pgfpoint{7.100000\du}{4.200000\du}}
\pgfpathcurveto{\pgfpoint{7.100000\du}{3.647715\du}}{\pgfpoint{7.249238\du}{3.200000\du}}{\pgfpoint{7.433333\du}{3.200000\du}}
\pgfusepath{stroke}
% setfont left to latex
\definecolor{dialinecolor}{rgb}{0.000000, 0.000000, 0.000000}
\pgfsetstrokecolor{dialinecolor}
\node at (8.100000\du,4.400000\du){};
\pgfsetlinewidth{0.100000\du}
\pgfsetdash{}{0pt}
\pgfsetdash{}{0pt}
\pgfsetbuttcap
\pgfsetmiterjoin
\pgfsetlinewidth{0.100000\du}
\pgfsetbuttcap
\pgfsetmiterjoin
\pgfsetdash{}{0pt}
\definecolor{dialinecolor}{rgb}{0.000000, 0.000000, 0.000000}
\pgfsetfillcolor{dialinecolor}
\pgfpathmoveto{\pgfpoint{5.083333\du}{7.450000\du}}
\pgfpathlineto{\pgfpoint{6.416667\du}{7.450000\du}}
\pgfpathcurveto{\pgfpoint{6.600762\du}{7.450000\du}}{\pgfpoint{6.750000\du}{7.897715\du}}{\pgfpoint{6.750000\du}{8.450000\du}}
\pgfpathcurveto{\pgfpoint{6.750000\du}{9.002285\du}}{\pgfpoint{6.600762\du}{9.450000\du}}{\pgfpoint{6.416667\du}{9.450000\du}}
\pgfpathlineto{\pgfpoint{5.083333\du}{9.450000\du}}
\pgfpathcurveto{\pgfpoint{4.899238\du}{9.450000\du}}{\pgfpoint{4.750000\du}{9.002285\du}}{\pgfpoint{4.750000\du}{8.450000\du}}
\pgfpathcurveto{\pgfpoint{4.750000\du}{7.897715\du}}{\pgfpoint{4.899238\du}{7.450000\du}}{\pgfpoint{5.083333\du}{7.450000\du}}
\pgfusepath{fill}
\definecolor{dialinecolor}{rgb}{0.000000, 0.000000, 0.000000}
\pgfsetstrokecolor{dialinecolor}
\pgfpathmoveto{\pgfpoint{5.083333\du}{7.450000\du}}
\pgfpathlineto{\pgfpoint{6.416667\du}{7.450000\du}}
\pgfpathcurveto{\pgfpoint{6.600762\du}{7.450000\du}}{\pgfpoint{6.750000\du}{7.897715\du}}{\pgfpoint{6.750000\du}{8.450000\du}}
\pgfpathcurveto{\pgfpoint{6.750000\du}{9.002285\du}}{\pgfpoint{6.600762\du}{9.450000\du}}{\pgfpoint{6.416667\du}{9.450000\du}}
\pgfpathlineto{\pgfpoint{5.083333\du}{9.450000\du}}
\pgfpathcurveto{\pgfpoint{4.899238\du}{9.450000\du}}{\pgfpoint{4.750000\du}{9.002285\du}}{\pgfpoint{4.750000\du}{8.450000\du}}
\pgfpathcurveto{\pgfpoint{4.750000\du}{7.897715\du}}{\pgfpoint{4.899238\du}{7.450000\du}}{\pgfpoint{5.083333\du}{7.450000\du}}
\pgfusepath{stroke}
% setfont left to latex
\definecolor{dialinecolor}{rgb}{0.000000, 0.000000, 0.000000}
\pgfsetstrokecolor{dialinecolor}
\node at (5.750000\du,8.650000\du){};
\pgfsetlinewidth{0.100000\du}
\pgfsetdash{}{0pt}
\pgfsetdash{}{0pt}
\pgfsetbuttcap
\pgfsetmiterjoin
\pgfsetlinewidth{0.100000\du}
\pgfsetbuttcap
\pgfsetmiterjoin
\pgfsetdash{}{0pt}
\definecolor{dialinecolor}{rgb}{1.000000, 1.000000, 1.000000}
\pgfsetfillcolor{dialinecolor}
\pgfpathmoveto{\pgfpoint{9.083333\du}{7.350000\du}}
\pgfpathlineto{\pgfpoint{10.416667\du}{7.350000\du}}
\pgfpathcurveto{\pgfpoint{10.600762\du}{7.350000\du}}{\pgfpoint{10.750000\du}{7.797715\du}}{\pgfpoint{10.750000\du}{8.350000\du}}
\pgfpathcurveto{\pgfpoint{10.750000\du}{8.902285\du}}{\pgfpoint{10.600762\du}{9.350000\du}}{\pgfpoint{10.416667\du}{9.350000\du}}
\pgfpathlineto{\pgfpoint{9.083333\du}{9.350000\du}}
\pgfpathcurveto{\pgfpoint{8.899238\du}{9.350000\du}}{\pgfpoint{8.750000\du}{8.902285\du}}{\pgfpoint{8.750000\du}{8.350000\du}}
\pgfpathcurveto{\pgfpoint{8.750000\du}{7.797715\du}}{\pgfpoint{8.899238\du}{7.350000\du}}{\pgfpoint{9.083333\du}{7.350000\du}}
\pgfusepath{fill}
\definecolor{dialinecolor}{rgb}{0.000000, 0.000000, 0.000000}
\pgfsetstrokecolor{dialinecolor}
\pgfpathmoveto{\pgfpoint{9.083333\du}{7.350000\du}}
\pgfpathlineto{\pgfpoint{10.416667\du}{7.350000\du}}
\pgfpathcurveto{\pgfpoint{10.600762\du}{7.350000\du}}{\pgfpoint{10.750000\du}{7.797715\du}}{\pgfpoint{10.750000\du}{8.350000\du}}
\pgfpathcurveto{\pgfpoint{10.750000\du}{8.902285\du}}{\pgfpoint{10.600762\du}{9.350000\du}}{\pgfpoint{10.416667\du}{9.350000\du}}
\pgfpathlineto{\pgfpoint{9.083333\du}{9.350000\du}}
\pgfpathcurveto{\pgfpoint{8.899238\du}{9.350000\du}}{\pgfpoint{8.750000\du}{8.902285\du}}{\pgfpoint{8.750000\du}{8.350000\du}}
\pgfpathcurveto{\pgfpoint{8.750000\du}{7.797715\du}}{\pgfpoint{8.899238\du}{7.350000\du}}{\pgfpoint{9.083333\du}{7.350000\du}}
\pgfusepath{stroke}
% setfont left to latex
\definecolor{dialinecolor}{rgb}{0.000000, 0.000000, 0.000000}
\pgfsetstrokecolor{dialinecolor}
\node at (9.750000\du,8.550000\du){};
\definecolor{dialinecolor}{rgb}{1.000000, 1.000000, 1.000000}
\pgfsetfillcolor{dialinecolor}
\fill (4.800000\du,11.750000\du)--(4.800000\du,13.650000\du)--(6.800000\du,13.650000\du)--(6.800000\du,11.750000\du)--cycle;
\pgfsetlinewidth{0.100000\du}
\pgfsetdash{}{0pt}
\pgfsetdash{}{0pt}
\pgfsetmiterjoin
\definecolor{dialinecolor}{rgb}{0.000000, 0.000000, 0.000000}
\pgfsetstrokecolor{dialinecolor}
\draw (4.800000\du,11.750000\du)--(4.800000\du,13.650000\du)--(6.800000\du,13.650000\du)--(6.800000\du,11.750000\du)--cycle;
% setfont left to latex
\definecolor{dialinecolor}{rgb}{0.000000, 0.000000, 0.000000}
\pgfsetstrokecolor{dialinecolor}
\node at (5.800000\du,12.895000\du){};
\pgfsetlinewidth{0.100000\du}
\pgfsetdash{}{0pt}
\pgfsetdash{}{0pt}
\pgfsetbuttcap
{
\definecolor{dialinecolor}{rgb}{0.000000, 0.000000, 0.000000}
\pgfsetfillcolor{dialinecolor}
% was here!!!
\pgfsetarrowsend{stealth}
\definecolor{dialinecolor}{rgb}{0.000000, 0.000000, 0.000000}
\pgfsetstrokecolor{dialinecolor}
\draw (7.520532\du,5.247974\du)--(6.329468\du,7.402026\du);
}
\pgfsetlinewidth{0.100000\du}
\pgfsetdash{}{0pt}
\pgfsetdash{}{0pt}
\pgfsetbuttcap
{
\definecolor{dialinecolor}{rgb}{0.000000, 0.000000, 0.000000}
\pgfsetfillcolor{dialinecolor}
% was here!!!
\pgfsetarrowsend{stealth}
\definecolor{dialinecolor}{rgb}{0.000000, 0.000000, 0.000000}
\pgfsetstrokecolor{dialinecolor}
\draw (8.517535\du,5.250165\du)--(9.332465\du,7.299835\du);
}
\pgfsetlinewidth{0.100000\du}
\pgfsetdash{}{0pt}
\pgfsetdash{}{0pt}
\pgfsetbuttcap
{
\definecolor{dialinecolor}{rgb}{0.000000, 0.000000, 0.000000}
\pgfsetfillcolor{dialinecolor}
% was here!!!
\pgfsetarrowsend{stealth}
\definecolor{dialinecolor}{rgb}{0.000000, 0.000000, 0.000000}
\pgfsetstrokecolor{dialinecolor}
\draw (5.762329\du,9.497974\du)--(5.788257\du,11.701831\du);
}
% setfont left to latex
\definecolor{dialinecolor}{rgb}{0.000000, 0.000000, 0.000000}
\pgfsetstrokecolor{dialinecolor}
\node[anchor=west] at (3.750000\du,7.450000\du){w};
\pgfsetlinewidth{0.100000\du}
\pgfsetdash{}{0pt}
\pgfsetdash{}{0pt}
\pgfsetbuttcap
{
\definecolor{dialinecolor}{rgb}{0.000000, 0.000000, 0.000000}
\pgfsetfillcolor{dialinecolor}
% was here!!!
\pgfsetarrowsend{stealth}
\definecolor{dialinecolor}{rgb}{0.000000, 0.000000, 0.000000}
\pgfsetstrokecolor{dialinecolor}
\draw (9.416667\du,9.350000\du)--(8.700000\du,11.700000\du);
}
\pgfsetlinewidth{0.100000\du}
\pgfsetdash{}{0pt}
\pgfsetdash{}{0pt}
\pgfsetbuttcap
{
\definecolor{dialinecolor}{rgb}{0.000000, 0.000000, 0.000000}
\pgfsetfillcolor{dialinecolor}
% was here!!!
\pgfsetarrowsend{stealth}
\definecolor{dialinecolor}{rgb}{0.000000, 0.000000, 0.000000}
\pgfsetstrokecolor{dialinecolor}
\draw (9.750000\du,9.350000\du)--(10.000000\du,11.750000\du);
}
\pgfsetlinewidth{0.100000\du}
\pgfsetdash{}{0pt}
\pgfsetdash{}{0pt}
\pgfsetbuttcap
{
\definecolor{dialinecolor}{rgb}{0.000000, 0.000000, 0.000000}
\pgfsetfillcolor{dialinecolor}
% was here!!!
\pgfsetarrowsend{stealth}
\definecolor{dialinecolor}{rgb}{0.000000, 0.000000, 0.000000}
\pgfsetstrokecolor{dialinecolor}
\draw (10.199890\du,9.399744\du)--(11.250000\du,11.850000\du);
}
\pgfsetlinewidth{0.100000\du}
\pgfsetdash{}{0pt}
\pgfsetdash{}{0pt}
\pgfsetbuttcap
{
\definecolor{dialinecolor}{rgb}{0.000000, 0.000000, 0.000000}
\pgfsetfillcolor{dialinecolor}
% was here!!!
\pgfsetarrowsend{stealth}
\definecolor{dialinecolor}{rgb}{0.000000, 0.000000, 0.000000}
\pgfsetstrokecolor{dialinecolor}
\pgfpathmoveto{\pgfpoint{12.049850\du}{7.599847\du}}
\pgfpatharc{136}{44}{3.261563\du and 3.261563\du}
\pgfusepath{stroke}
}
\pgfsetlinewidth{0.100000\du}
\pgfsetdash{}{0pt}
\pgfsetdash{}{0pt}
\pgfsetbuttcap
\pgfsetmiterjoin
\pgfsetlinewidth{0.100000\du}
\pgfsetbuttcap
\pgfsetmiterjoin
\pgfsetdash{}{0pt}
\definecolor{dialinecolor}{rgb}{1.000000, 1.000000, 1.000000}
\pgfsetfillcolor{dialinecolor}
\pgfpathmoveto{\pgfpoint{19.933333\du}{3.600000\du}}
\pgfpathlineto{\pgfpoint{21.266667\du}{3.600000\du}}
\pgfpathcurveto{\pgfpoint{21.450762\du}{3.600000\du}}{\pgfpoint{21.600000\du}{4.047715\du}}{\pgfpoint{21.600000\du}{4.600000\du}}
\pgfpathcurveto{\pgfpoint{21.600000\du}{5.152285\du}}{\pgfpoint{21.450762\du}{5.600000\du}}{\pgfpoint{21.266667\du}{5.600000\du}}
\pgfpathlineto{\pgfpoint{19.933333\du}{5.600000\du}}
\pgfpathcurveto{\pgfpoint{19.749238\du}{5.600000\du}}{\pgfpoint{19.600000\du}{5.152285\du}}{\pgfpoint{19.600000\du}{4.600000\du}}
\pgfpathcurveto{\pgfpoint{19.600000\du}{4.047715\du}}{\pgfpoint{19.749238\du}{3.600000\du}}{\pgfpoint{19.933333\du}{3.600000\du}}
\pgfusepath{fill}
\definecolor{dialinecolor}{rgb}{0.000000, 0.000000, 0.000000}
\pgfsetstrokecolor{dialinecolor}
\pgfpathmoveto{\pgfpoint{19.933333\du}{3.600000\du}}
\pgfpathlineto{\pgfpoint{21.266667\du}{3.600000\du}}
\pgfpathcurveto{\pgfpoint{21.450762\du}{3.600000\du}}{\pgfpoint{21.600000\du}{4.047715\du}}{\pgfpoint{21.600000\du}{4.600000\du}}
\pgfpathcurveto{\pgfpoint{21.600000\du}{5.152285\du}}{\pgfpoint{21.450762\du}{5.600000\du}}{\pgfpoint{21.266667\du}{5.600000\du}}
\pgfpathlineto{\pgfpoint{19.933333\du}{5.600000\du}}
\pgfpathcurveto{\pgfpoint{19.749238\du}{5.600000\du}}{\pgfpoint{19.600000\du}{5.152285\du}}{\pgfpoint{19.600000\du}{4.600000\du}}
\pgfpathcurveto{\pgfpoint{19.600000\du}{4.047715\du}}{\pgfpoint{19.749238\du}{3.600000\du}}{\pgfpoint{19.933333\du}{3.600000\du}}
\pgfusepath{stroke}
% setfont left to latex
\definecolor{dialinecolor}{rgb}{0.000000, 0.000000, 0.000000}
\pgfsetstrokecolor{dialinecolor}
\node at (20.600000\du,4.800000\du){};
\pgfsetlinewidth{0.100000\du}
\pgfsetdash{}{0pt}
\pgfsetdash{}{0pt}
\pgfsetbuttcap
\pgfsetmiterjoin
\pgfsetlinewidth{0.100000\du}
\pgfsetbuttcap
\pgfsetmiterjoin
\pgfsetdash{}{0pt}
\definecolor{dialinecolor}{rgb}{0.000000, 0.000000, 0.000000}
\pgfsetfillcolor{dialinecolor}
\pgfpathmoveto{\pgfpoint{18.483333\du}{7.650000\du}}
\pgfpathlineto{\pgfpoint{19.816667\du}{7.650000\du}}
\pgfpathcurveto{\pgfpoint{20.000762\du}{7.650000\du}}{\pgfpoint{20.150000\du}{8.097715\du}}{\pgfpoint{20.150000\du}{8.650000\du}}
\pgfpathcurveto{\pgfpoint{20.150000\du}{9.202285\du}}{\pgfpoint{20.000762\du}{9.650000\du}}{\pgfpoint{19.816667\du}{9.650000\du}}
\pgfpathlineto{\pgfpoint{18.483333\du}{9.650000\du}}
\pgfpathcurveto{\pgfpoint{18.299238\du}{9.650000\du}}{\pgfpoint{18.150000\du}{9.202285\du}}{\pgfpoint{18.150000\du}{8.650000\du}}
\pgfpathcurveto{\pgfpoint{18.150000\du}{8.097715\du}}{\pgfpoint{18.299238\du}{7.650000\du}}{\pgfpoint{18.483333\du}{7.650000\du}}
\pgfusepath{fill}
\definecolor{dialinecolor}{rgb}{0.000000, 0.000000, 0.000000}
\pgfsetstrokecolor{dialinecolor}
\pgfpathmoveto{\pgfpoint{18.483333\du}{7.650000\du}}
\pgfpathlineto{\pgfpoint{19.816667\du}{7.650000\du}}
\pgfpathcurveto{\pgfpoint{20.000762\du}{7.650000\du}}{\pgfpoint{20.150000\du}{8.097715\du}}{\pgfpoint{20.150000\du}{8.650000\du}}
\pgfpathcurveto{\pgfpoint{20.150000\du}{9.202285\du}}{\pgfpoint{20.000762\du}{9.650000\du}}{\pgfpoint{19.816667\du}{9.650000\du}}
\pgfpathlineto{\pgfpoint{18.483333\du}{9.650000\du}}
\pgfpathcurveto{\pgfpoint{18.299238\du}{9.650000\du}}{\pgfpoint{18.150000\du}{9.202285\du}}{\pgfpoint{18.150000\du}{8.650000\du}}
\pgfpathcurveto{\pgfpoint{18.150000\du}{8.097715\du}}{\pgfpoint{18.299238\du}{7.650000\du}}{\pgfpoint{18.483333\du}{7.650000\du}}
\pgfusepath{stroke}
% setfont left to latex
\definecolor{dialinecolor}{rgb}{0.000000, 0.000000, 0.000000}
\pgfsetstrokecolor{dialinecolor}
\node at (19.150000\du,8.850000\du){};
\pgfsetlinewidth{0.100000\du}
\pgfsetdash{}{0pt}
\pgfsetdash{}{0pt}
\pgfsetbuttcap
\pgfsetmiterjoin
\pgfsetlinewidth{0.100000\du}
\pgfsetbuttcap
\pgfsetmiterjoin
\pgfsetdash{}{0pt}
\definecolor{dialinecolor}{rgb}{0.000000, 0.000000, 0.000000}
\pgfsetfillcolor{dialinecolor}
\pgfpathmoveto{\pgfpoint{21.541667\du}{7.600000\du}}
\pgfpathlineto{\pgfpoint{23.308333\du}{7.600000\du}}
\pgfpathcurveto{\pgfpoint{23.552259\du}{7.600000\du}}{\pgfpoint{23.750000\du}{8.047715\du}}{\pgfpoint{23.750000\du}{8.600000\du}}
\pgfpathcurveto{\pgfpoint{23.750000\du}{9.152285\du}}{\pgfpoint{23.552259\du}{9.600000\du}}{\pgfpoint{23.308333\du}{9.600000\du}}
\pgfpathlineto{\pgfpoint{21.541667\du}{9.600000\du}}
\pgfpathcurveto{\pgfpoint{21.297741\du}{9.600000\du}}{\pgfpoint{21.100000\du}{9.152285\du}}{\pgfpoint{21.100000\du}{8.600000\du}}
\pgfpathcurveto{\pgfpoint{21.100000\du}{8.047715\du}}{\pgfpoint{21.297741\du}{7.600000\du}}{\pgfpoint{21.541667\du}{7.600000\du}}
\pgfusepath{fill}
\definecolor{dialinecolor}{rgb}{0.000000, 0.000000, 0.000000}
\pgfsetstrokecolor{dialinecolor}
\pgfpathmoveto{\pgfpoint{21.541667\du}{7.600000\du}}
\pgfpathlineto{\pgfpoint{23.308333\du}{7.600000\du}}
\pgfpathcurveto{\pgfpoint{23.552259\du}{7.600000\du}}{\pgfpoint{23.750000\du}{8.047715\du}}{\pgfpoint{23.750000\du}{8.600000\du}}
\pgfpathcurveto{\pgfpoint{23.750000\du}{9.152285\du}}{\pgfpoint{23.552259\du}{9.600000\du}}{\pgfpoint{23.308333\du}{9.600000\du}}
\pgfpathlineto{\pgfpoint{21.541667\du}{9.600000\du}}
\pgfpathcurveto{\pgfpoint{21.297741\du}{9.600000\du}}{\pgfpoint{21.100000\du}{9.152285\du}}{\pgfpoint{21.100000\du}{8.600000\du}}
\pgfpathcurveto{\pgfpoint{21.100000\du}{8.047715\du}}{\pgfpoint{21.297741\du}{7.600000\du}}{\pgfpoint{21.541667\du}{7.600000\du}}
\pgfusepath{stroke}
% setfont left to latex
\definecolor{dialinecolor}{rgb}{0.000000, 0.000000, 0.000000}
\pgfsetstrokecolor{dialinecolor}
\node at (22.425000\du,8.800000\du){};
\pgfsetlinewidth{0.100000\du}
\pgfsetdash{}{0pt}
\pgfsetdash{}{0pt}
\pgfsetbuttcap
{
\definecolor{dialinecolor}{rgb}{0.000000, 0.000000, 0.000000}
\pgfsetfillcolor{dialinecolor}
% was here!!!
\pgfsetarrowsend{stealth}
\definecolor{dialinecolor}{rgb}{0.000000, 0.000000, 0.000000}
\pgfsetstrokecolor{dialinecolor}
\draw (18.718420\du,9.700177\du)--(17.650000\du,12.300000\du);
}
\pgfsetlinewidth{0.100000\du}
\pgfsetdash{}{0pt}
\pgfsetdash{}{0pt}
\pgfsetbuttcap
{
\definecolor{dialinecolor}{rgb}{0.000000, 0.000000, 0.000000}
\pgfsetfillcolor{dialinecolor}
% was here!!!
\pgfsetarrowsend{stealth}
\definecolor{dialinecolor}{rgb}{0.000000, 0.000000, 0.000000}
\pgfsetstrokecolor{dialinecolor}
\draw (19.150000\du,9.650000\du)--(19.850000\du,12.500000\du);
}
\pgfsetlinewidth{0.100000\du}
\pgfsetdash{}{0pt}
\pgfsetdash{}{0pt}
\pgfsetbuttcap
{
\definecolor{dialinecolor}{rgb}{0.000000, 0.000000, 0.000000}
\pgfsetfillcolor{dialinecolor}
% was here!!!
\pgfsetarrowsend{stealth}
\definecolor{dialinecolor}{rgb}{0.000000, 0.000000, 0.000000}
\pgfsetstrokecolor{dialinecolor}
\draw (21.983333\du,9.600000\du)--(21.400000\du,12.150000\du);
}
\pgfsetlinewidth{0.100000\du}
\pgfsetdash{}{0pt}
\pgfsetdash{}{0pt}
\pgfsetbuttcap
{
\definecolor{dialinecolor}{rgb}{0.000000, 0.000000, 0.000000}
\pgfsetfillcolor{dialinecolor}
% was here!!!
\pgfsetarrowsend{stealth}
\definecolor{dialinecolor}{rgb}{0.000000, 0.000000, 0.000000}
\pgfsetstrokecolor{dialinecolor}
\draw (22.727795\du,9.648706\du)--(23.450000\du,12.150000\du);
}
\pgfsetlinewidth{0.100000\du}
\pgfsetdash{}{0pt}
\pgfsetdash{}{0pt}
\pgfsetbuttcap
{
\definecolor{dialinecolor}{rgb}{0.000000, 0.000000, 0.000000}
\pgfsetfillcolor{dialinecolor}
% was here!!!
\pgfsetarrowsend{stealth}
\definecolor{dialinecolor}{rgb}{0.000000, 0.000000, 0.000000}
\pgfsetstrokecolor{dialinecolor}
\draw (20.224402\du,5.649084\du)--(19.525598\du,7.600916\du);
}
\pgfsetlinewidth{0.100000\du}
\pgfsetdash{}{0pt}
\pgfsetdash{}{0pt}
\pgfsetbuttcap
{
\definecolor{dialinecolor}{rgb}{0.000000, 0.000000, 0.000000}
\pgfsetfillcolor{dialinecolor}
% was here!!!
\pgfsetarrowsend{stealth}
\definecolor{dialinecolor}{rgb}{0.000000, 0.000000, 0.000000}
\pgfsetstrokecolor{dialinecolor}
\draw (20.933333\du,5.600000\du)--(21.983333\du,7.600000\du);
}
% setfont left to latex
\definecolor{dialinecolor}{rgb}{0.000000, 0.000000, 0.000000}
\pgfsetstrokecolor{dialinecolor}
\node[anchor=west] at (24.100000\du,5.600000\du){keine };
% setfont left to latex
\definecolor{dialinecolor}{rgb}{0.000000, 0.000000, 0.000000}
\pgfsetstrokecolor{dialinecolor}
\node[anchor=west] at (24.100000\du,6.400000\du){Fortsetzung!};
\end{tikzpicture}

            \end{enumerate}

        \subsection{Laufzeitanalyse}
            Nach Spaltung hat der Baum eine bessere Struktur, deshalb:
            \emph{Amortisierte Analyse:}\\
            (= Kosten pro Operation gemittelt über n Operationen)

            \begin{lemma}
                In einem (2,4) Baum sind die amortisierten Kosten der Operationen Einfügen/Streichen $\LO(1)$.
            \end{lemma}
            \begin{proof}
                Beschreibung des Zustands des Baums T: \\
                $\textit{pot} (T)$
                \begin{eqnarray*}
                    &=& 2 \cdot \# \text{ Knoten von Grad } 1 \\
                    &+& 1 \cdot \# \text{ Knoten von Grad } 2 \\
                    &+& 0 \cdot \# \text{ Knoten von Grad } 3 \\
                    &+& 2 \cdot \# \text{ Knoten von Grad } 4 \\
                    &+& 4 \cdot \# \text{ Knoten von Grad } 5
                \end{eqnarray*}
                \emph{Invariante}: \\
                \begin{itemize}
                    \item $pot(T) \geq 0$ 
                    \item bei Spalten/Verschmelzen/Stehlen ist nur ein Knoten nicht auf Grad 2,3,4
                    \item Vor Einfügen /Streichen haben alle Knoten den Grad 2,3,4.
                \end{itemize}
                Einzeloperationen haben Kosten von $\LO(1)$.\\
                \emph{Behauptung:} Spalten / Verschmelzen verringern Potential. Stehlen erhöht es nicht. \\
                \emph{Beweis:} \\
                Stehlen. Knoten $w$ trug 2 Einheiten zum Potential bei und sein Nachbarknoten $p$. Danach trägt $w$ 1 Einheit bei, 
                Nachbar $\leq p +1$.
            \end{proof}

            \emph{Spalten}: $w$ trägt 4 bei zum Potential, der parent trägt p bei. Danach haben die beiden neuen Knoten $0$ und $1$, der
            parent hat $\leq  p +2 $ $\checkmark$.\\\

            \emph{Verschmelzen}: $w$ trägt 2 bei, Geschwisterknoten von $w$ trägt 1 bei, der parent trägt p bei \\
            $\Ra$ Danach 0 und $\leq p+1$ .\\\

            Amortisierte Laufzeit für Einfügen: Tatsächliche Kosten + Potentialerhöhung. \\
            Tatsächliche Kosten 1 + Potentialerhöhung $\leq 2$ \\
            Folge von $f$ Spaltungen: tatsächliche Kosten $f$, Potentialerhöhung $\leq f$. \\
            $\Ra$ Amortisierte Kosten von Einfügen: $\leq 3$ \\\\
            Streichen analog\\

        \subsection{Anwendung: Sortieren} 
            durch Einfügen: $\LO(n \log n)$ \
            vorsortierter Folgen: Mache Suchen billiger! \\
            Sei $x_1, \ldots, x_2$ Folge reeller Zahlen. \\
            $$F= \btl \{ (i,j) \mid i < j \text{ und } x_i > x_j \} \btr$$
            (Zahl der Inversionen). Es gilt $O \leq F \leq \frac{n^2}{2}$

            \begin{satz}
            $x_1, \ldots x_n$ kann in Zeit $\LO(n \log \frac{F}{n})$ sortiert werden. 
            \end{satz}

            \begin{proof}
                Sei $f_i = |\{ j \mid i < j , x_i > x_j \}|$ Es gilt $F= \sum \limits_{i} f_i$ \\
                Starte mit leerem (2,4) Baum. Füge Folge in umgekehrter Reihenfolge ein. \\
                % Graphic for TeX using PGF
% Title: /home/alex/WualaDrive/myfiles/Uni/4 Semester/Algorithmen/Skript/algorithmen/Diagramm24.dia
% Creator: Dia v0.97.1
% CreationDate: Tue May 18 15:50:34 2010
% For: alex
% \usepackage{tikz}
% The following commands are not supported in PSTricks at present
% We define them conditionally, so when they are implemented,
% this pgf file will use them.
\ifx\du\undefined
  \newlength{\du}
\fi
\setlength{\du}{15\unitlength}
\begin{tikzpicture}
\pgftransformxscale{1.000000}
\pgftransformyscale{-1.000000}
\definecolor{dialinecolor}{rgb}{0.000000, 0.000000, 0.000000}
\pgfsetstrokecolor{dialinecolor}
\definecolor{dialinecolor}{rgb}{1.000000, 1.000000, 1.000000}
\pgfsetfillcolor{dialinecolor}
\pgfsetlinewidth{0.100000\du}
\pgfsetdash{}{0pt}
\pgfsetdash{}{0pt}
\pgfsetbuttcap
\pgfsetmiterjoin
\pgfsetlinewidth{0.100000\du}
\pgfsetbuttcap
\pgfsetmiterjoin
\pgfsetdash{}{0pt}
\definecolor{dialinecolor}{rgb}{1.000000, 1.000000, 1.000000}
\pgfsetfillcolor{dialinecolor}
\fill (7.700000\du,10.100000\du)--(14.750000\du,10.100000\du)--(11.225000\du,2.650000\du)--cycle;
\definecolor{dialinecolor}{rgb}{0.000000, 0.000000, 0.000000}
\pgfsetstrokecolor{dialinecolor}
\draw (7.700000\du,10.100000\du)--(14.750000\du,10.100000\du)--(11.225000\du,2.650000\du)--cycle;
% setfont left to latex
\definecolor{dialinecolor}{rgb}{0.000000, 0.000000, 0.000000}
\pgfsetstrokecolor{dialinecolor}
\node at (11.225000\du,8.437500\du){};
% setfont left to latex
\definecolor{dialinecolor}{rgb}{0.000000, 0.000000, 0.000000}
\pgfsetstrokecolor{dialinecolor}
\node[anchor=west] at (6.800000\du,10.900000\du){x\_\{i+1\}};
% setfont left to latex
\definecolor{dialinecolor}{rgb}{0.000000, 0.000000, 0.000000}
\pgfsetstrokecolor{dialinecolor}
\node[anchor=west] at (13.650000\du,10.900000\du){x\_n};
\pgfsetlinewidth{0.100000\du}
\pgfsetdash{}{0pt}
\pgfsetdash{}{0pt}
\pgfsetbuttcap
{
\definecolor{dialinecolor}{rgb}{0.000000, 0.000000, 0.000000}
\pgfsetfillcolor{dialinecolor}
% was here!!!
\pgfsetarrowsend{stealth}
\definecolor{dialinecolor}{rgb}{0.000000, 0.000000, 0.000000}
\pgfsetstrokecolor{dialinecolor}
\draw (9.850000\du,12.250000\du)--(9.900000\du,10.900000\du);
}
\end{tikzpicture}

                Füge $x_n, \ldots, x_1$ ein. Einfügen $(x_i)$: Starte am linkesten Blatt; laufe hoch und drehe an der richtigen Stelle um,
                laufe runter und füge ein. Ist $x_i$ klein, so laufe nicht sehr hoch. \\\
                Es gilt (amortisiert): Kosten für Einfügen $= \LO(1 + \log f_i)$ \\
                Suchen$(x_i) = \LO(\log f_i)$ \\
                Einfügen: = $\LO(1)$. \\\\
                Laufe hoch bis Höhe $h$ (Umkehrpunkt): \\
                $$2^{h-1} \leq f_i \Ra h \leq 1 + \log f_i$$
                $\Ra$ Gesamtkosten: 
                \begin{align*}
                		\sum \limits_{i} \LO(1+ \log f_i) 	&= \LO(n + \sum \limits_{i=1}^n \log f_i) \\
                									&= \LO(n+\log \prod \limits_{i=1}^n f_i\\
                									&= \LO(n+\log \prod \limits_{i=1}^n \frac{F}{n}\\
                									&\leq \LO(n + n \log \frac{F}{n})
                \end{align*}
                $x_i$ wird nach dem $f_i$-ten Element in die bisherige Liste eingefügt.
            \end{proof}
        
        
        
    \section{B-Bäume (knotenorientiert)}
        \subsection*{Anmerkung}
            In allen Skizzen müssten die Pointer jeweils vor und nach einem Element stehen. Dies ist wegen des erhöhten Zeichenaufwandes vernachlässigt worden.
            
        \subsection{Idee und Definition}            
            Idee: Mehr Daten in einen Knoten 
            \begin{definition}
                B-Baum der Ordnung $k$. $k \geq 2$ ist ein Baum dessen: \\
                \begin{enumerate}[1]
                    \item Blätter alle gleiche Tiefe haben 
                    \item Wurzel $\geq 2$ Kinder und dessen andere inneren Knoten $\geq k$ Kinder Haben.
                    \item innere Knoten $\leq 2k-1$ Kinder haben. \\
                \end{enumerate}
            \end{definition}

            Höhe eines B-Baums: \\
            \begin{lemma}
                Sei $T$ ein Baum der Ordnung $k$ mit Höhe $h$ und $n$ Blättern. \\
                Dann gilt: \\
                \begin{center}
                    $2 \cdot k^{h-1} \leq n \leq (2k-1)^h$ 
                \end{center}
                \begin{proof}
                    Zahl der Blätter ist minimal, wenn jeweils die Minimal-Anzahl von Kindern vorkommen und maximal, wenn jeweils die Maximalzahl 
                    vorkommt.
                    $$2 \cdot \underbrace{k \cdot k \ldots \cdot k}_{h-1} \text{ mal} = n_{\min} \leq n \leq n_{\max} = (2k-1)^h$$
                \end{proof}

                Es gilt somit $\log_{(2k-1)} n \leq h \leq 1+ \log_k (\frac{n}{2})$ (logarithmieren!)\\
                Operationen: \textit{Suchen}(x), \textit{Einfügen}(x), \textit{Streichen}(x) \\
                Sei $u$ Knoten mit $k$ Kindern und Schlüsseln 
                $$s_1 < s_2 < \ldots < s_{l-1}$$ mit Unterbäumen $$T_1, \ldots , T_l$$
                       \begin{center}
            \begin{tikzpicture}[scale=1]
                \tikzstyle{bplus}=[rectangle split, rectangle split horizontal,
                    rectangle split ignore empty parts,rectangle split parts=5,draw,style={transform shape}]
                \tikzstyle{every node}=[bplus]
                \tikzstyle{level 1}=[sibling distance=20mm]
                \tikzstyle{level 2}=[sibling distance=15mm]
                \node {$s_ 1$ \nodepart{two} $s_2$ \nodepart{three} ... \nodepart{four} $s_{i-1}$ \nodepart{five} $s_i$} [->]
                    child {node {$T_1$}}
                    child {node {$T_2$}}
                    child {node {$...$}}
                    child {node {$...$}}
                    child {node {$T_{i-1}$}}
                    child {node {$T_i$}}
                ;
            \end{tikzpicture}
        \end{center}          
                Es gilt für alle $v \in T_i$ und Schlüssel $s$ in $v$: \\
                \begin{center}
                    $s \leq s_i$ falls $i=1$ \\
                    $s_{i-1} < s \leq s_i$ falls $1 < i < l$ \\
                    $s_{i-1} < s$ falls $i=l$ 
                \end{center}
            \end{lemma}

            Algorithmus: \\
            \begin{alltt}
                Suchen(x): Starte in w= Wurzel \\
                suche kleinstes \( s_i \) in w mit \( x \leq s_i \) \\
                falls \( s_i \) existiert, dann 
                if \( x= s_i \) then gefunden 
                else w \( \leftarrow \) i-tes Kind von w , suche dort weiter
                falls \( s_i \) nicht existiert, dann 
                w \( \leftarrow \) rechtestes Kind von w, suche dort weiter.
            \end{alltt}
            Laufzeit: $\LO(\log_k n \cdot k)$ (Höhe mal Suche innerhalb eines Knotens)\\

        \subsection{Einfügen(x)}
            Suchen(x) endet in Blatt b mit parent v, der l Kinder hat. Nun müssen 2 Fälle unterschieden werden.
            \begin{enumerate}
                \item In Blatt b ist noch genügend Platz für eine Einfügung vorhanden. Dann füge x in b ein.
                \item Nach Einfügung von x hätte Blatt b $2k-1$ Elemente: Teile auf.
            \end{enumerate}
            
            \emph{Aufteilen bei Einfügung:} Sollte bei einer Einfügung der 2. Fall aufgetreten sein, muss der Blattknoten geteilt werden. Sortiere dazu die Blattelemente aus b und entferne den Median $m$. Bilde aus den Elementen $e$ mit $e < m$ das Blatt $b$ und aus den verbleibenden Elementen das Blatt b'. Füge dann $m$ als neuen Seperator in den Vaterknoten $v$ ein. Die Blätter $b$ und $b$' stehen dann jeweils links und rechts von $m$. Ein Beispiel mit konkreten Werten und Erklärungen findet sich im nächsten Kapitel.
            
            \emph{Beachte:} Dabei kann der Vaterknoten $v$ ebenfalls mehr als $2k-2$ Elemente enthalten. Die Einfügung muss also rekursiv nach oben fortgesetzt werden. Wird dabei die Wurzel aufgespalten, so ergibt sich eine neue Wurzel mit 2 Kindern und Tiefe wächst um 1. Dies ist die einzige Möglichkeit, mit der die Tiefe eines B-Baums wachsen kann.
            
            
        \subsection{Konkrete Beispiele für die Operationen}
            \subsubsection{Einfügung mit Split}
                Es wird zu beginn vom folgenden B-Baum ausgegangen:
                \begin{center}
                        
        \begin{center}
            \begin{tikzpicture}
                \tikzstyle{bplus}=[rectangle split, rectangle split horizontal,
                    rectangle split ignore empty parts,draw]
                \tikzstyle{every node}=[bplus]
                \tikzstyle{level 1}=[sibling distance=60mm]
                \tikzstyle{level 2}=[sibling distance=15mm]
                \node {B \nodepart{two} D} [->]
                    child {node {A \nodepart{two} $\square$}}
                    child {node {C \nodepart{two} $\square$}}
                    child {node {H \nodepart{two} K}}
                ;
            \end{tikzpicture}
        \end{center} 
                \end{center}
                Es soll nun der Wert I eingefügt werden. Dies führt zu mehreren Splits und dem B-Baum
                \begin{center}
                    \begin{center}
            \begin{tikzpicture}
                \tikzstyle{bplus}=[rectangle split, rectangle split horizontal,
                    rectangle split ignore empty parts,draw]
                \tikzstyle{every node}=[bplus]
                \tikzstyle{level 1}=[sibling distance=60mm]
                \tikzstyle{level 2}=[sibling distance=15mm]
                \node {D \nodepart{two} $\square$} [->]
                    child {node {B \nodepart{two} $\square$}
                        child {node {A \nodepart{two} $\square$}}
                        child {node {C \nodepart{two} $\square$}}                    
                    }
                    child {node {I \nodepart{two} $\square$}
                        child {node {H \nodepart{two} $\square$}}
                        child {node {K \nodepart{two} $\square$}}                    
                    }                
                ;
            \end{tikzpicture}
        \end{center} 
                \end{center}
                Nachdem $I$ eingefügt wurde, war der Knoten $H,I,K$ überbelegt. Das mittlere Element wird nach oben geschoben. Auch dort ist der Knoten nun überbelegt und wird gesplittet. Dabei wächst die Tiefe des Baums um 1. 
                
            \subsubsection{Löschen mit Merge}
                Es wird zu beginn vom folgenden B-Baum ausgegangen:
                \begin{center}
                    \begin{center}
            \begin{tikzpicture}
                \tikzstyle{bplus}=[rectangle split, rectangle split horizontal,
                    rectangle split ignore empty parts,draw]
                \tikzstyle{every node}=[bplus]
                \tikzstyle{level 1}=[sibling distance=60mm]
                \tikzstyle{level 2}=[sibling distance=15mm]
                \node {B \nodepart{two} D} [->]
                    child {node {A \nodepart{two} $\square$}}
                    child {node {C \nodepart{two} $\square$}}
                    child {node {I \nodepart{two} K}}
                ;
            \end{tikzpicture}
        \end{center} 
                \end{center}
                Es soll nun der Wert B gelöscht werden
                \begin{center}
                           \begin{center}
            \begin{tikzpicture}
                \tikzstyle{bplus}=[rectangle split, rectangle split horizontal,
                    rectangle split ignore empty parts,draw]
                \tikzstyle{every node}=[bplus]
                \tikzstyle{level 1}=[sibling distance=60mm]
                \tikzstyle{level 2}=[sibling distance=15mm]
                \node {D \nodepart{two} $\square$} [->]
                    child {node {A \nodepart{two} C}}
                    child {node {I \nodepart{two} K}}
                ;
            \end{tikzpicture}
        \end{center} 
                \end{center}
                Nachdem B gelöscht wird, müssen Kindknoten gemerged werden.
                

    \section{Randomisierte Suchbäume}
        \begin{itemize}
            \item Treaps(Aragorn/Seidel)
            \item Skiplists (Pugh)
        \end{itemize}
        
        \subsection{Skiplists} 
            Gegeben: $S = \{ x_1, \ldots, x_n \}$ , $p=(\frac{1}{2})$ \\            
            Definiere Datenstruktur bezüglich Leveling aus Listen: \\
            Füge $\pm \infty$ als Spezialelemente in alle Listen ein. Elemente der Liste $L_i$ werden von links nach rechts verbunden. Verbinde zusätzlich alle Elemente $x$ vom höchsten Level bis Level 1. Element $x$ bildet also einen Turm mit $l(x)$ Knoten. Wir haben nun eine Baumstruktur mit Intervallen, die auf Level $i+1$ mehrere Intervalle auf Level $i$ umfassen. Dabei sei $c(I)$ die Zahl der Kinder von Intervall $I$           
            
            $S= L_1 \supseteq L_2 \supseteq \ldots \supseteq L_r=\emptyset$
            
            \subsubsection{Zufälliges Leveling:}
            
            
%            % 2013 anders, s.u.
%            \begin{alltt}
%            Starte mit \( L_1 = S \)  \\
%            Gehe zu \( L_2 \). Mache Münzwurf für jedes Element aus dem Level darunter
%            und kopiere es hoch, mit Wahrscheinlichkeit \( p \). \\
%            Level \( t \) sei das höchste mit  \( x \). Die Wahrscheinlichkeit dass Schlüssel \( x \)\\
%            auf höchstem Level \( k \) ist, ist 
%            $$p^{k-1} (1-p) $$
%            \end{alltt}
            
            $L_{i+1}$ ergibt sich aus $L_i$ indem jedes Element $x\in L_i$ mit Wahrscheinlichkeit $p=\frac{1}{2}$ an $L_{i+1}$ weitergereicht wird.
            $l(x)$ für $x\in S$ sind unabhängige Zufallsvariable nach einer Geometrischen Verteilung mit $p=\frac{1}{2}$.
            
            \emph{Geometische Verteilung:}\\
            Wirf Münze, bis Zahl fällt. X ist Anzahl der benötigten Würfe.
            
            $E(X)=\frac{1}{p}$, $Var(X)=\frac{1-p}{p}$\\\\
            
            Erwarteter Platzbedarf für zufällige Skiplist mit $|S|=n$ ist also $2n=\LO(n)$
            
            \begin{lemma}
            	Zahl der Level r bei zufälligem Leveling hat Erwartungswert $E(r)=\LO(\log n)$. Es gilt $r=\LO(\log n)$ mit hoher Wahrscheinlichkeit.
            \end{lemma}
            \begin{proof}
            	Sei $r=\max\{l(x)|x\in S\}$ und $l(x)$ sind geometrisch verteilte Zufallsvariablen mit $p=\frac{1}{2}$.
            	Sei $X_i$ die Zufallsvariable für $x_i\in S$.\\
            	Es gilt:\\
            	$P( X_i>t)<(1-p)^t$ und $P(\max\limits_{i} X_i>t)<n\cdot (1-p)^t=\frac{n}{2^t}$ für $p=\frac{1}{2}$\\
            	Mit $t=\alpha\log n$ und $r=1+\max\limits_i X_i \Ra P(r>\alpha\log n)\le \frac{1}{n^{\alpha\log n -1}}$ 
            \end{proof}


%		%quite strange...
%            Erwartungswert: 
%            $$\sum \limits_{k=1}^\infty k \cdot p^{k-1} (1-p) = (1-p) \sum \limits_{k=1}^\infty k \cdot p^{k-1} = (1-p) \cdot \frac{1}{(1-p)^2} = \frac{1}{1-p}$$
%
%            \underline{höchster Level:}(erwartet) \\
%            Sei $v = \max \{L(t) \mid t \in S \}$ \\
%            $L(x)$ hat den Erwartungswert $\frac{1}{1-p}$ \\
%            Wahrscheinlichkeit, dass $L(t) > t$ ist, ist $< p^t$ \\
%            Levelzuweisung der Elemente ist unabhängig, d.h. die Wahrscheinlichkeit dass ein Element (aus $n$ vielen) Level $2t$ hat, ist $< n \cdot p^t$ \\\
%            Setze $t = c \cdot \log n$, $p= \frac{1}{2}$ \\
%            $\Ra \textit{prob}(v > c \cdot \log n) \leq \frac{n}{2^{c \log n}} < \frac{1}{n}$ für $c > 1 $ \\
%            Höhe des Skipbits ist demnach $O (\log n)$ \\\\
%
%			%2013 anders
%            \emph{Suchen(x)} \\
%
%            Starte im Level $v$ mit Header. Laufe nach rechts in $L_v$, bis Element $> x$ gefunden wird. Im Element zuvor gehe nach unten. Iteriere. \\
%            Suche durchläuft Skipbits von rechts oder hin zu $x$. Strategie: Gehe möglichst nach rechtst, wenns nicht mehr geht, gehe nach unten. \\

	   \subsubsection{Suchen(x)}
	   	  laufe Intervalle levelweise (r$\ra$ 0) ab:\\
	   	  Sei $I_j(x)$ das Intervall auf Level j, das x enthält.\\
	   	  Liegt x auf dem Rand von zwei Intervallen, weise x dem linken Intervall zu.\\
	   	  Folge $I_r(x)\supseteq I_{r-i}(X)\supseteq\dots\supseteq I_1(x)$ lässt sich als Pfad von Wurzel zum Blatt ansehen.

        \subsubsection{Laufzeit}
        	    $$
        	        \LO\left(\sum\limits_{j=0}^r 1 + c\left(I_j(y)\right)\right)
        	    $$
            Anzahl Level: $\LO(\log n)$ , Anzahl Schritte nach unten. \\
            Der Suchpfad von $x$ zurück zu Level $v$ wird beschrieben durch: \\
            Gehe nach oben, wenn möglich; wenn nicht gehe nach links. \\
            Schritt nach links gibt es mit Wahrscheinlichkeit $\frac{1}{2}$ jeweils $\Ra$ Erwartete Anzahl der Schritte nach links ist
            genauso groß, wie erwartete Anzahl der Schritte nach oben. $\Ra$ Laufzeit $\LO(\log n)$ 
        
        \subsubsection{Einfügen(x)}
            Suche zuerst nach $x$. Füge es in $L_1$ ein und hänge zwei Zeiger um $\Ra \LO(\log n)$\\
            Laufe hoch und mache Zufallsexperiment, füge nacheinander eventuell Kopien ein.\\
            Ist $l(x)>r$, kreiere neues Level, $l(x)\gets r+1$, $r\gets r+1$. 
            Bestimme Suchpfade durch Intervalle von $l(x)$ aus und spalte durch Einfügen von x.
            %Gemeint ist, dass x bisher bestehende Intervalle teilt und so neue verknüpfungen nötig werden. Bessere Formulierung?
            Erwartete Zahl von Kopie ist 2 $\Ra \LO(1)$ erwartet. \\
            $\Ra $gesamt $ \LO(\log n)$
        
        \subsubsection{Streichen(x)}
            Während dem Suchlauf nach $x$ lösche von oben nach unten alle Vorkommen von $x$ durch Zeigerumhängen. Laufzeit: $\LO(\log n)$.\\
            
            \begin{satz}
        	        In zufälliger Skiplist für S der Größe n können Suchen/ Einfügen/ Streichen in erwarteter Zeit von $\LO(\log n)$ ausgeführt werden.
            \end{satz}
        
        \subsubsection{Weitere Operationen:}
        	    \begin{itemize}
        	  	    \item[join($S_1,x,S_2$):] ersetzte $S_1,S_2$ durch $S_1\cup\{x\}\cup S_2$ und $\forall y\in S_1, z\in S_2 : y<x<z$\\
        	      		Idee: hänge $L_i(S_1),L_i(S_2)$ aneinander $\forall i$ und füge dann x ein
        	      	\item[paste($S_1,S_2$):] analog
        	      	\item[split($x,S$):] Umkehrung von join()\\
        	      		Idee: streiche x, verweise auf $\pm\infty$ statt nächstes Element
        	    \end{itemize}

		    Alles in $\LO(\log n)$!
		    
		\subsubsection{Beispiel}
		    Es sollen beispielhaft die Elemente
            $$
                23, 16, 18, 35, 11, 28, 56, 13, 14
            $$
            mit Hilfe der Zufallsbitfolge
            $$
                1010011001010001101    
            $$
            in eine Skiplist eingefügt werden. Dies ergibt die folgende Skiplist.
            \begin{center}
                    \begin{tikzpicture}
        [->,>=stealth',shorten >=1pt,
        auto,minimum size=1.2cm,node distance=2cm,
        main node/.style={circle,draw},
        every node/.style={transform shape},
        scale=0.7]

        \node[main node] (ninfty_0) {$-\infty$};
            \node[main node] (ninfty_1)[above of=ninfty_0]  {$-\infty$};
            \node[main node] (ninfty_2)[above of=ninfty_1]  {$-\infty$};
            \node[main node] (ninfty_3)[above of=ninfty_2]  {$-\infty$};
        \node[main node] (11_0)[right of=ninfty_0] {$11$};
            \node[main node] (11_1)[above of=11_0] {$11$};
            \node[main node] (11_2)[above of=11_1] {$11$};
        \node[main node] (13_0)[right of=11_0] {$13$};
        \node[main node] (14_0)[right of=13_0] {$14$};
            \node[main node] (14_1)[above of=14_0] {$14$};
        \node[main node] (16_0)[right of=14_0] {$16$};
            \node[main node] (16_1)[above of=16_0] {$16$};
        \node[main node] (18_0)[right of=16_0] {$18$};
            \node[main node] (18_1)[above of=18_0] {$18$};
            \node[main node] (18_2)[above of=18_1] {$18$};
        \node[main node] (23_0)[right of=18_0] {$23$};
        \node[main node] (28_0)[right of=23_0] {$28$};
            \node[main node] (28_1)[above of=28_0] {$28$};
        \node[main node] (35_0)[right of=28_0] {$35$};
        \node[main node] (56_0)[right of=35_0] {$56$};
            \node[main node] (56_1)[above of=56_0] {$56$};
            \node[main node] (56_2)[above of=56_1] {$56$};
            \node[main node] (56_3)[above of=56_2] {$56$};
        \node[main node] (infty_0)[right of=56_0] {$\infty$};
            \node[main node] (infty_1)[above of=infty_0]  {$\infty$};
            \node[main node] (infty_2)[above of=infty_1]  {$\infty$};
            \node[main node] (infty_3)[above of=infty_2]  {$\infty$};

        % Vertikal
        
        %infty
        \path (infty_3) edge (infty_2)
        (infty_2) edge (infty_1)
        (infty_1) edge (infty_0);
        
        %-infty
        \path (ninfty_3) edge (ninfty_2)
        (ninfty_2) edge (ninfty_1)
        (ninfty_1) edge (ninfty_0);

        % 11
        \path (11_2) edge (11_1)
              (11_1) edge (11_0);

        % 14
        \path (14_1) edge (14_0);

        % 16
        \path (16_1) edge (16_0);

        % 18
        \path (18_2) edge (18_1)
              (18_1) edge (18_0);

        % 28
        \path (28_1) edge (28_0);

        % 56
        \path (56_3) edge (56_2)
              (56_2) edge (56_1)
              (56_1) edge (56_0);

        % Horizontal        
        %base
        \path
        (ninfty_0) edge (11_0)
        (11_0) edge (13_0)
        (13_0) edge (14_0)
        (14_0) edge (16_0)
        (16_0) edge (18_0)
        (18_0) edge (23_0)
        (23_0) edge (28_0)
        (28_0) edge (35_0)
        (35_0) edge (56_0)
        (56_0) edge (infty_0)
        ;
        % level 1
        \path
            (ninfty_1) edge (11_1)
            (11_1) edge (14_1)
            (14_1) edge (16_1)
            (16_1) edge (18_1)
            (18_1) edge (28_1)
            (28_1) edge (56_1)
            (56_1) edge (infty_1)
            ;

        % level 2
        \path
            (ninfty_2) edge (11_2)
            (11_2) edge (18_2)
            (18_2) edge (56_2)
            (56_2) edge (infty_2)
            ;

         % level 3
        \path
            (ninfty_3) edge (56_3)
            (56_3) edge (infty_3)
            ;
        
    \end{tikzpicture}
            \end{center}